\documentclass[12pt]{article}
\usepackage{pmmeta}
\pmcanonicalname{MyhillNerodeTheoremForSemigroups}
\pmcreated{2014-05-14 17:29:14}
\pmmodified{2014-05-14 17:29:14}
\pmowner{Ziosilvio}{18733}
\pmmodifier{Ziosilvio}{18733}
\pmtitle{Myhill-Nerode theorem for semigroups}
\pmrecord{5}{41748}
\pmprivacy{1}
\pmauthor{Ziosilvio}{18733}
\pmtype{Theorem}
\pmcomment{trigger rebuild}
\pmclassification{msc}{20M35}
\pmclassification{msc}{68Q70}
\pmdefines{recognizable subset of a semigroup}

% this is the default PlanetMath preamble.  as your knowledge
% of TeX increases, you will probably want to edit this, but
% it should be fine as is for beginners.

% almost certainly you want these
\usepackage{amssymb}
\usepackage{amsmath}
\usepackage{amsfonts}

% used for TeXing text within eps files
%\usepackage{psfrag}
% need this for including graphics (\includegraphics)
%\usepackage{graphicx}
% for neatly defining theorems and propositions
%\usepackage{amsthm}
% making logically defined graphics
%%%\usepackage{xypic}

% there are many more packages, add them here as you need them

% define commands here

\begin{document}
\newtheorem{theorem}{Theorem}

Let $S$ be a semigroup.
$X\subseteq S$ is \emph{recognizable}
if it is union of classes of a congruence $\chi$
such that $S/\chi$ is finite.

In the rest of the entry,
$\equiv_X$ will be the syntactic congruence of $X$,
and $\mathcal{N}_X$ its Nerode equivalence.
\begin{theorem}[Myhill-Nerode theorem for semigroups] \label{thm:mn}
Let $S$ be a semigroup and let $X\subseteq S$.
The following are equivalent.
\begin{enumerate}
\item \label{it:rec}
$X$ is recognizable.
\item \label{it:morph}
There exist a finite semigroup $T$ and a morphism $\phi:S\to T$
such that $X=\phi^{-1}(\phi(X))$.
\item \label{it:syn}
The syntactic semigroup of $X$ is finite.
\item \label{it:right}
There exists an equivalence relation $\sim$ on $S$ such that
\begin{itemize}
\item $S/\sim$ is finite and
\item $s_1\sim s_2$ implies $s_1t\sim s_2t$.
\end{itemize}
\item \label{it:nerode}
The quotient set $S/\mathcal{N}_X$ is finite.
\end{enumerate}
\end{theorem}
Theorem~\ref{thm:mn} generalizes Myhill-Nerode theorem for languages
to subsets of generic, not necessarily free, semigroups.
In fact, as a consequence of Theorem~\ref{thm:mn},
a language on a finite alphabet is recognizable in the sense given above
if and only if it is recognizable by a DFA.

The equivalence of points~\ref{it:rec}, \ref{it:morph} and~\ref{it:syn}
is attributed to John Myhill,
while the equivalence of points~\ref{it:rec}, \ref{it:right} and~\ref{it:nerode}
is attributed to Anil Nerode.

\emph{Proof of Theorem~\ref{thm:mn}.}

\ref{it:rec} $\Rightarrow$ \ref{it:morph}.
Given a congruence $\chi$ such that
$|S/\chi|<\infty$ and $X$ is union of classes of $\chi$,
choose $T$ as the quotient semigroup $S/\chi$
and $\phi$ as the natural homomorphism mapping $s$ to $[s]_\chi$.

\ref{it:morph} $\Rightarrow$ \ref{it:syn}.
Given a morphism of semigroups $\phi:S\to T$
with $T$ finite and $X=\phi^{-1}(\phi(X))$,
put $s\chi t$ iff $\phi(s)=\phi(t)$.

Since $\phi$ is a morphism, $\chi$ is a congruence;
moreover, $X=\phi^{-1}(\phi(X))$ means that
$X$ is union of classes of $\chi$-equivalence.
By the maximality property of syntactic congruence,
the number of classes of $\equiv_X$
does not exceed the number of classes of $\chi$,
which in turn does not exceed the number of elements of $T$.

\ref{it:syn} $\Rightarrow$ \ref{it:rec}.
Straightforward from $X$ being union of classes of $\equiv_X$.

\ref{it:syn} $\Rightarrow$ \ref{it:right}.
Straightforward from $\equiv_X$ satisfying the second requirement.

\ref{it:right} $\Rightarrow$ \ref{it:nerode}.
Straightforward from the maximality property of Nerode equivalence.

\ref{it:nerode} $\Rightarrow$ \ref{it:syn}.
Let
\begin{equation} \label{eq:nerode-finite}
S/\mathcal{N}_X
=\left\{[\xi_1]_{\mathcal{N}_X},\ldots,[\xi_K]_{\mathcal{N}_X}\right\}\;.
\end{equation}
Since $s_1\equiv_X s_2$ iff $\ell s_1\mathcal{N}_X \ell s_2$
for every $\ell \in S$,
determining $[s]_{\equiv_X}$
is the same as determining $[\ell s]_{\mathcal{N}_X}$
as $\ell$ varies in $S$,
\emph{plus the class $[s]_{\mathcal{N}_X}$}.
(This additional class takes into account the possibility
that $s \not \in [\ell s]_{\mathcal{N}_X}$ for any $\ell \in S$,
which cannot be excluded \emph{a priori}:
do not forget that $S$ is not required to be a monoid.)

But since $s_1\mathcal{N}_Xs_2$ implies $s_1t\mathcal{N}_Xs_2t$,
if $[\ell_1]_{\mathcal{N}_X}=[\ell_2]_{\mathcal{N}_X}$
then $[\ell_1 s]_{\mathcal{N}_X}=[\ell_2 s]_{\mathcal{N}_X}$ as well.
To determine $[s]_{\equiv_X}$
it is thus sufficient to determine the $[\xi_i s]_{\mathcal{N}_X}$
for $1\leq i\leq K$,
plus $[s]_{\mathcal{N}_X}$.

We can thus identify the class $[s]_{\equiv_X}$
with the sequence
\begin{math}
\left(
[\xi_1 s]_{\mathcal{N}_X},\ldots,[\xi_K s]_{\mathcal{N}_X},[s]_{\mathcal{N}_X}
\right).
\end{math}
Then the number of classes of $\equiv_X$
cannot exceed that of $(K+1)$-ples of classes of $\mathcal{N}_X$,
which is $K^{K+1}$.
\hfill$\Box$

\begin{thebibliography}{99}
A. de Luca and S. Varricchio.
\emph{Finiteness and Regularity in Semigroups and Formal Languages.}
Springer Verlag, Heidelberg 1999.
\end{thebibliography}

%%%%%
%%%%%
\end{document}
