\documentclass[12pt]{article}
\usepackage{pmmeta}
\pmcanonicalname{SymmetricGroupOnThreeLetters}
\pmcreated{2013-03-22 15:52:24}
\pmmodified{2013-03-22 15:52:24}
\pmowner{Wkbj79}{1863}
\pmmodifier{Wkbj79}{1863}
\pmtitle{symmetric group on three letters}
\pmrecord{13}{37870}
\pmprivacy{1}
\pmauthor{Wkbj79}{1863}
\pmtype{Example}
\pmcomment{trigger rebuild}
\pmclassification{msc}{20B30}

% this is the default PlanetMath preamble.  as your knowledge
% of TeX increases, you will probably want to edit this, but
% it should be fine as is for beginners.

% almost certainly you want these
\usepackage{amssymb}
\usepackage{amsmath}
\usepackage{amsfonts}

% used for TeXing text within eps files
%\usepackage{psfrag}
% need this for including graphics (\includegraphics)
%\usepackage{graphicx}
% for neatly defining theorems and propositions
%\usepackage{amsthm}
% making logically defined graphics
%%%\usepackage{xypic}

% there are many more packages, add them here as you need them

% define commands here
\begin{document}
\PMlinkescapeword{block}
\PMlinkescapeword{blocks}
\PMlinkescapeword{calculate}
\PMlinkescapeword{complete}
\PMlinkescapeword{right}
\PMlinkescapeword{similar}

This example is of the symmetric group on $3$ letters, usually denoted by $S_3$.  Here, we are considering the set of bijective functions on the set $A=\{1,2,3\}$ which naturally arise as the set of permutations on $A$.  Our binary operation is function composition which results in a new bijective function.  This example develops the \PMlinkescapetext{multiplication} table for $S_3$.  We start by listing the elements of our group.  These elements are listed according to the second method as described in the entry on permutation notation.

$$e={ 1\ 2\ 3 \choose 1\ 2\ 3} \hspace{20mm}  r ={ 1\ 2\ 3 \choose 2\ 1\ 3}$$
$$a={ 1\ 2\ 3 \choose 2\ 3\ 1} \hspace{20mm} s ={ 1\ 2\ 3 \choose 3\ 2\ 1}$$
$$b={ 1\ 2\ 3 \choose 3\ 1\ 2} \hspace{20mm} t ={ 1\ 2\ 3 \choose 1\ 3\ 2}$$

Here, our group is just $S_3=\{e,a,b,r,s,t\}$.  Now we can start to multiply and then fill in the table.
First, we calculate the square of each element.

$$a^2={ 1\ 2\ 3 \choose 2\ 3\ 1}{ 1\ 2\ 3 \choose 2\ 3\ 1}={ 1\ 2\ 3 \choose 3\ 1\ 2}= b$$
$$b^2={ 1\ 2\ 3 \choose 3\ 1\ 2}{ 1\ 2\ 3 \choose 3\ 1\ 2}={ 1\ 2\ 3 \choose 2\ 3\ 1}= a$$
$$r^2={ 1\ 2\ 3 \choose 2\ 1\ 3}{ 1\ 2\ 3 \choose 2\ 1\ 3}={ 1\ 2\ 3 \choose 1\ 2\ 3}=e$$
$$s^2={ 1\ 2\ 3 \choose 3\ 2\ 1}{ 1\ 2\ 3 \choose 3\ 2\ 1}={ 1\ 2\ 3 \choose 1\ 2\ 3}=e$$
$$t^2={ 1\ 2\ 3 \choose 1\ 3\ 2}{ 1\ 2\ 3 \choose 1\ 3\ 2}={ 1\ 2\ 3 \choose 1\ 2\ 3}=e$$

Next, we will fill in the upper right $3\operatorname{x}3$ block, we only need $ab$ and $ba$ since we can use the fact that there can be no repetition in any row or column.

$$ ab ={ 1\ 2\ 3 \choose 2\ 3\ 1}{ 1\ 2\ 3 \choose 3\ 1\ 2}={ 1\ 2\ 3 \choose 1\ 2\ 3}=e$$
$$ ba ={ 1\ 2\ 3 \choose 3\ 1\ 2}{ 1\ 2\ 3 \choose 2\ 3\ 1}={ 1\ 2\ 3 \choose 1\ 2\ 3}= e$$

The other $3\operatorname{x}3$ blocks are also similar.  Now continuing with the upper left 3 x 3 block, we go through the table again using the fact that there can be no repetition in any row or column.

$$ ar ={ 1\ 2\ 3 \choose 2\ 3\ 1}{ 1\ 2\ 3 \choose 2\ 1\ 3}={ 1\ 2\ 3 \choose 3\ 2\ 1}= s$$
$$ as ={ 1\ 2\ 3 \choose 2\ 3\ 1}{ 1\ 2\ 3 \choose 3\ 2\ 1}={ 1\ 2\ 3 \choose 1\ 3\ 2}= t$$

Similarly, we complete the final blocks of the table.

$$ ra ={ 1\ 2\ 3 \choose 2\ 1\ 3}{ 1\ 2\ 3 \choose 2\ 3\ 1}={ 1\ 2\ 3 \choose 1\ 3\ 2}= t$$
$$ rb ={ 1\ 2\ 3 \choose 2\ 1\ 3}{ 1\ 2\ 3 \choose 3\ 1\ 2}={ 1\ 2\ 3 \choose 3\ 2\ 1}= s$$

$$ sa ={ 1\ 2\ 3 \choose 3\ 2\ 1}{ 1\ 2\ 3 \choose 2\ 3\ 1}={ 1\ 2\ 3 \choose 2\ 1\ 3}= r$$
$$ sr ={ 1\ 2\ 3 \choose 3\ 2\ 1}{ 1\ 2\ 3 \choose 2\ 1\ 3}={ 1\ 2\ 3 \choose 2\ 3\ 1}= a$$

Finally, we fill in the table using the calculated values above.

\begin{center}
$\begin{array}{|c||c|c|c|c|c|c|}
\hline
 & e & a & b & r & s & t \\
\hline \hline
e & e & a & b & r & s & t \\
\hline
a & a & b & e & s & t & r \\
\hline
b & b & e & a & t & r & s \\
\hline
r & r & t & s & e & b & a \\
\hline
s & s & r & t & a & e & b \\
\hline
t & t & s & r & b & a & e \\
\hline \end{array}$
\end{center}
%%%%%
%%%%%
\end{document}
