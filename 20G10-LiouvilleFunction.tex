\documentclass[12pt]{article}
\usepackage{pmmeta}
\pmcanonicalname{LiouvilleFunction}
\pmcreated{2013-03-22 11:47:09}
\pmmodified{2013-03-22 11:47:09}
\pmowner{KimJ}{5}
\pmmodifier{KimJ}{5}
\pmtitle{Liouville function}
\pmrecord{12}{30257}
\pmprivacy{1}
\pmauthor{KimJ}{5}
\pmtype{Definition}
\pmcomment{trigger rebuild}
\pmclassification{msc}{20G10}
\pmclassification{msc}{11A25}
\pmclassification{msc}{81-00}
%\pmkeywords{number theory}

\endmetadata

\usepackage{amssymb}
\usepackage{amsmath}
\usepackage{amsfonts}
\usepackage{graphicx}
%%%%\usepackage{xypic}
\begin{document}
The \emph{Liouville function} is defined by $\lambda (1) = 1$ and $\lambda (n) = (-1)^{k_1 + k_2 + \cdots + k_r}$, if the prime factorization of $n > 1$ is $n = p_1^{k_1} p_2^{k_2} \cdots p_r^{k_r}$ (where each $p_i$ is positive). This function is completely multiplicative and \PMlinkescapetext{satisfies} the \PMlinkescapetext{identity}
\[
\sum_{d|n} \lambda (d) =
\begin{cases}
1 &\text{if $n=m^2$ for some integer $m$}\\
0 &\text{otherwise,}
\end{cases}
\]
where the sum runs over all positive divisors of $n$.
%%%%%
%%%%%
%%%%%
%%%%%
\end{document}
