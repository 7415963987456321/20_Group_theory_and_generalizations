\documentclass[12pt]{article}
\usepackage{pmmeta}
\pmcanonicalname{ProofOfClassEquationTheorem}
\pmcreated{2013-03-22 14:20:52}
\pmmodified{2013-03-22 14:20:52}
\pmowner{gumau}{3545}
\pmmodifier{gumau}{3545}
\pmtitle{proof of class equation theorem}
\pmrecord{4}{35823}
\pmprivacy{1}
\pmauthor{gumau}{3545}
\pmtype{Proof}
\pmcomment{trigger rebuild}
\pmclassification{msc}{20D20}

\endmetadata

% this is the default PlanetMath preamble.  as your knowledge
% of TeX increases, you will probably want to edit this, but
% it should be fine as is for beginners.

% almost certainly you want these
\usepackage{amssymb}
\usepackage{amsmath}
\usepackage{amsfonts}

% used for TeXing text within eps files
%\usepackage{psfrag}
% need this for including graphics (\includegraphics)
%\usepackage{graphicx}
% for neatly defining theorems and propositions
%\usepackage{amsthm}
% making logically defined graphics
%%%\usepackage{xypic}

% there are many more packages, add them here as you need them

% define commands here
\begin{document}
$X$ is a finite disjoint union of finite orbits:
$X = \cup_{i}Gx_{i}$. We can separate this union by considerating first only the orbits of 1 element and then the rest:
$X = \cup_{j=1}^{l}\{x_{i_{j}}\} \cup \cup_{k=1}^{s}Gx_{i_{k}}=G_{X} \cup_{k=1}^{s}Gx_{i_{k}}$
Then using the orbit-stabilizer theorem, we have $\#X=\#G_{X} + \sum_{k=1}^{s}[G:G_{x_{i_{k}}}]$ where for every $k$, $[G:G_{x_{i_{k}}}]\geq 2$, because if one of them were 1, then it would be associated to an orbit of 1 element, but we counted those orbits first. Then this stabilizers are not $G$. This finishes the proof.
%%%%%
%%%%%
\end{document}
