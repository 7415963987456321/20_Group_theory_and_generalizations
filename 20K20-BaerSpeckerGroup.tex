\documentclass[12pt]{article}
\usepackage{pmmeta}
\pmcanonicalname{BaerSpeckerGroup}
\pmcreated{2013-03-22 15:29:18}
\pmmodified{2013-03-22 15:29:18}
\pmowner{CWoo}{3771}
\pmmodifier{CWoo}{3771}
\pmtitle{Baer-Specker group}
\pmrecord{13}{37344}
\pmprivacy{1}
\pmauthor{CWoo}{3771}
\pmtype{Definition}
\pmcomment{trigger rebuild}
\pmclassification{msc}{20K20}
\pmsynonym{Specker group}{BaerSpeckerGroup}

\endmetadata

\usepackage{amssymb,amscd}
\usepackage{amsmath}
\usepackage{amsfonts}

% used for TeXing text within eps files
%\usepackage{psfrag}
% need this for including graphics (\includegraphics)
%\usepackage{graphicx}
% for neatly defining theorems and propositions
%\usepackage{amsthm}
% making logically defined graphics
%%%\usepackage{xypic}

% define commands here
\begin{document}
Let $A$ be a non-empty set, and $G$ an abelian group.  The set $K$
of all functions from $A$ to $G$ is an abelian group, with addition
defined elementwise by $(f+g)(x)=f(x)+g(x)$.  The zero element is
the function that sends all elements of $A$ into $0$ of $G$, and the
negative of an element $f$ is a function defined by
$(-f)(x)=-(f(x))$.

When $A=\mathbb{N}$, the set of natural numbers, and $G=\mathbb{Z}$,
$K$ as defined above is called the \emph{Baer-Specker group}.  Any
element of $K$, being a function from $\mathbb{N}$ to $\mathbb{Z}$,
can be expressed as an infinite sequence $(
x_1,x_2,\ldots,x_n,\ldots)$, and the elementwise addition on $K$ can
be realized as componentwise addition on the sequences: $$(
x_1,x_2,\ldots,x_n,\ldots)+( y_1,y_2,\ldots,y_n,\ldots)=
(x_1+y_1,x_2+y_2,\ldots,x_n+y_n,\ldots).$$ An alternative
characterization of the Baer-Specker group $K$ is that it can be
viewed as the countably infinite direct product of copies of
$\mathbb{Z}$:
$$K=\mathbb{Z}^{\mathbb{N}}\cong\mathbb{Z}^{\aleph_0}= \prod_{\aleph_0}\mathbb{Z}.$$

The Baer-Specker group is an important example of a torsion-free
abelian group whose rank is infinite.  It is not a free abelian
group, but any of its countable subgroup is free (abelian).

\begin{thebibliography}{6}
\bibitem{dh} P. A. Griffith, {\it Infinite Abelian Group Theory}, The University of Chicago Press (1970)
\end{thebibliography}
%%%%%
%%%%%
\end{document}
