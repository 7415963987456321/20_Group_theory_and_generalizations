\documentclass[12pt]{article}
\usepackage{pmmeta}
\pmcanonicalname{ExamplesOfGroups}
\pmcreated{2013-03-22 12:49:19}
\pmmodified{2013-03-22 12:49:19}
\pmowner{AxelBoldt}{56}
\pmmodifier{AxelBoldt}{56}
\pmtitle{examples of groups}
\pmrecord{34}{33144}
\pmprivacy{1}
\pmauthor{AxelBoldt}{56}
\pmtype{Example}
\pmcomment{trigger rebuild}
\pmclassification{msc}{20-00}
\pmclassification{msc}{20A05}
\pmrelated{ExamplesOfFiniteSimpleGroups}
\pmrelated{SpinGroup}
\pmrelated{ExamplesOfAlgebraicKTheoryGroups}
\pmrelated{QuantumGroups}
\pmrelated{GroupsOfSmallOrder}
\pmrelated{TriangleGroups}

% this is the default PlanetMath preamble.  as your knowledge
% of TeX increases, you will probably want to edit this, but
% it should be fine as is for beginners.

% almost certainly you want these
\usepackage{amssymb}
\usepackage{amsmath}
\usepackage{amsfonts}

% used for TeXing text within eps files
%\usepackage{psfrag}
% need this for including graphics (\includegraphics)
%\usepackage{graphicx}
% for neatly defining theorems and propositions
%\usepackage{amsthm}
% making logically defined graphics
%%%\usepackage{xypic} 

% there are many more packages, add them here as you need them

% define commands here
\begin{document}
\PMlinkescapeword{natural}
\PMlinkescapeword{algebraic}
\PMlinkescapeword{structure}
\PMlinkescapeword{relations}
\PMlinkescapeword{degree}
\PMlinkescapeword{groups}
\PMlinkescapeword{group}
\PMlinkescapeword{information}
\PMlinkescapeword{axiom}

  \PMlinkname{Groups}{Group} are ubiquitous throughout mathematics. Many ``naturally
  occurring'' groups are either groups of numbers (typically Abelian)
  or groups of symmetries (typically non-Abelian).

\paragraph{Groups of numbers}

\begin{itemize}

\item The most important group is the group of integers $\Bbb{Z}$ with
addition as operation and zero as identity element.

\item The integers modulo $n$, often denoted by $\Bbb{Z}_n$, form a
group under addition. Like $\Bbb{Z}$ itself, this is a cyclic group; any
cyclic group is isomorphic to one of these. 

\item The rational (or real, or complex) numbers form a group under
addition.

\item The positive rationals form a group under multiplication with identity element 1, and so
do the non-zero rationals. The same is true for the reals and real algebraic numbers.

\item The non-zero complex numbers form a group under multiplication.
So do the non-zero quaternions. The latter is our first example of a
non-Abelian group.

\item More generally, any (skew) field gives rise to two groups: the additive
group of all field elements with 0 as identity element, and the multiplicative group of all
non-zero field elements with 1 as identity element.

\item The complex numbers of absolute value 1 form a group under
multiplication, best thought of as the unit circle. The quaternions of
absolute value 1 form a group under multiplication, best thought of as
the three-dimensional unit sphere $S^3$. The two-dimensional sphere
$S^2$ however
is not a group in any natural way.

\item The positive integers less than $n$ which are
coprime to $n$ form a group if the operation is defined as
multiplication modulo $n$. This is an Abelian group whose order is given by the
Euler phi-function $\phi(n)$.

\item The units of the number ring $\mathbb{Z}[\sqrt{3}]$ form the multiplicative group consisting of all integer powers of $2+\sqrt{3}$ and their negatives (see units of quadratic fields).

\item Generalizing the last two examples, if $R$ is a ring with multiplicative identity 1, then the \PMlinkname{units of $R$}{GroupOfUnits} (the elements invertible with respect to multiplication) form a group with respect to ring multiplication and with identity element 1. See examples of rings.

\end{itemize}

Most groups of numbers carry natural topologies turning them into
topological groups.

\paragraph{Symmetry groups}

\begin{itemize}

\item The symmetric group of degree $n$, denoted by $S_n$, consists of
all permutations of $n$ items and has $n!$ elements. Every finite
group is isomorphic to a subgroup of some $S_n$ (Cayley's theorem).

\item An important subgroup of the symmetric group of degree $n$ is the 
alternating group, denoted $A_n$.  This consists of all \emph{even}
permutations on $n$ items.  A permutation is said to be even if it can
be written as the product of an even number of transpositions.  The alternating
group is normal in $S_n$, of index $2$, and it is an interesting fact that $A_n$
is simple for $n \geq 5$. See the proof on the simplicity of the alternating 
groups. See also examples of finite simple groups.

\item If any geometrical object is given, one can consider its
symmetry group consisting of all rotations and reflections which leave
the object unchanged. For example, the symmetry group of a cone is
isomorphic to $S^1$; the symmetry group of a square has eight elements and is isomorphic to the dihedral group $D_4$.

\item The set of all automorphisms of a given group (or field,
or graph, or topological space, or object in any category) forms a group with operation
given by the composition of homomorphisms. This is the automorphism group of the given object and captures its ``internal symmetries''.

\item In Galois theory, the symmetry groups of field extensions (or
equivalently: the symmetry groups of solutions to polynomial
equations) are the central object of study; they are called Galois groups.  One version of the inverse Galois problem asks whether every finite group can arise as the symmetry group of some algebraic extension of the rational numbers.  The answer is unknown.

\item Several matrix groups describe various aspects of the symmetry of $n$-space:
\begin{itemize}
\item The general linear group $\operatorname{GL}(n,\Bbb{R})$ of all real invertible
$n\times n$ matrices (with matrix multiplication as operation) contains
rotations, reflections, dilations, shear transformations, and their
combinations.
\item The orthogonal group $\operatorname{O}(n,\Bbb{R})$ of all real orthogonal
$n\times n$ matrices contains the rotations and reflections of $n$-space.
\item The special orthogonal group $\operatorname{SO}(n,\Bbb{R})$ of all real orthogonal
$n\times n$
matrices with determinant 1 contains the rotations of $n$-space.
\end{itemize}
 
All these matrix groups are Lie groups: groups which are differentiable manifolds
such that the group operations are smooth maps.

\end{itemize}

\paragraph{Other groups}

\begin{itemize}

\item The trivial group consists only of its identity element.

\item The Klein 4-group is a non-cyclic abelian group with four elements. For other small groups, see groups of small order.

\item If $X$ is a topological space and $x$ is a point of $X$, we can
define the fundamental group of $X$ at $x$. It consists of
(homotopy classes of) continuous
paths starting and ending at $x$ and describes the structure
of the ``holes'' in $X$ accessible from $x$. The fundamental group is generalized by the higher homotopy groups.

\item Other groups studied in algebraic topology are the homology groups of a topological space. In a different way, they also provide information about the ``holes'' of the space.

\item The free groups are important in algebraic topology. In a sense,
they are the most general groups, having only those relations among
their elements that are absolutely required by the group axioms. The free group on the set $S$ has as members all the finite strings that can be formed from elements of $S$ and their inverses; the operation comes from string concatenation.

\item If $A$ and $B$ are two Abelian groups (or modules over the same ring), then the set $\operatorname{Hom}(A,\,B)$ of all homomorphisms from $A$ to $B$ is an Abelian group. Note that the commutativity of $B$ is crucial here: without it, one couldn't prove that the sum of two homomorphisms is again a homomorphism.

\item Given any set $X$, the powerset ${\cal P}(X)$ of $X$ becomes an abelian group if we use the symmetric difference as operation. In this group, any element is its own inverse, which makes it into a vector space over $\Bbb{Z}_2$.

\item If $R$ is a ring with multiplicative identity, then the set of all invertible $n\times n$ matrices over $R$
forms a group under matrix multiplication with the identity matrix as identity element; this group is denoted by $\operatorname{GL}(n,R)$. It is the group of units of the ring of all $n\times n$ matrices over $R$.  For a given $n$, the groups $\operatorname{GL}(n,R)$ with commutative ring $R$ can be viewed as the points on the general linear group scheme $\operatorname{GL}_n$.

\item If $K$ is a number field, then multiplication of (equivalence classes of) non-zero ideals in the ring of algebraic integers $\cal{O}_K$ gives rise to the ideal class group of $K$.

\item The set of the equivalence classes of commensurability of the positive real numbers is an Abelian group with respect to the defined operation.

\item
The set of arithmetic functions that take a value other than 0 at 1 form an Abelian group under Dirichlet convolution.  They include as a subgroup the set of multiplicative functions.

\item Consider the curve $C=\{(x,\,y)\in K^2\mid y^2=x^3-x\}$, where $K$ is any field. Every straight line intersects this set in three points (counting a point twice if the line is tangent, and allowing for a point at infinity). If we require that those three points add up to zero for any straight line, then we have defined an abelian group structure on $C$. Groups like these are called abelian varieties.

\item Let $E$ be an elliptic curve defined over any field $F$. Then the set of $F$-rational points in the curve $E$, denoted by $E(F)$, can be given the structure of abelian group. If $F$ is a number field, then $E(F)$ is a finitely generated abelian group. The curve $C$ in the example above is an elliptic curve defined over $\mathbb{Q}$, thus $C(\mathbb{Q})$ is a finitely generated abelian group.

\item In the classification of all finite simple groups, several
``sporadic'' groups occur which don't follow any discernable pattern.
The largest of these is the monster group with about $ 8\cdot 10^{53} $
elements.

\end{itemize}
%%%%%
%%%%%
\end{document}
