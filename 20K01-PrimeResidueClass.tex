\documentclass[12pt]{article}
\usepackage{pmmeta}
\pmcanonicalname{PrimeResidueClass}
\pmcreated{2013-03-22 15:43:12}
\pmmodified{2013-03-22 15:43:12}
\pmowner{pahio}{2872}
\pmmodifier{pahio}{2872}
\pmtitle{prime residue class}
\pmrecord{18}{37668}
\pmprivacy{1}
\pmauthor{pahio}{2872}
\pmtype{Definition}
\pmcomment{trigger rebuild}
\pmclassification{msc}{20K01}
\pmclassification{msc}{13M99}
\pmclassification{msc}{11A07}
\pmsynonym{prime class}{PrimeResidueClass}
\pmrelated{MultiplicativeOrderOfAnIntegerModuloM}
\pmrelated{NonZeroDivisorsOfFiniteRing}
\pmrelated{GroupOfUnits}
\pmrelated{PrimitiveRoot}
\pmrelated{ResidueSystems}
\pmrelated{Klein4Group}
\pmrelated{EulerPhifunction}
\pmrelated{SummatoryFunctionOfArithmeticFunction}
\pmdefines{residue class group}

\endmetadata

% this is the default PlanetMath preamble.  as your knowledge
% of TeX increases, you will probably want to edit this, but
% it should be fine as is for beginners.

% almost certainly you want these
\usepackage{amssymb}
\usepackage{amsmath}
\usepackage{amsfonts}

% used for TeXing text within eps files
%\usepackage{psfrag}
% need this for including graphics (\includegraphics)
%\usepackage{graphicx}
% for neatly defining theorems and propositions
 \usepackage{amsthm}
% making logically defined graphics
%%%\usepackage{xypic}

% there are many more packages, add them here as you need them

% define commands here

\theoremstyle{definition}
\newtheorem*{thmplain}{Theorem}
\begin{document}
Let $m$ be a positive integer.  There are $m$ residue classes $a\!+\!m\mathbb{Z}$ modulo $m$.\, Such of them which have
                          $$\gcd(a,\,m) \;=\; 1,$$
are called the {\em prime residue classes} or {\em prime classes modulo $m$}, and they form an Abelian group with respect to the multiplication
   $$(a\!+\!m\mathbb{Z})\!\cdot\!(b\!+\!m\mathbb{Z}) \;:=\; ab\!+\!m\mathbb{Z}.$$
This group is called the {\em residue class group modulo $m$}.  Its order is $\varphi(m)$, where $\varphi$ means Euler's totient function.  For example, the prime classes modulo 8 (i.e. $1\!+\!8\mathbb{Z}$, $3\!+\!8\mathbb{Z}$, $5\!+\!8\mathbb{Z}$, $7\!+\!8\mathbb{Z}$) form a group isomorphic to the Klein 4-group.

The prime classes are the units of the residue class ring \,\,$\mathbb{Z}/m\mathbb{Z} = \mathbb{Z}_m$\,\, consisting of all residue classes modulo $m$.

Analogically, in the \PMlinkname{ring $R$ of integers}{ExamplesOfRingOfIntegersOfANumberField} of any algebraic number field, there are the residue classes and the prime residue classes modulo an ideal $\mathfrak{a}$ of $R$.  The number of all residue classes is 
$\mbox{N}(\mathfrak{a})$ and the number of the prime classes is also denoted by $\varphi(\mathfrak{a})$.\, It may be proved that
 $$\varphi(\mathfrak{a}) \;=\; 
\mbox{N}(\mathfrak{a})\prod_{\mathfrak{p}|\mathfrak{a}}\left(1-\frac{1}{\mbox{N}(\mathfrak{p})}\right);$$
$\mbox{N}$ is the absolute norm of ideal and $\mathfrak{p}$ runs all distinct prime ideals dividing $\mathfrak{a}$ (cf. the first formula in the entry ``\PMlinkname{Euler phi function}{EulerPhiFunction}'').\, Moreover, one has the result 
 $$\alpha^{\varphi(\mathfrak{a})} \;\equiv\; 1 \pmod{\mathfrak{a}}$$
for\, $((a),\,\mathfrak{a}) = (1)$,\, generalising the \PMlinkname{Euler--Fermat theorem}{EulerFermatTheorem}.
%%%%%
%%%%%
\end{document}
