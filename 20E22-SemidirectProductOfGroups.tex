\documentclass[12pt]{article}
\usepackage{pmmeta}
\pmcanonicalname{SemidirectProductOfGroups}
\pmcreated{2013-03-22 12:34:49}
\pmmodified{2013-03-22 12:34:49}
\pmowner{djao}{24}
\pmmodifier{djao}{24}
\pmtitle{semidirect product of groups}
\pmrecord{10}{32829}
\pmprivacy{1}
\pmauthor{djao}{24}
\pmtype{Definition}
\pmcomment{trigger rebuild}
\pmclassification{msc}{20E22}
\pmsynonym{semidirect product}{SemidirectProductOfGroups}
\pmsynonym{semi-direct product}{SemidirectProductOfGroups}

% this is the default PlanetMath preamble.  as your knowledge
% of TeX increases, you will probably want to edit this, but
% it should be fine as is for beginners.

% almost certainly you want these
\usepackage{amssymb}
\usepackage{amsmath}
\usepackage{amsfonts}

% used for TeXing text within eps files
%\usepackage{psfrag}
% need this for including graphics (\includegraphics)
%\usepackage{graphicx}
% for neatly defining theorems and propositions
\usepackage{amsthm}
% making logically defined graphics
%%%\usepackage{xypic} 

% there are many more packages, add them here as you need them

% define commands here
\newcommand{\C}{\mathbb{C}}
\newcommand{\N}{\mathbb{N}}
\newcommand{\Z}{\mathbb{Z}}
\newcommand{\Q}{\mathbb{Q}}
\newcommand{\R}{\mathbb{R}}
\renewcommand{\a}{\bf{a}}
\renewcommand{\b}{\bf{b}}
\renewcommand{\c}{\bf{c}}
\renewcommand{\d}{\bf{d}}
\renewcommand{\Re}{\operatorname{Re}}
\newcommand{\0}{\bf{0}}
\newcommand{\category}[1]{\mbox{\boldmath $\mathsf{{#1}}$}}
\renewcommand{\H}{\category{H}}
\newcommand{\A}{\category{A}}
\newcommand{\B}{\category{B}}
\newcommand{\<}{\langle}
\renewcommand{\>}{\rangle}
\newcommand{\lra}{\longrightarrow}
\newcommand{\ra}{\rightarrow}
\newcommand{\iso}{\cong}
\newcommand{\dsum}{\oplus}
\newcommand{\conv}{\circ}
\newcommand{\comp}{\circ}
\newcommand{\st}{\mid}
\newcommand{\intersect}{\cap}
\newcommand{\union}{\cup}
\renewcommand{\Im}{\operatorname{Im}}
\newcommand{\cross}{\times}
\newcommand{\Aut}{\operatorname{Aut}}
\newcommand{\semidirect}{\rtimes}

\newtheorem{theorem}{Theorem}
\newtheorem{proposition}[theorem]{Proposition}
\newtheorem{lemma}[theorem]{Lemma}
\newtheorem{corollary}[theorem]{Corollary}

\theoremstyle{definition}
\newtheorem{definition}[theorem]{Definition}
\begin{document}
The goal of this exposition is to carefully explain the correspondence
between the notions of external and internal semi--direct products of
groups, as well as the connection between semi--direct products and
short exact sequences.

Naturally, we start with the construction of semi--direct products.

\begin{definition}
Let $H$ and $Q$ be groups and let $\theta: Q \lra \Aut(H)$ be a group
homomorphism. The {\em semi--direct product} $H \semidirect_\theta Q$
is defined to be the group with underlying set $\{(h,q) \st h \in
H,\ q \in Q\}$ and group operation $(h,q)(h',q') := (h \theta(q) h',
qq')$.
\end{definition}

We leave it to the reader to check that $H \semidirect_\theta Q$ is
really a group. It helps to know that the inverse of $(h,q)$ is
$(\theta(q^{-1}) (h^{-1}), q^{-1})$.

For the remainder of this article, we omit $\theta$ from the notation
whenever this map is clear from the context.

Set $G := H \semidirect Q$. There exist canonical monomorphisms $H
\lra G$ and $Q \lra G$, given by
\begin{eqnarray*}
h \mapsto (h,1_Q),& \ & h \in H \\
q \mapsto (1_H,q),& \ & q \in Q
\end{eqnarray*}
where $1_H$ (resp. $1_Q$) is the identity element of $H$
(resp. $Q$). These monomorphisms are so natural that we will treat $H$
and $Q$ as subgroups of $G$ under these inclusions.

\begin{theorem}\label{ext-to-int}
Let $G := H \semidirect Q$ as above. Then:
\begin{itemize}
\item $H$ is a normal subgroup of $G$.
\item $HQ = G$.
\item $H \intersect Q = \{1_G\}$.
\end{itemize}
\end{theorem}
\begin{proof}
Let $p: G \lra Q$ be the projection map defined by $p(h,q) = q$. Then
$p$ is a homomorphism with kernel $H$. Therefore $H$ is a normal
subgroup of $G$.

%A harder proof of normality has been commented out.
%
%We already know that $H$ is a subgroup of $G$. The only issue is
%whether it is normal. Let $(h,q) \in G$ and $(h_0,1_Q) \in H$. Then
%\begin{eqnarray*}
%(h,q) (h_0,1_Q) (h,q)^{-1} & = & (h,q) (h_0,1_Q) (\theta(q^{-1})
%(h^{-1}),q^{-1}) \\
%& = & (h,q) (h_0 \theta(1_Q) (\theta(q^{-1}) (h^{-1})),q^{-1}) \\
%& = & (h \theta(q) (h_0 \theta(q^{-1})(h^{-1})),1_Q) \\
%& = & (h (\theta(q)(h_0)) h^{-1},1_Q).
%\end{eqnarray*}
%We have shown that $ghg^{-1} \in H$ for every $g \in G$ and $h \in H$,
%so $H$ is a normal subgroup of $G$.

Every $(h,q) \in G$ can be written as $(h,1_Q) (1_H,q)$. Therefore $HQ
= G$.

Finally, it is evident that $(1_H,1_Q)$ is the only element of $G$
that is of the form $(h,1_Q)$ for $h \in H$ and $(1_H,q)$ for $q \in
Q$.
\end{proof}

This result motivates the definition of internal semi--direct
products.

\begin{definition}
Let $G$ be a group with subgroups $H$ and $Q$. We say $G$ is the {\em
internal semi--direct product} of $H$ and $Q$ if:
\begin{itemize}
\item $H$ is a normal subgroup of $G$.
\item $HQ = G$.
\item $H \intersect Q = \{1_G\}$.
\end{itemize}
\end{definition}

We know an external semi--direct product is an internal semi--direct
product (Theorem~\ref{ext-to-int}). Now we prove a converse (Theorem~\ref{int-to-ext}), namely, that an internal semi--direct product is an external semi--direct product.

\begin{lemma}\label{unique}
Let $G$ be a group with subgroups $H$ and $Q$. Suppose $G=HQ$ and $H
\intersect Q = \{1_G\}$. Then every element $g$ of $G$ can be written
uniquely in the form $hq$, for $h \in H$ and $q \in Q$.
\end{lemma}
\begin{proof}
Since $G=HQ$, we know that $g$ can be written as $hq$. Suppose it can
also be written as $h'q'$. Then $hq=h'q'$ so ${h'}^{-1}h = q' q^{-1} \in
H \intersect Q = \{1_G\}$. Therefore $h=h'$ and $q=q'$.
\end{proof}

\begin{theorem}\label{int-to-ext}
Suppose $G$ is a group with subgroups $H$ and $Q$, and $G$ is the
internal semi--direct product of $H$ and $Q$. Then $G \iso H
\semidirect_\theta Q$ where $\theta: Q \lra \Aut(H)$ is given by
$$
\theta(q)(h) := qhq^{-1},\ q \in Q,\, h \in H.
$$
\end{theorem}
\begin{proof}
By Lemma~\ref{unique}, every element $g$ of $G$ can be written
uniquely in the form $hq$, with $h \in H$ and $q \in Q$. Therefore,
the map $\phi: H \semidirect Q \lra G$ given by $\phi(h,q) = hq$ is a
bijection from $G$ to $H \semidirect Q$. It only remains to show that
this bijection is a homomorphism.

Given elements $(h,q)$ and $(h',q')$ in $H \semidirect Q$, we have
$$
\phi((h,q)(h',q')) = \phi((h \theta(q)(h'),qq')) =
\phi(hqh'q^{-1},qq') = hqh'q' = \phi(h,q) \phi(h',q').
$$
Therefore $\phi$ is an isomorphism.
\end{proof}

Consider the external semi--direct product $G := H \semidirect_\theta
Q$ with subgroups $H$ and $Q$. We know from Theorem~\ref{int-to-ext}
that $G$ is isomorphic to the external semi--direct product $H
\semidirect_{\theta'} Q$, where we are temporarily writing $\theta'$
for the conjugation map $\theta'(q)(h) := qhq^{-1}$ of
Theorem~\ref{int-to-ext}. But in fact the two maps $\theta$ and
$\theta'$ are the same:
$$
\theta'(q)(h) = (1_H,q) (h,1_Q) (1_H, q^{-1}) = (\theta(q)(h),1_Q) =
\theta(q)(h).
$$
In summary, one may use Theorems~\ref{ext-to-int} and~\ref{int-to-ext} to pass freely between the notions of internal semi--direct product and external semi--direct product.

Finally, we discuss the correspondence between semi--direct products
and split exact sequences of groups.

\begin{definition}
An exact sequence of groups
$$
1 \lra H \stackrel{i}{\lra} G \stackrel{j}{\lra} Q \lra 1.
$$
is {\em split} if there exists a homomorphism $k: Q \lra G$ such that
$j \comp k$ is the identity map on $Q$.
\end{definition}

\begin{theorem}
Let $G$, $H$, and $Q$ be groups. Then $G$ is isomorphic to a
semi--direct product $H \semidirect Q$ if and only if there exists a
split exact sequence
$$
1 \lra H \stackrel{i}{\lra} G \stackrel{j}{\lra} Q \lra 1.
$$
\end{theorem}
\begin{proof}
First suppose $G \iso H \semidirect Q$. Let $i: H \lra G$ be the
inclusion map $i(h) = (h,1_Q)$ and let $j: G \lra Q$ be the
projection map $j(h,q) = q$. Let the splitting map $k: Q \lra G$ be
the inclusion map $k(q) = (1_H,q)$. Then the sequence above is clearly
split exact.

Now suppose we have the split exact sequence above. Let $k: Q \lra G$
be the splitting map. Then:
\begin{itemize}
\item $i(H) = \ker j$, so $i(H)$ is normal in $G$.
\item For any $g \in G$, set $q := k(j(g))$. Then $j(gq^{-1}) = j(g)
j(k(j(g)))^{-1} = 1_Q$, so $gq^{-1} \in \Im i$. Set $h :=
gq^{-1}$. Then $g=hq$. Therefore $G=i(H)k(Q)$.
\item Suppose $g \in G$ is in both $i(H)$ and $k(Q)$. Write $g =
k(q)$. Then $k(q) \in \Im i = \ker j$, so $q = j(k(q)) =
1_Q$. Therefore $g = k(q) = k(1_Q) = 1_G$, so $i(H) \intersect k(Q) =
\{1_G\}$.
\end{itemize}
This proves that $G$ is the internal semi--direct product of $i(H)$
and $k(Q)$. These are isomorphic to $H$ and $Q$,
respectively. Therefore $G$ is isomorphic to a semi--direct product $H
\semidirect Q$.
\end{proof}

Thus, not all normal subgroups $H \subset G$ give rise to an (internal)
semi--direct product $G = H \semidirect G/H$. More specifically, if
$H$ is a normal subgroup of $G$, we have the canonical exact sequence
$$
1 \lra H \lra G \lra G/H \lra 1.
$$
We see that $G$ can be decomposed into $H \semidirect G/H$ as an internal semi--direct product if and only if the canonical exact sequence splits.
%%%%%
%%%%%
\end{document}
