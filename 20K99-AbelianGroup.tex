\documentclass[12pt]{article}
\usepackage{pmmeta}
\pmcanonicalname{AbelianGroup}
\pmcreated{2013-03-22 14:01:55}
\pmmodified{2013-03-22 14:01:55}
\pmowner{yark}{2760}
\pmmodifier{yark}{2760}
\pmtitle{abelian group}
\pmrecord{25}{35107}
\pmprivacy{1}
\pmauthor{yark}{2760}
\pmtype{Definition}
\pmcomment{trigger rebuild}
\pmclassification{msc}{20K99}
\pmsynonym{commutative group}{AbelianGroup}
%\pmkeywords{group}
%\pmkeywords{abelian}
%\pmkeywords{commutative}
%\pmkeywords{quotient}
\pmrelated{Group}
\pmrelated{Klein4Group}
\pmrelated{CommutativeSemigroup}
\pmrelated{GeneralizedCyclicGroup}
\pmrelated{AbelianGroupsOfOrder120}
\pmrelated{FundamentalTheoremOfFinitelyGeneratedAbelianGroups}
\pmrelated{NonabelianGroup}
\pmrelated{Commutative}
\pmrelated{MetabelianGroup}
\pmdefines{abelian}
\pmdefines{commutative}

\endmetadata

\usepackage{amsfonts}
\usepackage{amsmath}
\usepackage{amsthm}

\newtheorem{thm}{Theorem}

\def\Z{\mathbb{Z}}
\begin{document}
\PMlinkescapeword{additive}
\PMlinkescapeword{homomorphism}
\PMlinkescapeword{module}
\PMlinkescapeword{modules}
\PMlinkescapeword{properties}
\PMlinkescapeword{subgroup}
\PMlinkescapeword{subgroups}
\PMlinkescapeword{theorem}
\PMlinkescapeword{unitary}
\PMlinkescapeword{word}

Let $(G,*)$ be a group. If for any $a,b\in G$ we have
$a*b=b*a$, we say that the group is \emph{abelian} (or \emph{commutative}).
Abelian groups are named after Niels Henrik Abel, but the word {\it abelian} is commonly written in lowercase.

Abelian groups are essentially the same thing as unitary $\Z$-\PMlinkname{modules}{Module}.
In fact, it is often more natural to treat abelian groups as modules rather than as groups, and for this reason they are commonly written using additive notation. 

Some of the basic properties of abelian groups are as follows:

\begin{thm}
Any \PMlinkname{subgroup}{Subgroup} of an abelian group is normal.
\end{thm}
\begin{proof}
Let $H$ be a subgroup of the abelian group $G$. Since $ah=ha$ for any $a\in G$ and any $h\in H$ we get $aH=Ha$. That is, $H$ is normal in $G$.
\end{proof}

\begin{thm}
Quotient groups of abelian groups are also abelian.
\end{thm}
\begin{proof}
Let $H$ be a subgroup of $G$. Since $G$ is abelian, $H$ is normal and we can get the quotient group $G/H$ whose elements are the equivalence classes for 
$a\sim b$ if $ab^{-1}\in H$.
The operation on the quotient group is given by $aH\cdot bH=(ab)H$. But $bH\cdot aH=(ba)H =(ab)H$, therefore the quotient group is also commutative.
\end{proof}

Here is another theorem concerning abelian groups:

\begin{thm} 
If $\varphi\colon G\to G$ defined by $ \varphi(x) =x^2$ is a \PMlinkname{homomorphism}{GroupHomomorphism}, then $G$ is abelian.
\end{thm}
\begin{proof}
If such a function were a homomorphism,
we would have
\[(xy)^2=\varphi(xy) = \varphi(x)\varphi(y)=x^2y^2\] that is, $xyxy=xxyy$.
Left-multiplying by $x^{-1}$ and right-multiplying by $y^{-1}$ we are led to
$yx=xy$ and thus the group is abelian.
\end{proof}

%%%%%
%%%%%
\end{document}
