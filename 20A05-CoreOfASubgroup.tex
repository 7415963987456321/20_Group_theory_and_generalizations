\documentclass[12pt]{article}
\usepackage{pmmeta}
\pmcanonicalname{CoreOfASubgroup}
\pmcreated{2013-03-22 15:37:22}
\pmmodified{2013-03-22 15:37:22}
\pmowner{yark}{2760}
\pmmodifier{yark}{2760}
\pmtitle{core of a subgroup}
\pmrecord{10}{37547}
\pmprivacy{1}
\pmauthor{yark}{2760}
\pmtype{Definition}
\pmcomment{trigger rebuild}
\pmclassification{msc}{20A05}
\pmsynonym{core}{CoreOfASubgroup}
\pmsynonym{normal core}{CoreOfASubgroup}
\pmsynonym{normal interior}{CoreOfASubgroup}
\pmrelated{NormalClosure2}
\pmdefines{core-free}
\pmdefines{corefree}
\pmdefines{normal-by-finite}
\pmdefines{core-free subgroup}
\pmdefines{corefree subgroup}
\pmdefines{normal-by-finite subgroup}

\usepackage{amssymb}
\usepackage{amsmath}
\usepackage{amsfonts}

\newcommand{\C}{\mathbb{C}}
\newcommand{\N}{\mathbb{N}}
\newcommand{\Q}{\mathbb{Q}}
\newcommand{\R}{\mathbb{R}}
\newcommand{\Z}{\mathbb{Z}}

\newtheorem{theorem}{Theorem}

\def\genby#1{{\left\langle #1\right\rangle}}
\def\normal{\trianglelefteq}
\def\subgroup{\leq}

\DeclareMathOperator{\Sym}{Sym}

\DeclareMathOperator{\coreop}{core}
\def\core#1#2{{\coreop}_{#1}(#2)}
\begin{document}
\PMlinkescapeword{argument}
\PMlinkescapeword{divides}
\PMlinkescapeword{finite}
\PMlinkescapeword{property}

Let $H$ be a subgroup of a group $G$.

The \emph{core} (or \emph{normal interior}, or \emph{normal core}) of $H$ in $G$ 
is the intersection of all conjugates of $H$ in $G$:

\[ \core{G}{H} = \bigcap_{x\in G}x^{-1}Hx. \]

It is not hard to show that
$\core{G}{H}$ is the largest normal subgroup of $G$ contained in $H$,
that is, $\core{G}{H}\normal G$ and
if $N\normal G$ and $N\subseteq H$ then $N\subseteq\core{G}{H}$.
For this reason, some authors denote the core by $H_G$
rather than $\core{G}{H}$,
by analogy with the notation $H^G$ for the normal closure.

If $\core{G}{H}=\{1\}$, 
then $H$ is said to be \emph{core-free}.

If $\core{G}{H}$ is of finite index in $H$,
then $H$ is said to be \emph{normal-by-finite}.

Let $\cal L$ be the set of left cosets of $H$ in $G$.
By considering the action of $G$ on $\cal L$ it can be shown that 
the \PMlinkname{quotient}{QuotientGroup} $G/\core{G}{H}$ embeds in the symmetric group $\Sym({\cal L})$.
A consequence of this is that if $H$ is of finite index in $G$,
then $\core{G}{H}$ is also of finite index in $G$,
and $[G:\core{G}{H}]$ divides $[G:H]!$ (the factorial of $[G:H]$).
In particular, if a simple group $S$ has a proper subgroup of finite index $n$,
then $S$ must be of finite order dividing $n!$, 
as the core of the subgroup is trivial.
It also follows that 
a group is virtually abelian if and only if it is abelian-by-finite,
because the core of an abelian subgroup of finite index 
is a normal abelian subgroup of finite index
(and the same argument applies if `abelian' is replaced by 
any other property that is inherited by subgroups).
%%%%%
%%%%%
\end{document}
