\documentclass[12pt]{article}
\usepackage{pmmeta}
\pmcanonicalname{LeftAndRightCosetsInADoubleCoset}
\pmcreated{2013-03-22 18:35:10}
\pmmodified{2013-03-22 18:35:10}
\pmowner{asteroid}{17536}
\pmmodifier{asteroid}{17536}
\pmtitle{left and right cosets in a double coset}
\pmrecord{7}{41312}
\pmprivacy{1}
\pmauthor{asteroid}{17536}
\pmtype{Theorem}
\pmcomment{trigger rebuild}
\pmclassification{msc}{20A05}

\endmetadata

% this is the default PlanetMath preamble.  as your knowledge
% of TeX increases, you will probably want to edit this, but
% it should be fine as is for beginners.

% almost certainly you want these
\usepackage{amssymb}
\usepackage{amsmath}
\usepackage{amsfonts}

% used for TeXing text within eps files
%\usepackage{psfrag}
% need this for including graphics (\includegraphics)
%\usepackage{graphicx}
% for neatly defining theorems and propositions
%\usepackage{amsthm}
% making logically defined graphics
%%%\usepackage{xypic}

% there are many more packages, add them here as you need them

% define commands here

\begin{document}
\PMlinkescapeword{right}

Let $H$ and $K$ be subgroups of a group $G$. Every double coset $HgK$, with $g \in G$, is a union of \PMlinkname{right}{Coset} or left cosets, since
\begin{align*}
HgK = \bigcup_{k \in K} Hgk\; = \bigcup_{h \in H} hgK,
\end{align*}
but these unions need not be disjoint. In particular, from the above equality we cannot say how many right (or left) cosets fit in a double coset.

The following proposition aims to clarify this.

$\,$

{\bf \PMlinkescapetext{Proposition} -} Let $H$ and $K$ be subgroups of a group $G$ and $g \in G$. We have that
\begin{align*}
HgK = \bigcup_{[k]\, \in\, (K \cap g^{-1}Hg) \backslash K} Hgk\; = \bigcup_{[h]\, \in\, H / (H \cap gKg^{-1})} hgK
\end{align*}
hold as disjoint unions. In particular, the number of right and left cosets in $HgK$ is respectively given by
\begin{align*}
\#(H \backslash HgK) = [K: K \cap g^{-1}Hg]\\
\#(HgK/K) =[H: H \cap gKg^{-1}]
\end{align*}

\subsection{Remarks}

\begin{itemize}
\item The number of right and left cosets in a double coset does not coincide in general, not \PMlinkescapetext{even} for double cosets of the form $HgH$.
\end{itemize}

\begin{thebibliography}{9}
\bibitem{krieg} A. Krieg, \emph{\PMlinkescapetext{Hecke algebras}}, Mem. Amer. Math. Soc., no. 435, vol. 87, 1990.
\end{thebibliography}
%%%%%
%%%%%
\end{document}
