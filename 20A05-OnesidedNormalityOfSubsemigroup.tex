\documentclass[12pt]{article}
\usepackage{pmmeta}
\pmcanonicalname{OnesidedNormalityOfSubsemigroup}
\pmcreated{2013-03-22 16:10:41}
\pmmodified{2013-03-22 16:10:41}
\pmowner{lars_h}{9802}
\pmmodifier{lars_h}{9802}
\pmtitle{one-sided normality of subsemigroup}
\pmrecord{6}{38265}
\pmprivacy{1}
\pmauthor{lars_h}{9802}
\pmtype{Definition}
\pmcomment{trigger rebuild}
\pmclassification{msc}{20A05}
\pmdefines{left-normal}
\pmdefines{right-normal}
\pmdefines{left-normalizer}
\pmdefines{right-normalizer}

% this is the default PlanetMath preamble.  as your knowledge
% of TeX increases, you will probably want to edit this, but
% it should be fine as is for beginners.

% almost certainly you want these
\usepackage{amssymb}
\usepackage{amsmath}
\usepackage{amsfonts}

% used for TeXing text within eps files
%\usepackage{psfrag}
% need this for including graphics (\includegraphics)
%\usepackage{graphicx}
% for neatly defining theorems and propositions
%\usepackage{amsthm}
% making logically defined graphics
%%%\usepackage{xypic}

% there are many more packages, add them here as you need them

% define commands here

\begin{document}
Let $S$ be a semigroup. A subsemigroup $N$ of $S$ is said to be 
\emph{left-normal} if $g N \subseteq N g$ for all $g \in S$ and 
it is said to be \emph{right-normal} if $g N \supseteq N g$ for all 
$g \in S$.
One may similarly define \emph{left-normalizers}
\[
  \mathrm{LN}_S(N) := \{ g \in S \,\mid\, gN \subseteq Ng \}
\]
and \emph{right-normalizers}
\[
  \mathrm{RN}_S(N) := \{ g \in S \,\mid\, Ng \subseteq gN \}
  \text{.}
\]

A left-normal subgroup $N$ of a \emph{group} $S$ is automatically 
normal, since
\[
  g N \subseteq
  N g =
  g g^{-1} N g \subseteq
  g N g^{-1} g = g N
  \text{.}
\]
In is similarly shown for general $S$ and $N$ that if some $g \in 
\mathrm{LN}_S(N)$ has an inverse $g^{-1}$ then $g^{-1} \in 
\mathrm{RN}_S(N)$ and vice versa. Left- and right-normalizers 
are always closed under multiplication (hence subsemigroups) 
and contain the identity element of $S$ if there is one.

An example of a left-normal but not right-normal $N \subseteq S$ 
can be constructed using matrices under multiplication, if one takes
\[
  S = \Biggl\{ \begin{pmatrix} k& m \\ 0& 1 \end{pmatrix} 
  \Biggm| k,m \in \mathbb{Z} \Biggr\}
  \qquad\text{and}\qquad
  N = \Biggl\{ \begin{pmatrix} 1& n \\ 0& 1 \end{pmatrix} 
  \Biggm| n \in \mathbb{Z} \Biggr\}
  \text{,}
\]
where one may note that $N$ is a group and $S$ is a monoid.
Since
\begin{align*}
  \begin{pmatrix} k& m \\ 0& 1 \end{pmatrix} 
  \begin{pmatrix} 1& n \\ 0& 1 \end{pmatrix} ={}&
  \begin{pmatrix} k& kn+m \\ 0& 1 \end{pmatrix} 
  \quad\text{and}\\
  \begin{pmatrix} 1& n \\ 0& 1 \end{pmatrix}
  \begin{pmatrix} k& m \\ 0& 1 \end{pmatrix} ={}& 
  \begin{pmatrix} k& n+m \\ 0& 1 \end{pmatrix} 
\end{align*}
it follows that $gN \subseteq Ng$ for all $g \in S$, with 
\PMlinkname{proper inclusion}{ProperSubset} 
when $k \neq \pm 1$.


The definition of left and 
\PMlinkescapetext{right} \PMlinkescapetext{normality} 
is somewhat arbitrary in the choice of whether to call 
something the \PMlinkescapetext{right} or left form. 
A reference supporting the choice documented here is:
\begin{thebibliography}{8}
\bibitem{MR0311830}
  Karl Heinrich \textsc{Hofmann} and Michael \textsc{Mislove}: 
  The centralizing theorem for left normal groups of units in 
  compact monoids,
  \textit{Semigroup Forum} \textbf{3} (1971/72),  no.~1, 31--42. 
\end{thebibliography}
It may also be observed that the combination `left normal' in semigroup 
theory frequently occurs as part of the phrase `left normal band', 
but in that case the etymology rather seems to be that `left' qualifies 
the phrase `normal band'.

%%%%%
%%%%%
\end{document}
