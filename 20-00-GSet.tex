\documentclass[12pt]{article}
\usepackage{pmmeta}
\pmcanonicalname{GSet}
\pmcreated{2013-03-22 17:55:44}
\pmmodified{2013-03-22 17:55:44}
\pmowner{jwaixs}{18148}
\pmmodifier{jwaixs}{18148}
\pmtitle{$G$-Set}
\pmrecord{5}{40423}
\pmprivacy{1}
\pmauthor{jwaixs}{18148}
\pmtype{Definition}
\pmcomment{trigger rebuild}
\pmclassification{msc}{20-00}
\pmrelated{GroupAction}

\endmetadata

% this is the default PlanetMath preamble.  as your knowledge
% of TeX increases, you will probably want to edit this, but
% it should be fine as is for beginners.

% almost certainly you want these
\usepackage{amssymb}
\usepackage{amsmath}
\usepackage{amsfonts}

% used for TeXing text within eps files
%\usepackage{psfrag}
% need this for including graphics (\includegraphics)
%\usepackage{graphicx}
% for neatly defining theorems and propositions
%\usepackage{amsthm}
% making logically defined graphics
%%%\usepackage{xypic}

% there are many more packages, add them here as you need them

% define commands here

\begin{document}
If $G$ is a group and $X$ a set, then $X$ is called a left $G$-Set if there exists a mapping $\lambda : G \times X \rightarrow X$ with

$$
  \lambda(g_1,\lambda(g_2,x)) = \lambda(g_1 g_2, x)
$$

or shorter with $\lambda(g,x) = gx$

$$
  g_1(g_2(x)) = (g_1 g_2)(x)
$$

for all $x \in X$ and $g_1,g_2 \in G$. And when $G$ acts on a set $X$, the set $X$ is always a $G$-set. \\ \\


$X$ is called a right $G$-Set if there exists a mapping $\lambda : X \times G \rightarrow X$ with

$$
  \lambda(\lambda(x,g_2),g_1) = \lambda(x,g_2 g_1)
$$
%%%%%
%%%%%
\end{document}
