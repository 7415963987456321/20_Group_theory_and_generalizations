\documentclass[12pt]{article}
\usepackage{pmmeta}
\pmcanonicalname{Coset}
\pmcreated{2013-03-22 12:08:10}
\pmmodified{2013-03-22 12:08:10}
\pmowner{djao}{24}
\pmmodifier{djao}{24}
\pmtitle{coset}
\pmrecord{9}{31306}
\pmprivacy{1}
\pmauthor{djao}{24}
\pmtype{Definition}
\pmcomment{trigger rebuild}
\pmclassification{msc}{20A05}
\pmdefines{index}
\pmdefines{left coset}
\pmdefines{right coset}

\endmetadata

\usepackage{amssymb}
\usepackage{amsmath}
\usepackage{amsfonts}
\usepackage{graphicx}
%%%\usepackage{xypic}
\begin{document}
Let $H$ be a subgroup of a group $G$, and let $a \in G$. The {\em left coset} of $a$ with respect to $H$ in $G$ is defined to be the set
$$
aH := \{ ah \mid h \in H \}.
$$
The {\em right coset} of $a$ with respect to $H$ in $G$ is defined to be the set
$$
Ha := \{ ha \mid h \in H \}.
$$
Two left cosets $aH$ and $bH$ of $H$ in $G$ are either identical or disjoint. Indeed, if $c \in aH \cap bH$, then $c = ah_1$ and $c = bh_2$ for some $h_1,h_2 \in H$, whence $b^{-1} a = h_2 h_1^{-1} \in H$. But then, given any $ah \in aH$, we have $ah = (bb^{-1})ah = b(b^{-1}a) h \in bH$, so $aH \subset bH$, and similarly $bH \subset aH$. Therefore $aH = bH$.

Similarly, any two right cosets $Ha$ and $Hb$ of $H$ in $G$ are either identical or disjoint. Accordingly, the collection of left cosets (or right cosets) partitions the group $G$; the corresponding equivalence relation for left cosets can be described succintly by the relation $a \sim b$ if $a^{-1} b \in H$, and for right cosets by $a \sim b$ if $ab^{-1} \in H$.

The {\em index} of $H$ in $G$, denoted $[G:H]$, is the cardinality of the set $G/H$ of left cosets of $H$ in $G$.
%%%%%
%%%%%
%%%%%
\end{document}
