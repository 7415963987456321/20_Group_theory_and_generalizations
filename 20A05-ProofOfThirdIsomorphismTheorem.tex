\documentclass[12pt]{article}
\usepackage{pmmeta}
\pmcanonicalname{ProofOfThirdIsomorphismTheorem}
\pmcreated{2013-03-22 15:35:09}
\pmmodified{2013-03-22 15:35:09}
\pmowner{Thomas Heye}{1234}
\pmmodifier{Thomas Heye}{1234}
\pmtitle{proof of third isomorphism theorem}
\pmrecord{5}{37496}
\pmprivacy{1}
\pmauthor{Thomas Heye}{1234}
\pmtype{Proof}
\pmcomment{trigger rebuild}
\pmclassification{msc}{20A05}

\endmetadata

% this is the default PlanetMath preamble.  as your knowledge
% of TeX increases, you will probably want to edit this, but
% it should be fine as is for beginners.

% almost certainly you want these
\usepackage{amssymb}
\usepackage{amsmath}
\usepackage{amsfonts}

% used for TeXing text within eps files
%\usepackage{psfrag}
% need this for including graphics (\includegraphics)
%\usepackage{graphicx}
% for neatly defining theorems and propositions
%\usepackage{amsthm}
% making logically defined graphics
%%%\usepackage{xypic}

% there are many more packages, add them here as you need them

% define commands here
\begin{document}
We'll give a proof of  the third isomorphism theorem using the Fundamental homomorphism theorem.

Let $G$ be a group, and let $K\subseteq H$ be normal subgroups of $G$. Define $p,q$ to be the natural homomorphisms from $G$ to $G/H$, $G/K$ respectively:
\[p(g)=gH, q(g)=gK\;\forall\;g \in G.\]
$K$ is a subset of $\ker(p)$, so there exists a unique homomorphism $\varphi\colon G/K \to G/H$ so that $\varphi \circ q=p$.

$p$ is surjective, so $\varphi$ is surjective as well; hence $\operatorname{im}\varphi=G/H$. The kernel of $\varphi$ is $\ker(p)/K=H/K$. So by the first isomorphism theorem we have
\[(G/K) / \ker(\varphi)=(G/K) / (H/K) \approx \operatorname{im}\varphi=G/H.\]
%%%%%
%%%%%
\end{document}
