\documentclass[12pt]{article}
\usepackage{pmmeta}
\pmcanonicalname{Generator}
\pmcreated{2013-03-22 13:30:39}
\pmmodified{2013-03-22 13:30:39}
\pmowner{Wkbj79}{1863}
\pmmodifier{Wkbj79}{1863}
\pmtitle{generator}
\pmrecord{10}{34094}
\pmprivacy{1}
\pmauthor{Wkbj79}{1863}
\pmtype{Definition}
\pmcomment{trigger rebuild}
\pmclassification{msc}{20A05}
\pmrelated{GeneratingSetOfAGroup}
\pmrelated{ProperGeneratorTheorem}

\endmetadata

% this is the default PlanetMath preamble.  as your knowledge
% of TeX increases, you will probably want to edit this, but
% it should be fine as is for beginners.

% almost certainly you want these
\usepackage{amssymb}
\usepackage{amsmath}
\usepackage{amsfonts}

% used for TeXing text within eps files
%\usepackage{psfrag}
% need this for including graphics (\includegraphics)
%\usepackage{graphicx}
% for neatly defining theorems and propositions
%\usepackage{amsthm}
% making logically defined graphics
%%%\usepackage{xypic}

% there are many more packages, add them here as you need them

% define commands here
\begin{document}
If $G$ is a cyclic group and $g \in G$, then $g$ is a {\sl generator \/} of $G$ if $\langle g \rangle =G$.

All infinite cyclic groups have exactly $2$ generators.  To see this, let $G$ be an infinite cyclic group and $g$ be a generator of $G$.  Let $z \in \mathbb{Z}$ such that $g^z$ is a generator of $G$.  Then $\langle g^z \rangle =G$.  Then $g \in G= \langle g^z \rangle$.  Thus, there exists $n \in {\mathbb Z}$ with $g=(g^z)^n=g^{nz}$.  Therefore, $g^{nz-1}=e_G$.  Since $G$ is infinite and $|g|=|\langle g \rangle |=|G|$ must be infinity, $nz-1=0$.  Since $nz=1$ and $n$ and $z$ are integers, either $n=z=1$ or $n=z=-1$.  It follows that the only generators of $G$ are $g$ and $g^{-1}$.

A finite cyclic group of order $n$ has exactly $\varphi(n)$ generators, where $\varphi$ is the Euler totient function.  To see this, let $G$ be a finite cyclic group of order $n$ and $g$ be a generator of $G$.  Then $|g|=|\langle g \rangle |=|G|=n$.  Let $z \in \mathbb{Z}$ such that $g^z$ is a generator of $G$.  By the division algorithm, there exist $q,r \in \mathbb{Z}$ with $0 \le r<n$ such that $z=qn+r$.  Thus, $g^z=g^{qn+r}=g^{qn}g^r=(g^n)^qg^r=(e_G)^qg^r=e_Gg^r=g^r$.  Since $g^r$ is a generator of $G$, it must be the case that $\langle g^r \rangle =G$.  Thus, $\displaystyle n=|G|=|\langle g^r \rangle|=|g^r|=\frac{|g|}{\gcd(r,|g|)}=\frac {n}{\gcd(r,n)}$.  Therefore, $\gcd(r,n)=1$, and the result follows.
%%%%%
%%%%%
\end{document}
