\documentclass[12pt]{article}
\usepackage{pmmeta}
\pmcanonicalname{NonisomorphicGroupsOfGivenOrder}
\pmcreated{2013-03-22 18:56:38}
\pmmodified{2013-03-22 18:56:38}
\pmowner{pahio}{2872}
\pmmodifier{pahio}{2872}
\pmtitle{non-isomorphic groups of given order}
\pmrecord{5}{41800}
\pmprivacy{1}
\pmauthor{pahio}{2872}
\pmtype{Theorem}
\pmcomment{trigger rebuild}
\pmclassification{msc}{20A05}
\pmrelated{BinomialCoefficient}
\pmrelated{PropertiesOfConjugacy}
\pmdefines{Landau's theorem}

\endmetadata

% this is the default PlanetMath preamble.  as your knowledge
% of TeX increases, you will probably want to edit this, but
% it should be fine as is for beginners.

% almost certainly you want these
\usepackage{amssymb}
\usepackage{amsmath}
\usepackage{amsfonts}

% used for TeXing text within eps files
%\usepackage{psfrag}
% need this for including graphics (\includegraphics)
%\usepackage{graphicx}
% for neatly defining theorems and propositions
 \usepackage{amsthm}
% making logically defined graphics
%%%\usepackage{xypic}

% there are many more packages, add them here as you need them

% define commands here

\theoremstyle{definition}
\newtheorem*{thmplain}{Theorem}

\begin{document}
\textbf{Theorem.}\, For every positive integer $n$, there exists only a finite amount of non-isomorphic groups of order $n$.\\

This assertion follows from Cayley's theorem, according to which any group of order $n$ is isomorphic with a subgroup of the symmetric group $\mathfrak{S}_n$.\, The number of non-isomorphic subgroups of $\mathfrak{S}_n$ cannot be greater than
$${n!\!-\!1 \choose n\!-\!1}.$$\\

The above theorem may be used in proving the following Landau's theorem:

\textbf{Theorem (Landau).}\, For every positive integer $n$, there exists only a finite amount of finite non-isomorphic groups which contain exactly $n$ conjugacy classes of elements.\\

One needs also the

\textbf{Lemma.}\, If\, $n \in \mathbb{Z}_+$\, and\, $0 < r \in \mathbb{R}$,\, then there is at most a finite amount of the vectors \,$(m_1,\,m_2,\,\ldots,\,m_n)$\, consisting of positive integers such that
$$\sum_{j=1}^n\frac{1}{m_j} \;=\; r.$$
The lemma is easily proved by induction on $n$.





%%%%%
%%%%%
\end{document}
