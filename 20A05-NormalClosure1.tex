\documentclass[12pt]{article}
\usepackage{pmmeta}
\pmcanonicalname{NormalClosure1}
\pmcreated{2013-03-22 14:41:50}
\pmmodified{2013-03-22 14:41:50}
\pmowner{yark}{2760}
\pmmodifier{yark}{2760}
\pmtitle{normal closure}
\pmrecord{9}{36307}
\pmprivacy{1}
\pmauthor{yark}{2760}
\pmtype{Definition}
\pmcomment{trigger rebuild}
\pmclassification{msc}{20A05}
\pmsynonym{normal subgroup generated by}{NormalClosure1}
\pmsynonym{conjugate closure}{NormalClosure1}
\pmsynonym{smallest normal subgroup containing}{NormalClosure1}
\pmrelated{Normalizer}
\pmrelated{CoreOfASubgroup}
\pmdefines{nearly normal}

\usepackage{amssymb}
\usepackage{amsmath}
\usepackage{amsfonts}

\def\normal{\trianglelefteq}
\begin{document}
Let $S$ be a subset of a group $G$.
The \emph{normal closure} of $S$ in $G$ is the intersection of all normal subgroups of $G$ that contain $S$, that is
\[\bigcap_{S\subseteq N\normal G}\!\!N.\]
The normal closure of $S$ is the smallest normal subgroup of $G$ that contains $S$, and so is also called the \emph{normal subgroup generated by} $S$.

It is not difficult to show that the normal closure of $S$ is the subgroup generated by all the conjugates of elements of $S$.

The normal closure of $S$ in $G$ is variously denoted by 
$\langle S^G\rangle$ or $\langle S\rangle^G$ or $S^G$.

If $H$ is a subgroup of $G$,
and $H$ is of finite index in its normal closure,
then $H$ is said to be \emph{nearly normal}.
%%%%%
%%%%%
\end{document}
