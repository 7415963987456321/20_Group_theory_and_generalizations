\documentclass[12pt]{article}
\usepackage{pmmeta}
\pmcanonicalname{InnerAutomorphism}
\pmcreated{2013-03-22 12:49:53}
\pmmodified{2013-03-22 12:49:53}
\pmowner{rmilson}{146}
\pmmodifier{rmilson}{146}
\pmtitle{inner automorphism}
\pmrecord{12}{33155}
\pmprivacy{1}
\pmauthor{rmilson}{146}
\pmtype{Definition}
\pmcomment{trigger rebuild}
\pmclassification{msc}{20A05}
\pmsynonym{inner}{InnerAutomorphism}
\pmdefines{conjugation}
\pmdefines{outer}
\pmdefines{outer automorphism}
\pmdefines{automorphism group}

\endmetadata

\usepackage{amsmath}
\usepackage{amsfonts}
\usepackage{amssymb}
\newcommand{\reals}{\mathbb{R}}
\newcommand{\natnums}{\mathbb{N}}
\newcommand{\cnums}{\mathbb{C}}
\newcommand{\znums}{\mathbb{Z}}
\newcommand{\lp}{\left(}
\newcommand{\rp}{\right)}
\newcommand{\lb}{\left[}
\newcommand{\rb}{\right]}
\newcommand{\supth}{^{\text{th}}}
\newtheorem{proposition}{Proposition}
\newtheorem{definition}[proposition]{Definition}

\newtheorem{theorem}[proposition]{Theorem}
\begin{document}
Let $G$ be a group.  For every $x\in G$, we define a
mapping
$$\phi_x:G\rightarrow G,\quad y\mapsto x y x^{-1},\quad y\in G,$$
called conjugation by $x$.
It is easy to show the conjugation map is in fact, a group automorphism.

An automorphism of $G$ that corresponds to conjugation by some
$x\in G$ is called \emph{inner}. An automorphism that isn't inner is called
an \emph{outer} automorphism.  

The composition operation gives the set of all automorphisms of $G$
the structure of a group, $\operatorname{Aut}(G)$.  The inner
automorphisms also form a group, $\operatorname{Inn}(G)$, which is a
normal subgroup of $\operatorname{Aut}(G)$.  Indeed, if $\phi_x,\;
x\in G$ is an inner automorphism and $\pi:G\rightarrow G$ an arbitrary
automorphism, then
$$\pi\circ \phi_x \circ\pi^{-1} = \phi_{\pi(x)}.$$
Let us also note that the mapping 
$$x\mapsto \phi_x,\quad x\in G$$
is a surjective group homomorphism with kernel
$\operatorname{Z}(G)$, the centre subgroup. Consequently,
$\operatorname{Inn}(G)$ is naturally isomorphic to the quotient of
$G/\operatorname{Z}(G)$.

Note:  the above definitions and assertions hold, mutatis mutandi, if we define 
the conjugation action of $x\in G$ on $B$ to be the right action
\[ y\mapsto x^{-1} y x,\quad y\in G,\]
rather than the left action given above.
%%%%%
%%%%%
\end{document}
