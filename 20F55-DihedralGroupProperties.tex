\documentclass[12pt]{article}
\usepackage{pmmeta}
\pmcanonicalname{DihedralGroupProperties}
\pmcreated{2013-03-22 16:06:35}
\pmmodified{2013-03-22 16:06:35}
\pmowner{Algeboy}{12884}
\pmmodifier{Algeboy}{12884}
\pmtitle{dihedral group properties}
\pmrecord{10}{38175}
\pmprivacy{1}
\pmauthor{Algeboy}{12884}
\pmtype{Topic}
\pmcomment{trigger rebuild}
\pmclassification{msc}{20F55}
\pmrelated{GeneralizedQuaternionGroup}

\usepackage{latexsym}
\usepackage{amssymb}
\usepackage{amsmath}
\usepackage{amsfonts}
\usepackage{amsthm}

%%\usepackage{xypic}

%-----------------------------------------------------

%       Standard theoremlike environments.

%       Stolen directly from AMSLaTeX sample

%-----------------------------------------------------

%% \theoremstyle{plain} %% This is the default

\newtheorem{thm}{Theorem}

\newtheorem{coro}[thm]{Corollary}

\newtheorem{lem}[thm]{Lemma}

\newtheorem{lemma}[thm]{Lemma}

\newtheorem{prop}[thm]{Proposition}

\newtheorem{conjecture}[thm]{Conjecture}

\newtheorem{conj}[thm]{Conjecture}

\newtheorem{defn}[thm]{Definition}

\newtheorem{remark}[thm]{Remark}

\newtheorem{ex}[thm]{Example}



%\countstyle[equation]{thm}



%--------------------------------------------------

%       Item references.

%--------------------------------------------------


\newcommand{\exref}[1]{Example-\ref{#1}}

\newcommand{\thmref}[1]{Theorem-\ref{#1}}

\newcommand{\defref}[1]{Definition-\ref{#1}}

\newcommand{\eqnref}[1]{(\ref{#1})}

\newcommand{\secref}[1]{Section-\ref{#1}}

\newcommand{\lemref}[1]{Lemma-\ref{#1}}

\newcommand{\propref}[1]{Prop\-o\-si\-tion-\ref{#1}}

\newcommand{\corref}[1]{Cor\-ol\-lary-\ref{#1}}

\newcommand{\figref}[1]{Fig\-ure-\ref{#1}}

\newcommand{\conjref}[1]{Conjecture-\ref{#1}}


% Normal subgroup or equal.

\providecommand{\normaleq}{\unlhd}

% Normal subgroup.

\providecommand{\normal}{\lhd}

\providecommand{\rnormal}{\rhd}
% Divides, does not divide.

\providecommand{\divides}{\mid}

\providecommand{\ndivides}{\nmid}


\providecommand{\union}{\cup}

\providecommand{\bigunion}{\bigcup}

\providecommand{\intersect}{\cap}

\providecommand{\bigintersect}{\bigcap}

\DeclareMathOperator{\Aut}{Aut }







\begin{document}
\section{Properties of Dihedral Groups}


A group generated by two involutions is a dihedral group.  When
the group is finite it is possible to show that the group has
order $2n$ for some $n>0$ and takes the presentation
\[D_{2n}=\langle a,b~|~a^n=1, b^2=1, a^b=a^{-1}\rangle.\]

\begin{remark}
Contemporary group theorists prefer $D_{2n}$ over $D_n$ as the notation for the
dihedral group of order $2n$.  Although this notation is overly explicit, it does help to resolve the ambiguity with the Lie type $D_l$ which corresponds to the orthogonal group $\Omega^+(2l,q)$.  However, introductory texts in algebra still make use of the more appropriate $D_{n}$ notation to emphasize the connection to the symmetries of a regular $n$-gon ($n$-sided polygon).
\end{remark}

$D_2\cong\mathbb{Z}_2$, $D_4\cong\mathbb{Z}_2\times \mathbb{Z}_2$ are the two abelian examples of dihedral groups.  They can be considered as dihedral groups of the respective order because they satisfy the relations, though the geometric interpretations are slightly modified.  Often $D_2$ can be termed the symmetries of a line segment, and the $D_4$ the symmetries of a non-square rectangle, as the symmetry groups of each of these object is isomorphic to $D_2$ and $D_4$ respectively.  These exceptions cause problems for most theorems on dihedral groups so it is convenient to insist that $n>2$ for theorems.

\begin{prop}
\[D_{2n}=\{a^i~|~i\in \mathbb{Z}_n\}\sqcup \{a^ib ~|~i\in \mathbb{Z}_n\}\]
is a irredundant list of the elements of $D_{2n}$.  Moreover the conjugacy
classes of $D_{2n}$ are $\{a^i, a^{-i}\}$ for all $i\in \mathbb{Z}_n$ and
\begin{itemize}
\item $2|n$, $\{a^{2i}b~|~i\in \mathbb{Z}_n\}$ and $\{a^{2i+1}b ~|~ i\in \mathbb{Z}_n\}$
\item $2\ndivides n$, $\{a^ib~|~i\in \mathbb{Z}_n\}$.
\end{itemize}
Consequently when $n>2$ the center of $D_{2n}$ is $1$ when $2\ndivides n$ and 
$Z(D_{2n})=\langle a^{n/2}\rangle$ when $2|n$.
Furthermore $C_n:=\langle a\rangle$ is a characteristic subgroup of $D_{2n}$,
provided $n\neq 2$.  
\end{prop}

\begin{proof}
The conjugation relation $a^b=a^{-1}$ allows us to place every element
in the normal form $a^i b^j$.  If $a^i b^j = a^k b^l$ then 
$a^{i-k}= b^{l-j}$.  Yet $\langle a\rangle \intersect \langle b\rangle=1$
so $i-k\equiv 0\pmod{n}$ and $l-j\equiv 0\pmod{2}$.  Thus we have and
irredundant list as required.

For the conjugacy classes note that $(a^i)^{a^j b}=(a^i)^b=a^{-i}$ so
that these conjugacy classes are established.  Next 
\[(a^i b)^{a^j}=a^{-j+i}ba^j=a^{-j+i}a^{-j}b=a^{-2j+i}b\]
for all $i\in \mathbb{Z}$.  When $2\ndivides n$ we have $(2,n)=1$ so $2$ is invertible modulo $n$.  We let $j=2^{-1}(k-i)$ for any $k\in \mathbb{Z}$ and we see that $a^i b$ is conjugate to any $a^k b$.  However, when $2|n$ we have a parity constraint that
so far creates the two classes.  We need to also verify conjugation by
$a^jb$ does not fuse the two classes.  Indeed
\[(a^i b)^{a^j b}=(a^{-2j+i}b)^b=ba^{-2j+i}b=a^{2j-i}b,\]
thus we retain two conjugacy classes amongst the reflections.

Finally, the order of the elements $(a^i b)$ is 2 -- a fact used already.
Thus the only cyclic subgroup of order $n$, when $n>2$, is $C_n$ and
thus by its uniqueness it is characteristic.
\end{proof}

\begin{prop}
The maximal subgroups of $D_{2n}$ are dihedral or cyclic.  In particular,
the unique maximal cyclic group is $C_n=\langle a\rangle$ and the
maximal dihedral groups are those of the form $\langle a^{n/p}, a^ib\rangle$
for primes $p$ dividing $n$.
\end{prop}

We will prove this with a more general claim.  First we pause to note that as a corollary to these two propositions we can determine the entire lattice of normal and characteristic subgroups of a dihedral group.  

\begin{coro}
A proper subgroup $H$ of $D_{2n}$ is normal in $D_{2n}$ if and only if $H\leq \langle a\rangle$ or $2|n$, and $H$ is one the following two maximal subgroups
of index 2:
\[M_1=\langle a^2,b\rangle,\qquad M_2=\langle a^2,ab\rangle.\]
The proper characteristic subgroups of $D_{2n}$ are all the subgroups of $\langle a\rangle$.
\end{coro}
\begin{proof}
If $H$ is normal and contains an element of the form $a^i b$, then it contains 
the entire conjugacy class of $a^i b$.  If $n$ is odd then all reflections are
conjugate to $a^ib$ so $H$ contains all reflections of $D_{2n}$ and so $H$ is $D_{2n}$ as the relfections generate $D_{2n}$.

If instead $n$ is even then $H$ is forced only to contain one of the two conjugacy classes of reflections.  If $i$ is even then $H$ contains $b$ and $a^2b$ so it contains $a^2$.  If $i$ is odd then $H$ contains $ab$ and $a^3b$
so it contains $a^2=aba^3b$ (note $n>3$ as $n>2$ and $2|n$).

The two maximal subgroups of index $2$ which can exist when $n$ is even can be interchanged by an outer automorphism which maps $a\mapsto a^{-1}$ and 
$b\mapsto ab$ so these two are not characterisitic.  The subgroups of a characterisitic cyclic group are necessarily characteristic.
\end{proof}

\begin{prop}
Quotient groups of dihedral groups are dihedral, and subgroups of dihedral
groups are dihedral or cyclic.
\end{prop}

\begin{proof}
The homomorphic image of a dihedral group has two generators $\hat a$
and $\hat b$ which satisfy the conditions $\hat a^{\hat b}=\hat a^{-1}$
and $\hat a^n=1$ and $\hat b^2=1$, therefore the image is a dihedral group.

For subgroups we proceed by induction.  When $n=1$ the result is clear.
Now suppose that $D_{2n}$ has some proper subgroup $H$ that is not dihedral
or cyclic.  $H$ is contained in some maximal subgroup $M$ of $D_{2n}$.  However
the maximal subgroups of $D_{2n}$ are cyclic or dihedral so $H$ falls to 
the induction step for $M$ -- together with the fact that subgroups of
cyclic groups are cyclic.  Thus $H$ must actually be dihedral or cyclic
to avoid contradictions.
\end{proof}

\begin{prop}\label{prop:dihedral:nilpotent}
$D_n$ is nilpotent if and only if $n=2^i$ for some $i\geq 0$.
\end{prop}

\begin{prop}\label{prop:dihedral:solvable}
$D_{2n}$ is solvable for all $n\geq 1$.
\end{prop}

\begin{proof}
When $n=1$, $D_{2n}\cong \mathbb{Z}_2$ which is nilpotent and so also solvable.
Now let $n>1$.  Then $D_{2n}/\langle a\rangle\cong \mathbb{Z}_2$ and $\langle a\rangle\cong \mathbb{Z}_n$.  Both $\mathbb{Z}_n$ and $\mathbb{Z}_2$ are nilpotent and so they are both solvable.
As extensions of solvable groups are solvable, $D_{2n}$ is solvable for all $n>0$.
\end{proof}

\subsection{Automorphisms of $D_{2n}$}

\begin{thm}
Let $n>2$.
The automorphism group of $D_{2n}$ is isomorphic to 
$\mathbb{Z}_n^\times \ltimes \mathbb{Z}_n$, with the canonical action of 
$1:\mathbb{Z}_n^\times \rightarrow \Aut \mathbb{Z}_n = \mathbb{Z}_n^\times$.
Explicitly, 
\[\Aut D_{2n}=\{\gamma_{s,t}~|~s\in \mathbb{Z}_n^\times, t\in \mathbb{Z}_n\}\]
with $\gamma_{s,t}$ defined as
\[(a^i)\gamma_{s,t}=\alpha^{is},\qquad (a^i b)\gamma_{s,t}=a^{is+t}b.\]
\end{thm}

\begin{proof}
We apply the needle-in-the-haystack heuristic and search first to
explain why these are the only possible forms for the automorphisms.
We will then prove all such are indeed automorphisms.

Given $\gamma\in \Aut D_{2n}$, we know $\langle a\rangle$ is characteristic
in $D_{2n}$ so $a\gamma=a^s$ for some $s\in \mathbb{Z}_n$.  But $\gamma$ is invertible
so indeed $(s,n)=1$ so that $s\in \mathbb{Z}_n^\times$.  Next $b\gamma=a^t b$
as $b$ cannot be sent to $\langle a\rangle$.

Now we claim $\gamma=\gamma_{s,t}$.
\[(a^i)\gamma=a^{is}=(a^i)\gamma_{s,t}\]
and 
\[(a^i b)\gamma=a^{is} a^tb=a^{is+t}b=(a^ib)\gamma_{s,t}.\]
Now we must show all $\gamma_{s,t}$ are indeed homomorphisms when 
$s\in \mathbb{Z}_n^\times$ and $t\in \mathbb{Z}_n$.  First we note that $\gamma$ is well-defined as we have an irredundant listing of the elements.  Next we verify the homomorphism cases.
\begin{eqnarray*}
(a^i a^j)\gamma_{s,t} & = & a^{(i+j)s} =a^{is}a^{js}=(a^i)\gamma_{s,t}(a^j)\gamma_{s,t}.\\
(a^i a^j b)\gamma_{s,t} & = & a^{(i+j)s +t}b=a^{is} (a^{js+t}b)
	=(a^i)\gamma_{s,t}(a^i b)\gamma_{s,t}.\\
(a^ib a^j)\gamma_{s,t} & = & (a^{i-j}b)\gamma_{s,t}=a^{(i-j)s+t}b = a^{is+t}b a^{js}
	= (a^ib)\gamma_{s,t} (a^j)\gamma_{s,t}.\\
(a^iba^jb)\gamma_{s,t} & = & (a^{i-j})\gamma_{s,t} = a^{is-js}=a^{is +t -t -is}\\
	& = & (a^{is+t}b)(ba^{-t-is})=(a^{is+t}b)(a^{js+t}b)=(a^ib)\gamma_{s,t}(a^jb)\gamma_{s,t}.\\
\end{eqnarray*}
So indeed $\gamma_{s,t}$ is a homomorphism.

Finally, we show the composition of two such maps both to identify
the automorphism group and to show that each $\gamma_{s,t}$ is invertible.
\[(a^ib)\gamma_{s,t}\gamma_{u,v}=(a^{is+t}b)\gamma_{u,v}=a^{isu+tu+v}b.\]
Hence, $\gamma_{s,t}\gamma_{u,v}=\gamma_{su,tu+v}$.  This agrees on $a^i$'s as well.
This reveals the isomorphism desired: $\Aut D_{2n}\rightarrow\mathbb{Z}_n^\times \ltimes \mathbb{Z}_n$ by $\gamma_{s,t}\mapsto (s,t)$ where we see the multiplications agree as
\[(s,t)(u,v)=(su,tu+v).\]
In fact this demonstrates that the inverse of $\gamma_{s,t}$ is simply
$\gamma_{s^{-1},-ts^{-}}$ and the identity map is $\gamma_{1,0}$.
\end{proof}


%%%%%
%%%%%
\end{document}
