\documentclass[12pt]{article}
\usepackage{pmmeta}
\pmcanonicalname{FreeGroup}
\pmcreated{2013-03-22 12:28:39}
\pmmodified{2013-03-22 12:28:39}
\pmowner{yark}{2760}
\pmmodifier{yark}{2760}
\pmtitle{free group}
\pmrecord{43}{32687}
\pmprivacy{1}
\pmauthor{yark}{2760}
\pmtype{Definition}
\pmcomment{trigger rebuild}
\pmclassification{msc}{20E05}
\pmrelated{FreeModule}
\pmrelated{Subgroup}
\pmrelated{Group}
\pmrelated{FreeProduct}
\pmdefines{freely generates}
\pmdefines{freely generated}
\pmdefines{rank}
\pmdefines{empty word}

\endmetadata

\usepackage{amssymb}
\usepackage{amsmath}
\usepackage{amsfonts}

\def\FA{\mathcal{F}[A]}
\def\WA{\mathcal{W}[A]}
\def\Z{\mathbb{Z}}
\begin{document}
\PMlinkescapeword{alphabet}
\PMlinkescapeword{contraction}
\PMlinkescapeword{contractions}
\PMlinkescapeword{equivalent}
\PMlinkescapeword{expansion}
\PMlinkescapeword{expansions}
\PMlinkescapeword{generates}
\PMlinkescapeword{index}
\PMlinkescapeword{inverse}
\PMlinkescapeword{minimal}
\PMlinkescapephrase{multiplicative group}
\PMlinkescapeword{occurrence}
\PMlinkescapeword{order}
\PMlinkescapeword{rank}
\PMlinkescapeword{ranks}
\PMlinkescapeword{states}
\PMlinkescapeword{structure}
\PMlinkescapeword{type}
\PMlinkescapeword{word}
\PMlinkescapeword{words}

\section*{Definition}

If $F$ is a group with a subset $A$ such that for every group $G$
every function $\psi\colon A\to G$ extends to a unique \PMlinkname{homomorphism}{GroupHomomorphism}
$\phi\colon F\to G$, then $F$ is said to be a \emph{free group} of \emph{rank} $|A|$,
and we say that $A$ \emph{freely generates} $F$.

\section*{Examples}

A group with only one element is a free group of rank $0$, freely generated by the empty set.

The infinite cyclic group $\Z$ is a free group of rank $1$, freely generated by either $\{1\}$ or $\{-1\}$.

An example of a free group of rank $2$ is the multiplicative group of $2\times2$ integer matrices generated by
\[ \left( \begin{array}{cc}
1 & 2 \\
0 & 1 \end{array} \right)\phantom{.}\] 
and
\[ \left( \begin{array}{cc}
1 & 0 \\
2 & 1 \end{array} \right).\] 

If $F$ is a free group freely generated by a set $A$, where $|A|>1$, then for distinct $a,b\in A$ the set $\{a^nb^n\mid 0<n\in\Z\}$ generates a free group of countably infinite rank.

\section*{Properties}

If a free group $F$ is freely generated by $A$, then $A$ is a minimal generating set for $F$, and no set of smaller cardinality than $A$ can generate $F$.
It follows that if $F$ is freely generated by both $A$ and $B$, then $|A|=|B|$. So the rank of a free group is a well-defined concept, and free groups of different ranks are non-isomorphic.

For every cardinal number $\kappa$ there is, up to isomorphism, exactly one free group of rank $\kappa$.
The abelianization of a free group of rank $\kappa$ is a free abelian group of rank $\kappa$.

Every group is a homomorphic image of some free group.
More precisely, if $G$ is a group generated by a set of cardinality $\kappa$, then $G$ is a homomorphic image of every free group of rank $\kappa$ or more.

The \PMlinkname{Nielsen-Schreier Theorem}{NielsenSchreierTheorem} states that every \PMlinkname{subgroup}{Subgroup} of a free group is itself free.

\section*{Construction}

For any set $A$, the following construction gives a free group of rank $|A|$.

Let $A$ be a set with elements $a_{i}$ for $i$ in some index set $I$. 
We refer to $A$ as an \emph{alphabet} and the elements of $A$ as \emph{letters}.
A \emph{syllable} is a symbol of the form $a_{i}^n$ for $n \in \mathbb{Z}$.
It is customary to write $a$ for $a^1$. Define a \emph{word} to be a finite sequence of syllables. For example, 
$$ a_2^3a_1a_4^{-1}a_3^2a_2^{-3}$$
is a five-syllable word. Notice that there exists a unique \emph{empty word}, i.e., the word with no syllables, usually written simply as $1$. 
Denote the set of all words formed from elements of $A$ by $\WA$. 

Define a binary operation, called the product, on $\WA$ by concatenation of words. To illustrate, if $a_2^{3}a_1$ and $a_1^{-1}a_3^{4}$ are elements of $\WA$ then their product is simply $a_2^{3}a_1a_1^{-1}a_3^{4}$. 
This gives $\WA$ the structure of a monoid. 
The empty word $1$ acts as a \PMlinkname{right and left identity}{LeftIdentityAndRightIdentity} in $\WA$, and is the only element which has an inverse. In order to give $\WA$ the structure of a group, two more ideas are needed.

If $v=u_1a_i^0u_2$ is a word where $u_1,u_2$ are also words and $a_i$ is some element of $A$, an \emph{elementary contraction of type I} replaces the occurrence of $a^0$ by $1$. Thus, after this type of contraction we get another word $w=u_1u_2$. If $v=u_1a_i^pa_i^qu_2$ is a word, an \emph{elementary contraction of type II} replaces the occurrence of $a_i^pa_i^q$ by $a_i^{p+q}$ which results in $w=u_1a_i^{p+q}u_2$. In either of these cases, we also say that $w$ is obtained from $v$ by an elementary contraction, or that $v$ is obtained from $w$ by an elementary expansion.

Call two words $u,v$ equivalent (denoted $u \sim v$) if one can be obtained from the other by a finite sequence of elementary contractions or expansions. This is an equivalence relation on $\WA$.  Let $\FA$ be the set of equivalence classes of words in $\WA$. Then $\FA$ is group under the operation $$[u][v] = [uv]$$
where $[u] \in \FA$. The inverse $[u]^{-1}$ of an element $[u]$ is obtained by reversing the order of the syllables of $[u]$ and changing the sign of each syllable. For example, if $[u] = [a_1a_3^{2}]$, then $[u]^{-1} = [a_3^{-2}a_1^{-1}]$. 

It can be shown that $\FA$ is a free group freely generated by the set $\{[x]\mid x\in A\}$. Moreover, a group is free if and only if it is isomorphic to $\FA$ for some set $A$.
%%%%%
%%%%%
\end{document}
