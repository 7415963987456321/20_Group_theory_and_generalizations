\documentclass[12pt]{article}
\usepackage{pmmeta}
\pmcanonicalname{GeneralizedQuaternionGroup}
\pmcreated{2013-03-22 16:27:41}
\pmmodified{2013-03-22 16:27:41}
\pmowner{Algeboy}{12884}
\pmmodifier{Algeboy}{12884}
\pmtitle{generalized quaternion group}
\pmrecord{7}{38620}
\pmprivacy{1}
\pmauthor{Algeboy}{12884}
\pmtype{Derivation}
\pmcomment{trigger rebuild}
\pmclassification{msc}{20A99}
\pmsynonym{quaternion groups}{GeneralizedQuaternionGroup}
\pmrelated{DihedralGroupProperties}
\pmdefines{generalized quaternion}

\endmetadata

\usepackage{latexsym}
\usepackage{amssymb}
\usepackage{amsmath}
\usepackage{amsfonts}
\usepackage{amsthm}

%%\usepackage{xypic}

%-----------------------------------------------------

%       Standard theoremlike environments.

%       Stolen directly from AMSLaTeX sample

%-----------------------------------------------------

%% \theoremstyle{plain} %% This is the default

\newtheorem{thm}{Theorem}

\newtheorem{coro}[thm]{Corollary}

\newtheorem{lem}[thm]{Lemma}

\newtheorem{lemma}[thm]{Lemma}

\newtheorem{prop}[thm]{Proposition}

\newtheorem{conjecture}[thm]{Conjecture}

\newtheorem{conj}[thm]{Conjecture}

\newtheorem{defn}[thm]{Definition}

\newtheorem{remark}[thm]{Remark}

\newtheorem{ex}[thm]{Example}



%\countstyle[equation]{thm}



%--------------------------------------------------

%       Item references.

%--------------------------------------------------


\newcommand{\exref}[1]{Example-\ref{#1}}

\newcommand{\thmref}[1]{Theorem-\ref{#1}}

\newcommand{\defref}[1]{Definition-\ref{#1}}

\newcommand{\eqnref}[1]{(\ref{#1})}

\newcommand{\secref}[1]{Section-\ref{#1}}

\newcommand{\lemref}[1]{Lemma-\ref{#1}}

\newcommand{\propref}[1]{Prop\-o\-si\-tion-\ref{#1}}

\newcommand{\corref}[1]{Cor\-ol\-lary-\ref{#1}}

\newcommand{\figref}[1]{Fig\-ure-\ref{#1}}

\newcommand{\conjref}[1]{Conjecture-\ref{#1}}


% Normal subgroup or equal.

\providecommand{\normaleq}{\unlhd}

% Normal subgroup.

\providecommand{\normal}{\lhd}

\providecommand{\rnormal}{\rhd}
% Divides, does not divide.

\providecommand{\divides}{\mid}

\providecommand{\ndivides}{\nmid}


\providecommand{\union}{\cup}

\providecommand{\bigunion}{\bigcup}

\providecommand{\intersect}{\cap}

\providecommand{\bigintersect}{\bigcap}










\begin{document}
The groups given by the presentation
\[Q_{4n}=\langle a,b : a^n=b^2, a^{2n}=1, b^{-1}ab=a^{-1}\rangle\]
are the \emph{generalized quaternion groups}.  Generally one insists that $n>1$
as the properties of generalized quaternions become more uniform at this stage.
However if $n=1$ then one observes $a=b^2$ so $Q_{4n}\cong\mathbb{Z}_4$.
Dihedral group properties are strongly related to generalized quaternion
group properties because of their highly related presentations.  We will see this in many of our results.

\begin{prop}
\begin{enumerate}
\item $|Q_{4n}|=4n$.
\item $Q_{4n}$ is abelian if and only if $n=1$.
\item Every element $x\in Q_{4n}$ can be written uniquely as $x=a^i b^j$
where $0\leq i<2n$ and $j=0,1$.
\item $Z(Q_{4n})=\langle a^n\rangle\cong \mathbb{Z}_2$.
\item $Q_{4n}/Z(Q_{4n})\cong D_{2n}$.
\end{enumerate}
\end{prop}
\begin{proof}
Given the relation $b^{-1} ab=a^{-1}$ (rather treating it as $ab=ba^{-1}$) then 
as with dihedral groups we can 
shuffle words in $\{a,b\}$ to group all the $a's$ at the beginning and the
$b's$ at the end.  So every word takes the form $a^i b^j$.  As $|a|=2n$
and $|b|=4$ we have $0\leq i<2n$ and $0\leq j<4$.  However we have an 
added relation that $a^n=b^2$ so we can write $a^i b^2=a^{i+2}$ and
also $a^i b^3=a^{i+2}b$ so we restrict to $j=0,1$.  This gives us $4n$ elements
of this form which makes the order of $Q_{4n}$ at most $4n$.

As $a^n=b^2$ it follows $[a^n,a^i b^j]=[a^n,b^j]=[b^2,b^j]=1$.  So
$a^n$ is central.  If we quotient by $\langle a^n\rangle$ then we have
the presentation
\[\langle a,b:a^n=1, b^2=1, b^{-1} ab=a^{-1}\rangle\]
which we recognize as the presentation of the dihedral group.  Thus
$Q_{4n}/\langle a^n\rangle\cong D_{2n}$.  This prove the order of $Q_{4n}$
is exactly $4n$.  Moreover, given $a^i b^j\in Z(Q_{4n})$ we have
\[1=[a^i b^j, b]=b^{-j}a^{-i} b^{-1} a^{i} b^j b
   = b^{-j} a^{-i} a^{-i} b^{-1} b^j b
   = b^{-j} a^{-2i} b^j
   = a^{2i}.\]
So we have $i=n$.  So $a^n b^j=b^{j+2}$.  Then $1=[b^{j+2},a]$ forces $j=0,2$.
This means $Z(Q_{4n})=\langle a^n\rangle=\langle b^2\rangle$.  
\end{proof}


\section{Examples}

As mentioned, if $n=1$ then $Q_4\cong \mathbb{Z}_4$.  If $n=2$ then we have the
usual quaternion group $Q_8$.  Because of the genesis of quaternions, this group is often denoted with $i,j,k$ relations as follows:
\[Q_8=\langle -1, i,j,k : i^2=j^2=k^2=-1, ij=k=-1 ji\rangle.\]
These relations are responsible for many useful results such as defining cross products for three-dimensional manipulations, and are also responsible for the
most common example of a division ring.  As a group, $Q_8$ is a curious specimen of a $p$-group in that it has only normal subgroups yet is non-abelian, it has a unique minimal subgroup and cannot be represented faithfully except by a regular representation -- thus requiring degree 8.  [To see this note that the unique minmal subgroup is necessarily normal, thus if a proper subgroup is the stabilizer of an action, then the minimal normal subgroup is in the kernel so the representation is not faithful.]

A common work around is to use $2\times 2$ matrices over $\mathbb{C}$ but to treat these as matrices over $\mathbb{R}$.
\[
-1=
\begin{bmatrix}
-1 & 0 \\
0 & -1
\end{bmatrix},
\quad
i=
\begin{bmatrix}
i & 0 \\
0 & -i
\end{bmatrix},
\quad
j=
\begin{bmatrix}
0 & i \\
i & 0
\end{bmatrix},
k=
\begin{bmatrix}
0 & -1 \\
1 & 0
\end{bmatrix}.
\]

A worthwhile additional example is $n=3$.  For this
produces a group order 12 which is often overlooked.

\section{Subgroup structure}

\begin{prop}
$Q_{4n}$ is Hamiltonian -- meaning all a non-abelian group whose subgroups are normal -- if and only 
if $n=2$.
\end{prop}
\begin{proof}
As $Q_{4n}/Z(Q_{4n})\cong D_{2n}$, then if $Q_{4n}$ is Hamiltonian then we require
$D_{2n}$ to be as well.  However when $n>2$ we know $D_{2n}$ has non-normal subgroups,
for example $\langle ab\rangle$.  So we require $n\leq 2$.  If $n=1$ then $Q_{4n}$
is cyclic and so trivially Hamiltonian.  When $n=2$ we have the usual quaternion
group of order 8 which is Hamiltonian by direct inspection: the conjugacy classes
are $\{1\}$, $\{a^2\}$, $\{a,a^3\}$, $\{b,a^2b\}$ and $\{ab,a^3 b\}$, more commonly
described by $\{1\}$, $\{-1\}$, $\{i,-i\}$, $\{j,-j\}$ and $\{k,-k\}$.  In any case,
all subgroups are normal.
\end{proof}

By way of converse it can be shown that the only finite Hamiltonian groups are
$A\oplus Q_8$ where $A$ is abelian without an element of order 4.
One sees already in $\mathbb{Z}_4\oplus Q_8$ that the subgroup $\langle (1,i)\rangle$  is conjugate to the distinct subgroup $\langle (1,-i)\rangle$ and so such groups are not Hamiltonian.


\begin{prop}
\begin{enumerate}
\item $|a^i|=2n/i$ for $1<i\leq 2n$ and $|a^i b|=4$ for all $i$.
\item Every subgroup of $Q_{4n}$ is either cyclic or a generalized
quaternion.
\item The normal subgroups of $Q_{4n}$ are either subgroups of
$\langle a\rangle$ or $n=2^i$ and it is maximal subgroups (of index 2) of
which there are 2 acyclic ones.
\end{enumerate}
\end{prop}
\begin{proof}
The order of elements of $\langle a\rangle$ follows from standard cyclic group theory.
Now for $a^i b$ we simply compute: $(a^i b)^2=a^i b a^i b=a^i a^{-i} b^2=b^2$.
So $|a^i b|=4$.

Now let $H$ be a subgroup of $Q_{4n}$.  If $Z(Q_{4n})\leq H$ then $H/Z(Q_{4n})$ is
a subgroup of $D_{2n}$.  We know the subgroups of $D_{2n}$ are either cyclic or
dihedral.  If $H/Z(Q_{4n})$ is cyclic then $H$ is cyclic (indeed it is a subgroup
of $\langle a\rangle$ or $H=\langle a^i b\rangle$).  So assume that $H/Z(Q_{4n})$ is
dihedral.  Then we have a dihedral presentation 
$\langle x,y:x^m=1, y^2=1, y^{-1}xy=x^{-1}\rangle$ for $H/Z(Q_{4n})$.  Now pullback
this presentation to $H$ and we find $H$ is quaternion.  

Finally, if $H$ does not contain $Z(Q_{4n})$ then $H$ does not contain an element
of the form $a^i b$, so $H\leq \langle a\rangle$ and so it is cyclic.

For the normal subgroup structure, from the relation $b^{-1} ab=a^{-1}$ we
see $\langle a\rangle$ is normal.  Thus all subgroups of $\langle a\rangle$
are normal as $\langle a\rangle$ is a normal cyclic subgroup.  Next suppose
$H$ is a normal subgroup not contained in $\langle a\rangle$.  Then $H$ contains
some $a^i b$, and so $H$ contains $Z(Q_{4n})$.  Thus $H/Z(Q_{4n})$ is a normal
subgroup of $D_{2n}$.  We know this forces $H/Z(Q_{4n})$ to be contained in
$\langle a\rangle/Z(Q_{4n})$, a contradiction on our assumptions on $H$, or
$n=2^i$ and $H/Z(Q_{4n})$ is a maximal subgroup (of index 2).
\end{proof}

\begin{prop}
$Q_{4n}$ has a unique minimal subgroup if and only if $n=2^i$.
\end{prop}
\begin{proof}
If $p|n$ and $p>2$ then $a^{2n/p}$ has order $p$ and so the subgroup 
$\langle a^{2n/p}\rangle$ is of order $p$, so it is minimal.  As the center
is also a minimal subgroup of order 2, then we do not have a unique minimal subgroup
in these conditions.  Thus $n=2^i$.

Now suppose $n=2^i$ then $Q_{4n}$ is a $2$-group so the minimal subgroups must all
be of order 2.  So we locate the elements of order 2.  We have shown $|a^i b|=4$ for
any $i$, and furthermore that $(a^i b)^2=b^2=a^n$.  The only other minimal subgroups
will be generated by $a^i$ for some $i$, and as $|a|=2^{i+1}$ there is a unique minimal
subgroup.
\end{proof}

It can also be shown that any finite group with a unique minimal subgroup is either
cyclic of prime power order, or $Q_{4n}$ for some $n=2^i$.
  We note that these groups have only regular faithful representations.
%%%%%
%%%%%
\end{document}
