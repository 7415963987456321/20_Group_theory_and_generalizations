\documentclass[12pt]{article}
\usepackage{pmmeta}
\pmcanonicalname{Exponentiation}
\pmcreated{2013-03-22 19:08:44}
\pmmodified{2013-03-22 19:08:44}
\pmowner{pahio}{2872}
\pmmodifier{pahio}{2872}
\pmtitle{exponentiation}
\pmrecord{9}{42047}
\pmprivacy{1}
\pmauthor{pahio}{2872}
\pmtype{Topic}
\pmcomment{trigger rebuild}
\pmclassification{msc}{20-00}
\pmrelated{ContinuityOfNaturalPower}
\pmdefines{power law}
\pmdefines{power of product}

\endmetadata

% this is the default PlanetMath preamble.  as your knowledge
% of TeX increases, you will probably want to edit this, but
% it should be fine as is for beginners.

% almost certainly you want these
\usepackage{amssymb}
\usepackage{amsmath}
\usepackage{amsfonts}

% used for TeXing text within eps files
%\usepackage{psfrag}
% need this for including graphics (\includegraphics)
%\usepackage{graphicx}
% for neatly defining theorems and propositions
 \usepackage{amsthm}
% making logically defined graphics
%%%\usepackage{xypic}

% there are many more packages, add them here as you need them

% define commands here

\theoremstyle{definition}
\newtheorem*{thmplain}{Theorem}

\begin{document}
\PMlinkescapeword{exponents}

\begin{itemize}
\item In the entry general associativity, the notion of the \emph{power} $a^n$ for elements $a$ of a set having an associative binary operation ``$\cdot$'' and for positive integers $n$ as \PMlinkname{exponents}{GeneralPower} was defined as a generalisation of the operation.\, Then the two \emph{power laws}
$$a^m\!\cdot\!a^n \;=\; a^{m+n}, \quad (a^m)^n \;=\; a^{mn}$$
are \PMlinkescapetext{valid}.\, For the validity of the third well-known power law, 
$$(a\!\cdot\!b)^n \;=\; a^n\!\cdot\!b^n,$$
the law of \emph{power of product}, the commutativity of the operation is needed.

\textbf{Example.}\,In the symmetric group $S_3$, where the group operation is not commutative, we get different results from
$$[(123)(13)]^2 \;=\; (23)^2 \;=\; (1)$$ 
and
$$(123)^2(13)^2 \;=\; (132)(1) \;=\; (132)$$
(note that in these ``products'', which \PMlinkescapetext{mean} compositions of mappings, the latter ``factor'' acts first).
\end{itemize}


\begin{itemize}
\item Extending the power notion for zero and negative integer exponents requires the existence of \PMlinkid{neutral}{10539} and inverse elements ($e$ and $a^{-1}$):
$$a^0 \;:=\; e, \qquad a^{-n} \;:=\; (a^{-1})^n$$
The two first power laws then remain in \PMlinkescapetext{force} for all integer exponents, and if the operation is commutative, also the \PMlinkescapetext{third power law is universal}.
\end{itemize}

When the operation in question is the multiplication of real or complex numbers, the power notion may be extended for other than integer exponents.

\begin{itemize}
\item One step is to introduce \PMlinkname{fractional}{FractionalNumber} exponents by using \PMlinkname{roots}{NthRoot}; see the fraction power.
\item The following step would be the irrational exponents, which are \PMlinkescapetext{contained} in the power functions.\, The irrational exponents are possible to introduce by utilizing the exponential function and logarithms; another way would be to define $a^\varrho$ as limit of a sequence 
$$a^{r_1},\,a^{r_2},\,\ldots$$
where the limit of the rational number sequence \,$r_1,\,r_2,\,\ldots$\, is $\varrho$.\, The sequence 
$a^{r_1},\,a^{r_2},\,\ldots$ may be shown to be a Cauchy sequence.
\item The last step were the imaginary (non-real complex) exponents $\mu$, when also the base of the power may be other than a positive real number; the one gets the so-called \emph{general power}.
\end{itemize}

%%%%%
%%%%%
\end{document}
