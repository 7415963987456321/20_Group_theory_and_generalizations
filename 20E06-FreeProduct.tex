\documentclass[12pt]{article}
\usepackage{pmmeta}
\pmcanonicalname{FreeProduct}
\pmcreated{2013-03-22 14:53:34}
\pmmodified{2013-03-22 14:53:34}
\pmowner{yark}{2760}
\pmmodifier{yark}{2760}
\pmtitle{free product}
\pmrecord{12}{36574}
\pmprivacy{1}
\pmauthor{yark}{2760}
\pmtype{Definition}
\pmcomment{trigger rebuild}
\pmclassification{msc}{20E06}
\pmrelated{FreeProductWithAmalgamatedSubgroup}
\pmrelated{FreeGroup}
\pmdefines{free factor}

\endmetadata

\usepackage{amssymb}
\usepackage{amsmath}

\def\genby#1{{\langle #1\rangle}}
\def\isomorphic{\cong}
\begin{document}
\PMlinkescapeword{generates}
\PMlinkescapeword{homomorphism}
\PMlinkescapeword{obvious}
\PMlinkescapeword{quotient}
\PMlinkescapeword{subgroup}
\PMlinkescapeword{subgroups}

\section*{Definition}

Let $G$ be a group, and let $(A_i)_{i\in I}$ be 
a family of \PMlinkname{subgroups}{Subgroup} of $G$.
Then $G$ is said to be a \emph{free product} of the subgroups $A_i$
if given any group $H$ and 
a \PMlinkname{homomorphism}{GroupHomomorphism} 
$f_i\colon A_i\to H$ for each $i\in I$,
there is a unique homomorphism $f\colon G\to H$
such that $f|_{A_i}=f_i$ for all $i\in I$.
The subgroups $A_i$ are then called the \emph{free factors} of $G$.

If $G$ is the free product of $(A_i)_{i\in I}$,
and $(K_i)_{i\in I}$ is a family of groups such that $K_i\isomorphic A_i$
for each $i\in I$,
then we may also say that $G$ is the free product of $(K_i)_{i\in I}$.
With this definition, every family of groups has a free product,
and the free product is unique up to isomorphism.

The free product is the coproduct in the category of groups.

\section*{Construction}

Free groups are simply the free products of infinite cyclic groups,
and it is possible to generalize the construction given in the free group
article to the case of arbitrary free products.
But we will instead construct
the free product as a \PMlinkname{quotient}{QuotientGroup} of a free group.

Let $(K_i)_{i\in I}$ be a family of groups.
For each $i\in I$,
let $X_i$ be a set and $\gamma_i\colon X_i\to K_i$ a function
such that $\gamma_i(X_i)$ generates $K_i$.
The $X_i$ should be chosen to be pairwise disjoint;
for example, we could take $X_i=K_i\times\{i\}$,
and let $\gamma_i$ be the obvious bijection.
Let $F$ be a free group freely generated by $\bigcup_{i\in I}X_i$.
For each $i\in I$,
the subgroup $\genby{X_i}$ of $F$ is freely generated by $X_i$,
so there is a homomorphism $\phi_i\colon\genby{X_i}\to K_i$
extending $\gamma_i$.
Let $N$ be the normal closure of $\bigcup_{i\in I}\ker{\phi_i}$ in $F$.

Then it can be shown that $F/N$ is 
the free product of the family of subgroups $(\genby{X_i}N/N)_{i\in I}$,
and $K_i\isomorphic\genby{X_i}N/N$ for each $i\in I$.
%%%%%
%%%%%
\end{document}
