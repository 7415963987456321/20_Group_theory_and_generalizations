\documentclass[12pt]{article}
\usepackage{pmmeta}
\pmcanonicalname{ExamplesOfNoncommutativeOperations}
\pmcreated{2013-03-22 15:03:04}
\pmmodified{2013-03-22 15:03:04}
\pmowner{yark}{2760}
\pmmodifier{yark}{2760}
\pmtitle{examples of non-commutative operations}
\pmrecord{10}{36768}
\pmprivacy{1}
\pmauthor{yark}{2760}
\pmtype{Example}
\pmcomment{trigger rebuild}
\pmclassification{msc}{20-00}

\endmetadata

\usepackage{amsmath}
\usepackage{amsfonts}

\newcommand{\Z}{\mathbb{Z}}
\newcommand{\C}{\mathbb{C}}
\newcommand{\R}{\mathbb{R}}
\newcommand{\Q}{\mathbb{Q}}
\begin{document}
A standard example of a non-commutative operation is matrix multiplication. Consider the following two integer matrices:
\[
A=\begin{pmatrix}
1 & 1\\
0&1
\end{pmatrix},\qquad
B=\begin{pmatrix}
0 & 1\\
0 & 1
\end{pmatrix}
\]

If we compute $AB$ we get 
\[
AB=\begin{pmatrix}
0 & 2 \\
0 & 1
\end{pmatrix}
\]
but if we compute $BA$ we have
\[
BA=\begin{pmatrix}
0 & 1 \\
0 & 1
\end{pmatrix}.
\]

Since $AB\neq BA$ we conclude that matrix product is not commutative.

Operations do not necessarily have to operate on numbers. Another classic example is function composition. Let $f$ and $g$ be real functions given by
\[
f(x) = x^2,\qquad g(x) = 2x.
\]

We see that
\[
(f\circ g)(x) = f(g(x)) = (2x)^2 = 4x^2,
\]
but 
\[
(g \circ f )(x) = g(f(x)) = 2(x^2) = 2x^2.
\]
Since we got different functions, we conclude that function composition is not commutative.
%%%%%
%%%%%
\end{document}
