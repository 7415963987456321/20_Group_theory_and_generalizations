\documentclass[12pt]{article}
\usepackage{pmmeta}
\pmcanonicalname{GroupsWithAbelianInnerAutomorphismGroup}
\pmcreated{2013-03-22 17:25:30}
\pmmodified{2013-03-22 17:25:30}
\pmowner{rm50}{10146}
\pmmodifier{rm50}{10146}
\pmtitle{groups with abelian inner automorphism group}
\pmrecord{7}{39799}
\pmprivacy{1}
\pmauthor{rm50}{10146}
\pmtype{Topic}
\pmcomment{trigger rebuild}
\pmclassification{msc}{20A05}

\endmetadata

% this is the default PlanetMath preamble.  as your knowledge
% of TeX increases, you will probably want to edit this, but
% it should be fine as is for beginners.

% almost certainly you want these
\usepackage{amssymb}
\usepackage{amsmath}
\usepackage{amsfonts}

% used for TeXing text within eps files
%\usepackage{psfrag}
% need this for including graphics (\includegraphics)
%\usepackage{graphicx}
% for neatly defining theorems and propositions
%\usepackage{amsthm}
% making logically defined graphics
%%%\usepackage{xypic}

% there are many more packages, add them here as you need them

% define commands here
\DeclareMathOperator{\Inn}{Inn}
\newcommand{\cD}{\mathcal{D}}
\begin{document}
The inner automorphism group $\Inn(G)$ is isomorphic to the central quotient of $G$, $G/Z(G)$. \PMlinkescapetext{Even} if $\Inn(G)$ is abelian, one cannot conclude that $G$ itself is abelian. For example, let $G=\cD_8$, the dihedral group of symmetries of the square.
\[G=\langle r,s \mid r^4=s^2=1, rs = sr^3\rangle\]
and $Z(G)=\{1,r^2\}$. Representatives of the cosets of $Z(G)$ are $\{1,r,s,rs\}$; these define a group of order $4$ that is isomorphic to the \PMlinkname{Klein $4$-group}{Klein4Group} $V_4$. Thus the central quotient is abelian, but the group itself is not.

However, if the central quotient is \emph{cyclic}, it does follow that $G$ is abelian. For, choose a representative $x$ in $G$ of a generator for $G/Z(G)$. Each element of $G$ is thus of the form $x^az$ for $z\in Z(G)$. So given $g,h\in G$,
\[gh = x^{a_1}z_1x^{a_2}z_2=x^{a_1}x^{a_2}z_1z_2 = x^{a_1+a_2}z_1z_2=x^{a_2}x^{a_1}z_2z_1=x^{a_2}z_2x^{a_1}z_1=hg\]
where the various manipulations are justified by the fact that the $z_i\in Z(G)$ and that powers of $x$ commute with themselves.

Finally, note that if $\Inn(G)$ is non-trivial, then $G$ is nonabelian, for $\Inn(G)$ nontrivial implies that for some $g\in G$, conjugation by $g$ is not the identity, so there is some element of $G$ with which $g$ does not commute. So by the above argument, $\Inn(G)$, if non-trivial, cannot be cyclic (else $G$ would be abelian).
%%%%%
%%%%%
\end{document}
