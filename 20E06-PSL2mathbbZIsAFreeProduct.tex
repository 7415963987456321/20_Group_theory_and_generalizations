\documentclass[12pt]{article}
\usepackage{pmmeta}
\pmcanonicalname{PSL2mathbbZIsAFreeProduct}
\pmcreated{2013-03-22 17:35:21}
\pmmodified{2013-03-22 17:35:21}
\pmowner{rm50}{10146}
\pmmodifier{rm50}{10146}
\pmtitle{$PSL_2(\mathbb{Z})$ is a free product}
\pmrecord{7}{40001}
\pmprivacy{1}
\pmauthor{rm50}{10146}
\pmtype{Example}
\pmcomment{trigger rebuild}
\pmclassification{msc}{20E06}

% this is the default PlanetMath preamble.  as your knowledge
% of TeX increases, you will probably want to edit this, but
% it should be fine as is for beginners.

% almost certainly you want these
\usepackage{amssymb}
\usepackage{amsmath}
\usepackage{amsfonts}

% used for TeXing text within eps files
%\usepackage{psfrag}
% need this for including graphics (\includegraphics)
%\usepackage{graphicx}
% for neatly defining theorems and propositions
%\usepackage{amsthm}
% making logically defined graphics
%%%\usepackage{xypic}
\usepackage{pst-plot}
\usepackage{psfrag}

% there are many more packages, add them here as you need them

% define commands here
\newcommand{\Nats}{\mathbb{N}}
\newcommand{\Ints}{\mathbb{Z}}
\newcommand{\Reals}{\mathbb{R}}
\newcommand{\Complex}{\mathbb{C}}
\newcommand{\Rats}{\mathbb{Q}}
\newcommand{\Half}{\mathbb{H}}
\newcommand{\Gal}{\operatorname{Gal}}
\newcommand{\Cl}{\operatorname{Cl}}
\newcommand{\Alg}{\mathcal{O}}
\newcommand{\ol}{\overline}
\newcommand{\Leg}[2]{\left(\frac{#1}{#2}\right)}
\newrgbcolor{LightGreen}{0.7 1 0.7}
\newrgbcolor{LightYellow}{1 1 0.7}
\newrgbcolor{LightBlue}{0.7 1 1}
\newrgbcolor{LightRed}{1 0.7 0.7}

\begin{document}
We know that the modular group $\Gamma=\mathrm{PSL}_2(\Ints)$ is equal to the subgroup generated by the two elements $S, T\in\Gamma$ with
\begin{gather*}
S=\begin{pmatrix}0 & -1\\1 & 0\end{pmatrix} = \frac{-1}{z}\\
T=\begin{pmatrix}1 & 1\\0 & 1\end{pmatrix} = z+1
\end{gather*}
or, equivalently, to the subgroup generated by $S$ and $ST$. 

This article shows that in fact $\Gamma$ is the free product of the subgroups $\langle S\rangle$ and $\langle ST\rangle$, denoted $\langle S\rangle \star \langle ST\rangle$. To see this, we use the theorem relating free products and group actions (q.v.). Let $\Half$ represent the upper half-plane, $\Half=\{z\in\Complex\ \mid\ \Im(z)>0\}$, and define (see figure)
\begin{align*}
S_1 &=\{z\in \Half\ \mid\ \Re(z)>0\}\\
S_2 &=\{z\in \Half\ \mid\ \Re(z)<-1/2\}\cup \{z\in\Half\ \mid\ \lvert z+1\rvert<1\}\\
S_3 &= \Half - S_1 - S_2
\end{align*}

\begin{center}
\begin{pspicture}(-6,0)(6,6)
\psset{unit=3cm}
\psframe[linestyle=none,fillstyle=solid,fillcolor=LightYellow](0,0)(2,2)
\psframe[linestyle=none,fillstyle=solid,fillcolor=LightGreen](-0.5,0)(0,2)
\psframe[linestyle=none,fillstyle=solid,fillcolor=LightRed](-2,0)(-0.5,2)
\pswedge[linestyle=none,fillstyle=solid,fillcolor=LightRed](-1,0){1}{0}{90}
\psset{linecolor=blue}
\psarc(-1,0){1}{0}{180}
\psarc(0,0){1}{0}{180}
\psarc(1,0){1}{0}{180}
\psarc(-2,0){1}{0}{90}
\psarc(2,0){1}{90}{180}
\qline(-1.5,0)(-1.5,2)
\qline(-0.5,0)(-0.5,2)
\qline(0.5,0)(0.5,2)
\qline(1.5,0)(1.5,2)
\psarc(-1.6667,0){0.333}{0}{180}
\psarc(-1.3333,0){0.333}{0}{180}
\psarc(-0.6667,0){0.333}{0}{180}
\psarc(-0.3333,0){0.333}{0}{180}
\psarc( 0.3333,0){0.333}{0}{180}
\psarc( 0.6667,0){0.333}{0}{180}
\psarc( 1.3333,0){0.333}{0}{180}
\psarc( 1.6667,0){0.333}{0}{180}
\rput(1.,1.5){\LARGE $S_1$}
\rput(-1.5,1.5){\LARGE $S_2$}
\rput{L}(-0.25,1.3){\LARGE $\Half-S_1-S_2$}
\rput(2,2.1){.}
\psdots(0,1)(-0.5,0.866)(0.5,0.866)
\rput(-.05,.95){$i$}
\rput(-.55,.95){$\rho$}
\rput(.55,.95){$-\overline{\rho}$}
\pscustom[fillstyle=hlines,linestyle=none]{\psarc(0,0){1}{60}{120} \psline(-0.5,2)(.5,2)}
\psset{linecolor=black}
\psaxes[labels=none]{-}(0,0)(-2,0.1)(2,0.1)
\multido{\n=-2+1}{5}{\rput(\n,-.1){\small \n}}
\end{pspicture}
\end{center}

Note first that $S:S_2\cup S_3\to S_1$, since $S$ reverses the sign of the real part of its argument. Thus in particular all nontrivial elements in $\langle S\rangle$ map $S_2\to S_1$.

Second, recall from the article on the modular group that $ST$ rotates the half-plane around the point $\rho$, and that under that rotation, $ST(S_1\cup S_3)\subsetneq S_2$ and $(ST)^2(S_1\cup S_3)\subsetneq S_2$. Thus in particular all nontrivial elements of $\langle ST\rangle$ map $S_1\to S_2$.

Finally, by the above, we see that $S:S_3\to S_1$ and $ST, (ST)^2:S_3\to S_2$.

Since $\Gamma=\langle S,ST\rangle$, we can apply the theorem relating free products and group actions, choosing $s$ to be any point in $S_3$, to conclude that
\[\Gamma =\langle S\rangle \star \langle ST\rangle\cong \Ints/2\Ints \star \Ints/3\Ints\]
%%%%%
%%%%%
\end{document}
