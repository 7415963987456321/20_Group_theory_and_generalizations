\documentclass[12pt]{article}
\usepackage{pmmeta}
\pmcanonicalname{PrefixSet}
\pmcreated{2013-03-22 16:11:56}
\pmmodified{2013-03-22 16:11:56}
\pmowner{Mazzu}{14365}
\pmmodifier{Mazzu}{14365}
\pmtitle{prefix set}
\pmrecord{6}{38292}
\pmprivacy{1}
\pmauthor{Mazzu}{14365}
\pmtype{Definition}
\pmcomment{trigger rebuild}
\pmclassification{msc}{20M05}
%\pmkeywords{free semigroup}
%\pmkeywords{word}
\pmdefines{prefix}
\pmdefines{prefix set}
\pmdefines{proper prefix}
\pmdefines{prefix closed}
\pmdefines{prefix closure}
\pmdefines{prefix free}

% this is the default PlanetMath preamble.  as your knowledge
% of TeX increases, you will probably want to edit this, but
% it should be fine as is for beginners.

% almost certainly you want these
\usepackage{amssymb}
\usepackage{amsmath}
\usepackage{amsfonts}

% used for TeXing text within eps files
%\usepackage{psfrag}
% need this for including graphics (\includegraphics)
%\usepackage{graphicx}
% for neatly defining theorems and propositions
%\usepackage{amsthm}
% making logically defined graphics
%%%\usepackage{xypic}

% there are many more packages, add them here as you need them

% define commands here

\begin{document}
\newcommand{\prefi}{\mathrm{pref}}

Let $X$ be a set, and $w\in X^*$ be a word, i.e. an element of the free  monoid on $X$. A word $v\in X^*$ is called \emph{prefix} of $w$ when a second word $z\in X^*$ exists such that $x=vz$.  A \emph{proper prefix} of a word u is a prefix v of u not equal to u (sometimes v is required to be non-empty).

Note that the empty word $\varepsilon$ and $w$ are prefix of $w$, and a proper prefix of $w$ if $w$ is non-empty.

The \emph{prefix set} of $w$ is the set $\prefi(w)$ of prefixes of $w$, i.e. if $w=w_1w_2...w_n$ with $w_j\in X$ for each $j\in\{1,...,n\}$ we have $$\prefi(w)=\{\varepsilon,\ w_1,\ w_1w_2,\ ... ,\ w_1w_2...w_{n-1},\ w\}.$$

Some closely related concepts are:

\begin{enumerate}
\item 
A set of words is \emph{prefix closed} if for every word in the set, any of its prefix is also in the set.
\item 
The \emph{prefix closure} of a set S is the smallest prefix closed set containing S, or, equivalently, the union of the prefix sets of words in S.
\item
A set S is \emph{prefix free} if for any word in S, no proper prefixes of the word are in S.
\end{enumerate}
%%%%%
%%%%%
\end{document}
