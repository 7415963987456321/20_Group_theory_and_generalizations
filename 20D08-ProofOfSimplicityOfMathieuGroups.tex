\documentclass[12pt]{article}
\usepackage{pmmeta}
\pmcanonicalname{ProofOfSimplicityOfMathieuGroups}
\pmcreated{2013-03-22 18:44:08}
\pmmodified{2013-03-22 18:44:08}
\pmowner{monster}{22721}
\pmmodifier{monster}{22721}
\pmtitle{proof of simplicity of Mathieu groups}
\pmrecord{8}{41505}
\pmprivacy{1}
\pmauthor{monster}{22721}
\pmtype{Proof}
\pmcomment{trigger rebuild}
\pmclassification{msc}{20D08}
\pmclassification{msc}{20B20}

\endmetadata

% this is the default PlanetMath preamble.  as your knowledge
% of TeX increases, you will probably want to edit this, but
% it should be fine as is for beginners.

% almost certainly you want these
\usepackage{amssymb}
\usepackage{amsmath}
\usepackage{amsfonts}

% used for TeXing text within eps files
%\usepackage{psfrag}
% need this for including graphics (\includegraphics)
%\usepackage{graphicx}
% for neatly defining theorems and propositions
\usepackage{amsthm}
% making logically defined graphics
%%%\usepackage{xypic}

% there are many more packages, add them here as you need them

% define commands here
\newtheorem{thm}{Theorem}
\newtheorem{lem}[thm]{Lemma}
\newtheorem{cor}[thm]{Corollary}
\newcommand{\F}{\mathbb{F}}
\begin{document}
\PMlinkescapeword{restricted}
\PMlinkescapeword{classes}
\PMlinkescapeword{nor}
\PMlinkescapeword{conclusion}
\PMlinkescapeword{complete}
\PMlinkescapeword{way}
\PMlinkescapeword{multiplication}
\PMlinkescapeword{point}

We give a uniform proof of the simplicity of the Mathieu groups $M_{22}$, $M_{23}$, and $M_{24}$, and the alternating groups $A_n$ (for $n>5$), assuming the simplicity of $M_{21} \cong PSL(3,\mathbb{F}_4)$ and $A_5 \cong PSL(2,\mathbb{F}_4)$.  (Essentially, we are assuming that the simplicity of the projective special linear groups is known.)

\begin{lem}
Let $G$ act transitively on a set $S$.  If $H$ is a normal subgroup of $G$, then the transitivity classes of the action,
restricted to $H$, form a set of blocks for the action of $G$.
\end{lem}

\begin{proof}
If $T$, $U$ are any transitivity classes for the restricted action, let $t \in T$, $u \in U$, and $g \in G$ such
that $gt = u$.  Then $x \mapsto gx$ is a bijective map from $T$ onto $U$ (here we use normality).  Hence any element of $G$ maps transitivity classes to transitivity classes.
\end{proof}

Hence it follows:

\begin{cor}
Let $G$ act \PMlinkname{primitively}{PrimativeTransitivePermutationGroupOnASet} on a set $S$.  If $H$ is a normal subgroup of $G$, then either $H$ acts transitively on $S$, or $H$ lies in the kernel of the action.  If the action is faithful, then either $H = \{1\}$ or $H$ is transitive.
\end{cor}

\begin{thm}
Let $G$ be a group acting primitively and faithfully on a set $S$.  Let $K$ be the stabilizer of some point $s_0 \in S$, and assume that $K$ is simple.  Then if $H$ is a nontrivial proper normal subgroup of $G$, then $G$ is isomorphic to the semidirect product of $H$ by $K$.  $H$ can be identified with $S$ in such a way that $1 \in H$ is identified with $s_0$, the action of $H$ becomes left multiplication, and the action of $K$ becomes conjugation.
\end{thm}

\begin{proof}
Since $H \cap K$ is a normal subgroup of $K$, it is either $\{1\}$ or $K$.

If $H \cap K = K$, then $K \subset H$, and since $K$ is maximal and $H$ is proper, we have $K=H$.  Since $H$ is normal and $H$ stabilizes $s_0$, then $H$ stabilizes every point (since the action is transitive).  Since the action is faithful, $K = H = \{1\}$, a contradiction.  (This contradiction can also be reached by applying the corollary.)

Therefore, $H \cap K = \{1\}$.  So no element of $H$, other than 1, fixes $s_0$.  Thus $H$ acts freely and transitively on $S$.  For any $g \in G$, if $g s_0 = s$ and $h s = s_0$, then $h g s_0 = s_0$, hence $h g$ is in $K$.  Thus $G$ is generated by $H$ and $K$.  Since $H$ is normal and $H \cap K = \{1\}$, $G$ is the (internal) semidirect product of $H$ by $K$.
\end{proof}

Now we come to the main theorem from which we will deduce the simplicity results.

\begin{thm}
Let $G$ be a group acting faithfully on a set $S$.  Let $s_0 \in S$ and let $K$ be the stabilizer of $s_0$.  Assume $K$ is simple.

1. Assume the action of $G$ is doubly transitive, and let $H$ be a nontrivial proper normal subgroup of $G$.  Then $H$ is an elementary abelian $p$-group for some prime $p$.  Furthermore, $K$ is isomorphic to a subgroup of $GL(n,\F_p)$, and $G$ is isomorphic to a subgroup of $AGL(n,\F_p)$, the group of affine transformations of $H$.

2. If the action of $G$ is triply transitive and $|S|>3$, then any nontrivial proper normal subgroup of $G$ is an elementary abelian 2-group.

3. If the action of $G$ is quadruply transitive and $|S|>4$, then $G$ is simple.
\end{thm}

\begin{proof}
For part 1, use the identification of $H$ with $S$ given by the previous theorem.  Since the action is doubly transitive, the  action by conjugation of $K$ is transitive on $H - \{1\}$.  Therefore, all non-identity elements of $H$ have the same order, which must therefore be some prime $p$.  Hence $H$ is a $p$-group.  The center $Z(H)$ is nontrivial, and is preserved by all automorphisms.  By double transitivity again, there is an automorphism taking any nontrivial element to any other; hence $H$ is abelian.  Therefore $H$ is an elementary abelian $p$-group.

For part 2, we know from part 1 that $H$ is isomorphic to an elementary abelian $p$-group and $K$ acts as linear transformations of $H$.  Since the action of $G$ is triply transitive, the action of $K$ on the nonzero elements is doubly transitive.  However, if $p>2$, then the linearity of the action disallows double transitivity (if $x \mapsto y$, then $2x \mapsto 2y$ so we do not have complete freedom for any two elements since $H$ \PMlinkescapetext{contains} some element besides 0, $y$ and $2y$.)

(We note that when $|S|=3$, we have the example $G=S_3$, $H=A_3$, $K=S_2$.)

Here is an example illustrating part 2.  The group $AGL(n,\F_2)$ acts triply transitively on $\F_2^n$, and the stabilizer of a point is $GL(n,\F_2)$, which is simple if $n>2$.  $AGL(n, \F_2)$ contains the normal subgroup of translations, an elementary abelian 2-group.

For part 3, note that the action of $GL(n,\F_2)$ on $\F_2^n$, $n>2$, is not triply transitive on nonzero elements, so the only conclusion left is that $G$ is simple.
\end{proof} 

\begin{cor}
The Mathieu groups $M_{21}$, $M_{22}$, $M_{23}$, and $M_{24}$ are simple.
\end{cor}

\begin{proof}
We take it as known that $M_{21} \cong PSL(3, \F_4)$ is simple.  Since $M_n$ has $M_{n-1}$ as point stabilizer, and has a triply transitive action on a set of $n$ elements, we may work our way inductively up to $M_{24}$, using the previous theorem.  The \PMlinkname{index}{Coset} of $M_{n-1}$ in $M_n$ is $n$, which is not a power of 2. Hence in all cases, $M_n$ is simple.
\end{proof}

\begin{cor}
The alternating groups $A_n$ are simple for $n \ge 5$.
\end{cor}

\begin{proof}
Since the natural action of $A_n$ on $n$ letters is quadruply transitive for $n \ge 6$, and the point stabilizer of $A_n$ is $A_{n-1}$, we may apply the theorem to deduce the simplicity of the alternating groups $A_n$, $n \ge 5$, from the simplicity of  $A_5 \cong PSL(2,\F_4) \cong PSL(2,\F_5)$.
\end{proof}

%%%%%
%%%%%
\end{document}
