\documentclass[12pt]{article}
\usepackage{pmmeta}
\pmcanonicalname{AbelianGroupsOfOrder120}
\pmcreated{2013-03-22 13:54:17}
\pmmodified{2013-03-22 13:54:17}
\pmowner{alozano}{2414}
\pmmodifier{alozano}{2414}
\pmtitle{abelian groups of order $120$}
\pmrecord{5}{34654}
\pmprivacy{1}
\pmauthor{alozano}{2414}
\pmtype{Example}
\pmcomment{trigger rebuild}
\pmclassification{msc}{20E34}
\pmrelated{FundamentalTheoremOfFinitelyGeneratedAbelianGroups}
\pmrelated{AbelianGroup2}

% this is the default PlanetMath preamble.  as your knowledge
% of TeX increases, you will probably want to edit this, but
% it should be fine as is for beginners.

% almost certainly you want these
\usepackage{amssymb}
\usepackage{amsmath}
\usepackage{amsthm}
\usepackage{amsfonts}

% used for TeXing text within eps files
%\usepackage{psfrag}
% need this for including graphics (\includegraphics)
%\usepackage{graphicx}
% for neatly defining theorems and propositions
%\usepackage{amsthm}
% making logically defined graphics
%%%\usepackage{xypic}

% there are many more packages, add them here as you need them

% define commands here

\newtheorem{thm}{Theorem}
\newtheorem{defn}{Definition}
\newtheorem{prop}{Proposition}
\newtheorem{lemma}{Lemma}
\newtheorem{cor}{Corollary}

% Some sets
\newcommand{\Nats}{\mathbb{N}}
\newcommand{\Ints}{\mathbb{Z}}
\newcommand{\Reals}{\mathbb{R}}
\newcommand{\Complex}{\mathbb{C}}
\newcommand{\Rats}{\mathbb{Q}}
\begin{document}
Here we present an application of the fundamental theorem of
finitely generated abelian groups.

{\bf Example (Abelian groups of order $120$)}:

Let $G$ be an abelian group of order $n=120$. Since the group is
finite it is obviously finitely generated, so we can apply the
theorem. There exist $n_1,n_2,\ldots,n_s$ with
$$G\cong
\Ints/n_1\Ints\oplus\Ints/n_2\Ints\oplus\ldots\oplus\Ints/n_s\Ints$$
$$\forall i, n_i\geq 2;\quad n_{i+1}\mid n_i\ \text{for }
1\leq i\leq s-1$$ Notice that in the case of a finite group, $r$,
as in the statement of the theorem, must be equal to $0$. We have
$$n=120=2^3\cdot3\cdot5=\prod_{i=1}^s n_i=n_1\cdot n_2\cdot \ldots \cdot n_s$$
and by the divisibility properties of $n_i$ we must have that
every prime divisor of $n$ must divide $n_1$. Thus the
possibilities for $n_1$ are the following
$$2\cdot 3\cdot 5,\quad 2^2\cdot 3 \cdot 5,\quad 2^3\cdot 3\cdot 5$$
If $n_1=2^3\cdot 3\cdot 5=120$ then $s=1$. In the case that
$n_1=2^2\cdot 3 \cdot 5$ then $n_2=2$ and $s=2$. It remains to
analyze the case $n_1=2\cdot 3 \cdot 5$. Now the only possibility for
$n_2$ is $2$ and $n_3=2$ as well.

Hence if $G$ is an abelian group of order $120$ it must be ({\bf
up to isomorphism}) one of the following:
$$\Ints/120\Ints,\quad \Ints/60\Ints\oplus \Ints/2\Ints,\quad
\Ints/30\Ints\oplus\Ints/2\Ints\oplus\Ints/2\Ints$$ Also notice
that they are all non-isomorphic. This is because
$$\Ints/(n\cdot m)\Ints \cong \Ints/n\Ints\oplus \Ints/m\Ints
\Leftrightarrow \operatorname{gcd}(n,m)=1$$ which is due to the
Chinese Remainder theorem.
%%%%%
%%%%%
\end{document}
