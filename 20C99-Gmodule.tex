\documentclass[12pt]{article}
\usepackage{pmmeta}
\pmcanonicalname{Gmodule}
\pmcreated{2013-03-22 14:57:53}
\pmmodified{2013-03-22 14:57:53}
\pmowner{rspuzio}{6075}
\pmmodifier{rspuzio}{6075}
\pmtitle{$G$-module}
\pmrecord{6}{36663}
\pmprivacy{1}
\pmauthor{rspuzio}{6075}
\pmtype{Definition}
\pmcomment{trigger rebuild}
\pmclassification{msc}{20C99}
%\pmkeywords{group}
%\pmkeywords{representation}
\pmrelated{GroupRepresentation}
\pmrelated{Group}

\endmetadata

\usepackage{graphicx}
%%%\usepackage{xypic} 
\usepackage{bbm}
\newcommand{\Z}{\mathbbmss{Z}}
\newcommand{\C}{\mathbbmss{C}}
\newcommand{\R}{\mathbbmss{R}}
\newcommand{\Q}{\mathbbmss{Q}}
\newcommand{\mathbb}[1]{\mathbbmss{#1}}
\newcommand{\figura}[1]{\begin{center}\includegraphics{#1}\end{center}}
\newcommand{\figuraex}[2]{\begin{center}\includegraphics[#2]{#1}\end{center}}
\newtheorem{dfn}{Definition}
\begin{document}
Let $V$ a vector space over some field $K$ (usually $K=\Q$ or $K=\C$).  Let $G$ 
be a group which acts on $V$.  This means that there is an operation 
$\psi \colon G\times V \to V$ such that
\begin{enumerate}
\item $gv \in V$.
\item $g(hv) = (gh)v$
\item $ev = v$
\end{enumerate}
where $gv$ stands for $\psi(g,v)$ and $e$ is the identity element of $G$.

If in addition, 
\[ g(cv + dw) = c(gv)+d(gw)\]
for any $g\in G$, $v,w \in V$, $c,d\in K$, we say that $V$ is a $G$-module. 
This is equivalent with the existence of a group representation from $G$ to $GL(V)$.
%%%%%
%%%%%
\end{document}
