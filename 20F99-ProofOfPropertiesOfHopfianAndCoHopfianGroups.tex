\documentclass[12pt]{article}
\usepackage{pmmeta}
\pmcanonicalname{ProofOfPropertiesOfHopfianAndCoHopfianGroups}
\pmcreated{2013-03-22 18:31:17}
\pmmodified{2013-03-22 18:31:17}
\pmowner{joking}{16130}
\pmmodifier{joking}{16130}
\pmtitle{proof of properties of Hopfian and co-Hopfian groups}
\pmrecord{6}{41214}
\pmprivacy{1}
\pmauthor{joking}{16130}
\pmtype{Proof}
\pmcomment{trigger rebuild}
\pmclassification{msc}{20F99}

% this is the default PlanetMath preamble.  as your knowledge
% of TeX increases, you will probably want to edit this, but
% it should be fine as is for beginners.

% almost certainly you want these
\usepackage{amssymb}
\usepackage{amsmath}
\usepackage{amsfonts}

% used for TeXing text within eps files
%\usepackage{psfrag}
% need this for including graphics (\includegraphics)
%\usepackage{graphicx}
% for neatly defining theorems and propositions
%\usepackage{amsthm}
% making logically defined graphics
%%%\usepackage{xypic}

% there are many more packages, add them here as you need them

% define commands here

\begin{document}
\textbf{Proposition}. A group $G$ is Hopfian if and only if every surjective homomorphism $G\to G$ is an automorphism.

\textit{Proof}. ``$\Rightarrow$'' Assume that $\psi:G\to G$ is a surjective homomorpism such that $\psi$ is not an automorphism, which means that $\mathrm{Ker}(\psi)$ is nontrivial. Then (due to the First Isomorphism Theorem) $G/\mathrm{Ker}(\psi)$ is isomorphic to $\mathrm{Im}(\psi)=G$. Contradiction, since $G$ is Hopfian.

``$\Leftarrow$'' Assume that $G$ is not Hopfian. Then there exists nontrivial normal subgroup $H$ of $G$ and an isomorphism $\phi:G/H\to G$. Let $\pi:G\to G/H$ be the quotient homomorphism. Then obviously $\pi\circ\phi:G\to G$ is a surjective homomorphism, but $\mathrm{Ker}(\pi\circ\phi)=H$ is nontrivial, therefore $\pi\circ\phi$ is not an automorphism. Contradiction. $\square$


\textbf{Proposition}. A group $G$ is co-Hopfian if and only if every injective homomorphism $G\to G$ is an automorphism.

\textit{Proof}. ``$\Rightarrow$'' Assume that $\psi:G\to G$ is an injective homomorphism which is not an automorphism. Therefore $\mathrm{Im}(\psi)$ is a proper subgroup of $G$, therefore (since $\mathrm{Ker}(\psi)=\{e\}$ and due to the First Isomorphism Theorem) $G$ is isomorphic to its proper subgroup, namely $\mathrm{Im}(\psi)$. Contradiction, since $G$ is co-Hopfian.

``$\Leftarrow$'' Assume that $G$ is not co-Hopfian. Then there exists a proper subgroup $H$ of $G$ and an isomorphism $\phi:G\to H$. Let $i:H\to G$ be an inclusion homomorphism. Then $i\circ\phi:G\to G$ is an injective homomorphism which is not onto (because $i$ is not). Contradiction. $\square$
%%%%%
%%%%%
\end{document}
