\documentclass[12pt]{article}
\usepackage{pmmeta}
\pmcanonicalname{CyclicallyReduced}
\pmcreated{2013-03-22 17:34:04}
\pmmodified{2013-03-22 17:34:04}
\pmowner{CWoo}{3771}
\pmmodifier{CWoo}{3771}
\pmtitle{cyclically reduced}
\pmrecord{7}{39978}
\pmprivacy{1}
\pmauthor{CWoo}{3771}
\pmtype{Definition}
\pmcomment{trigger rebuild}
\pmclassification{msc}{20F10}
\pmclassification{msc}{20F05}
\pmclassification{msc}{20F06}
\pmdefines{cyclic reduction}
\pmdefines{cyclic conjugation}

\endmetadata

\usepackage{amssymb,amscd}
\usepackage{amsmath}
\usepackage{amsfonts}
\usepackage{mathrsfs}

% used for TeXing text within eps files
%\usepackage{psfrag}
% need this for including graphics (\includegraphics)
%\usepackage{graphicx}
% for neatly defining theorems and propositions
\usepackage{amsthm}
% making logically defined graphics
%%\usepackage{xypic}
\usepackage{pst-plot}
\usepackage{psfrag}

% define commands here
\newtheorem{prop}{Proposition}
\newtheorem{thm}{Theorem}
\newtheorem{ex}{Example}
\newcommand{\real}{\mathbb{R}}
\newcommand{\pdiff}[2]{\frac{\partial #1}{\partial #2}}
\newcommand{\mpdiff}[3]{\frac{\partial^#1 #2}{\partial #3^#1}}
\begin{document}
Let $M(X)$ be a free monoid with involution $^{-1}$ on $X$.  A word $w\in M(X)$ is said to be \emph{cyclically reduced} if every cyclic conjugate of it is reduced.  In other words, $w$ is cyclically reduced iff $w$ is a reduced word and that if $w=uvu^{-1}$ for some words $u$ and $v$, then $w=v$.

For example, if $X=\lbrace a,b,c\rbrace$, then words such as $$c^{-1}bc^2a\qquad\mbox{and}\qquad abac^2ba^2$$ are cyclically reduced, where as words $$a^2bca^{-1}\qquad\mbox{and}\qquad cb^2b^3c$$ are not, the former is reduced, but of the form $a(abc)a^{-1}$, while the later is not even a reduced word.

\textbf{Remarks}.  The concept of cyclically reduced words carries over to words in groups.  We consider words in a group $G$.
\begin{itemize}
\item If a word is cyclically reduced, so is its inverse and all of its cyclic conjugates.
\item A word $v$ is a \emph{cyclic reduction} of a word $w$ if $w=uvu^{-1}$ for some word $u$, and $v$ is cyclically reduced.  Clearly, every word and its cyclic reduction are conjugates of each other.  Furthermore, any word has a unique cyclic reduction. 
\item Every group $G$ has a presentation $\langle S| R\rangle$ such that 
\begin{enumerate}
\item $R$ is cyclically reduced (meaning every element of $R$ is cyclically reduced),
\item closed under inverses (meaning if $u\in R$, then $u^{-1}\in R$), and 
\item closed under cyclic conjugation (meaning any cyclic conjugate of an element in $R$ is in $R$).
\end{enumerate}
Furthermore, if $G$ is finitely presented, $R$ above can be chosen to be finite.
\begin{proof}  Every group $G$ has a presentation $\langle S| R'\rangle$.  There is an isomorphism from $F(S)/N(R')$ to $G$, where $F(S)$ is the free group freely generated by $S$, and $N(R')$ is the normalizer of the subset $R'\subseteq F(S)$ in $F(S)$.  Let $R''$ be the set of all cyclic reductions of words in $R'$.  Then $N(R'')=N(R')$, since any word not cyclically reduced in $R'$ is conjugate to its cyclic reduction in $R''$, and hence in $N(R'')$.  Next, for each $u\in R''$, toss in its inverse and all of its cyclic conjugates.  The resulting set $R$ is still cyclically reduced.  Furthermore, $R$ satisfies the remaining conditions above.  Finally, $N(R)=N(R'')$, as any cyclic conjugate $v$ of a word $w$ is clearly a conjugate of $w$.  Therefore, $G$ has presentation $\langle S| R\rangle$.

If $G$ is finitely presented, then $S$ and $R'$ above can be chosen to be finite sets.  Therefore, $R''$ and $R$ are both finite.  $R$ is finite because the number of cyclic conjugates of a word is at most the length of the word, and hence finite.
\end{proof}
\end{itemize}
%%%%%
%%%%%
\end{document}
