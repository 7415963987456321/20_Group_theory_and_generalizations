\documentclass[12pt]{article}
\usepackage{pmmeta}
\pmcanonicalname{ErdHosGinzburgZivTheorem}
\pmcreated{2013-03-22 13:40:00}
\pmmodified{2013-03-22 13:40:00}
\pmowner{bbukh}{348}
\pmmodifier{bbukh}{348}
\pmtitle{Erd\H{o}s-Ginzburg-Ziv theorem}
\pmrecord{7}{34327}
\pmprivacy{1}
\pmauthor{bbukh}{348}
\pmtype{Theorem}
\pmcomment{trigger rebuild}
\pmclassification{msc}{20D60}
\pmclassification{msc}{11B50}
\pmsynonym{EGZ theorem}{ErdHosGinzburgZivTheorem}
%\pmkeywords{zero-sum}

\endmetadata

\usepackage{amssymb}
\usepackage{amsmath}
\usepackage{amsfonts}

\makeatletter
\@ifundefined{bibname}{}{\renewcommand{\bibname}{References}}
\makeatother
\begin{document}
If $a_1, a_2,\dotsc, a_{2n-1}$ is a set of integers, then there exists a subset $a_{i_1}, a_{i_2},\dotsc,a_{i_n}$ of $n$ integers such that
\begin{equation*}
a_{i_1}+ a_{i_2}+\dotsb+a_{i_n}\equiv 0 \pmod n.
\end{equation*}
The theorem is also known as the EGZ theorem.

\begin{thebibliography}{1}

\bibitem{cite:nathanson_classicalbases}
Melvyn~B. Nathanson.
\newblock {\em Additive Number Theory: Inverse Problems and Geometry of
Sumsets}, volume 165 of {\em GTM}.
\newblock Springer, 1996.
\newblock \PMlinkexternal{Zbl 0859.11003}{http://www.emis.de/cgi-bin/zmen/ZMATH/en/quick.html?type=html&an=0859.11003}.
\bibitem{cite:haopan}
\newblock Hao,P. {\em On a Congruence modulo a Prime}
\newblock Amer. Math. Monthly, vol. 113, (2006), 652-654

\end{thebibliography}

%@BOOK{cite:nathanson_inverseprob,
% author = {Melvyn B. Nathanson},
% title = {Additive Number Theory: Inverse Problems and Geometry of Sumsets},
% series = {GTM},
% volume = 165,
% year = 1996,
% publisher = {Springer}
%}
%
%%%%%
%%%%%
\end{document}
