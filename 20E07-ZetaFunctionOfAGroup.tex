\documentclass[12pt]{article}
\usepackage{pmmeta}
\pmcanonicalname{ZetaFunctionOfAGroup}
\pmcreated{2013-03-22 15:16:00}
\pmmodified{2013-03-22 15:16:00}
\pmowner{avf}{9497}
\pmmodifier{avf}{9497}
\pmtitle{Zeta function of a group}
\pmrecord{7}{37053}
\pmprivacy{1}
\pmauthor{avf}{9497}
\pmtype{Definition}
\pmcomment{trigger rebuild}
\pmclassification{msc}{20E07}
\pmclassification{msc}{20F69}
\pmclassification{msc}{20F18}
\pmrelated{Group}

\endmetadata

\usepackage{amssymb}
\usepackage{amsmath}
\usepackage{amsfonts}
\usepackage{amsthm}
\usepackage[matrix,arrow,curve]{xy}
\usepackage[only,trianglelefteqslant]{stmaryrd}

\newcommand{\sgp}{\leqslant}
\newcommand{\sgpf}{\leqslant_{\mathrm{f}}}
\newcommand{\nsgp}{\trianglelefteqslant}
\newcommand{\nsgpf}{\trianglelefteqslant_{\mathrm{f}}}
\newcommand{\nsgpo}{\trianglelefteqslant_{\mathrm{o}}}
\newcommand{\hir}{\mathrm{h}}
\newcommand{\stab}{\mathrm{Stab}}
\begin{document}
Let $G$ be a finitely generated group and let $\mathcal{X}$ be a
family of finite index subgroups of $G$. Define
\[
        a_n(\mathcal{X}) = |\{H \in \mathcal{X} \mid |G : H| = n\}|.
\]
Note that these numbers are finite since a finitely generated group
has only finitely many subgroups of a given index. We define the
\textbf{zeta function} of the family $\mathcal{X}$ to be the formal
Dirichlet series
\[
        \zeta_{\mathcal{X}}(s) = \sum_{n=1}^\infty a_n(\mathcal{X})
        n^{-s}.
\]

Two important special cases are the zeta function counting all
subgroups and the zeta function counting normal subgroups. Let
$\mathcal{S}(G)$ and $\mathcal{N}(G)$ be the families of all finite
index subgroups of $G$ and of all finite index normal subgroups of
$G$, respectively. We write $a_n^\sgp(G) = a_n(\mathcal{S}(G))$ and
$a_n^\nsgp(G) = a_n(\mathcal{N}(G))$ and define
\[
        \zeta_G^\sgp(s) = \zeta_{\mathcal{S}(G)}(s) = \sum_{H \sgpf G}
        |G : H|^{-s},
\]
and
\[
        \zeta_G^\nsgp(s) = \zeta_{\mathcal{N}(G)}(s) = \sum_{N \nsgpf G}
        |G : N|^{-s}.
\]

If, in addition, $G$ is nilpotent, then $\zeta_G^\sgp$ has a
decomposition as a formal Euler product
\[
        \zeta_G^\sgp(s) = \prod_{p \text{ prime}}
        \zeta_{G,p}^\sgp(s),
\]
where
\[
        \zeta_{G,p}^\sgp(s) = \sum_{i=0}^\infty a_{p^i}^\sgp(G)
        p^{-is}.
\]
An analogous result holds for the normal zeta function
$\zeta_G^\nsgp$. The result for both $\zeta_G^\sgp$ and
$\zeta_G^\nsgp$ can be proved using properties of the profinite
completion of $G$. However, a simpler proof for the normal zeta
function is provided by the fact that a finite nilpotent group
decomposes into a direct product of its Sylow subgroups. These results
allow the zeta functions to be expressed in terms of $p$-adic
integrals, which can in turn be used to prove (using some high-powered
machinery) that $\zeta_{G,p}^\sgp(s)$ and $\zeta_{G,p}^\nsgp(s)$ are
rational functions in $p$ and $p^{-s}$.

In the case when $G$ is a $\mathcal{T}$-group, that is, $G$ is
finitely generated, torsion free, and nilpotent, define $\alpha_G^\sgp$
to be the abscissa of convergence of $\zeta_G^\sgp$. That is,
$\alpha_G^\sgp$ is the smallest $\alpha \in \mathbb{R}$ such that
$\zeta_G^\sgp$ defines a holomorphic function in the right half-plane
$\{z \in \mathbb{C} \mid \Re(z) > \alpha\}$. It can then be shown that
$\alpha_G^\sgp \leqslant \hir(G)$, where $\hir(G)$ is the Hirsch number
of $G$. Therefore, if $G$ is a $\mathcal{T}$-group, $\zeta_G^\sgp$
defines a holomorphic function in some right half-plane.

\begin{thebibliography}{99}

\bibitem{gss}
F.~J. Grunewald, D.~Segal, and G.~C. Smith, \emph{Subgroups of finite index in
  nilpotent groups}, Invent. math. \textbf{93} (1988), 185--223.

\bibitem{ennui}
M.~P.~F. du~Sautoy, \emph{Zeta functions of groups: the quest for order versus the flight
  from ennui}, Groups St.\ Andrews 2001 in Oxford. Vol. I, London Math. Soc.
  Lecture Note Ser., vol. 304, Cambridge Univ. Press, 2003, pp.~150--189.

\end{thebibliography}
%%%%%
%%%%%
\end{document}
