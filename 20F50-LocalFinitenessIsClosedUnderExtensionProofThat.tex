\documentclass[12pt]{article}
\usepackage{pmmeta}
\pmcanonicalname{LocalFinitenessIsClosedUnderExtensionProofThat}
\pmcreated{2013-03-22 15:36:53}
\pmmodified{2013-03-22 15:36:53}
\pmowner{yark}{2760}
\pmmodifier{yark}{2760}
\pmtitle{local finiteness is closed under extension, proof that}
\pmrecord{6}{37538}
\pmprivacy{1}
\pmauthor{yark}{2760}
\pmtype{Proof}
\pmcomment{trigger rebuild}
\pmclassification{msc}{20F50}
%\pmkeywords{locally finite}
\pmrelated{LocallyFiniteGroup}

\endmetadata

\usepackage{amssymb}
\usepackage{amsmath}
\usepackage{amsfonts}

\def\genby#1{{\left\langle #1\right\rangle}}
\def\subgroup{\leq}
\begin{document}
\PMlinkescapeword{completes}
\PMlinkescapeword{finite}
\PMlinkescapeword{mapping}

Let $G$ be a group and $N$ a normal subgroup of $G$
such that $N$ and $G/N$ are both locally finite.
We aim to show that $G$ is locally finite.
Let $F$ be a finite subset of $G$.
It suffices to show that $F$ is contained in a finite subgroup of $G$.

Let $R$ be a set of coset representatives of $N$ in $G$,
chosen so that $1\in R$.
Let $r\colon G/N\to R$ be the function mapping cosets to their representatives,
and let $s\colon G\to N$ be defined by $s(x)=r(xN)^{-1}x$ for all $x\in G$.
Let $\pi\colon G\to G/N$ be the canonical projection.
Note that for any $x\in G$ we have $x=r(xN)s(x)$.

Put $A=r(\genby{\pi(F)})$, which is finite as $G/N$ is locally finite.
Let $B=s(F\cup AA\cup A^{-1})$, let $C=B\cup B^{-1}$
and let $$D=\{a^{-1}ca\mid a\in A\hbox{ and }c\in C\}\subseteq N.$$
Put $H=\genby{D}$, which is finite as $N$ is locally finite.
Note that $1\in A\subseteq R$
and $1\in B\subseteq C\subseteq D\subseteq H\subgroup N$.

For any $a_1,a_2\in A$ we have $a_1a_2=r(a_1a_2N)s(a_1a_2)\in AB$.
Note that $D^{-1}=D$,
and so every element of $H$ is a product of elements of $D$.
So any element of the form $a^{-1}ha$, where $a\in A$ and $h\in H$,
is a product of elements of the form $a^{-1}a_1^{-1}ca_1a$
for $a_1\in A$ and $c\in C$;
but $a_1a=a_2b$ for some $a_2\in A$ and $b\in B$,
so $a^{-1}ha$ is a product of elements of the form 
$b^{-1}a_2^{-1}ca_2b=b^{-1}(a_2^{-1}ca_2)b\in CDB\subseteq H$,
and therefore $a^{-1}ha\in H$.

We claim that $AH\subgroup G$.
Let $a_1,a_2\in A$ and $h_1,h_2\in H$.
We have $(a_1h_1)(a_2h_2)=a_1a_2(a_2^{-1}h_1a_2)h_2$.
But, by the previous paragraph, $a_1a_2\in AB$ and $a_2^{-1}h_1a_2\in H$,
so $a_1a_2(a_2^{-1}h_1a_2)h_2\in ABHH\subseteq AH$.
Thus $AHAH\subseteq AH$.
Also, $(a_1h_1)^{-1}=h_1^{-1}a_1^{-1}\in Ha_1^{-1}$.
But $a_1^{-1}=r(a_1^{-1}N)s(a_1^{-1})\in AB$,
so $Ha_1^{-1}\subseteq HAB\subseteq AHAH\subseteq AH$.
Thus $(AH)^{-1}\subseteq AH$.
It follows that $AH$ is a subgroup of $G$, and it is clearly finite.

For any $x\in F$ we have $x=r(xN)s(x)\in AB$.
So $F\subseteq AH$, which completes the proof.
%%%%%
%%%%%
\end{document}
