\documentclass[12pt]{article}
\usepackage{pmmeta}
\pmcanonicalname{GroupsOfSmallOrder}
\pmcreated{2013-03-22 14:47:54}
\pmmodified{2013-03-22 14:47:54}
\pmowner{Daume}{40}
\pmmodifier{Daume}{40}
\pmtitle{groups of small order}
\pmrecord{15}{36451}
\pmprivacy{1}
\pmauthor{Daume}{40}
\pmtype{Example}
\pmcomment{trigger rebuild}
\pmclassification{msc}{20A05}
\pmclassification{msc}{20-00}
\pmrelated{ExamplesOfGroups}

\endmetadata

% As your knowledge of TeX increases,
% you will probably want to edit this, but
% it should be fine as is for beginners.

% almost certainly you want these
\usepackage{amssymb}
\usepackage{amsmath}
\usepackage{amsfonts}

% used for TeXing text within eps files
%\usepackage{psfrag}
% need this for including graphics (\includegraphics)
%\usepackage{graphicx}
% for neatly defining theorems and propositions
%\usepackage{amsthm}
% making logically defined graphics
%%%\usepackage{xypic} 

% there are many more packages, add them here as you need them

% define commands here
\DeclareMathOperator{\Dic}{Dic}

% The below lines should work as the command
% \renewcommand{\bibname}{References}
% without creating havoc when rendering an entry in
% the page-image mode.
\makeatletter
\@ifundefined{bibname}{}{\renewcommand{\bibname}{References}}
\makeatother
\begin{document}
\PMlinkescapeword{degree}

Below is a list of all possible groups per order up to isomorphism.

\textbf{Groups of prime order:}
\begin{itemize}
\item All groups of prime order are isomorphic to a cyclic group of that order.
\end{itemize}

\textbf{Groups of prime square order:}
\begin{itemize}
\item All groups of order $p^2$, where $p$ is a prime, are isomorphic to one of the following:
\begin{itemize}
\item $C_{p^2}$\textit{(Abelian)}: cyclic group of order $p^2$.
\item $C_p\times C_p$\textit{(Abelian)}: elementary abelian group of order $p^2$.
\end{itemize}
\end{itemize}

\textbf{Groups of order 1:}
\begin{itemize}
\item trivial group \textit{(i.e. $\{ e\}$)}.
\end{itemize}

\textbf{Groups of order 6:}
\begin{itemize}
\item $C_6$\textit{(Abelian)}: cyclic group of order 6.
\item $S_3$\textit{(non-Abelian)}: symmetric group where $n=3$.
\end{itemize}

\textbf{Groups of order 8:}
\begin{itemize}
\item $C_8$\textit{(Abelian)}: cyclic group of order 8.
\item $C_4\times C_2$\textit{(Abelian)}: direct product of two groups of a cyclic group of order 4 and a cyclic group of order 2.
\item $C_2\times C_2\times C_2$\textit{(Abelian)}: direct product of three groups of a cyclic group of order 2.
\item $D_4$\textit{(non-Abelian)}: octic group; dihedral group of degree 4.
\item $Q_8$\textit{(non-Abelian)}: quaternion group.
\end{itemize}

\textbf{Groups of order 10:}
\begin{itemize}
\item $C_{10}$\textit{(Abelian)}: cyclic group of order 10.
\item $D_5$\textit{(non-Abelian)}: dihedral group of degree 5.
\end{itemize}

\textbf{Groups of order 12:}
\begin{itemize}
\item $C_{12}$\textit{(Abelian)}: cyclic group of order 12.
\item $C_2\times C_6$\textit{(Abelian)}.
\item $A_4$\textit{(non-Abelian)}: alternating group of degree 4.
\item $D_6$\textit{(non-Abelian)}: dihedral group of degree 6.
\item $\Dic(C_6)$\textit{(non-Abelian)}: dicyclic group of order 12.
This is a generalized quaternion group $Q_{12}$.
\end{itemize}

\textbf{Groups of order 14:}
\begin{itemize}
\item $C_{14}$\textit{(Abelian)}: cyclic group of order 14.
\item $D_7$\textit{(non-Abelian)}: dihedral group of degree 7.
\end{itemize}

\textbf{Groups of order 15:}
\begin{itemize}
\item $C_{15}$\textit{(Abelian)}: cyclic group of order 15.
\end{itemize}

\begin{thebibliography}{1}
\bibitem[PJ]{PJ} Pedersen, John: Groups of small order. \PMlinkexternal{http://www.math.usf.edu/~eclark/algctlg/small_groups.html}{http://www.math.usf.edu/~eclark/algctlg/small_groups.html}
\end{thebibliography}
%%%%%
%%%%%
\end{document}
