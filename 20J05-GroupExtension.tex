\documentclass[12pt]{article}
\usepackage{pmmeta}
\pmcanonicalname{GroupExtension}
\pmcreated{2013-03-22 15:24:25}
\pmmodified{2013-03-22 15:24:25}
\pmowner{CWoo}{3771}
\pmmodifier{CWoo}{3771}
\pmtitle{group extension}
\pmrecord{11}{37246}
\pmprivacy{1}
\pmauthor{CWoo}{3771}
\pmtype{Definition}
\pmcomment{trigger rebuild}
\pmclassification{msc}{20J05}
\pmrelated{HNNExtension}
\pmdefines{split extension}
\pmdefines{abelian extension}
\pmdefines{central extension}
\pmdefines{cyclic extension}
\pmdefines{extension problem}

\endmetadata

\usepackage{amssymb,amscd}
\usepackage{amsmath}
\usepackage{amsfonts}

% used for TeXing text within eps files
%\usepackage{psfrag}
% need this for including graphics (\includegraphics)
%\usepackage{graphicx}
% for neatly defining theorems and propositions
%\usepackage{amsthm}
% making logically defined graphics
%%\usepackage{xypic}

% define commands here
\begin{document}
\PMlinkescapeword{extension}
\PMlinkescapeword{extensions}

Let $G$ and $H$ be groups.  A group $E$ is called an \emph{extension
of $G$ by $H$} if
\begin{enumerate}
\item $G$ is isomorphic to a normal subgroup $N$ of $E$, and
\item $H$ is isomorphic to the quotient group $E/N$.
\end{enumerate}
The definition is well-defined and it is convenient sometimes to
regard $G$ as a normal subgroup of $E$.  The definition can be
alternatively defined: $E$ is an extension of $G$ by $H$ if there is
a short exact sequence of groups:
$$1\longrightarrow G\longrightarrow E\longrightarrow H\longrightarrow 1.$$
In fact, some authors define an extension (of a group by a group) to
be a short exact sequence of groups described above.  Also, many authors
prefer the reverse terminology, calling the group $E$ an extension of $H$
by $G$.

\textbf{Remarks}
\begin{itemize}
\item Given any groups $G$ and $H$, an extension of $G$ by $H$
exists: take the direct product of $G$ and $H$.
\item An intermediate concept between an extension a direct product is that of a \emph{semidirect product} of two groups: If $G$ and $H$ are groups, and $E$
is an extension of $G$ by $H$ (identifying $G$ with a normal subgroup
of $E$), then $E$ is called a semidirect product of $G$ by $H$ if
\begin{enumerate}
\item $H$ is isomorphic to a subgroup of $E$, thus viewing $H$ as a
subgroup of $E$,
\item $E=GH$, and
\item $G\cap H=\langle1\rangle$.
\end{enumerate}
Equivalently, $E$ is a semidirect product of $G$ and $H$ if the
short exact sequence
$$1\longrightarrow G\longrightarrow E\stackrel{\alpha}{\longrightarrow} H\longrightarrow 1$$
splits.  That is, there is a group homomorphism $\phi\colon H\to E$ such
that the composition
$$H\stackrel{\phi}{\longrightarrow}E\stackrel{\alpha}{\longrightarrow}H$$
gives the identity map.
Thus, a semidirect product is also known as a \emph{split
extension}.  That a semidirect product $E$ of $G$ by $H$ is also an extension of $G$ by $H$ can be seen via the isomorphism $h\mapsto hG$.  

Furthermore, if $H$ happens to be normal in $E$, then $E$ is isomorphic to the direct product of $G$ and $H$.  (We need to show that $(g,h)\mapsto gh$ is an isomorphism.  It is not hard to see that the map is a bijection.  The trick is to show that it is a homomorphism, which boils down to showing that every element of $G$ commutes with every element of $H$.  To show the last step, suppose $ghg^{-1}=\overline{h}\in H$.  Then $gh=\overline{h}g$, so $gh\overline{h}^{-1}=\overline{h}g\overline{h}^{-1}= \overline{g}\in G$, or that $h\overline{h}^{-1}=g^{-1}\overline{g}$.  Therefore, $h=\overline{h}$.)
\item The \emph{extension problem} in group theory is the
classification of all extension groups of a given group $G$ by a
given group $H$.  Specifically, it is a problem of finding all
``inequivalent'' extensions of $G$ by $H$.  Two extensions $E_1$ and
$E_2$ of $G$ by $H$ are \emph{equivalent} if there is a homomorphism
$e\colon E_1\to E_2$ such that the following diagram of two short exact
sequences is commutative:
$$\xymatrix{1\ar@{=}[d]\ar[r]&G\ar@{=}[d]\ar[r]&E_1\ar[d]^e\ar[r]&H\ar@{=}[d]
\ar[r]&1\ar@{=}[d]\\1\ar[r]&G\ar[r]&E_2\ar[r]&H\ar[r]&1.}$$
According to the 5-lemma, $e$ is actually an isomorphism.  Thus equivalences
of extensions are well-defined.
\item Like split extensions, special extensions are formed when
certain conditions are imposed on $G$, $H$, or even $E$:
\begin{enumerate}
\item If all the groups involved are abelian (only that $E$ is abelian is
necessary here), then we have an \emph{abelian extension}.
\item If $G$, considered as a normal subgroup of $E$, actually lies within
the center of $E$, then $E$ is called a \emph{central extension}.  A central extension that is also a semidirect product is a direct product.  Indeed, if $E$ is both a central extension and a semidirect product of $G$ by $H$, we observe that $(g\overline{h})h(g\overline{h})^{-1}=\overline{h}h\overline{h}^{-1}\in H$ so that $H$ is normal in $E$.  Applying this result to the previous discussion and we have $E\cong G\times H$.
\item If $G$ is a cyclic group, then the extensions in question are
called \emph{cyclic extensions}.
\end{enumerate}
\end{itemize}
%%%%%
%%%%%
\end{document}
