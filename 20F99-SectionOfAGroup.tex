\documentclass[12pt]{article}
\usepackage{pmmeta}
\pmcanonicalname{SectionOfAGroup}
\pmcreated{2013-03-22 17:15:04}
\pmmodified{2013-03-22 17:15:04}
\pmowner{yark}{2760}
\pmmodifier{yark}{2760}
\pmtitle{section of a group}
\pmrecord{11}{39584}
\pmprivacy{1}
\pmauthor{yark}{2760}
\pmtype{Definition}
\pmcomment{trigger rebuild}
\pmclassification{msc}{20F99}
\pmsynonym{section}{SectionOfAGroup}
\pmsynonym{quotient of a subgroup}{SectionOfAGroup}
\pmdefines{involved in}

\endmetadata


\begin{document}
\PMlinkescapeword{means}
\PMlinkescapeword{relation}
\PMlinkescapeword{structure}

A \emph{section} of a group $G$ is
a \PMlinkname{quotient}{QuotientGroup} of a subgroup of $G$.
That is, a section of $G$ is a group of the form $H/N$,
where $H$ is a subgroup of $G$, and $N$ is a normal subgroup of $H$.

A group $G$ is said to be \emph{involved in} a group $K$
if $G$ is isomorphic to a section of $K$.

The relation `is involved in' is \PMlinkname{transitive}{Transitive3},
that is, if $G$ is involved in $K$ and $K$ is involved in $L$,
then $G$ is involved in $L$.

Intuitively, `$G$ is involved in $K$'
means that all of the structure of $G$ can be found inside $K$.
%%%%%
%%%%%
\end{document}
