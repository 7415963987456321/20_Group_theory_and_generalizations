\documentclass[12pt]{article}
\usepackage{pmmeta}
\pmcanonicalname{ArchimedeanSemigroup}
\pmcreated{2013-03-22 13:08:06}
\pmmodified{2013-03-22 13:08:06}
\pmowner{mclase}{549}
\pmmodifier{mclase}{549}
\pmtitle{Archimedean semigroup}
\pmrecord{4}{33572}
\pmprivacy{1}
\pmauthor{mclase}{549}
\pmtype{Definition}
\pmcomment{trigger rebuild}
\pmclassification{msc}{20M14}
\pmrelated{ArchimedeanProperty}
\pmdefines{divides}
\pmdefines{Archimedean}

\endmetadata

% this is the default PlanetMath preamble.  as your knowledge
% of TeX increases, you will probably want to edit this, but
% it should be fine as is for beginners.

% almost certainly you want these
\usepackage{amssymb}
\usepackage{amsmath}
\usepackage{amsfonts}

% used for TeXing text within eps files
%\usepackage{psfrag}
% need this for including graphics (\includegraphics)
%\usepackage{graphicx}
% for neatly defining theorems and propositions
%\usepackage{amsthm}
% making logically defined graphics
%%%\usepackage{xypic}

% there are many more packages, add them here as you need them

% define commands here
\begin{document}
Let $S$ be a commutative semigroup.  We say an element $x$ \emph{divides} an element $y$, written $x \mid y$, if there is an element $z$ such that $xz = y$.

An \emph{Archimedean semigroup} $S$ is a commutative semigroup with the property that for all $x, y \in S$ there is a natural number $n$ such that $x \mid y^n$.

This is related to the Archimedean property of positive real numbers $\mathbb{R}^+$: if $x, y > 0$ then there is a natural number $n$ such that $x < ny$.  Except that the notation is additive rather than multiplicative, this is the same as saying that $(\mathbb{R}^+, +)$ is an Archimedean semigroup.
%%%%%
%%%%%
\end{document}
