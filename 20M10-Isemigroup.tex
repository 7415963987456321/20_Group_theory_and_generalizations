\documentclass[12pt]{article}
\usepackage{pmmeta}
\pmcanonicalname{Isemigroup}
\pmcreated{2013-03-22 16:11:27}
\pmmodified{2013-03-22 16:11:27}
\pmowner{Mazzu}{14365}
\pmmodifier{Mazzu}{14365}
\pmtitle{I-semigroup}
\pmrecord{5}{38282}
\pmprivacy{1}
\pmauthor{Mazzu}{14365}
\pmtype{Definition}
\pmcomment{trigger rebuild}
\pmclassification{msc}{20M10}
%\pmkeywords{semigroup}
\pmrelated{SemigroupWithInvolution}
\pmdefines{I-semigroup}
\pmdefines{I-monoid}

\endmetadata

% this is the default PlanetMath preamble.  as your knowledge
% of TeX increases, you will probably want to edit this, but
% it should be fine as is for beginners.

% almost certainly you want these
\usepackage{amssymb}
\usepackage{amsmath}
\usepackage{amsfonts}

% used for TeXing text within eps files
%\usepackage{psfrag}
% need this for including graphics (\includegraphics)
%\usepackage{graphicx}
% for neatly defining theorems and propositions
%\usepackage{amsthm}
% making logically defined graphics
%%%\usepackage{xypic}

% there are many more packages, add them here as you need them

% define commands here

\begin{document}
An \emph{$I$-semigroup} [resp. \emph{$I$-monoid}] is a semigroup $S$ [resp. a monoid $M$] with a unary operation $x\mapsto x^{-1}$ defined on $S$ [resp. on $M$] such that for each $x,y\in S$ [resp. for each $x,y\in M$] $$(x^{-1})^{-1}=x,\ \ \ x=xx^{-1}x.$$

Notice that $$x^{-1}xx^{-1}=x^{-1}(x^{-1})^{-1}x^{-1}=x^{-1},$$ so $x^{-1}$ is an inverse of $x$.

The class of $I$-semigroups [resp. $I$-monoids] strictly contains the class of inverse semigroups [resp. inverse monoids]. In fact, the class of inverse semigroups [resp. inverse monoids] is precisely the class of $I$-semigroups with involution [resp. $I$-monoids with involution], i.e. the class of $I$-semigroups  [resp. $I$-monoids] in which the unary operation $^{-1}$ is also an involution.

\begin{thebibliography}{9}
\bibitem{b:howie} J.M. Howie, \emph{Fundamentals of Semigroup Theory}, Oxford University Press, Oxford, 1991.
\end{thebibliography}
%%%%%
%%%%%
\end{document}
