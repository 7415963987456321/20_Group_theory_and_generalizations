\documentclass[12pt]{article}
\usepackage{pmmeta}
\pmcanonicalname{SemidirectFactorAndQuotientGroup}
\pmcreated{2013-03-22 15:10:22}
\pmmodified{2013-03-22 15:10:22}
\pmowner{yark}{2760}
\pmmodifier{yark}{2760}
\pmtitle{semi-direct factor and quotient group}
\pmrecord{8}{36924}
\pmprivacy{1}
\pmauthor{yark}{2760}
\pmtype{Theorem}
\pmcomment{trigger rebuild}
\pmclassification{msc}{20E22}

\endmetadata

\usepackage{amssymb}
\usepackage{amsmath}
\usepackage{amsfonts}
\usepackage{amsthm}

\theoremstyle{definition}
\newtheorem*{thmplain}{Theorem}
\begin{document}
\begin{thmplain}
If the group $G$ is a semi-direct product of its subgroups $H$ and $Q$,
then the semi-direct \PMlinkescapetext{factor} $Q$
is isomorphic to the quotient group $G/H$.
\end{thmplain}

{\em Proof.} Every element $g$ of $G$ has the unique representation $g = hq$
with $h\in H$ and $q\in Q$.
We therefore can define the mapping
\[
  g\mapsto q
\]
from $G$ to $Q$.
The mapping is surjective since any element $y$ of $Q$ is the image of $ey$.
The mapping is also a homomorphism since if $g_1 = h_1q_1$ and $g_2 = h_2q_2$, then we obtain
\[
  f(g_1g_2) = f(h_1q_1h_2q_2) = f(h_1h_2q_1q_2) = q_1q_2 = f(g_1)f(g_2).
\]
Then we see that $\ker{f} = H$ because all elements $h = he$ of $H$
are mapped to the identity element $e$ of $Q$.
Consequently we get, according to the  first isomorphism theorem, the result
\[
  G/H \cong Q.
\]

\textbf{Example.}
The multiplicative group $\mathbb{R}^{\times}$ of reals
is the semi-direct product of the subgroups
$\{1,\,-1\} = \{\pm1\}$ and $\mathbb{R}_+$.
The quotient group $\mathbb{R}^{\times}/\{\pm1\}$ consists of all cosets
\[
  x\{\pm1\} = \{x,\,-x\}
\]
where $x\neq 0$, and is obviously isomorphic with
$\mathbb{R}_+ = \{x\mid x > 0\}$.
%%%%%
%%%%%
\end{document}
