\documentclass[12pt]{article}
\usepackage{pmmeta}
\pmcanonicalname{TarskiGroup}
\pmcreated{2013-03-22 15:46:00}
\pmmodified{2013-03-22 15:46:00}
\pmowner{yark}{2760}
\pmmodifier{yark}{2760}
\pmtitle{Tarski group}
\pmrecord{10}{37722}
\pmprivacy{1}
\pmauthor{yark}{2760}
\pmtype{Definition}
\pmcomment{trigger rebuild}
\pmclassification{msc}{20F50}
\pmdefines{Tarski monster}

\endmetadata

\usepackage{amssymb}
\usepackage{amsmath}
\usepackage{amsfonts}

% The below lines should work as the command
% \renewcommand{\bibname}{References}
% without creating havoc when rendering an entry in
% the page-image mode.
\makeatletter
\@ifundefined{bibname}{}{\renewcommand{\bibname}{References}}
\makeatother
\begin{document}
\PMlinkescapeword{lattice}
\PMlinkescapeword{properties}
\PMlinkescapeword{satisfies}
\PMlinkescapeword{subgroup}

A \emph{Tarski group} is an infinite group $G$
such that every non-trivial proper subgroup of $G$ is of prime order.

Tarski groups are also called \emph{Tarski monsters},
especially in the case when 
all the proper non-trivial subgroups are of the same order 
(that is, when the Tarski group is 
a \PMlinkname{$p$-group}{PGroup4} for some prime $p$).

Alexander Ol'shanskii\cite{ol1,ol2} showed that Tarski groups exist,
and that there is a Tarski $p$-group for every prime $p > 10^{75}$.

From the definition one can easily deduce 
a number of properties of Tarski groups.
For example,
every Tarski group is a simple group, 
it satisfies the minimal condition and the maximal condition,
it can be generated by just two elements,
it is periodic but not locally finite,
and its \PMlinkname{subgroup lattice}{LatticeOfSubgroups} is \PMlinkname{modular}{ModularLattice}.

\begin{thebibliography}{9}
\bibitem{ol1}
 A.\ Yu.\ Olshanskii,
 {\it An infinite group with subgroups of prime orders},
 Math.\ USSR Izv.\ 16 (1981), 279--289;
 translation of Izvestia Akad.\ Nauk SSSR Ser.\ Matem.\ 44 (1980), 309--321. 
\bibitem{ol2}
 A.\ Yu.\ Olshanskii,
 {\it Groups of bounded period with subgroups of prime order}, 
 Algebra and Logic 21 (1983), 369--418;
 translation of Algebra i Logika 21 (1982), 553--618.
\end{thebibliography}
%%%%%
%%%%%
\end{document}
