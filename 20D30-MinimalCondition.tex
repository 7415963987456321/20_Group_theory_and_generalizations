\documentclass[12pt]{article}
\usepackage{pmmeta}
\pmcanonicalname{MinimalCondition}
\pmcreated{2013-03-22 13:58:49}
\pmmodified{2013-03-22 13:58:49}
\pmowner{mclase}{549}
\pmmodifier{mclase}{549}
\pmtitle{minimal condition}
\pmrecord{4}{34753}
\pmprivacy{1}
\pmauthor{mclase}{549}
\pmtype{Definition}
\pmcomment{trigger rebuild}
\pmclassification{msc}{20D30}
\pmsynonym{descending chain condition}{MinimalCondition}
\pmrelated{ChernikovGroup}

\endmetadata

% this is the default PlanetMath preamble.  as your knowledge
% of TeX increases, you will probably want to edit this, but
% it should be fine as is for beginners.

% almost certainly you want these
\usepackage{amssymb}
\usepackage{amsmath}
\usepackage{amsfonts}

% used for TeXing text within eps files
%\usepackage{psfrag}
% need this for including graphics (\includegraphics)
%\usepackage{graphicx}
% for neatly defining theorems and propositions
%\usepackage{amsthm}
% making logically defined graphics
%%%\usepackage{xypic}

% there are many more packages, add them here as you need them

% define commands here
\begin{document}
A group is said to satisfy the \emph{minimal condition} if every strictly descending chain of subgroups
$$G_1 \supset G_2 \supset G_3 \supset \cdots$$
is finite.

This is also called the \emph{descending chain condition}.

A group which satisfies the minimal condition is necessarily periodic.  For if it contained an element $x$ of infinite order, then $$\langle x \rangle \supset \langle x^2 \rangle \supset \langle x^4 \rangle \supset \cdots \supset \langle x^{2^n} \rangle \supset \cdots$$ is an infinite descending chain of subgroups.

Similar properties are useful in other classes of algebraic structures: see for example the Artinian condition for rings and modules.
%%%%%
%%%%%
\end{document}
