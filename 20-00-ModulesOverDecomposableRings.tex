\documentclass[12pt]{article}
\usepackage{pmmeta}
\pmcanonicalname{ModulesOverDecomposableRings}
\pmcreated{2013-03-22 18:50:05}
\pmmodified{2013-03-22 18:50:05}
\pmowner{joking}{16130}
\pmmodifier{joking}{16130}
\pmtitle{modules over decomposable rings}
\pmrecord{4}{41639}
\pmprivacy{1}
\pmauthor{joking}{16130}
\pmtype{Theorem}
\pmcomment{trigger rebuild}
\pmclassification{msc}{20-00}
\pmclassification{msc}{16-00}
\pmclassification{msc}{13-00}

% this is the default PlanetMath preamble.  as your knowledge
% of TeX increases, you will probably want to edit this, but
% it should be fine as is for beginners.

% almost certainly you want these
\usepackage{amssymb}
\usepackage{amsmath}
\usepackage{amsfonts}

% used for TeXing text within eps files
%\usepackage{psfrag}
% need this for including graphics (\includegraphics)
%\usepackage{graphicx}
% for neatly defining theorems and propositions
%\usepackage{amsthm}
% making logically defined graphics
%%%\usepackage{xypic}

% there are many more packages, add them here as you need them

% define commands here

\begin{document}
Let $R_1,R_2$ be two, nontrivial, unital rings and $R=R_1\oplus R_2$. If $M_1$ is a $R_1$-module and $M_2$ is a $R_2$-module, then obviously $M_1\oplus M_2$ is a $R$-module via $(r,s)\cdot (m_1,m_2)=(r\cdot m_1,s\cdot m_2)$. We will show that every $R$-module can be obtain in this way.

\textbf{Proposition.} If $M$ is a $R$-module, then there exist submodules $M_1,M_2\subseteq M$ such that $M=M_1\oplus M_2$ and for any $r\in R_1$, $s\in R_2$, $m_1\in M_1$ and $m_2\in M_2$ we have 
$$(r,s)\cdot m_1=(r,0)\cdot m_1\ \ \ \ (r,s)\cdot m_2=(0,s)\cdot m_2,$$
i.e. ring action on $M_1$ (respectively $M_2$) does not depend on $R_2$ (respectively $R_1$).

\textit{Proof.} Let $e=(1,0)\in R$ and $f=(0,1)\in R$. Of course both $e,f$ are idempotents and $(1,1)=e+f$. Moreover $ef=fe=0$ and $e,f$ are central, i.e. $e,f\in \{c\in R\ \big| \ \forall_{x\in R}\ cx=xc\}$. We will use $e,f$ to construct submodules $M_1,M_2$. More precisely, let $M_1=eM$ and $M_2=fM$. Because $e,f$ are central, then it is clear that both $M_1$ and $M_2$ are submodules. We will show that $M_1+M_2=M$. Indeed, let $m\in M$. Then we have
$$m=(1,1)\cdot m=(e+f)\cdot m=e\cdot m + f\cdot m.$$
Thus $M_1+M_2=M$. Furthermore, assume that $m\in M_1\cap M_2$. Then there exist $m_1,m_2\in M$ such that
$$e\cdot m_1=m=f\cdot m_2$$
and therefore 
$$e\cdot m_1 - f\cdot m_2=0.$$
Now, after multiplying both sides by $e$ we obtain that
$$0=(ee)\cdot m_1 - (ef)\cdot m_2=e\cdot m_1-0\cdot m_2=e\cdot m_1=m,$$
thus $M_1\cap M_2=0$. This shows that $M=M_1\oplus M_2$. To finish the proof, we need to show that the ring action on $M_1$ does not depend on $R_2$ (the other case is analogous). But this is clear, since for any $(r,s)\in R$ and $m\in M$ we have
$$(r,s)\cdot (e\cdot m)=\big( (r,s)(1,0)\big) \cdot m=(r,0)\cdot m=\big( (r,0)(1,0)\big)\cdot m=(r,0)\cdot(e\cdot m).$$
This completes the proof. $\square$
%%%%%
%%%%%
\end{document}
