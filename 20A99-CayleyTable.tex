\documentclass[12pt]{article}
\usepackage{pmmeta}
\pmcanonicalname{CayleyTable}
\pmcreated{2013-03-22 13:06:44}
\pmmodified{2013-03-22 13:06:44}
\pmowner{akrowne}{2}
\pmmodifier{akrowne}{2}
\pmtitle{Cayley table}
\pmrecord{11}{33540}
\pmprivacy{1}
\pmauthor{akrowne}{2}
\pmtype{Definition}
\pmcomment{trigger rebuild}
\pmclassification{msc}{20A99}
\pmsynonym{Cayley-table}{CayleyTable}

\endmetadata

\usepackage{amssymb}
\usepackage{amsmath}
\usepackage{amsfonts}

%\usepackage{psfrag}
%\usepackage{graphicx}
%%%\usepackage{xypic}
\begin{document}
A \emph{Cayley table} for a group is essentially the ``multiplication table'' of the group.\footnote{A caveat to novices in group theory: multiplication is usually used notationally to represent the group operation, but the operation needn't resemble multiplication in the reals.  Hence, you should take ``multiplication table'' with a grain or two of salt.}  The columns and rows of the table (or matrix) are labeled with the elements of the group, and the cells represent the result of applying the group operation to the row-th and column-th elements.

Formally, let $G$ be our group, with operation $\circ$ the group operation.  Let $C$ be the Cayley table for the group, with $C(i,j)$ denoting the element at row $i$ and column $j$.  Then

$$ C(i,j) = e_i \circ e_j $$

where $e_i$ is the $i$th element of the group, and $e_j$ is the $j$th element.

Note that for an Abelian group, we have $e_i \circ e_j = e_j \circ e_i$, hence the Cayley table is a symmetric matrix.

All Cayley tables for isomorphic groups are isomorphic (that is, the same, invariant of the labeling and ordering of group elements). 

\subsection{Examples.}

\begin{itemize}

\item The Cayley table for $\mathbb{Z}_4$, the group of integers modulo 4 (under addition), would be

$$ \left(\begin{array}{c|cccc}
     & [0] & [1] & [2] & [3] \\
\hline
\; [0] & [0] & [1] & [2] & [3] \\
\; [1] & [1] & [2] & [3] & [0] \\
\; [2] & [2] & [3] & [0] & [1] \\
\; [3] & [3] & [0] & [1] & [2]
\end{array}\right) $$

\item The Cayley table for $S_3$, the permutation group of order 3, is
$$ \left(\begin{array}{c|cccccc}
& (1) & (123) & (132) & (12) & (13) & (23) \\
\hline
(1)  & (1) & (123) & (132) & (12) & (13) & (23) \\
(123) & (123) & (132) & (1) & (13) & (23) & (12) \\
(132) & (132) & (1) & (123) & (23) & (12) & (13) \\
(12) & (12) & (23) & (13) & (1) & (132) & (123) \\
(13) & (13) & (12) & (23) & (123) & (1) & (132) \\
(23) & (23) & (13) & (12) & (132) & (123) & (1)
\end{array}\right) $$

\end{itemize}
%%%%%
%%%%%
\end{document}
