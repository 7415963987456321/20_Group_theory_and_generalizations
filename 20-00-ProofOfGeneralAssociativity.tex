\documentclass[12pt]{article}
\usepackage{pmmeta}
\pmcanonicalname{ProofOfGeneralAssociativity}
\pmcreated{2014-05-11 15:12:33}
\pmmodified{2014-05-11 15:12:33}
\pmowner{pahio}{2872}
\pmmodifier{pahio}{2872}
\pmtitle{proof of general associativity}
\pmrecord{10}{39962}
\pmprivacy{1}
\pmauthor{pahio}{2872}
\pmtype{Proof}
\pmcomment{trigger rebuild}
\pmclassification{msc}{20-00}
\pmrelated{PowerSet}
\pmrelated{Cardinality}

% this is the default PlanetMath preamble.  as your knowledge
% of TeX increases, you will probably want to edit this, but
% it should be fine as is for beginners.

% almost certainly you want these
\usepackage{amssymb}
\usepackage{amsmath}
\usepackage{amsfonts}

% used for TeXing text within eps files
%\usepackage{psfrag}
% need this for including graphics (\includegraphics)
%\usepackage{graphicx}
% for neatly defining theorems and propositions
 \usepackage{amsthm}
% making logically defined graphics
%%%\usepackage{xypic}

% there are many more packages, add them here as you need them

% define commands here

\theoremstyle{definition}
\newtheorem*{thmplain}{Theorem}

\begin{document}
\PMlinkescapeword{fixed} \PMlinkescapeword{force}

We suppose that ``$\cdot$'' is an associative binary operation of the set $S$.\\

Let\, $f_1\!:\,S \to 2^S$\, be the mapping
         $$f_1(a_1) \;:=\; \{a_1\} \quad \forall a_1 \in S.$$
We define recursively the mapping
$$f_n\!:\,\underbrace{S\!\times\!S\!\times\!\ldots\!\times\!S}_n \to 2^S$$
such that
\begin{align}
f_n(a_1,\ldots,\,a_n) \;:=\; \bigcup_{r=1}^{n-1}f_r(a_1,\ldots,\,a_r)\cdot f_{n-r}(a_{r+1},\ldots,\,a_n)
\end{align}
for\, $n = 2,\,3,\,4,\,\ldots$

For example,
$$f_2(a_1,a_2) \;=\; \{a_1\}\cdot\{a_2\} \;=\; \{a_1\cdot a_2\},$$
$$f_3(a_1,a_2,a_3) \;=\; \{a_1\cdot(a_2\cdot a_3)\}\cup\{(a_1\cdot a_2)\cdot a_3\} \;=\; 
\{a_1\cdot(a_2\cdot a_3),\,(a_1\cdot a_2)\cdot a_3\} \;=\; \{(a_1\cdot a_2)\cdot a_3\};$$
the last equality due to the associativity.\, It's clear that always
$$|f_1(a_1)|\;=\; 1,\qquad |f_2(a_1,a_2)| \;=\; 1,
\qquad |f_3(a_1,a_2,a_3)| \;=\; 1.$$

We shall show by induction that
\begin{align}
|f_n(a_1,a_2,\ldots,\,a_n)| \;=\; 1
\end{align}
for each positive integer $n$.\, This means that all groupings of the $n$ fixed elements using parentheses in forming the products with ``$\cdot$'' yield one single element.\\

We make the induction hypothesis, that (2) is true for all\, $n < k.$

Now let $z$ and $z'$ be arbitrary elements of\, $f_k(a_1,\ldots,\,a_k)$.\, 
Then there exist the elements $x,y,x',y'$ of $S$ and the 
integers\, $r, s \in \{1,\,\ldots,\,k\!-\!1\}$\, such that 
$$z \;=\; x\cdot y,\quad x \in f_r(a_1,\ldots,\,a_r),\quad y \in f_{k-r}(a_{r+1},\ldots,\, a_k),$$
$$z' \;=\; x'\cdot y',\quad x' \in f_s(a_1,\ldots,\,a_s),\quad y' \in f_{k-s}(a_{s+1},\ldots,\, a_k).$$
If specially\, $r = s$,\, then, by the induction hypothesis,\, $x = x'$\, and\, $y = y'$,\, whence\, $z = x\cdot y = x'\cdot y' = z'$.\, If on the contrary,\, $r \neq s$,\, e.g.\, $r < s$,\, then the induction hypothesis guarantees the existence of an element $v$ of $S$ such that 
$$f_{s-r}(a_{r+1},\ldots,\,a_s) \;=\; \{v\}$$
and
$$x' \;=\; x\cdot v,\quad y \;=\; v\cdot y'.$$
Since ``$\cdot$'' is associative, we have
$$z \;=\; x\cdot y \;=\; x\cdot(v\cdot y') \;=\; (x\cdot v)\cdot y' \;=\; x'\cdot y' \;=\; z'.$$
Thus the equation (2) is in force for\, $n = k$.




%%%%%
%%%%%
\end{document}
