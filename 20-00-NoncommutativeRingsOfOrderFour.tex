\documentclass[12pt]{article}
\usepackage{pmmeta}
\pmcanonicalname{NoncommutativeRingsOfOrderFour}
\pmcreated{2013-03-22 17:09:24}
\pmmodified{2013-03-22 17:09:24}
\pmowner{Wkbj79}{1863}
\pmmodifier{Wkbj79}{1863}
\pmtitle{non-commutative rings of order four}
\pmrecord{12}{39467}
\pmprivacy{1}
\pmauthor{Wkbj79}{1863}
\pmtype{Topic}
\pmcomment{trigger rebuild}
\pmclassification{msc}{20-00}
\pmclassification{msc}{16B99}
\pmrelated{Klein4Ring}
\pmrelated{OppositeRing}
\pmrelated{ExampleOfKlein4Ring}

\endmetadata

\usepackage{amssymb}
\usepackage{amsmath}
\usepackage{amsfonts}
\usepackage{pstricks}
\usepackage{psfrag}
\usepackage{graphicx}
\usepackage{amsthm}
%%\usepackage{xypic}

\begin{document}
\PMlinkescapeword{order}
\PMlinkescapeword{multiplication}

Up to isomorphism, there are two non-commutative rings of \PMlinkname{order}{OrderRing} four.  Since all cyclic rings are \PMlinkname{commutative}{CommutativeRing}, one can immediately deduce that a ring of order four must have an additive group that is isomorphic to $\mathbb{F}_2 \oplus \mathbb{F}_2$.

One of the two non-commutative rings of order four is the Klein 4-ring, whose multiplication table is given by:

$$\begin{array}{c|cccc}
\cdot & 0 & a & b & c \\
\hline
0 & 0 & 0 & 0 & 0 \\
a & 0 & a & 0 & a \\
b & 0 & b & 0 & b \\
c & 0 & c & 0 & c \end{array}$$

The other is closely related to the Klein 4-ring.  In fact, it is anti-isomorphic to the Klein 4-ring; that is, its multiplication table is obtained by swapping the \PMlinkescapetext{rows and columns} of the multiplication table for the Klein 4-ring:

$$\begin{array}{c|cccc}
\cdot & 0 & a & b & c \\
\hline
0 & 0 & 0 & 0 & 0 \\
a & 0 & a & b & c \\
b & 0 & 0 & 0 & 0 \\
c & 0 & a & b & c \end{array}$$
%%%%%
%%%%%
\end{document}
