\documentclass[12pt]{article}
\usepackage{pmmeta}
\pmcanonicalname{HamiltonianGroup}
\pmcreated{2013-03-22 15:36:14}
\pmmodified{2013-03-22 15:36:14}
\pmowner{yark}{2760}
\pmmodifier{yark}{2760}
\pmtitle{Hamiltonian group}
\pmrecord{15}{37520}
\pmprivacy{1}
\pmauthor{yark}{2760}
\pmtype{Definition}
\pmcomment{trigger rebuild}
\pmclassification{msc}{20F24}
\pmclassification{msc}{20F18}
\pmclassification{msc}{20F50}
\pmsynonym{Hamilton group}{HamiltonianGroup}
\pmdefines{Dedekind group}
\pmdefines{quasi-Hamiltonian group}
\pmdefines{Hamiltonian}

\endmetadata

\usepackage{amssymb}
\usepackage{amsmath}
\usepackage{amsthm}

\newtheorem*{cor*}{Corollary}
\newtheorem*{thm*}{Theorem}

% The below lines should work as the command
% \renewcommand{\bibname}{References}
% without creating havoc when rendering an entry in 
% the page-image mode.
\makeatletter
\@ifundefined{bibname}{}{\renewcommand{\bibname}{References}}
\makeatother
\begin{document}
\PMlinkescapeword{class}
\PMlinkescapeword{form}
\PMlinkescapeword{finite}
\PMlinkescapeword{hamiltonian}
\PMlinkescapeword{infinite}
\PMlinkescapeword{length}
\PMlinkescapeword{structure}
\PMlinkescapeword{subgroup}
\PMlinkescapeword{subgroups}
\PMlinkescapeword{theorem}

A \emph{Hamiltonian group} is a non-abelian group in which all \PMlinkname{subgroups}{Subgroup} are normal.

Richard Dedekind investigated finite Hamiltonian groups in 1895, and proved that they all contain a copy of the quaternion group $Q_8$ of order $8$ (see the structure theorem below). He named them in honour of William Hamilton, the discoverer of quaternions.

Groups in which all subgroups are normal (that is, groups that are either abelian or Hamiltonian) are sometimes called \emph{Dedekind groups}, or \emph{quasi-Hamiltonian groups}.

The following structure theorem was proved in its full form by Baer\cite{baer}, but Dedekind already came close to it in his original paper\cite{dedekind}.

\begin{thm*}
A group is Hamiltonian if and only if it is isomorphic to $Q_8\times P$
for some periodic abelian group $P$ that has no element of order $4$.
\end{thm*}

In particular, Hamiltonian groups are always periodic (in fact, locally finite), nilpotent of class $2$, and solvable of length $2$.

From the structure theorem one can also see that the only Hamiltonian \PMlinkname{$p$-groups}{PGroup4} are $2$-groups of the form $Q_8\times B$,
where $B$ is an elementary abelian $2$-group.

\begin{thebibliography}{9}
\bibitem{baer}
 R.\ Baer,
 {\it Situation der Untergruppen und Struktur der Gruppe},
 S.\ B.\ Heidelberg.\ Akad.\ Wiss. 2 (1933), 12--17.
\bibitem{dedekind}
 R.\ Dedekind,
 {\it Ueber Gruppen, deren s\"ammtliche Theiler Normaltheiler sind},
 Mathematische Annalen 48 (1897), 548--561. (This paper is
\PMlinkexternal{available from GDZ}{http://gdz.sub.uni-goettingen.de/dms/resolveppn/?GDZPPN002256258}.)
\end{thebibliography}
%%%%%
%%%%%
\end{document}
