\documentclass[12pt]{article}
\usepackage{pmmeta}
\pmcanonicalname{TheHistoryOfHavingSettledToAccomplishStudies}
\pmcreated{2013-11-27 10:59:44}
\pmmodified{2013-11-27 10:59:44}
\pmowner{jacou}{1000048}
\pmmodifier{}{0}
\pmtitle{The History of Having Settled To Accomplish Studies}
\pmrecord{14}{40669}
\pmprivacy{1}
\pmauthor{jacou}{0}
\pmtype{Proof}
\pmcomment{trigger rebuild}
\pmclassification{msc}{20A05}

\endmetadata


\begin{document}
\newtheorem*{theorem}{Theorem}
\begin{theorem}
A group homomorphism preserves inverses elements. That is, for groups $(G,\ast)$ and $(K,\star)$, and a homomorphism $\phi\colon G \to K$, $\phi (x^{-1}) = \phi (x)^{-1}$.
\end{theorem}

\begin{proof}
Fix an $x \in G$. Observe that

\begin{equation}\label{eq:idG}
\phi (x \ast x^{-1}) = \phi (1_G) = \phi (x) \star \phi (x^{-1})
\end{equation}

Recall that, for any group homomorphism $\phi\colon G \to K$,

\begin{equation}\label{eq:idK}
\phi (1_G) = 1_K  
\end{equation}

In other \PMlinkescapetext{words}, homomorphisms preserve identity. \footnote{A proof for that statement is attached to the \PMlinkescapetext{parent}.} It follows from (\ref{eq:idG}) and (\ref{eq:idK}) that

\begin{equation}
\phi (x) \star \phi (x^{-1}) = 1_K
\end{equation}

Because the inverse of any group is unique, the only value of $\phi (x^{-1})$ whose product with $\phi (x)$ is $1_K$ is, of course, $\phi (x)^{-1}$. Therefore, all group homomorphisms preserve the inverse.
\end{proof}
%%%%%
%%%%%
\end{document}
