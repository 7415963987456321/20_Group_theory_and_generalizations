\documentclass[12pt]{article}
\usepackage{pmmeta}
\pmcanonicalname{LocallycalP}
\pmcreated{2013-03-22 14:18:57}
\pmmodified{2013-03-22 14:18:57}
\pmowner{yark}{2760}
\pmmodifier{yark}{2760}
\pmtitle{locally $\cal P$}
\pmrecord{5}{35782}
\pmprivacy{1}
\pmauthor{yark}{2760}
\pmtype{Definition}
\pmcomment{trigger rebuild}
\pmclassification{msc}{20E25}
\pmrelated{GeneralizedCyclicGroup}
\pmrelated{LocallyFiniteGroup}
\pmrelated{LocallyNilpotentGroup}

\endmetadata

\usepackage{amssymb}
\usepackage{amsmath}
\usepackage{amsfonts}

%\usepackage{psfrag}
%\usepackage{graphicx}
%\usepackage{amsthm}
%%%\usepackage{xypic}

\renewcommand{\le}{\leqslant}
\renewcommand{\ge}{\geqslant}
\renewcommand{\leq}{\leqslant}
\renewcommand{\geq}{\geqslant}
\begin{document}
\PMlinkescapeword{property}

Let $\cal P$ be a property of groups.
A group $G$ is said to be \emph{locally $\cal P$} if every nontrivial finitely generated subgroup of $G$ has property $\cal P$.

For example, the locally infinite groups are precisely the torsion-free groups.
Other classes of groups defined this way include locally finite groups and locally cyclic groups.
%%%%%
%%%%%
\end{document}
