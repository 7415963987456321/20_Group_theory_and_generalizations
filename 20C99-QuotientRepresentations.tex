\documentclass[12pt]{article}
\usepackage{pmmeta}
\pmcanonicalname{QuotientRepresentations}
\pmcreated{2013-03-22 16:37:59}
\pmmodified{2013-03-22 16:37:59}
\pmowner{rm50}{10146}
\pmmodifier{rm50}{10146}
\pmtitle{quotient representations}
\pmrecord{6}{38834}
\pmprivacy{1}
\pmauthor{rm50}{10146}
\pmtype{Definition}
\pmcomment{trigger rebuild}
\pmclassification{msc}{20C99}

\endmetadata

% this is the default PlanetMath preamble.  as your knowledge
% of TeX increases, you will probably want to edit this, but
% it should be fine as is for beginners.

% almost certainly you want these
\usepackage{amssymb}
\usepackage{amsmath}
\usepackage{amsfonts}

% used for TeXing text within eps files
%\usepackage{psfrag}
% need this for including graphics (\includegraphics)
%\usepackage{graphicx}
% for neatly defining theorems and propositions
%\usepackage{amsthm}
% making logically defined graphics
%%%\usepackage{xypic}

% there are many more packages, add them here as you need them

% define commands here
\newtheorem{defn}{Definition}
\begin{document}
\PMlinkescapeword{stable}
We assume that all representations ($G$-modules) are finite-dimensional.

\begin{defn} If $N_1$ and $N_2$ are $G$-modules over a field $k$ (i.e. representations of $G$ in $N_1$ and $N_2$), then a map $\varphi:N_1\to N_2$ is a \emph{$G$-map} if $\varphi$ is $k$-linear and preserves the $G$-action, i.e. if 
\[\varphi(\sigma \cdot x)=\sigma\cdot \varphi(x)\] 
\end{defn}

$G$-maps have subrepresentations, also called $G$-submodules, as their kernel and image. To see this, let $\varphi:N_1\to N_2$ be a $G$-map; let $M_1\subset N_1$ and $M_2\subset N_2$ be the kernel and image respectively of $\varphi$. $M_1$ is a submodule of $N_1$ if it is stable under the action of $G$, but
\[x\in M_1\Rightarrow \varphi(\sigma\cdot x)=\sigma\cdot\varphi(x)=0\Rightarrow \sigma\cdot x\in M_1\]
$M_2$ is a submodule of $N_2$ if it is stable under the action of $G$, but
\[y=\varphi(x)\in M_2\Rightarrow \sigma\cdot y=\sigma\cdot \varphi(x)=\varphi(\sigma\cdot x)\Rightarrow \sigma\cdot y\in M_2\]

Finally, we define the intuitive concept of a quotient $G$-module. Suppose $N'\subset N$ is a $G$-submodule. Then $N/N'$ is a finite-dimensional vector space. We can define an action of $G$ on $N/N'$ via $\sigma(n+N')=\sigma(n)+\sigma(N')=\sigma(n)+N'$, so that $n+N'$ is well-defined under the action and $N/N'$ is a $G$-module.
%%%%%
%%%%%
\end{document}
