\documentclass[12pt]{article}
\usepackage{pmmeta}
\pmcanonicalname{Centralizer}
\pmcreated{2013-03-22 12:35:01}
\pmmodified{2013-03-22 12:35:01}
\pmowner{drini}{3}
\pmmodifier{drini}{3}
\pmtitle{centralizer}
\pmrecord{14}{32833}
\pmprivacy{1}
\pmauthor{drini}{3}
\pmtype{Definition}
\pmcomment{trigger rebuild}
\pmclassification{msc}{20-00}
\pmsynonym{centraliser}{Centralizer}
\pmrelated{Normalizer}
\pmrelated{GroupCentre}
\pmrelated{ClassEquationTheorem}

\usepackage{amssymb}
\usepackage{amsmath}
\usepackage{amsfonts}
\begin{document}
Let $G$ be a group.  The {\em centralizer} of an element $a \in G$ is defined to be the set $$C(a) = \{x \in G \mid xa = ax\}$$
Observe that, by definition, $e \in C(a)$, and that if $x, y \in C(a)$, then $xy^{-1}a = xy^{-1}a(yy^{-1})=xy^{-1}yay^{-1}=xay^{-1} = axy^{-1}$, so that $xy^{-1} \in C(a)$. Thus $C(a)$ is a subgroup of $G$.  For $a \neq e$, the subgroup is non-trivial, containing at least $\{e, a\}$.

To illustrate an application of this concept we prove the following lemma.

{\bf Lemma:}\\
There exists a bijection between the right cosets of $C(a)$ and the conjugates of $a$.

{\bf Proof:}\\
If $x,y \in G$ are in the same right coset, then $y = cx$ for some $c \in C(a)$. Thus $y^{-1}ay = x^{-1}c^{-1}acx = x^{-1}c^{-1}cax = x^{-1}ax$.
Conversely, if $y^{-1}ay = x^{-1}ax$ then $xy^{-1}a = axy^{-1}$ and $xy^{-1} \in C(a)$ giving $x,y$ are in the same right coset.
Let $[a]$ denote the conjugacy class of $a$. It follows that $|[a]| = [G : C(a)]$ and  $|[a]| \mid |G|$.\\

We remark that $a \in Z(G) \iff C(a) = G \iff |[a]| = 1$, where $Z(G)$ denotes the center of $G$.

Now let $G$ be a $p$-group, i.e. a finite group of order $p^n$,
where $p$ is a prime and $n$ is a positive integer.
Let $z = |Z(G)|$.
Summing over elements in distinct conjugacy classes,
we have $p^n = \sum{|[a]|} = z + \sum_{a \notin Z(G)}{|[a]|}$
since the center consists precisely of the conjugacy classes of
cardinality $1$.
But $|[a]| \mid p^n$, so $p \mid z$.
However, $Z(G)$ is certainly non-empty, so we conclude that every
$p$-group has a non-trivial center.

The groups $C(gag^{-1})$ and $C(a)$, for any $g$, are isomorphic.
%%%%%
%%%%%
\end{document}
