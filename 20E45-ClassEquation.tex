\documentclass[12pt]{article}
\usepackage{pmmeta}
\pmcanonicalname{ClassEquation}
\pmcreated{2013-03-22 13:10:41}
\pmmodified{2013-03-22 13:10:41}
\pmowner{yark}{2760}
\pmmodifier{yark}{2760}
\pmtitle{class equation}
\pmrecord{9}{33624}
\pmprivacy{1}
\pmauthor{yark}{2760}
\pmtype{Theorem}
\pmcomment{trigger rebuild}
\pmclassification{msc}{20E45}
\pmsynonym{conjugacy class formula}{ClassEquation}
\pmrelated{ConjugacyClass}

\usepackage{amssymb}
\usepackage{amsmath}
\usepackage{amsfonts}

\begin{document}
The conjugacy classes of a group form a partition of its elements.
In a finite group, this means that the order of the group is
the sum of the number of elements of the distinct conjugacy classes.
For an element $g$ of group $G$,
we denote the centralizer in $G$ of $g$ by $C_G(g)$.
The number of elements in the conjugacy class of $g$ is $[G:C_G(g)]$,
the index of $C_G(g)$ in $G$.
For an element $g$ of the center $Z(G)$ of $G$,
the conjugacy class of $g$ consists of the singleton $\{g\}$.
Putting this together gives us the \emph{class equation}
\[
  |G| = |Z(G)| + \sum_{i=1}^m [G:C_G(x_i)]
\]
where the $x_i$ are elements of
the distinct conjugacy classes contained in $G\setminus Z(G)$.
%%%%%
%%%%%
\end{document}
