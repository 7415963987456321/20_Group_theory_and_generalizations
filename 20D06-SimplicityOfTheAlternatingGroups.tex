\documentclass[12pt]{article}
\usepackage{pmmeta}
\pmcanonicalname{SimplicityOfTheAlternatingGroups}
\pmcreated{2013-03-22 13:07:57}
\pmmodified{2013-03-22 13:07:57}
\pmowner{rmilson}{146}
\pmmodifier{rmilson}{146}
\pmtitle{simplicity of the alternating groups}
\pmrecord{16}{33569}
\pmprivacy{1}
\pmauthor{rmilson}{146}
\pmtype{Result}
\pmcomment{trigger rebuild}
\pmclassification{msc}{20D06}
\pmclassification{msc}{20E32}
%\pmkeywords{simple}
%\pmkeywords{alternating}
%\pmkeywords{group}
%\pmkeywords{simplicity}
\pmrelated{ExamplesOfFiniteSimpleGroups}

\endmetadata

\usepackage{amsmath}
\usepackage{amsfonts}
\usepackage{amssymb}
\usepackage{amsthm}

\newtheorem{theorem}{Theorem}
\newtheorem{lemma}[theorem]{Lemma}

\begin{document}
\PMlinkescapeword{fix}
\PMlinkescapeword{normality}
\PMlinkescapeword{type}
\PMlinkescapeword{decompositions}
\PMlinkescapeword{covers}
\PMlinkescapeword{observation}
\PMlinkescapeword{necessary}

\begin{theorem}
  If $n \geq 5$, then the alternating group
  on $n$ symbols, $A_n$, is simple.
\end{theorem}
\noindent
Throughout the proof we extensively employ the cycle notation, with
composition on the left, as is usual.  The symmetric group on $n$
symbols is denoted by $S_n$.  

The following observation will be useful.  Let $\pi$ be a permutation
written as disjoint cycles
\[
\pi = (a_1, a_2, \ldots, a_k)(b_1, b_2, \ldots, b_l)\ldots
(c_1,\ldots, c_m)
\]
It is easy to check that for every permutation $\sigma \in S_n$ we
have
\[
\sigma \pi \sigma^{-1} = (\sigma(a_1), \sigma(a_2), \ldots, \sigma(a_k))\,
                         (\sigma(b_1),\sigma(b_2), \ldots
                         \sigma(b_l))\, \ldots (\sigma(c_1),\ldots,
                         \sigma(c_m)) 
\]
As a consequence, two permutations of $S_n$ are conjugate exactly when
they have the same cycle type.

Two preliminary results  will also be necessary.  
\begin{lemma}
The set of cycles of length $3$ generates $A_n$.
\end{lemma}
\begin{proof}
A product of $3$-cycles is an even permutation, so the subgroup generated by all $3$-cycles is therefore contained in $A_n$.  For the reverse inclusion, by definition every even permutation is the product of even number of transpositions.  Thus, it suffices to show that the product of two transpositions can be written as a product of $3$-cycles.  There are two possibilities.  Either the two transpositions move an element in common, say $(a,b)$ and $(a,c)$, or the two transpositions are disjoint, say $(a,b)$ and $(c,d)$.  In the former case,
\[
(a,b)(a,c) = (a,c,b),
\]
and in the latter,
\[
(a,b)(c,d) = (a,b,d)(c,b,d).
\]
This establishes the first lemma.
\end{proof}

\begin{lemma}
\label{FirstLemma}
If a normal subgroup $N \triangleleft A_n$ contains a $3$-cycle, then
$N = A_n$.
\end{lemma}
\begin{proof}
We will show that if $(a,b,c) \in N$, then the assumption of normality implies that any other $(a',b',c') \in N$.  This is easy to show, because there is some permutation in $\sigma \in S_n$ that under conjugation takes $(a,b,c)$ to $(a',b',c')$, that is
\[
\sigma (a,b,c) \sigma^{-1} = (\sigma(a), \sigma(b), \sigma(c)) = (a',b',c').
\]
In case $\sigma$ is odd, then (because $n \geq 5$) we can choose some transposition $(d,e) \in A_n$ disjoint from $(a',b',c')$ so that
\[
\sigma (a,b,c) \sigma^{-1} =  (d,e)(a',b',c')(d,e),
\]
that is,
\[
\sigma' (a,b,c) \sigma'^{-1} = (d,e)\sigma (a,b,c) \sigma^{-1}(d,e) = (a',b',c')
\]
where $\sigma'$ is even.  This means that $N$ contains all $3$-cycles,
as $N \triangleleft A_n$. Hence, by previous lemma $N = A_n$ as
required.
\end{proof}

\paragraph{Proof of theorem.}
  Let $N \triangleleft A_n$ be a non-trivial normal subgroup.  We will
  show that $N = A_n$.  The proof now proceeds by cases.  In each
  case, the normality of $N$ will allow us to reduce the proof to
  Lemma 2 or to one of the previous cases.

\paragraph{Case 1.}
Suppose that there exists a $\pi\in N$ that, when written as disjoint
cycles, has a cycle of length at least $4$, say
\[
\pi = (a_1, a_2, a_3, a_4, \ldots)\ldots
\]
Upon conjugation by $(a_1, a_2, a_3) \in A_n$, we obtain
\[
\pi' = (a_1, a_2, a_3) \pi (a_3, a_2, a_1) = (a_2, a_3, a_1, a_4,
\ldots) \ldots
\]
Hence, $\pi' \in N$, and hence $\pi' \pi^{-1} = (a_1, a_2, a_4) \in N$
also.  Notice that the rest of the cycles cancel.  By Lemma
\ref{FirstLemma}, $N = A_n$.

\paragraph{Case 2.}
Suppose that there exists a $\pi \in N$ whose disjoint cycle
decomposition has at least two cycles of length 3, say
\[ \pi = (a,b,c)(d,e,f) \ldots \] Conjugation by $(c,d,e)\in A_n$
implies that $N$ also contains
\[
\pi' = (c,d,e)\pi(e,d,c) = (a,b,d)(e,c,f)\ldots
\]
Hence, $N$ also contains $\pi' \pi = (a,d,c,b,f)\ldots$. This reduces
the proof to Case 1.

\paragraph{Case 3.}
Suppose that there exists a $\pi \in N$ whose disjoint cycle
decomposition consists of exactly one $3$-cycle and an even (possibly
zero) number of transpositions.  Hence, $\pi \pi$ is a $3$-cycle.
Lemma \ref{FirstLemma} can then be applied to complete the proof.

\paragraph{Case 4.}
Suppose there exists a $\pi \in N$ of the form $\pi = (a,b)(c,d)$.
Conjugating by $(a,e,b)$ with $e$ distinct from $a,~b,~c,~d$ (at least
one such $e$ exists, as $n \geq 5$) yields
\[
\pi' = (a,e,b)\pi(b,e,a) = (a,e)(c,d) \in N.
\]
Hence $\pi' \pi = (a,b,e) \in N$.  Again, Lemma \ref{FirstLemma} applies.

\paragraph{Case 5.}
Suppose that $N$ contains a permutation of the form 
\[\pi = (a_1,
b_1)(a_2, b_2)(a_3, b_3)(a_4, b_4)\ldots\] This time we conjugate by
$(a_2, b_1)(a_3, b_2)$.
\[
\pi' = (a_2,b_1)(a_3,b_2) \pi (a_3, b_2)(a_2,b_1) = (a_1,
a_2)(a_3,b_1)(b_2,b_3)(a_4,b_4) \ldots.
\]
Observe that 
\[ \pi' \pi = (a_1, a_3, b_2)(a_2, b_3, b_1) \ldots , \]
which reduces the proof to Case 2.

Since there exists at least one non-identity $\pi\in N$, and since this
$\pi$ is covered by one of the above cases, we conclude that $N =
A_n$, as was to be shown.

%%%%%
%%%%%
\end{document}
