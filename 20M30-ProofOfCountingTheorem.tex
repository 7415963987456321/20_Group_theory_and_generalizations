\documentclass[12pt]{article}
\usepackage{pmmeta}
\pmcanonicalname{ProofOfCountingTheorem}
\pmcreated{2013-03-22 12:47:07}
\pmmodified{2013-03-22 12:47:07}
\pmowner{n3o}{216}
\pmmodifier{n3o}{216}
\pmtitle{proof of counting theorem}
\pmrecord{5}{33099}
\pmprivacy{1}
\pmauthor{n3o}{216}
\pmtype{Proof}
\pmcomment{trigger rebuild}
\pmclassification{msc}{20M30}

\endmetadata

% this is the default PlanetMath preamble.  as your knowledge
% of TeX increases, you will probably want to edit this, but
% it should be fine as is for beginners.

% almost certainly you want these
\usepackage{amssymb}
\usepackage{amsmath}
\usepackage{amsfonts}

% used for TeXing text within eps files
%\usepackage{psfrag}
% need this for including graphics (\includegraphics)
%\usepackage{graphicx}
% for neatly defining theorems and propositions
%\usepackage{amsthm}
% making logically defined graphics
%%%\usepackage{xypic}

% there are many more packages, add them here as you need them

% define commands here
\begin{document}
Let $N$ be the cardinality of the set of all the couples $(g,x)$ such that $g \cdot x = x$. For each $g \in G$, there exist $\operatorname{stab}_g(X)$ couples with $g$ as the first element, while for each $x$, there are $|G_x|$ couples with $x$ as the second element. Hence the following equality holds:
\[ N = \sum_{g \in G} \operatorname{stab}_g(X) = \sum_{x \in X} |G_x|. \] 
From the orbit-stabilizer theorem it follows that:
\[ N = |G| \sum_{x \in X} \frac{1}{|G(x)|}. \]
Since all the $x$ belonging to the same orbit $G(x)$ contribute with
\[ |G(x)| \frac{1}{|G(x)|} = 1 \]
in the sum, then $\sum_{x\in X} 1/|G(x)|$ precisely equals the number of distinct orbits $s$. We have therefore
\[ \sum_{g \in G} \operatorname{stab}_g(X) = |G| s, \]
which proves the theorem.
%%%%%
%%%%%
\end{document}
