\documentclass[12pt]{article}
\usepackage{pmmeta}
\pmcanonicalname{ProofOfCayleysTheorem}
\pmcreated{2013-03-22 12:30:50}
\pmmodified{2013-03-22 12:30:50}
\pmowner{Evandar}{27}
\pmmodifier{Evandar}{27}
\pmtitle{proof of Cayley's theorem}
\pmrecord{5}{32751}
\pmprivacy{1}
\pmauthor{Evandar}{27}
\pmtype{Proof}
\pmcomment{trigger rebuild}
\pmclassification{msc}{20B99}

% this is the default PlanetMath preamble.  as your knowledge
% of TeX increases, you will probably want to edit this, but
% it should be fine as is for beginners.

% almost certainly you want these
\usepackage{amssymb}
\usepackage{amsmath}
\usepackage{amsfonts}

% used for TeXing text within eps files
%\usepackage{psfrag}
% need this for including graphics (\includegraphics)
%\usepackage{graphicx}
% for neatly defining theorems and propositions
%\usepackage{amsthm}
% making logically defined graphics
%%%\usepackage{xypic} 

% there are many more packages, add them here as you need them

% define commands here
\begin{document}
Let $G$ be a group, and let $S_G$ be the permutation group of the underlying set $G$.  For each $g\in G$, define $\rho_g : G\rightarrow G$ by $\rho_g (h) = gh$.  Then $\rho_g$ is invertible with inverse $\rho_{g^{-1}}$, and so is a permutation of the set $G$.

Define $\Phi :G\rightarrow S_G$ by $\Phi (g) = \rho_g$.  Then $\Phi$ is a homomorphism, since $$(\Phi (gh))(x) = \rho_{gh}(x) = ghx = \rho_g(hx) = (\rho_g\circ\rho_h)(x) = ((\Phi (g))(\Phi (h)))(x)$$

And $\Phi$ is injective, since if $\Phi (g) = \Phi (h)$ then $\rho_g = \rho_h$, so $gx = hx$ for all $x\in X$, and so $g=h$ as required.

So $\Phi$ is an embedding of $G$ into its own permutation group.  If $G$ is finite of order $n$, then simply numbering the elements of $G$ gives an embedding from $G$ to $S_n$.
%%%%%
%%%%%
\end{document}
