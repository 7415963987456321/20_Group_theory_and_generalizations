\documentclass[12pt]{article}
\usepackage{pmmeta}
\pmcanonicalname{NoncommutingGraph}
\pmcreated{2013-03-22 15:18:50}
\pmmodified{2013-03-22 15:18:50}
\pmowner{GrafZahl}{9234}
\pmmodifier{GrafZahl}{9234}
\pmtitle{non-commuting graph}
\pmrecord{11}{37117}
\pmprivacy{1}
\pmauthor{GrafZahl}{9234}
\pmtype{Definition}
\pmcomment{trigger rebuild}
\pmclassification{msc}{20D60}
\pmclassification{msc}{05C25}
\pmsynonym{non-commuting graph of a group}{NoncommutingGraph}
\pmrelated{NonAbelianTheories}
\pmrelated{NonAbelianStructures}
\pmrelated{CommutativeVsNonCommutativeDynamicModelingDiagrams}
\pmrelated{GeneralizedToposesTopoiWithManyValuedLogicSubobjectClassifiers}

\endmetadata

% this is the default PlanetMath preamble.  as your knowledge
% of TeX increases, you will probably want to edit this, but
% it should be fine as is for beginners.

% almost certainly you want these
\usepackage{amssymb}
\usepackage{amsmath}
\usepackage{amsfonts}

% used for TeXing text within eps files
%\usepackage{psfrag}
% need this for including graphics (\includegraphics)
\usepackage{graphicx}
% for neatly defining theorems and propositions
\usepackage{amsthm}
% making logically defined graphics
%%%\usepackage{xypic}

% there are many more packages, add them here as you need them

% define commands here
\newcommand{\<}{\langle}
\renewcommand{\>}{\rangle}
\newcommand{\Bigcup}{\bigcup\limits}
\newcommand{\DirectSum}{\bigoplus\limits}
\newcommand{\Prod}{\prod\limits}
\newcommand{\Sum}{\sum\limits}
\newcommand{\h}{\widehat}
\newcommand{\mbb}{\mathbb}
\newcommand{\mbf}{\mathbf}
\newcommand{\mc}{\mathcal}
\newcommand{\mmm}[9]{\left(\begin{array}{rrr}#1&#2&#3\\#4&#5&#6\\#7&#8&#9\end{array}\right)}
\newcommand{\mf}{\mathfrak}
\newcommand{\ol}{\overline}

% Math Operators/functions
\DeclareMathOperator{\Aut}{Aut}
\DeclareMathOperator{\End}{End}
\DeclareMathOperator{\Frob}{Frob}
\DeclareMathOperator{\cwe}{cwe}
\DeclareMathOperator{\id}{id}
\DeclareMathOperator{\mult}{mult}
\DeclareMathOperator{\we}{we}
\DeclareMathOperator{\wt}{wt}
\begin{document}
\PMlinkescapeword{associate}
\PMlinkescapeword{square}
Let $G$ be a non-abelian group with center $Z(G)$.  Associate a graph
$\Gamma_G$ with $G$ whose vertices are the non-central elements
$G\setminus Z(G)$ and whose edges join those vertices $x,y\in
G\setminus Z(G)$ for which $xy\neq yx$. Then $\Gamma_G$ is said to be
the \emph{non-commuting graph} of $G$.
The non-commuting graph $\Gamma_G$ was first considered by Paul
Erd\"os, when he posed the following problem in 1975 [B.H. Neumann,  {\sl A problem of Paul Erd\"os on
groups,}  J. Austral. Math. Soc. Ser. A {\bf 21} (1976), 467-472]:\\
 Let
$G$ be a group whose non-commuting graph has no infinite complete
subgraph. Is it true that there is a finite bound on the
cardinalities of complete subgraphs   of $\Gamma_G$?\\
B. H. Neumann  answered positively Erd\"os' question.\\
The non-commuting graph $\Gamma_G$ of a non-abelian group $G$ is always connected with diameter 2 and girth 3.
It is also Hamiltonian.  $\Gamma_G$ is planar if and only if $G$ is isomphic to the symmetric group $S_3$, or 
the dihedral group $\mc{D}_8$ of order $8$ or the quaternion group $Q_8$ of order $8$.\\
See 
[Alireza Abdollahi, S. Akbari and H.R. Maimani, {\sl Non-commuting graph of a group}, Journal of Algbera, {\bf 298} (2006) 468-492.] for  proofs of these properties of $\Gamma_G$.  


\subsection*{Examples}

\subsubsection*{Symmetric group $S_3$}
The symmetric group $S_3$ is the smallest non-abelian group. In cycle
notation, we have
\begin{equation*}
S_3=\{(),(12),(13),(23),(123),(132)\}.
\end{equation*}
The center is trivial: $Z(S_3)=\{()\}$. The non-commuting graph in
Figure~\ref{fig:s3} contains all possible edges except one.

\begin{figure}
\label{fig:s3}
\begin{center}
\includegraphics{NonCommutingGraphOfAGroup.s3.eps}
\end{center}
\sf\caption{Non-commuting graph of the symmetric group $S_3$}
\end{figure}

\subsubsection*{Octic group}
The dihedral group $\mc{D}_8$, generally known as the octic group, is
the symmetry group of the \PMlinkid{square}{1086}. If you label the vertices of the
square from $1$ to $4$ going along the edges, the octic group may be
seen as a \PMlinkid{subgroup}{1045} of the symmetric group $S_4$:
\begin{equation*}
\mc{D}_8:=\{(),(13),(24),(12)(34),(13)(24),(14)(23),(1234),(1432)\}.
\end{equation*}
So the octic group has \PMlinkid{order}{2871} $8$ (hence its name), and its center
consists of the identity element and the simultaneous flip around both
diagonals $(13)(24)$. Its non-commuting graph is given in
Figure~\ref{fig:d4}.

\begin{figure}
\label{fig:d4}
\begin{center}
\includegraphics{NonCommutingGraphOfAGroup.d4.eps}
\end{center}
\sf\caption{Non-commuting graph of the octic group}
\end{figure}
%%%%%
%%%%%
\end{document}
