\documentclass[12pt]{article}
\usepackage{pmmeta}
\pmcanonicalname{IdealOfElementsWithFiniteOrder}
\pmcreated{2013-03-22 17:52:30}
\pmmodified{2013-03-22 17:52:30}
\pmowner{pahio}{2872}
\pmmodifier{pahio}{2872}
\pmtitle{ideal of elements with finite order}
\pmrecord{8}{40355}
\pmprivacy{1}
\pmauthor{pahio}{2872}
\pmtype{Theorem}
\pmcomment{trigger rebuild}
\pmclassification{msc}{20A05}
\pmclassification{msc}{16D25}
\pmrelated{OrderGroup}
\pmrelated{Lcm}
\pmrelated{Multiple}
\pmrelated{OrdersOfElementsInIntegralDomain}
\pmrelated{CharacteristicOfFiniteRing}

\endmetadata

% this is the default PlanetMath preamble.  as your knowledge
% of TeX increases, you will probably want to edit this, but
% it should be fine as is for beginners.

% almost certainly you want these
\usepackage{amssymb}
\usepackage{amsmath}
\usepackage{amsfonts}

% used for TeXing text within eps files
%\usepackage{psfrag}
% need this for including graphics (\includegraphics)
%\usepackage{graphicx}
% for neatly defining theorems and propositions
 \usepackage{amsthm}
% making logically defined graphics
%%%\usepackage{xypic}

% there are many more packages, add them here as you need them

% define commands here

\theoremstyle{definition}
\newtheorem*{thmplain}{Theorem}
\DeclareMathOperator{\lcm}{lcm}
\begin{document}
\textbf{Theorem.}\, The set of all elements of a ring, which have a finite order in the additive group of the ring, is a (two-sided) ideal of the ring.\\

{\em Proof.}\, Let $S$ be the set of the elements with finite order in the ring $R$.\, Denote by $o(x)$ the order of $x$.\, Take arbitrary elements $a,\,b$ of the set $S$.

If\; $\lcm(o(a),\,o(b)) = n = ko(a) = lo(b)$,\, then
$$n(a-b) = na-nb = ko(a)a-lo(b)b = k\cdot0-l\cdot0 = 0-0 = 0.$$
Thus\, $o(a-b) \leqq n < \infty$\, and so\, $a-b \in S$.

For any element $r$ of $R$ we have
$$o(a)(ra) = \underbrace{ra+ra+\ldots+ra}_{o(a)} = r(\underbrace{a+a+\ldots+a}_{o(a)}) =
r(o(a)a) = r\cdot0 = 0.$$
Therefore, $o(ra) \leqq o(a) < \infty$\, and $ra \in S$.\, Similarly,\, $ar \in S$.

Since $S$ satisfies the conditions for an ideal, the theorem has been proven.
%%%%%
%%%%%
\end{document}
