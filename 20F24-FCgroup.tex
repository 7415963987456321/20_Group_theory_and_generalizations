\documentclass[12pt]{article}
\usepackage{pmmeta}
\pmcanonicalname{FCgroup}
\pmcreated{2013-03-22 14:52:28}
\pmmodified{2013-03-22 14:52:28}
\pmowner{yark}{2760}
\pmmodifier{yark}{2760}
\pmtitle{FC-group}
\pmrecord{22}{36551}
\pmprivacy{1}
\pmauthor{yark}{2760}
\pmtype{Definition}
\pmcomment{trigger rebuild}
\pmclassification{msc}{20F24}
\pmsynonym{FC group}{FCgroup}
%\pmkeywords{group}
%\pmkeywords{conjugacy}
\pmdefines{FC}
\pmdefines{locally normal}
\pmdefines{locally normal group}
\pmdefines{locally finite and normal}
\pmdefines{locally finite and normal group}
\pmdefines{BFC-group}
\pmdefines{BFC group}
\pmdefines{BFC}
\pmdefines{finite-by-abelian}
\pmdefines{finite-by-abelian group}
\pmdefines{centre-by-finite group}
\pmdefines{center-by-finite group}
\pmdefines{central-by-finite group}
\pmdefines{centre-by}

\usepackage{amssymb}
\usepackage{amsmath}
\usepackage{amsthm}

\newtheorem{corollary}{Corollary}
\newtheorem{theorem}{Theorem}

\DeclareMathOperator{\Tor}{Tor}

% The below lines should work as the command
% \renewcommand{\bibname}{References}
% without creating havoc when rendering an entry in
% the page-image mode.
\makeatletter
\@ifundefined{bibname}{}{\renewcommand{\bibname}{References}}
\makeatother
\begin{document}
\PMlinkescapeword{equivalent}
\PMlinkescapeword{fixed}
\PMlinkescapeword{hamiltonian}
\PMlinkescapeword{index}
\PMlinkescapephrase{locally finite}
\PMlinkescapeword{properties}
\PMlinkescapeword{quotient}
\PMlinkescapeword{subgroup}
\PMlinkescapeword{term}
\PMlinkescapeword{theorem}

An \emph{FC-group} is a group in which every element has only finitely many conjugates. Equivalently, a group $G$ is an FC-group if and only if the centralizer $C_G(x)$ is of finite index in $G$ for each $x\in G$.

All finite groups and all abelian groups are obviously FC-groups.
Further examples of FC-groups can be obtained by taking restricted direct products of such groups.

The term {\it FC-group} was introduced by Baer\cite{baer};
the {\it FC} is simply a mnemonic for the definition involving finite conjugacy classes.

\section{Some theorems}

\begin{theorem}
Every \PMlinkname{subgroup}{Subgroup} of an FC-group is an FC-group.
\end{theorem}

\begin{theorem}
Every homomorphic image of an FC-group is an FC-group.
\end{theorem}

\begin{theorem}
Every restricted direct product of FC-groups is an FC-group.
\end{theorem}

\begin{theorem}
Every periodic FC-group is \PMlinkname{locally finite}{LocallyFiniteGroup}.
\end{theorem}

\begin{theorem}
Let $G$ be an FC-group.
The elements of finite order in $G$ form a subgroup,
which will be denoted by $\Tor(G)$.
The subgroup $\Tor(G)$ is a periodic FC-group,
and the \PMlinkname{quotient}{QuotientGroup} $G/\Tor(G)$ is a torsion-free abelian group.
\end{theorem}

\begin{corollary}
Every torsion-free FC-group is abelian.
\end{corollary}

\begin{theorem}
If $G$ is a finitely generated FC-group,
then $G/Z(G)$ and $\Tor(G)$ are both finite.
\end{theorem}

\begin{theorem}
Every FC-group is a subdirect product of a periodic FC-group
and a torsion-free abelian group.
\end{theorem}

From Theorem 4 above it follows that a group $G$ is a periodic FC-group
if and only if every finite subset of $G$ has a finite normal closure.
For this reason, periodic FC-groups are sometimes called \emph{locally normal} (or \emph{locally finite and normal}) groups.

\section*{Stronger properties}

The following two properties are sometimes encountered,
both of which are somewhat stronger than being an FC-group.
For finitely generated groups they are in fact equivalent to being an FC-group,
by Theorem 6 above.

A \emph{BFC-group} is a group $G$ such that every conjugacy class of elements of $G$ has at most $n$ elements, for some fixed integer $n$.
B.~H.~Neumann showed\cite{neumann} that $G$ is a BFC-group if and only if its commutator subgroup $[G,G]$ is finite
(which in turn is easily shown to be equivalent to $G$ being \emph{finite-by-abelian}, that is,
having a finite normal subgroup $N$ such that $G/N$ is abelian).

A \emph{centre-by-finite} (or \emph{central-by-finite}) group
is a group $G$ such that the central quotient $G/Z(G)$ is finite.
A centre-by-finite group is necessarily a BFC-group,
because the centralizer of any element contains the centre.

\begin{thebibliography}{9}
\bibitem{baer}
 R.\ Baer,
 {\it Finiteness properties of groups},
 Duke Math.\ J.\ 15 (1948), 1021--1032.
\bibitem{neumann}
 B.\ H.\ Neumann,
 {\it Groups covered by permutable subsets},
 J.\ London Math.\ Soc. 29 (1954), 236--248.
\end{thebibliography}
%%%%%
%%%%%
\end{document}
