\documentclass[12pt]{article}
\usepackage{pmmeta}
\pmcanonicalname{ExamplesOfPrimitiveGroupsThatAreNotDoublyTransitive}
\pmcreated{2013-03-22 17:22:37}
\pmmodified{2013-03-22 17:22:37}
\pmowner{rm50}{10146}
\pmmodifier{rm50}{10146}
\pmtitle{examples of primitive groups that are not doubly transitive}
\pmrecord{7}{39742}
\pmprivacy{1}
\pmauthor{rm50}{10146}
\pmtype{Example}
\pmcomment{trigger rebuild}
\pmclassification{msc}{20B15}

\endmetadata

% this is the default PlanetMath preamble.  as your knowledge
% of TeX increases, you will probably want to edit this, but
% it should be fine as is for beginners.

% almost certainly you want these
\usepackage{amssymb}
\usepackage{amsmath}
\usepackage{amsfonts}

% used for TeXing text within eps files
%\usepackage{psfrag}
% need this for including graphics (\includegraphics)
%\usepackage{graphicx}
% for neatly defining theorems and propositions
%\usepackage{amsthm}
% making logically defined graphics
%%%\usepackage{xypic}

% there are many more packages, add them here as you need them

% define commands here
\newcommand{\cD}{\mathcal{D}}
\begin{document}
\PMlinkescapeword{c}
\PMlinkescapeword{entire}
The group $\cD_{2n}, n\geq 3$, the dihedral group of order $2n$, is the symmetry group of the regular $n$-gon. (Note that we use the more common notation $\cD_{2n}$ for this group rather than $\cD_n$).

$\cD_{2n}$ is clearly not doubly transitive for $n\geq 4$, since it preserves ``adjacency'' in the vertices. Thus, for example, clearly no element of $\cD_{2n}$ can take $(1,2)$ to $(1,3)$. ($\cD_{2\cdot 3}=\cD_6$, the symmetry group of the triangle, is, however, doubly transitive).

We show that for $p$ prime, $\cD_{2p}$ is primitive. To prove this, we need only verify that any block containing two distinct elements is the entire set of vertices. Number the vertices consecutively  $\{0,\ldots,p-1\}$, and let $r$ be the element of $\cD_{2n}$ that takes each vertex into its successor $\pmod p$. Now, suppose a block contains two distinct elements $a,b$; assume wlog that $b\neq 0$. Iteratively apply $\displaystyle r^{b-a}$ to these elements to get
\begin{center}
\begin{tabular}{c c}
$a$&$b$\\
$b$&$2b-a$\\
$2b-a$&$3b-a$\\
$\ldots$&$\ldots$
\end{tabular}
\end{center}
Since blocks are either equal or disjoint, we see that the block in question contains $a,b$, and $nb-a$ for each $n$. But $a\neq b$, so $nb-a$ runs through all \PMlinkname{residues}{ResidueSystems} $\pmod p$ and thus the block contains each vertex. Thus $D_{2p}$ is primitive.

For nonprime $n$, $\cD_{2n}$ is not primitive. In this case, if $d$ is a divisor of $n$, then the set of vertices that are multiples of $d$ form a block.

%%%%%
%%%%%
\end{document}
