\documentclass[12pt]{article}
\usepackage{pmmeta}
\pmcanonicalname{CenterOfAGroup}
\pmcreated{2013-03-22 12:23:38}
\pmmodified{2013-03-22 12:23:38}
\pmowner{yark}{2760}
\pmmodifier{yark}{2760}
\pmtitle{center of a group}
\pmrecord{20}{32191}
\pmprivacy{1}
\pmauthor{yark}{2760}
\pmtype{Definition}
\pmcomment{trigger rebuild}
\pmclassification{msc}{20A05}
\pmsynonym{center}{CenterOfAGroup}
\pmsynonym{centre}{CenterOfAGroup}
\pmrelated{CenterOfARing}
\pmrelated{Centralizer}
\pmdefines{central quotient}

\endmetadata

\usepackage{amssymb}
\usepackage{amsmath}
\usepackage{amsfonts}

\DeclareMathOperator{\Inn}{Inn}
\begin{document}
\PMlinkescapeword{entire}
\PMlinkescapeword{properties}

The \emph{center} of a group $G$ is the subgroup consisting of those elements that commute with every other element. Formally,
$$\operatorname{Z}(G) = \{x \in G \mid xg = gx\hbox{ for all }g \in G\}.$$

It can be shown that the center has the following properties:
\begin{itemize}
\item It is a normal subgroup (in fact, a characteristic subgroup).
\item It consists of those conjugacy classes containing just one element.
\item The center of an abelian group is the entire group.
\item For every prime $p$, every non-trivial finite \PMlinkname{$p$-group}{PGroup4} has a non-trivial center.
(\PMlinkname{Proof of a stronger version of this theorem.}{ProofOfANontrivialNormalSubgroupOfAFinitePGroupGAndTheCenterOfGHaveNontrivialIntersection})
\end{itemize}

A subgroup of the center of a group $G$
is called a \emph{central subgroup} of $G$.
All central subgroups of $G$ are normal in $G$.

For any group $G$, the \PMlinkname{quotient}{QuotientGroup} $G/\operatorname{Z}(G)$ is called the \emph{central quotient} of $G$,
and is isomorphic to the inner automorphism group $\Inn(G)$.
%%%%%
%%%%%
\end{document}
