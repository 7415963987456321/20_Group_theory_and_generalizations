\documentclass[12pt]{article}
\usepackage{pmmeta}
\pmcanonicalname{FiniteNilpotentGroups}
\pmcreated{2013-03-22 15:46:40}
\pmmodified{2013-03-22 15:46:40}
\pmowner{Algeboy}{12884}
\pmmodifier{Algeboy}{12884}
\pmtitle{finite nilpotent groups}
\pmrecord{18}{37735}
\pmprivacy{1}
\pmauthor{Algeboy}{12884}
\pmtype{Topic}
\pmcomment{trigger rebuild}
\pmclassification{msc}{20D15}
\pmclassification{msc}{20E15}
\pmclassification{msc}{20E34}
%\pmkeywords{Sylow subgroups}
%\pmkeywords{nilpotent}
%\pmkeywords{commutator}
\pmrelated{NilpotentGroup}
\pmrelated{DirectProductAndRestrictedDirectProductOfGroups}
\pmrelated{ClassificationOfFiniteNilpotentGroups}

\endmetadata

\usepackage{latexsym}
\usepackage{amssymb}
\usepackage{amsmath}
\usepackage{amsfonts}
\usepackage{amsthm}

%%\usepackage{xypic}

%-----------------------------------------------------

%       Standard theoremlike environments.

%       Stolen directly from AMSLaTeX sample

%-----------------------------------------------------

%% \theoremstyle{plain} %% This is the default

\newtheorem{thm}{Theorem}

\newtheorem{coro}[thm]{Corollary}

\newtheorem{lem}[thm]{Lemma}

\newtheorem{lemma}[thm]{Lemma}

\newtheorem{prop}[thm]{Proposition}

\newtheorem{conjecture}[thm]{Conjecture}

\newtheorem{conj}[thm]{Conjecture}

\newtheorem{defn}[thm]{Definition}

\newtheorem{remark}[thm]{Remark}

\newtheorem{ex}[thm]{Example}



%\countstyle[equation]{thm}



%--------------------------------------------------

%       Item references.

%--------------------------------------------------


\newcommand{\exref}[1]{Example-\ref{#1}}

\newcommand{\thmref}[1]{Theorem-\ref{#1}}

\newcommand{\defref}[1]{Definition-\ref{#1}}

\newcommand{\eqnref}[1]{(\ref{#1})}

\newcommand{\secref}[1]{Section-\ref{#1}}

\newcommand{\lemref}[1]{Lemma-\ref{#1}}

\newcommand{\propref}[1]{Prop\-o\-si\-tion-\ref{#1}}

\newcommand{\corref}[1]{Cor\-ol\-lary-\ref{#1}}

\newcommand{\figref}[1]{Fig\-ure-\ref{#1}}

\newcommand{\conjref}[1]{Conjecture-\ref{#1}}


% Normal subgroup or equal.

\providecommand{\normaleq}{\unlhd}

% Normal subgroup.

\providecommand{\normal}{\lhd}

\providecommand{\rnormal}{\rhd}
% Divides, does not divide.

\providecommand{\divides}{\mid}

\providecommand{\ndivides}{\nmid}


\providecommand{\union}{\cup}

\providecommand{\bigunion}{\bigcup}

\providecommand{\intersect}{\cap}

\providecommand{\bigintersect}{\bigcap}

\newenvironment{example}{\textbf{Example.}  }{$\Box$}
\begin{document}
The study of finite nilpotent groups mostly centers around the study of $p$-groups.  This is because of the following two theorems.

\begin{thm}
\PMlinkname{Finite}{Finite} $p$-groups are nilpotent.
\end{thm}
\begin{proof}
From the class equation we know the center of a finite $p$-group is non-trivial.
Thus by induction the upper central series of a $p$-group $P$ terminates at $P$.
So $P$ is nilpotent.
\end{proof}


\begin{example}
Infinite $p$-groups may not always be nilpotent.  In the extreme there are counterexamples like the Tarski monsters $T_p$.  These are infinite $p$-groups in which every proper subgroup has order $p$.  Therefore given any two non-trivial elements $x,y$ in which $y\notin \langle x\rangle$ generate $T_p$.  In particular, the only central element is 1 so that the upper central series is trivial and therefore $T_p$ is not nilpotent. 

Indeed, Tarski monsters are not in fact solvable groups which is a weaker property than nilpotent.
\end{example}

\begin{example}
Some infinite $p$-groups are nilpotent.  Indeed, some infinite $p$-groups are even abelian such as $\mathbb{Z}_p^\infty$ -- the countable dimension vector space over the field $\mathbb{Z}_p$ -- and the Pr\"ufer group $\mathbb{Z}_{p^\infty}$ -- the inductive limit of $\mathbb{Z}_{p^n}$.
\end{example}

\begin{thm}\label{thm:nil}
Let $G$ be a finite group.  Then all the following are equivalent.
\begin{enumerate}
\item\label{nil:1} $G$ is nilpotent.
\item\label{nil:2} Every Sylow subgroup of $G$ is normal.
\item\label{nil:3} For every prime $p\big{|}|G|$, there exists a unique Sylow $p$-subgroup of $G$.
\item\label{nil:4} $G$ is the direct product of its Sylow subgroups.
\end{enumerate}
\end{thm}

For the proof recal the following consequence of the Sylow theorems:
\begin{prop}
If $G$ is a finite group and $P$ a Sylow $p$-subgroup of $G$ then
 \[ N_G(N_G(P)) = N_G(P).\]
\end{prop}
(See Subgroups Containing The Normalizers Of Sylow Subgroups Normalize Themselves)


Now we prove Theorem \ref{thm:nil}
\begin{proof}
(\ref{nil:1}) implies (\ref{nil:2}).  Suppose that $G$ is nilpotent and that
$P$ is a Sylow $p$-subgroup of $G$.  Then as $G$ is nilpotent, every subgroup of $G$ is subnormal in $G$, meaning, if $H$ is properly contained in $G$ then $N_G(H)$ properly contains $H$.  Thus $N_G(N_G(P))$ is larger than $N_G(P)$ or $N_G(P)=G$.  However because $P$ is a Sylow $p$-subgroup we know $N_G(P)=N_G(N_G(P))$ so we conclude $N_G(P)=G$.  Therefore every Sylow $p$-subgroup of $G$ is normal in $G$.

(\ref{nil:2}) implies (\ref{nil:3}).  Suppose every Sylow subgroup of $G$ is normal in $G$.  Then by the Sylow theorems we know that for every prime $p$ dividing $|G|$ there is exactly one Sylow $p$-subgroup of $G$ -- as all Sylow $p$-subgroups are conjugate and here by assumption all are also normal.

(\ref{nil:3}) implies (\ref{nil:4}).  Suppose that there is a unique Sylow $p$-subgroup of $G$ for every $p\big{|}|G|$.  Then by the Sylow theorems every Sylow subgroup of $G$ is normal in $G$.  Furthermore, if $P$ and $Q$ are two distinct Sylow subgroups then they they are Sylow subgroups for different primes so that by Lagrange's theorem their intersection is trivial.   Let $P_1,\dots, P_k$ the Sylow subgroups of $G$.  Then as each $P_i$ is normal in $G$ we have
$G=P_1\cdots P_k$ and we have also demonstrated $P_1\cdots P_i\intersect P_{i+1}=1$ for $2\leq i\leq k$ therefore $G$ is the direct product of $P_1,\dots, P_k$.

(\ref{nil:4}) implies (\ref{nil:1}).  Suppose that $G$ is a product of its Sylow subgroups.  Then as every Sylow subgroup is a $p$-group, $G$ is a product of nilpotent groups so $G$ itself is nilpotent.
\end{proof}
%%%%%
%%%%%
\end{document}
