\documentclass[12pt]{article}
\usepackage{pmmeta}
\pmcanonicalname{ExampleOfNonpermutableSubgroup}
\pmcreated{2013-03-22 16:15:56}
\pmmodified{2013-03-22 16:15:56}
\pmowner{Algeboy}{12884}
\pmmodifier{Algeboy}{12884}
\pmtitle{example of non-permutable subgroup}
\pmrecord{10}{38375}
\pmprivacy{1}
\pmauthor{Algeboy}{12884}
\pmtype{Example}
\pmcomment{trigger rebuild}
\pmclassification{msc}{20E07}

\endmetadata

\usepackage{latexsym}
\usepackage{amssymb}
\usepackage{amsmath}
\usepackage{amsfonts}
\usepackage{amsthm}


\newtheorem{claim}{Claim}
%%\usepackage{xypic}

%-----------------------------------------------------

%       Standard theoremlike environments.

%       Stolen directly from AMSLaTeX sample

%-----------------------------------------------------

%% \theoremstyle{plain} %% This is the default

\newtheorem{thm}{Theorem}

\newtheorem{coro}[thm]{Corollary}

\newtheorem{lem}[thm]{Lemma}

\newtheorem{lemma}[thm]{Lemma}

\newtheorem{prop}[thm]{Proposition}

\newtheorem{conjecture}[thm]{Conjecture}

\newtheorem{conj}[thm]{Conjecture}

\newtheorem{defn}[thm]{Definition}

\newtheorem{remark}[thm]{Remark}

\newtheorem{ex}[thm]{Example}



%\countstyle[equation]{thm}



%--------------------------------------------------

%       Item references.

%--------------------------------------------------


\newcommand{\exref}[1]{Example-\ref{#1}}

\newcommand{\thmref}[1]{Theorem-\ref{#1}}

\newcommand{\defref}[1]{Definition-\ref{#1}}

\newcommand{\eqnref}[1]{(\ref{#1})}

\newcommand{\secref}[1]{Section-\ref{#1}}

\newcommand{\lemref}[1]{Lemma-\ref{#1}}

\newcommand{\propref}[1]{Prop\-o\-si\-tion-\ref{#1}}

\newcommand{\corref}[1]{Cor\-ol\-lary-\ref{#1}}

\newcommand{\figref}[1]{Fig\-ure-\ref{#1}}

\newcommand{\conjref}[1]{Conjecture-\ref{#1}}


% Normal subgroup or equal.

\providecommand{\normaleq}{\unlhd}

% Normal subgroup.

\providecommand{\normal}{\lhd}

\providecommand{\rnormal}{\rhd}
% Divides, does not divide.

\providecommand{\divides}{\mid}

\providecommand{\ndivides}{\nmid}


\providecommand{\union}{\cup}

\providecommand{\bigunion}{\bigcup}

\providecommand{\intersect}{\cap}

\providecommand{\bigintersect}{\bigcap}










\begin{document}
\begin{ex}
There are groups (even finitely generated) with subnormal subgroups which are
not permutable.
\end{ex}
\begin{proof}
Let $D_8$ be the dihedral group of order 8.  The classic presentation is
\[D_8=\langle a,b : a^4=b^2=1, bab=a^{-1}\rangle.\]
As the group is nilpotent we know every subgroup is subnormal; however, not every subgroup is permutable.  In particular, observe that for two general subgroups $H$ and $K$ of $D_8$, it may be possible that $HK$ is not a subgroup.  In this situation we find our counterexample.
\[\langle b\rangle\langle ab\rangle=\{1,b,ab,bab\}=\{1,b,ab,a^{-1}\}.\]
Yet
\[\langle ab\rangle\langle b\rangle=\{1,ab,b,abb\}=\{1,b,ab,a\}.\]
More generally, in any dihedral group
\[D_{2n}=\langle a,b : a^n=b^2=1, bab=a^{-1}\rangle,\]
for $n>2$, then 
\[\langle b\rangle\langle ab\rangle\neq \langle ab\rangle\langle b\rangle\]
and both are subnormal whenever $n=2^i$.
\end{proof}

However, we do observe the competing observation that the group generated
by $\langle b\rangle$ and $\langle ab\rangle$ is the same as the group
generated by $\langle ab\rangle$ and $\langle b\rangle$, namely $D_8$.
Indeed in any group with subgroups $H$ and $K$, $\langle H,K\rangle=\langle K,H\rangle$ so the condition of permutability is one which must be tested as complexes (sets $HK$), not as subgroups.  This is a consequence of the following general result:

\begin{claim}
$HK=KH$ if and only if $HK=\langle H,K\rangle$.
\end{claim}
\begin{proof}
We will show that every element in $\langle H,K\rangle$ can be written in
the form $hk$ for some $h\in H$ and $k\in K$.  To see this first note
every element in $\langle H,K\rangle$ is a word over elements in $H$ and in 
$K$.  If the word involves only elements in $H$ or only elements in $K$ then
we are done. Now for induction suppose all words of length $m$ in $\langle H,K\rangle$ can be expressed in the form $hk$ for some $h\in H$ and $k\in K$.
Then given a word of length $m+1$ we have either $h'w$ for $h'\in H$ and $w$
a word of length $m$, in which case we are done, or $k'w$ for some $k'\in K$
and $w$ a word of length $m$.  Then by induction $w=hk$ form some $h\in H$ and
$k\in K$.  Hence $k'w=k'hk$.  Then $k'h\in KH=HK$ so there exists $h'\in H$ and $k''\in K$ such that $k'h=h'k''$.  Thus $k'w=h'k''k=h'(k''k)$ is of the 
desired format.  Hence $HK\subseteq \langle H,K\rangle\subseteq HK$ so 
$HK=\langle H,K\rangle$.

For the converse suppose $HK=\langle H,K\rangle$.  Then $KH\subseteq HK$.
This means for every $k\in K$ and $h\in H$ there exists $h'\in H$ and
$k'\in K$ such that $k^{-1}h^{-1}=h'k'$.  Thus 
\[hk=(k^{-1} h^{-1})^{-1}=(h' k')^{-1}=(k')^{-1}(h')^{-1}\in KH.\]
Thus $HK\subseteq KH$ and $KH=HK$.
\end{proof}

This helps illustrate how permutability is such a useful condition in the study of subgroup lattices (one of Ore's main research interests).  For these are the subgroups whose complexes are also subgroups.  Thus we can relate the order of $\langle H,K\rangle$ to the order of $H$ and $K$ and many other combinatorial relations.

%%%%%
%%%%%
\end{document}
