\documentclass[12pt]{article}
\usepackage{pmmeta}
\pmcanonicalname{AnAssociativeQuasigroupIsAGroup}
\pmcreated{2013-03-22 18:28:50}
\pmmodified{2013-03-22 18:28:50}
\pmowner{CWoo}{3771}
\pmmodifier{CWoo}{3771}
\pmtitle{an associative quasigroup is a group}
\pmrecord{7}{41156}
\pmprivacy{1}
\pmauthor{CWoo}{3771}
\pmtype{Derivation}
\pmcomment{trigger rebuild}
\pmclassification{msc}{20N05}
\pmrelated{Group}

\usepackage{amssymb,amscd}
\usepackage{amsmath}
\usepackage{amsfonts}
\usepackage{mathrsfs}

% used for TeXing text within eps files
%\usepackage{psfrag}
% need this for including graphics (\includegraphics)
%\usepackage{graphicx}
% for neatly defining theorems and propositions
\usepackage{amsthm}
% making logically defined graphics
%%\usepackage{xypic}
\usepackage{pst-plot}

% define commands here
\newcommand*{\abs}[1]{\left\lvert #1\right\rvert}
\newtheorem{prop}{Proposition}
\newtheorem{thm}{Theorem}
\newtheorem{ex}{Example}
\newcommand{\real}{\mathbb{R}}
\newcommand{\pdiff}[2]{\frac{\partial #1}{\partial #2}}
\newcommand{\mpdiff}[3]{\frac{\partial^#1 #2}{\partial #3^#1}}
\begin{document}
\begin{prop} Let $G$ be a set and $\cdot$ a binary operation on $G$.  Write $ab$ for $a\cdot b$.  The following are equivalent:
\begin{enumerate}
\item $(G,\cdot)$ is an associative quasigroup.
\item $(G,\cdot)$ is an associative loop.
\item $(G,\cdot)$ is a group.
\end{enumerate}
\end{prop}
\begin{proof}  We will prove this in the following direction $(1)\Rightarrow (2) \Rightarrow (3) \Rightarrow (1)$.
\begin{description}
\item[$(1)\Rightarrow (2)$.] Let $x\in G$, and $e_1,e_2\in G$ such that $xe_1=x=e_2x$.  So $xe_1^2=xe_1=x$, which shows that $e_1^2=e_1$.  Let $a\in G$ be such that $e_1a=x$.  Then $e_2e_1a=e_2x=x=e_1a$, so that $e_2e_1=e_1=e_1^2$, or $e_2=e_1$.  Set $e=e_1$.  For any $y\in G$, we have $ey=e^2y$, so $y=ey$.  Similarly, $ye=ye^2$ implies $y=ye$.  This shows that $e$ is an identity of $G$.
\item[$(2)\Rightarrow (3)$.]
First note that all of the group axioms are automatically satisfied in $G$ under $\cdot$, except the existence of an (two-sided) inverse element, which we are going to verify presently.  For every $x\in G$, there are unique elements $y$ and $z$ such that $x y=z x=e$.  Then $y = e y = (z x)  y = z (x y)= z e = z$.  This shows that $x$ has a unique two-sided inverse $x^{-1}:=y=z$.  Therefore, $G$ is a group under $\cdot$.
\item[$(3)\Rightarrow (1)$.]
Every group is clearly a quasigroup, and the binary operation is associative.
\end{description}
This completes the proof.
\end{proof}

\textbf{Remark}.  In fact, if $\cdot$ on $G$ is flexible, then every element in $G$ has a unique inverse: for $z (x z) = (z x) z = e z = z = z e$, so by left division (by $z$), we get $x z=e=x y$, and therefore $z=y$, again by left division (by $x$).  However, $G$ may no longer be a group, because associativity may longer hold.
%%%%%
%%%%%
\end{document}
