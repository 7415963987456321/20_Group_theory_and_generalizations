\documentclass[12pt]{article}
\usepackage{pmmeta}
\pmcanonicalname{SubgroupsOfS4}
\pmcreated{2013-03-22 17:40:54}
\pmmodified{2013-03-22 17:40:54}
\pmowner{rm50}{10146}
\pmmodifier{rm50}{10146}
\pmtitle{subgroups of $S_4$}
\pmrecord{13}{40121}
\pmprivacy{1}
\pmauthor{rm50}{10146}
\pmtype{Topic}
\pmcomment{trigger rebuild}
\pmclassification{msc}{20B30}
\pmclassification{msc}{20B35}

% this is the default PlanetMath preamble.  as your knowledge
% of TeX increases, you will probably want to edit this, but
% it should be fine as is for beginners.

% almost certainly you want these
\usepackage{amssymb}
\usepackage{amsmath}
\usepackage{amsfonts}

% used for TeXing text within eps files
%\usepackage{psfrag}
% need this for including graphics (\includegraphics)
%\usepackage{graphicx}
% for neatly defining theorems and propositions
%\usepackage{amsthm}
% making logically defined graphics
%%\usepackage{xypic}

% there are many more packages, add them here as you need them

% define commands here
\newcommand{\subgroup}{\leq}
\newcommand{\Order}[1]{\lvert #1\rvert}
\newcommand{\Ints}{\mathbb{Z}}
\begin{document}
\PMlinkescapeword{addition}
\PMlinkescapeword{summing}
The symmetric group on $4$ letters, $S_4$, has $24$ elements. Listed by cycle type, they are:
\begin{center}
\begin{tabular}{ccl}
Cycle type & Number of elements & elements\\
$1,1,1,1$ & $1$ & $()$\\
$2,1,1$ & $6$ & $(12),\ (13),\ (14),\ (23),\ (24),\ (34)$\\
$3,1$ & $8$ & $(123),\ (132),\ (124),\ (142),\ (134),\ (143),\ (234),\ (243)$\\
$2,2$ & $3$ & $(12)(34),\ (13)(24),\ (14)(23)$\\
$4$ & $6$ & $(1234),\ (1243),\ (1324),\ (1342),\ (1423),\ (1432)$
\end{tabular}
\end{center}
Any subgroup of $S_4$ must be generated by some subset of these elements, and must have \PMlinkname{order}{OrderGroup} dividing $24$, so must be one of $1,2,3,4,6,8$, or $12$.

Think of $S_4$ as acting on the set of ``letters'' $\Omega=\{1,2,3,4\}$ by permuting them. Then each subgroup $G$ of $S_4$ acts either transitively or intransitively. If $G$ is transitive, then by the orbit-stabilizer theorem, since there is only one orbit we have that the order of $G$ is a multiple of $\Order{\Omega}=4$. Thus all the transitive subgroups are of orders $4, 8$, or $12$. If $G$ is intransitive, then $G$ has at least two orbits on $\Omega$. If one orbit is of size $k$ for $1\leq k<4$, then $G$ can naturally be thought of as (isomorphic to) a subgroup of $S_k\times S_{n-k}$. Thus all intransitive subgroups of $S_4$ are isomorphic to subgroups of
\begin{gather*}
S_2\times S_2\cong \Ints/2\Ints\times\Ints/2\Ints\cong V_4\\
S_1\times S_3\cong S_3
\end{gather*}

Looking first at subgroups of order $12$, we note that $A_4$ is one such subgroup (and must be transitive, by the above analysis):
\[A_4=\{e,(12)(34),(13)(24),(14)(23),(123),(132),(124),(142),(134),(143),(234),(243)\}\]
Any other subgroup $G$ of order $12$ must contain at least one element of order $3$, and must also contain an element of order $2$. It is easy to see that if $G$ contains two elements of order three that are not inverses, then $G=A_4$, while if $G$ contains exactly two elements of order three which are inverses, then it contains at least one element with cycle type $2,2$. But any such element together with a $3$-cycle generates $A_4$. Thus $A_4$ is the only subgroup of $S_4$ of order $12$.

We look next at order $8$ subgroups. These subgroups are $2$-Sylow subgroups of $S_4$, so they are all conjugate and thus isomorphic. The number of them is odd and divides $24/8=3$, so is either $1$ or $3$. But $S_4$ has three conjugate subgroups of order $8$ that are all isomorphic to $D_8$, the dihedral group with $8$ elements:
\begin{gather*}
\{e,(1324),\ (1423),\ (12)(34),\ (14)(23),\ (13)(24),\ (12),\ (34)\}\\
\{e,(1234),\ (1432),\ (13)(24),\ (12)(34),\ (14)(23),\ (13),\ (24)\}\\
\{e,(1342),\ (1243),\ (14)(23),\ (13)(24),\ (12)(34),\ (14),\ (23)\}
\end{gather*}
and so these are the only subgroups of order $8$ (which must also be transitive).

All subgroups of order $6$ must be intransitive by the above analysis since $4\nmid 6$, so by the above, a subgroup of order $6$ must be isomorphic to $S_3$ and thus must be the image of an embedding of $S_3$ into $S_4$. $S_3$ is generated by transpositions (as is $S_n$ for any $n$), so we can determine embeddings of $S_3$ into $S_4$ by looking at the image of transpositions. But the images of the three transpositions in $S_3$ are determined by the images of $(12)$ and $(13)$ since $(23)=(12)(13)(12)$. So we may send $(12)$ and $(13)$ to any pair of transpositions in $S_4$ with a common element; there are four such pairs and thus four embeddings. These correspond to four distinct subgroups of $S_4$, all conjugate, and all isomorphic to $S_3$:
\begin{gather*}
\{e,(12),\ (13),\ (23),\ (123),\ (132)\}\\
\{e,(13),\ (14),\ (34),\ (134),\ (143)\}\\
\{e,(23),\ (24),\ (34),\ (234),\ (243)\}\\
\{e,(12),\ (14),\ (24),\ (124),\ (142)\}
\end{gather*}
(The fact that transpositions in $S_3$ must be mapped to transpositions in $S_4$ rather than elements of cycle type $2,2$ is left to the reader).

We shall see that some subgroups of order $4$ are transitive while others are intransitive. A subgroup of order four is clearly isomorphic to either $\Ints/4\Ints$ or to $\Ints/2\Ints\times\Ints/2\Ints$. The only elements of order $4$ are the $4$-cycles, so each $4$-cycle generates a subgroup isomorphic to $\Ints/4\Ints$, which also contains the inverse of the $4$-cycle. Since there are six $4$-cycles, $S_4$ has three cyclic subgroups of order $4$, and each is obviously transitive:
\begin{gather*}
\{e,\ (1234),\ (13)(24),\ (1432)\}\\
\{e,\ (1243),\ (14)(23),\ (1342)\}\\
\{e,\ (1324),\ (12)(34),\ (1423)\}
\end{gather*} 

A subgroup isomorphic to $\Ints/2\Ints\times\Ints/2\Ints$ has, in addition to the identity, three elements $\sigma_1,\sigma_2,\sigma_3$ of order $2$, and thus of cycle types $2,1,1$ or $2,2$. There are several possibilities:
\begin{itemize}
\item All three of the $\sigma_i$ are of cycle type $2,1,1$. Then the product of any two of those is a $3$-cycle or of cycle type $2,2$, which is a contradiction.
\item Two of the $\sigma_i$ are of cycle type $2,2$. Then the third is as well, since the product of any pair of the elements of $S_4$ of cycle type $2,2$ is the third such. In this case, the group is
\[\{e,\ (12)(34),\ (13)(24),\ (14)(23)\}\]
This group acts transitively.
\item One of the $\sigma_i$ is of cycle type $2,2$ and the other two are of cycle type $2,1,1$. In this case, the two $2$-cycles must be disjoint, since otherwise their product is a $3$-cycle, so the group looks like
\[\{e,\ (12),\ (34),\ (12)(34)\}\]
or one of its conjugates (of which there are two). These groups are intransitive, each having two orbits of size $2$.
\end{itemize}

Finally, we have a number of subgroups of order $2$ and $3$ generated by elements of those orders; all of these are intransitive.

Summing up, $S_4$ has the following subgroups up to isomorphism and conjugation:
\begin{center}
\begin{tabular}{ccl}
Order & Conjugates & Group\\
\hline
$12$ & $1$ & $A_4$ (transitive)\\
$8$ & $3$ & $\{e,(1324),(1423),(12)(34),(14)(23),(13)(24),(12),(34)\}\cong D_8$ (transitive)\\
$6$ & $4$ & $\{e,\ (12),\ (13),\ (23),\ (123),\ (132)\}\cong S_3$ (intransitive)\\
$4$ & $3$ & $\{e,\ (1234),\ (13)(24),\ (1432)\}\cong \Ints/4\Ints$ (transitive)\\
$4$ & $1$ & $\{e,\ (12)(34),\ (13)(24),\ (14)(23)\}\cong V_4$ (transitive)\\ 
$4$ & $3$ & $\{e,\ (12),\ (34),\ (12)(34)\}\cong V_4$ (intransitive)\\
$3$ & $4$ & $\{e,\ (123),\ (132)\}$ (intransitive)\\
$2$ & $6$ & $\{e,\ (12)\}$ (intransitive)\\
$2$ & $3$ & $\{e,\ (12)(34)\}$ (intransitive)\\
$1$ & $1$ & $\{e\}$ (intransitive)
\end{tabular}
\end{center}
Of these, the only proper nontrivial normal subgroups of $S_4$ are $A_4$ and the group $\{e,\ (12)(34),\ (13)(24),\ (14)(23)\}\cong V_4$ (see the article on normal subgroups of the symmetric groups).

The subgroup lattice of $S_4$ is thus (listing only one group in each conjugacy class, and taking liberties identifying isomorphic images as subgroups):
\[
\xymatrix @R1pc@C1pc{
&&S_4\ar@{-}[llddd]\ar@{-}[d]\ar@{-}[rrdd] & & & & (24)\\
&&A_4\ar@{-}[ldddd]\ar@{-}[rddd] & & & & (12)\\
&&&& D_8\ar@{-}[ldd]\ar@{-}[dd]\ar@{-}[rdd] & & (8)\\
S_3\ar@{-}[rdd]\ar@{-}[rrddd] & & & & & & (6)\\
&&& \langle(12)(34),(13)(24)\rangle\ar@{-}[rdd] & \langle(12),(34)\rangle\ar@{-}[dd]\ar@{-}[lldd] & \langle(1234)\rangle\ar@{-}[ldd] & (4)\\
&\langle(123)\rangle\ar@{-}[rdd] & & & & & (3)\\
&& \langle(12)\rangle\ar@{-}[d] && \langle(12)(34)\rangle\ar@{-}[lld] & & (2)\\
&&\{e\} & & & & (1) 
}
\]
%%%%%
%%%%%
\end{document}
