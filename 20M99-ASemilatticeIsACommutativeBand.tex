\documentclass[12pt]{article}
\usepackage{pmmeta}
\pmcanonicalname{ASemilatticeIsACommutativeBand}
\pmcreated{2013-03-22 12:57:28}
\pmmodified{2013-03-22 12:57:28}
\pmowner{mclase}{549}
\pmmodifier{mclase}{549}
\pmtitle{a semilattice is a commutative band}
\pmrecord{6}{33320}
\pmprivacy{1}
\pmauthor{mclase}{549}
\pmtype{Proof}
\pmcomment{trigger rebuild}
\pmclassification{msc}{20M99}
\pmclassification{msc}{06A12}
\pmrelated{Lattice}

\endmetadata

% this is the default PlanetMath preamble.  as your knowledge
% of TeX increases, you will probably want to edit this, but
% it should be fine as is for beginners.

% almost certainly you want these
\usepackage{amssymb}
\usepackage{amsmath}
\usepackage{amsfonts}

% used for TeXing text within eps files
%\usepackage{psfrag}
% need this for including graphics (\includegraphics)
%\usepackage{graphicx}
% for neatly defining theorems and propositions
%\usepackage{amsthm}
% making logically defined graphics
%%%\usepackage{xypic}

% there are many more packages, add them here as you need them

% define commands here
\begin{document}
This note explains how a semilattice is the same as a commutative band.

Let $S$ be a semilattice, with partial order $<$ and each pair of elements $x$ and $y$ having a greatest lower bound $x \wedge y$.
Then it is easy to see that the operation $\wedge$ defines a binary operation on $S$ which makes it a commutative semigroup, and that every element is idempotent since $x \wedge x = x$.

Conversely, if $S$ is such a semigroup, define $x \leq y$ iff $x = xy$.  Again, it is easy to see that this defines a partial order on $S$, and that greatest lower bounds exist with respect to this partial order, and that in fact $x \wedge y = xy$.
%%%%%
%%%%%
\end{document}
