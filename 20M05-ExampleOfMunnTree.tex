\documentclass[12pt]{article}
\usepackage{pmmeta}
\pmcanonicalname{ExampleOfMunnTree}
\pmcreated{2013-03-22 16:12:02}
\pmmodified{2013-03-22 16:12:02}
\pmowner{Mazzu}{14365}
\pmmodifier{Mazzu}{14365}
\pmtitle{example of Munn tree}
\pmrecord{20}{38294}
\pmprivacy{1}
\pmauthor{Mazzu}{14365}
\pmtype{Example}
\pmcomment{trigger rebuild}
\pmclassification{msc}{20M05}
\pmclassification{msc}{20M18}

% this is the default PlanetMath preamble. as your knowledge
% of TeX increases, you will probably want to edit this, but
% it should be fine as is for beginners.

% almost certainly you want these
\usepackage{amssymb}
\usepackage{amsmath}
\usepackage{amsfonts}

% used for TeXing text within eps files
%\usepackage{psfrag}
% need this for including graphics (\includegraphics)
%\usepackage{graphicx}
% for neatly defining theorems and propositions
%\usepackage{amsthm}
% making logically defined graphics
%%\usepackage{xypic} 

% there are many more packages, add them here as you need them

\newtheorem{thm}{Theorem}


% define commands here 

\begin{document}
\PMlinkescapeword{inverse}
\PMlinkescapeword{graph}
\PMlinkescapeword{right}
\PMlinkescapeword{theory}
\PMlinkescapeword{argument}
\PMlinkescapeword{opposite}
\PMlinkescapeword{representation}


\PMlinkescapeword{reduce}
\PMlinkescapeword{reduction}
\PMlinkescapeword{reduced}

\newcommand{\redu}{\mathrm{red}}




\newcommand{\prefi}{\mathrm{pref}}


\newcommand{\V}{\mathrm{V}}
\newcommand{\E}{\mathrm{E}}
\newcommand{\schG}{\mathcal{S}\Gamma}

\newcommand{\e}{\mathrm{e}}
\newcommand{\co}{\mathrm{c}}

\newcommand{\cbra}[1]{\left( #1 \right)}
\newcommand{\qbra}[1]{\left[ #1 \right]}
\newcommand{\gbra}[1]{\left\{ #1 \right\}}
\newcommand{\abra}[1]{\left\langle #1 \right\rangle}

\newcommand{\mipres}[2]{\mathrm{Inv}^1\abra{#1 | #2}}
\newcommand{\sipres}[2]{\mathrm{Inv}\abra{#1 | #2}}
\newcommand{\mt}{\mathrm{MT}}

\newcommand{\double}[1]{\cbra{#1\amalg #1^{-1}}}
\newcommand{\doubles}[1]{\cbra{#1\amalg #1^{-1}}^\ast}
\newcommand{\doublep}[1]{\cbra{#1\amalg #1^{-1}}^+}


Let $X = \gbra{a, b, c}$, $w = aaa^{-1}a^{-1}a^{-1}abb^{-1}ab^{-1}bcaa^{-1}cc^{-1}$. The reduced prefix set of $w$ is
$$\redu (\prefi(w)) = \gbra{\varepsilon, a, aa, a^{-1}, b, ab^{-1}, ac, aca, acc}.$$
The Munn tree $\mt(w)$ is the following. 
$$
\xymatrix{
                  & b         & {ab^{-1}}  \ar[d]_{b}    &  \\
{a^{-1}} \ar[r]^{a} &  \varepsilon \ar[u]^{b} \ar[r]^{a} &  a \ar[d]^{c} \ar[r]^{a} &  {aa}\\
 & & {ac} \ar[r]^{c} \ar[d]^{a}& {acc}\\
 & & {aca} &
}
$$


Note that we have drawn only  edges of the form $(v_1,x,v_2)$ (i.e. $\xymatrix{v_1 \ar[r]^{x} &  v_2}$) with $x\in X$, leaving implicit the existence of the opposite edges $(v_2,x^{-1},v_1)$ (i.e. $\xymatrix{v_2 \ar[r]^{x^{-1}} &  v_1}$), as usual in the diagram representation of inverse word graphs.
%%%%%
%%%%%
\end{document}
