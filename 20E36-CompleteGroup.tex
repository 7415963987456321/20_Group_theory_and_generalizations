\documentclass[12pt]{article}
\usepackage{pmmeta}
\pmcanonicalname{CompleteGroup}
\pmcreated{2013-03-22 15:21:46}
\pmmodified{2013-03-22 15:21:46}
\pmowner{CWoo}{3771}
\pmmodifier{CWoo}{3771}
\pmtitle{complete group}
\pmrecord{5}{37186}
\pmprivacy{1}
\pmauthor{CWoo}{3771}
\pmtype{Definition}
\pmcomment{trigger rebuild}
\pmclassification{msc}{20E36}
\pmclassification{msc}{20F28}

\usepackage{amssymb,amscd}
\usepackage{amsmath}
\usepackage{amsfonts}

% used for TeXing text within eps files
%\usepackage{psfrag}
% need this for including graphics (\includegraphics)
%\usepackage{graphicx}
% for neatly defining theorems and propositions
%\usepackage{amsthm}
% making logically defined graphics
%%%\usepackage{xypic}

% define commands here
\begin{document}
A \emph{complete group} is a group $G$ that is
\begin{enumerate}
\item centerless (center $Z(G)$ of $G$ is the trivial group), and
\item any of its automorphism $g\colon G\to G$ is an inner automorphism.
\end{enumerate}
If a group $G$ is complete, then its group of automorphisms,
$\operatorname{Aut}(G)$, is isomorphic to $G$.  Here's a quick
proof.  Define $\phi\colon G\to \operatorname{Aut}(G)$ by
$\phi(g)=g^{\#}$, where $g^{\#}(x)=gxg^{-1}$.  For $g,h\in G$,
$(gh)^{\#}(x)=(gh)x(gh)^{-1}=g(hxh^{-1})g^{-1}=(g^{\#}h^{\#})(x)$,
so $\phi$ is a homomorphism.  It is onto because every $\alpha\in
\operatorname{Aut}(G)$ is inner, (=$g^{\#}$ for some $g\in G$). Finally,
if $g^{\#}(x)=h^{\#}(x)$, then $gxg^{-1}=hxh^{-1}$, which means
$(h^{-1}g)x=x(h^{-1}g)$, for all $x\in G$.  This implies that
$h^{-1}g\in Z(G)=\langle e \rangle$, or $h=g$.  $\phi$ is
one-to-one.
\\\\
It can be shown that all symmetric groups on $n$ letters are
complete groups, except when $n=2$ and $6$.

\begin{thebibliography}{8}
\bibitem{rot} J. Rotman, {\em The Theory of Groups, An Introduction},
Allyn and Bacon, Boston (1965).
\end{thebibliography}
%%%%%
%%%%%
\end{document}
