\documentclass[12pt]{article}
\usepackage{pmmeta}
\pmcanonicalname{ProofOfCauchysTheoremInAbelianCase}
\pmcreated{2013-03-22 14:30:28}
\pmmodified{2013-03-22 14:30:28}
\pmowner{kshum}{5987}
\pmmodifier{kshum}{5987}
\pmtitle{proof of Cauchy's theorem in abelian case}
\pmrecord{7}{36045}
\pmprivacy{1}
\pmauthor{kshum}{5987}
\pmtype{Proof}
\pmcomment{trigger rebuild}
\pmclassification{msc}{20D99}
\pmclassification{msc}{20E07}

% this is the default PlanetMath preamble.  as your knowledge
% of TeX increases, you will probably want to edit this, but
% it should be fine as is for beginners.

% almost certainly you want these
\usepackage{amssymb}
\usepackage{amsmath}
\usepackage{amsfonts}

% used for TeXing text within eps files
%\usepackage{psfrag}
% need this for including graphics (\includegraphics)
%\usepackage{graphicx}
% for neatly defining theorems and propositions
%\usepackage{amsthm}
% making logically defined graphics
%%%\usepackage{xypic}

% there are many more packages, add them here as you need them

% define commands here
\begin{document}
Suppose $G$ is abelian and the order of $G$ is $h$. Let $g_1$, $g_2,\ldots, g_h$ be the elements of $G$, and for $i=1,\ldots, h$, let $a_i$ be the order of $g_i$.

Consider the direct sum
\[
  H = \bigoplus_{i=1}^h \mathbb{Z}/a_i\mathbb{Z}.
\]
The order of $H$ is obviously $a_1a_2\cdots a_h$. We can define a group homomorphism $\theta$ from $H$ to $G$ by
\[
  (x_1,\ldots,x_h) \mapsto g_1^{x_1}\cdots g_h^{x_h}.
\]
$\theta$ is certainly surjective. So $|H| = |G|\cdot|\ker(\theta)|$. Since $p$ is a prime factor of $G$, $p$ divides |H|, and therefore must divide one of the $a_i$'s, say $a_1$. Then $g_1^{a_1/p}$ is an element of order $p$.
%%%%%
%%%%%
\end{document}
