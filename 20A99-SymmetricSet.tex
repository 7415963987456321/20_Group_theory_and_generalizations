\documentclass[12pt]{article}
\usepackage{pmmeta}
\pmcanonicalname{SymmetricSet}
\pmcreated{2013-03-22 13:48:26}
\pmmodified{2013-03-22 13:48:26}
\pmowner{Koro}{127}
\pmmodifier{Koro}{127}
\pmtitle{symmetric set}
\pmrecord{7}{34528}
\pmprivacy{1}
\pmauthor{Koro}{127}
\pmtype{Definition}
\pmcomment{trigger rebuild}
\pmclassification{msc}{20A99}
\pmclassification{msc}{22A05}
\pmclassification{msc}{15-00}
\pmclassification{msc}{46-00}

\endmetadata

% this is the default PlanetMath preamble.  as your knowledge
% of TeX increases, you will probably want to edit this, but
% it should be fine as is for beginners.

% almost certainly you want these
\usepackage{amssymb}
\usepackage{amsmath}
\usepackage{amsfonts}

% used for TeXing text within eps files
%\usepackage{psfrag}
% need this for including graphics (\includegraphics)
%\usepackage{graphicx}
% for neatly defining theorems and propositions
%\usepackage{amsthm}
% making logically defined graphics
%%%\usepackage{xypic}

% there are many more packages, add them here as you need them

% define commands here

\newcommand{\sR}[0]{\mathbb{R}}
\newcommand{\sC}[0]{\mathbb{C}}
\newcommand{\sN}[0]{\mathbb{N}}
\newcommand{\sZ}[0]{\mathbb{Z}}
\begin{document}
{\bf Definition} A subset $A$ of a group $G$ is said to be \emph{symmetric} if $A = A^{-1}$, where $A^{-1} = \{ a^{-1} : a \in A\}$. In other \PMlinkescapetext{words}, $A$ is symmetric if $a^{-1}\in A$ whenever $a\in A$.

If $A$ is a subset of a vector space, then $A$ is said to be \emph{symmetric} if it is symmetric with respect to the additive group structure of the vector space; that is, if $A=\{-a: a\in A\}$ \cite{cristescu}.

\subsubsection{Examples}
\begin{enumerate}
\item In $\sR$, examples of symmetric sets are
intervals of the type $(-k,k)$ with $k>0$, and the sets $\mathbb{Z}$ and $\{-1,1\}$.
\item Any vector subspace in a vector space is a symmetric set.
\item If $A$ is any subset of a group, then $A\cap A^{-1}$ 
and $A\cup A^{-1}$ are symmetric sets. 
\end{enumerate}

\begin{thebibliography}{9}
 \bibitem{cristescu} R. Cristescu, \emph{Topological vector spaces},
 Noordhoff International Publishing, 1977.
 \bibitem{rudin_fap}
 W. Rudin, \emph{Functional Analysis},
 McGraw-Hill Book Company, 1973.
 \end{thebibliography}
%%%%%
%%%%%
\end{document}
