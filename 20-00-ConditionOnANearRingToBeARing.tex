\documentclass[12pt]{article}
\usepackage{pmmeta}
\pmcanonicalname{ConditionOnANearRingToBeARing}
\pmcreated{2013-03-22 17:19:54}
\pmmodified{2013-03-22 17:19:54}
\pmowner{CWoo}{3771}
\pmmodifier{CWoo}{3771}
\pmtitle{condition on a near ring to be a ring}
\pmrecord{14}{39684}
\pmprivacy{1}
\pmauthor{CWoo}{3771}
\pmtype{Theorem}
\pmcomment{trigger rebuild}
\pmclassification{msc}{20-00}
\pmclassification{msc}{16-00}
\pmclassification{msc}{13-00}
\pmrelated{UnitalRing}

\usepackage{amssymb,amscd}
\usepackage{amsmath}
\usepackage{amsfonts}
\usepackage{mathrsfs}

% used for TeXing text within eps files
%\usepackage{psfrag}
% need this for including graphics (\includegraphics)
%\usepackage{graphicx}
% for neatly defining theorems and propositions
\usepackage{amsthm}
% making logically defined graphics
%%\usepackage{xypic}
\usepackage{pst-plot}
\usepackage{psfrag}

% define commands here
\newtheorem{prop}{Proposition}
\newtheorem{thm}{Theorem}
\newtheorem{ex}{Example}
\newcommand{\real}{\mathbb{R}}
\newcommand{\pdiff}[2]{\frac{\partial #1}{\partial #2}}
\newcommand{\mpdiff}[3]{\frac{\partial^#1 #2}{\partial #3^#1}}
\begin{document}
\PMlinkescapeword{near}
\PMlinkescapeword{combination}

Every ring is a near-ring.  The converse is true only when additional conditions are imposed on the near-ring.

\begin{thm} Let $(R,+,\cdot)$ be a near ring with a multiplicative identity $1$ such that the $\cdot$ also left distributes over $+$; that is, $c\cdot (a+b)=c\cdot a+c\cdot b$.  Then $R$ is a ring.\end{thm}

In short, a distributive near-ring with $1$ is a ring.

Before proving this, let us list and prove some general facts about a near ring:
\begin{enumerate}
\item Every near ring has a unique additive identity: if both $0$ and $0'$ are additive identities, then $0=0+0'=0'$.
\item Every element in a near ring has a unique additive inverse.  The additive inverse of $a$ is denoted by $-a$.
\begin{proof} If $b$ and $c$ are additive inverses of $a$, then $b+a=0=a+c$ and $b=b+0=b+(a+c)=(b+a)+c=0+c=c$. \end{proof}
\item $-(-a)=a$, since $a$ is the (unique) additive inverse of $-a$.
\item There is no ambiguity in defining ``subtraction'' $-$ on a near ring $R$ by $a-b:=a+(-b)$.
\item \label{c} $a-b=0$ iff $a=b$, which is just the combination of the above three facts.
\item If a near ring has a multiplicative identity, then it is unique.  The proof is identical to the one given for the first Fact.
\item \label{w} If a near ring has a multiplicative identity $1$, then $(-1)a=-a$.
\begin{proof} $a+(-1)a=1a+(-1)a=(1+(-1))a=0a=0$.  Therefore $(-1)a=-a$ since $a$ has a unique additive inverse. \end{proof}
\end{enumerate}
We are now in the position to prove the theorem.
\begin{proof}  Set $r=a+b$ and $s=b+a$.  Then 
\begin{alignat*}{2}
r-s &=r-(b+a) &\quad \qquad \text{substitution} \\
&=r+(-1)(b+a) &\quad \qquad \text{by Fact \ref{w} above} \\
&= r+((-1)b+(-1)a) &\quad \qquad \text{by left distributivity} \\
&= r+(-b+(-a)) &\quad \qquad \text{by Fact \ref{w} above} \\
&= (a+b)+(-b+(-a)) &\quad \qquad \text{substitution} \\
&= ((a+b)+(-b))+(-a) &\quad \qquad \text{additive associativity} \\
&= (a+(b+(-b))+(-a) &\quad \qquad \text{additive associativity} \\
&= (a+0)+(-a) &\quad \qquad -b \text{ is the additive inverse of } b\\
&= a+(-a) &\quad \qquad 0 \text{ is the additive identity} \\
&= 0 &\quad \qquad \text{same reason as above}
\end{alignat*}
Therefore, $a+b=r=s=b+a$ by Fact \ref{c} above.
\end{proof}
%%%%%
%%%%%
\end{document}
