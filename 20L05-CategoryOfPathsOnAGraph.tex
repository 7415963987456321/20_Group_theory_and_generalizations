\documentclass[12pt]{article}
\usepackage{pmmeta}
\pmcanonicalname{CategoryOfPathsOnAGraph}
\pmcreated{2013-03-22 16:45:54}
\pmmodified{2013-03-22 16:45:54}
\pmowner{rspuzio}{6075}
\pmmodifier{rspuzio}{6075}
\pmtitle{category of paths on a graph}
\pmrecord{18}{38992}
\pmprivacy{1}
\pmauthor{rspuzio}{6075}
\pmtype{Example}
\pmcomment{trigger rebuild}
\pmclassification{msc}{20L05}
\pmclassification{msc}{18B40}
\pmrelated{IndexOfCategories}

\endmetadata

% this is the default PlanetMath preamble.  as your knowledge
% of TeX increases, you will probably want to edit this, but
% it should be fine as is for beginners.

% almost certainly you want these
\usepackage{amssymb}
\usepackage{amsmath}
\usepackage{amsfonts}

% used for TeXing text within eps files
%\usepackage{psfrag}
% need this for including graphics (\includegraphics)
%\usepackage{graphicx}
% for neatly defining theorems and propositions
%\usepackage{amsthm}
% making logically defined graphics
%%%\usepackage{xypic}

% there are many more packages, add them here as you need them

% define commands here

\begin{document}
A nice class of illustrative examples of some notions of category theory 
is provided by categories of paths on a graph.

Let $G$ be an undirected graph.  Denote the set of vertices of $G$ by ``$V$'' 
and denote the set of edges of $G$ by ``$E$''.  

A path of the graph $G$ is an ordered tuplet of vertices $(x_1, x_2, \ldots x_n)$
such that, for all $i$ between $1$ and $n-1$, there exists an edge connecting
$x_i$ and $x_{i+i}$.  As a special case, we allow trivial paths which consist
of a  single vertex --- soon we will see that these in fact play an important 
role as identity elements in our category.

In our category, the vertices of the graph will be the objects and the 
morphisms will be paths; given two of these objects $a$ and $b$, we set 
$\operatorname{Hom}(a,b)$ to be the set of all paths $(x_1, x_2, \ldots x_n)$ 
such that $x_1 = a$ and $x_n = b$.  Given an object $a$, we set $1_a = (a)$, 
the trivial path mentioned above.

To finish specifying our category, we need to specify the composition operation.
This operation will be the concatenation of paths, which is defined as follows:
Given a path $(x_1, x_2, \ldots, x_n) \in \operatorname{Hom}(a,b)$ and a path
$(y_1, y_2, \ldots, y_m) \in \operatorname{Hom}(a,b)$, we set 
\[
a \circ b = (x_1, x_2, \ldots x_n, y_2, \ldots, y_m).\] 
(Remember that $x_n = y_1 = b$.)  To have a bona fide category, we need to 
check that this choice satisfies the defining properties (A1 - A3 in the 
entry \PMlinkid{category}{965}).  This is rather easily verified.  

\textbf{A1:} Given a morphism $(x_1, x_2, \ldots x_n)$, it can only
belong to $\operatorname{Hom}(a,b)$ if $x_1 = a$ and $x_n = b$, hence
$\operatorname{Hom}(a,b) \cup \operatorname{Hom}(c,d) = \emptyset$ unless
$a = c$ and $b = d$.

\textbf{A2:}  Suppose that we have four objects $a,b,c,d$ and three 
morphisms, $(x_1, x_2, \ldots x_n) \in \operatorname{Hom}(a,b)$,
$(y_1, y_2, \ldots y_m) \in \operatorname{Hom}(b,c)$, and 
$(z_1, z_2, \ldots z_k) \in \operatorname{Hom}(c,d)$.  Then,
by the definition of the operation $\circ$ given above,
\begin{align*}
((x_1, x_2, \ldots, x_n) \circ 
(&y_1, y_2, \ldots, y_m)) \circ
(z_1, z_2, \ldots, z_k) \\ &= 
(x_1, x_2, \ldots, x_n, y_2, \ldots, y_m) \circ
(z_1, z_2, \ldots, z_k) \\ &=
(x_1, x_2, \ldots, x_n, y_2, \ldots, y_m, z_2, \ldots, z_k) \\
(x_1, x_2, \ldots, x_n) \circ 
(&(y_1, y_2, \ldots, y_m)  \circ
(z_1, z_2, \ldots, z_k)) \\ &= 
(x_1, x_2, \ldots, x_n) \circ 
(y_1, y_2, \ldots, y_m, z_2, \ldots, z_k) \\ &=
(x_1, x_2, \ldots, x_n, y_2, \ldots, y_m, z_2, \ldots, z_k).
\end{align*}
Since these two quantities are equal, the operation is associative.

\textbf{A3:}  It is easy enough to check that paths with a single
vertex act as identity elements:
\begin{align*}
(x_1) \circ (x_1, x_2, \ldots, x_n) &= (x_1, x_2, \ldots, x_n) \\
(x_1, x_2, \ldots, x_n) \circ (x_n) &= (x_1, x_2, \ldots, x_n)
\end{align*}

It is also possible to consider the equivalence class of paths 
modulo retracing.  To introduce this category, we first define
a binary relation $\approx$ on the class of paths as follows:
Let $A$ and $B$ be any two paths such that the right endpoint 
of $A$ is the same as the left endpoint of $B$, i.e. $A \in 
\operatorname{Hom}(a,b)$ and $B \in \operatorname{Hom}(b,c)$
for some vertices $a,b,c$ of our graph.  Let $d$ be any vertex 
which shares an edge with $d$.  Then we set $A \circ B \approx
A \circ (c, d, c) \circ B$.

Let $\sim$ be the smallest equivalence relations which contains
$\approx$.  We will call this equivalence relation \emph{retracing}.

As defined above, it may not intuitively obvious what this 
equivalence amounts to.  To this end, we may consider a different
description.  Define the \emph{reversal} of a path to be the
path obtained by reversing the order of the vertices traversed:
\[
 (x_1, x_2, \ldots, x_{n-1}, x_n) ^{-1} = 
 (x_n, x_{n-1}, \ldots, x_2, x_1)
\]
Then we may show that two paths are equivalent under retracing
if they may both be obtained from a third path by inserting
terms of the form $XX^{-1}$.  In symbols, we claim that $A \sim B$
if there exists an integer $n>0$ and paths $X_1, \ldots X_{n+1}, 
Y_1, \ldots Y_{n-1}, Z_1, \ldots Z_n$ such that 
\[
 A = X_1 \circ X_1^{-1} \circ Z_1 \circ X_2 \circ X_2^{-1} \circ 
 \cdots \circ X_{n-1} \circ X_{n-1}^{-1} \circ Z_n \circ X_n \circ
 X_n^{-1} \circ Z_n \circ X_{n+1} \circ X_{n+1}^{-1}
\]
and
\[
 B = Y_1 \circ Y_1^{-1} \circ Z_1 \circ Y_2 \circ Y_2^{-1} \circ 
 \cdots \circ Y_{n-1} \circ Y_{n-1}^{-1} \circ Z_n \circ Y_n \circ
 Y_n^{-1} \circ Z_n \circ Y_{n+1} \circ Y_{n+1}^{-1}
\]
This characterization explains the choice of the term ``retracing'' ---
we do not change the equivalence class of the path if we happen to 
wander off somewhere in the course of following the path but then
backtrack and pick the path up again where we left off on our
digression.

Rather than presenting a detailed formal proof, we will sketch how
the two definitions may be shown to be equivalent.
%%%%%
%%%%%
\end{document}
