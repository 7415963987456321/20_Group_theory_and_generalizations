\documentclass[12pt]{article}
\usepackage{pmmeta}
\pmcanonicalname{Semilattice}
\pmcreated{2013-03-22 12:57:23}
\pmmodified{2013-03-22 12:57:23}
\pmowner{mclase}{549}
\pmmodifier{mclase}{549}
\pmtitle{semilattice}
\pmrecord{6}{33317}
\pmprivacy{1}
\pmauthor{mclase}{549}
\pmtype{Definition}
\pmcomment{trigger rebuild}
\pmclassification{msc}{20M99}
\pmclassification{msc}{06A12}
\pmrelated{Lattice}
\pmrelated{Poset}
\pmrelated{Idempotent2}
\pmrelated{Join}
\pmrelated{Meet}
\pmrelated{CompleteSemilattice}
\pmdefines{lower semilattice}
\pmdefines{upper semilattice}

\endmetadata

% this is the default PlanetMath preamble.  as your knowledge
% of TeX increases, you will probably want to edit this, but
% it should be fine as is for beginners.

% almost certainly you want these
\usepackage{amssymb}
\usepackage{amsmath}
\usepackage{amsfonts}

% used for TeXing text within eps files
%\usepackage{psfrag}
% need this for including graphics (\includegraphics)
%\usepackage{graphicx}
% for neatly defining theorems and propositions
%\usepackage{amsthm}
% making logically defined graphics
%%%\usepackage{xypic}

% there are many more packages, add them here as you need them

% define commands here
\begin{document}
A \emph{lower semilattice} is a partially ordered set S in which each pair of elements has a greatest lower bound.

A \emph{upper semilattice} is a partially ordered set S in which each pair of elements has a least upper bound.

Note that it is not normally necessary to distinguish lower from upper semilattices, because one may be converted to the other by reversing the partial order.  It is normal practise to refer to either structure as a \emph{semilattice} and it should be clear from the context whether greatest lower bounds or least upper bounds exist.

Alternatively, a semilattice can be considered to be a commutative band, that is a semigroup which is commutative, and in which every element is idempotent.  In this context, semilattices are important elements of semigroup theory and play a key role in the structure theory of commutative semigroups.

A partially ordered set which is both a lower semilattice and an upper semilattice is a lattice.
%%%%%
%%%%%
\end{document}
