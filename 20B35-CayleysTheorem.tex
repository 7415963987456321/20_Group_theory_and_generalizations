\documentclass[12pt]{article}
\usepackage{pmmeta}
\pmcanonicalname{CayleysTheorem}
\pmcreated{2013-03-22 12:23:13}
\pmmodified{2013-03-22 12:23:13}
\pmowner{vitriol}{148}
\pmmodifier{vitriol}{148}
\pmtitle{Cayley's theorem}
\pmrecord{7}{32174}
\pmprivacy{1}
\pmauthor{vitriol}{148}
\pmtype{Theorem}
\pmcomment{trigger rebuild}
\pmclassification{msc}{20B35}
\pmrelated{CayleysTheoremForSemigroups}

\endmetadata

% this is the default PlanetMath preamble.  as your knowledge
% of TeX increases, you will probably want to edit this, but
% it should be fine as is for beginners.

% almost certainly you want these
\usepackage{amssymb}
\usepackage{amsmath}
\usepackage{amsfonts}

% used for TeXing text within eps files
%\usepackage{psfrag}
% need this for including graphics (\includegraphics)
%\usepackage{graphicx}
% for neatly defining theorems and propositions
%\usepackage{amsthm}
% making logically defined graphics
%%%\usepackage{xypic}

% there are many more packages, add them here as you need them

% define commands here
\begin{document}
Let $G$ be a group, then $G$ is isomorphic to a subgroup of the permutation group $S_{G}$

If $G$ is finite and of order $n$, then $G$ is isomorphic to a subgroup of the permutation group $S_{n}$

Furthermore, suppose $H$ is a proper subgroup of $G$. Let $X = \{Hg | g \in G\}$ be the set of right cosets in $G$. The map $\theta:G \to S_{X}$ given by $\theta(x)(Hg) = Hgx$ is a homomorphism. The kernel is the largest normal subgroup of $H$. We note that $|S_X| = [G : H]!$. Consequently if $|G|$ doesn't divide $[G :  H]!$ then $\theta$ is not an isomorphism so $H$ contains a non-trivial normal subgroup, namely the kernel of $\theta$.
%%%%%
%%%%%
\end{document}
