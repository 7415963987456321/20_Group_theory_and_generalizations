\documentclass[12pt]{article}
\usepackage{pmmeta}
\pmcanonicalname{HomogeneousSpace}
\pmcreated{2013-03-22 13:28:07}
\pmmodified{2013-03-22 13:28:07}
\pmowner{rmilson}{146}
\pmmodifier{rmilson}{146}
\pmtitle{homogeneous space}
\pmrecord{6}{34038}
\pmprivacy{1}
\pmauthor{rmilson}{146}
\pmtype{Definition}
\pmcomment{trigger rebuild}
\pmclassification{msc}{20A05}
\pmdefines{action on cosets}
\pmdefines{isotropy subgroup}

\usepackage{amsmath}
\usepackage{amsfonts}
\usepackage{amssymb}
\newcommand{\reals}{\mathbb{R}}
\newcommand{\natnums}{\mathbb{N}}
\newcommand{\cnums}{\mathbb{C}}
\newcommand{\znums}{\mathbb{Z}}
\newcommand{\lp}{\left(}
\newcommand{\rp}{\right)}
\newcommand{\lb}{\left[}
\newcommand{\rb}{\right]}
\newcommand{\supth}{^{\text{th}}}
\newtheorem{proposition}{Proposition}
\newtheorem{definition}[proposition]{Definition}

\newtheorem{theorem}[proposition]{Theorem}

\newcommand{\Perm}{\operatorname{Perm}}
\newcommand{\hg}{\hat{g}}
\begin{document}
\paragraph{Overview and definition.}
Let $G$ be a group acting transitively on a set $X$.  In other words,
we consider a homomorphism $\phi:G\to\Perm(X),$ where the latter
denotes the group of all bijections of $X$. If we consider $G$ as
being, in some sense, the automorphisms of $X$, the transitivity
assumption means that it is impossible to distinguish a particular
element of $X$ from any another element.  Since the elements of $X$
are indistinguishable, we call $X$ a \emph{homogeneous space}.
Indeed, the concept of a homogeneous space, is logically equivalent to
the concept of a transitive group action.

\paragraph{Action on cosets.}
Let $G$ be a group, $H<G$ a subgroup, and let $G/H$ denote the set of
left cosets, as above.  For every $g\in G$ we consider the mapping
$\psi_H(g):G/H \to G/H$ with action 
$$a H \to ga H,\quad a\in G.$$
\begin{proposition}
The mapping $\psi_H(g)$ is a bijection.  The
corresponding mapping $\psi_H:G\to\Perm(G/H)$ is a group homomorphism,
specifying a transitive group action of $G$ on $G/H$.
\end{proposition}
Thus, $G/H$ has the natural structure of a homogeneous space.  Indeed,
we shall see that every homogeneous space $X$ is isomorphic to $G/H$,
for some subgroup $H$.

N.B. In geometric applications, the want the homogeneous space $X$ to
have some extra structure, like a topology or a differential
structure.  Correspondingly, the group of automorphisms is either a
continuous group or a Lie group.  In order for the quotient space $X$
to have a Hausdorff topology, we need to assume that the subgroup $H$
is closed in $G$.

\paragraph{The isotropy subgroup and the basepoint identification.}
Let $X$ be a homogeneous space. For $x\in X$, the subgroup
$$H_x = \{ h\in G: hx = x \},$$
consisting of all $G$-actions that fix
$x$, is called the isotropy subgroup at the basepoint $x$.  We
identify the space of cosets $G/H_x$ with the homogeneous space by
means of the mapping $\tau_{x}: G/H_x \to X$, defined by
$$\tau_{x}(aH_x) = ax,\quad a\in G.$$
\begin{proposition}
  The above mapping is a well-defined bijection.
\end{proposition}
To show that $\tau_x$ is well defined, let $a,b\in G$ be members of
the same left coset, i.e.  there exists an $h\in H_x$ such that
$b=ah$.  Consequently
$$bx = a(hx) = ax,$$
as desired.  The mapping $\tau_{x}$ is onto because the action of
$G$ on $X$ is 
assumed to be transitive.  To show that $\tau_x$ is one-to-one, consider
two cosets $aH_x, bH_x,\; a,b\in G$ such that 
$ax=bx$.  It follows that $b^{-1}a$ fixes $x$, and hence is an
element of $H_x$. Therefore $aH_x$ and $bH_x$ are the same coset.


\paragraph{The homogeneous space as a quotient.}  Next, let us show
that $\tau_{x}$ is equivariant relative to the action of $G$ on $X$
and the action of $G$ on the quotient $G/H_x$. 
\begin{proposition}
We have that
$$\phi(g)\circ\tau_x = \tau_x\circ\psi_{H_x}(g)$$
for all $g\in G$.
\end{proposition}
To prove this, let $g,a\in G$ be given, and note that
$$\psi_{H_x}(g)(aH_x)=gaH_x.$$
The latter coset corresponds under $\tau_{x}$ to the
point $gax$, as desired.

Finally, let us note that $\tau_x$ identifies the point $x\in X$ with
the coset of the identity element $eH_x$, that is to say, with the
subgroup $H_x$ itself.  For this reason, the point $x$ is often called
the basepoint of the identification $\tau_x: G/H_x \to X$.

\paragraph{The choice of basepoint.}
Next, we consider the effect of the choice of basepoint on the
quotient structure of a homogeneous space.  Let $X$ be a homogeneous
space. 
\begin{proposition}
The set of all isotropy subgroups $\{ H_x : x\in X\}$ forms a
single conjugacy class of subgroups in $G$.    
\end{proposition}
To show this, let $x_0,
x_1\in X$ be given.  By the transitivity of the action we may choose a
$\hg\in G$ such that $x_1 = \hg x_0$.  Hence, for all $h\in G$
satisfying $hx_0 = x_0$, we have
$$(\hg h \hg^{-1}) x_1 = \hg (h ( \hg^{-1} x_1)) = \hg x_0 = x_1.$$
Similarly, for all $h\in H_{x_1}$ we have that $\hg^{-1} h \hg$ fixes $x_0$.
Therefore,
$$\hg (H_{x_0}) \hg^{-1} = H_{x_1};$$
or what is equivalent, for all $x\in X$ and $g\in G$ we have
$$g H_x g^{-1} = H_{gx}.$$

\paragraph{Equivariance.}
Since we can identify a homogeneous space $X$ with $G/H_x$ for every
possible $x\in X$, it stands to reason that there exist equivariant
bijections between the different $G/H_x$.  To  describe these, let
$H_0, H_1<G$ be conjugate subgroups with
$$H_1 = \hg H_0 \hg^{-1}$$
for some fixed $\hg\in G$.  Let us set
$$X=G/H_0,$$
and let $x_0$ denote the identity coset $H_0$, and $x_1$
the coset $\hg H_0$.  What is the subgroup of $G$ that fixes $x_1$?
In other words, what are all the $h\in G$ such that
$$h  \hg H_0 = \hg H_0,$$
or what is equivalent, all $h\in G$ such that
$$\hg^{-1} h \hg \in H_0.$$
The collection of all such $h$ is
precisely the subgroup $H_1$.  Hence, $\tau_{x_1}: G/H_1\to G/H_0$ is
the desired equivariant bijection.  This is a well defined mapping
from the set of $H_1$-cosets to the set of $H_0$-cosets, with action
given by
$$\tau_{x_1}(a H_1)=a \hg H_0,\quad a\in G.$$

Let $\psi_0:
G\to \Perm(G/H_0)$ and $\psi_1:G\to \Perm(G/H_1)$ denote the
corresponding coset $G$-actions.  
\begin{proposition}
  For all $g\in G$ we have that
  $$\tau_{x_1}\circ\psi_1(g)  =  \psi_0(g)\circ \tau_{x_1}.$$
  \end{proposition}
%%%%%
%%%%%
\end{document}
