\documentclass[12pt]{article}
\usepackage{pmmeta}
\pmcanonicalname{HomomorphismsOfSimpleGroups}
\pmcreated{2013-03-22 15:41:59}
\pmmodified{2013-03-22 15:41:59}
\pmowner{rspuzio}{6075}
\pmmodifier{rspuzio}{6075}
\pmtitle{homomorphisms of simple groups}
\pmrecord{4}{37644}
\pmprivacy{1}
\pmauthor{rspuzio}{6075}
\pmtype{Theorem}
\pmcomment{trigger rebuild}
\pmclassification{msc}{20E32}

\endmetadata

% this is the default PlanetMath preamble.  as your knowledge
% of TeX increases, you will probably want to edit this, but
% it should be fine as is for beginners.

% almost certainly you want these
\usepackage{amssymb}
\usepackage{amsmath}
\usepackage{amsfonts}

% used for TeXing text within eps files
%\usepackage{psfrag}
% need this for including graphics (\includegraphics)
%\usepackage{graphicx}
% for neatly defining theorems and propositions
%\usepackage{amsthm}
% making logically defined graphics
%%%\usepackage{xypic}

% there are many more packages, add them here as you need them

% define commands here
\begin{document}
If a group $G$ is simple, and $H$ is an arbitrary group then any
homomorphism of $G$ to $H$ must either map all elements of $G$ to the
identity of $H$ or be one-to-one.

The kernel of a homomorphism must be a normal subgroup.  Since $G$ is
simple, there are only two possibilities: either the kernel is all of
$G$ of it consists of the identity.  In the former case, the
homomorphism will map all elements of $G$ to the identity.  In the
latter case, we note that a group homomorphism is injective iff the kernel
is trivial.

This is important in the context of representation theory.  In that
case, $H$ is a linear group and this result may be restated as saying
that representations of a simple group are either trivial or faithful.
%%%%%
%%%%%
\end{document}
