\documentclass[12pt]{article}
\usepackage{pmmeta}
\pmcanonicalname{PermutationRepresentation}
\pmcreated{2013-03-22 14:53:59}
\pmmodified{2013-03-22 14:53:59}
\pmowner{drini}{3}
\pmmodifier{drini}{3}
\pmtitle{permutation representation}
\pmrecord{6}{36582}
\pmprivacy{1}
\pmauthor{drini}{3}
\pmtype{Definition}
\pmcomment{trigger rebuild}
\pmclassification{msc}{20Cxx}
\pmrelated{MatrixRepresentation}

\usepackage{graphicx}
%%%\usepackage{xypic} 
\usepackage{bbm}
\newcommand{\Z}{\mathbbmss{Z}}
\newcommand{\C}{\mathbbmss{C}}
\newcommand{\R}{\mathbbmss{R}}
\newcommand{\Q}{\mathbbmss{Q}}
\newcommand{\mathbb}[1]{\mathbbmss{#1}}
\newcommand{\figura}[1]{\begin{center}\includegraphics{#1}\end{center}}
\newcommand{\figuraex}[2]{\begin{center}\includegraphics[#2]{#1}\end{center}}
\newtheorem{dfn}{Definition}
\newcommand{\vct}[1]{\mathbf{#1}}
\begin{document}
Let $G$ be a group, and $S$ any finite set on which $G$ acts.

That means that for any $g,h\in G$; $\vct{v},\vct{w}\in S$
\begin{itemize}
\item $g\vct{v}\in V$,
\item $(gh)\vct v = g(h\vct v)$,
\item $e\vct v = \vct v$.
\end{itemize}

Notice that we almost have what it takes to make $S$ a representation of $G$, but $S$ is no vector space. We can however obtain a $G$-module (a vector space carrying a representation of $G$) as follows.

Let $S=\{\vct{s}_1,\vct{s}_2,\ldots,\vct{s}_n\}$. And let $\C S = \C[\vct{s}_1,\vct{s}_2,\ldots,\vct{s}_n]$ be the vector space generated by $S$ over $\C$. in other words, $\C S$ is made of all formal linear combinations
$c_1\vct{s}_1+c_2\vct{s}_2+\cdots+c_n\vct{s}_n$ with $c_j\in\C$.
The sum is defined coordinate-wise as is scalar multiplication.

Then the action of $G$ in $S$ can be extended linearly to $\C S$ as
\[
g(c_1\vct{s}_1+c_2\vct{s}_2+\cdots+c_n\vct{s}_n)=
c_1(g\vct{s}_1)+c_2(g\vct{s}_2)+\cdots+c_n(g\vct{s}_n)
\]
and then the map
$\rho : G\to GL(\C S)$ where $\rho$ is such that
$\rho(g)(\vct v) = g\vct v$ makes $\C S$ into a $G$-module. The $G$-module $\C S$ is known as the \emph{permutation representation} associated with $S$.


\textbf{Example.}\\
If $G=S_n$ acts on $S=\{\vct 1,\vct 2,\ldots,\vct n\}$, then
\[
\C S = \{ c_1 \vct{1} + c_2 \vct{2} + \cdots+ c_n\vct n\}.
\]
If $\sigma \in S_n$, the action becomes
\[
\sigma(c_1 \vct{1} + c_2 \vct{2} + \cdots+ c_n\vct n)
=
c_1\sigma(\vct 1)+c_2\sigma(\vct 2) + \cdots + c_n\sigma(\vct n).
\]
Since $S$ forms a basis for this space, we can compute the matrices corresponding to the defining permutation and we will see that the corresponding permutation matrices are obtained.

\bigskip
\textbf{References.}
Bruce E. Sagan. \emph{The Symmetric Group: Representations, Combinatorial Algorithms and Symmetric Functions.}  2a Ed. 2000. Graduate Texts in Mathematics. Springer.
%%%%%
%%%%%
\end{document}
