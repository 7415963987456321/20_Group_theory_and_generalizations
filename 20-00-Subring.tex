\documentclass[12pt]{article}
\usepackage{pmmeta}
\pmcanonicalname{Subring}
\pmcreated{2013-03-22 12:30:19}
\pmmodified{2013-03-22 12:30:19}
\pmowner{yark}{2760}
\pmmodifier{yark}{2760}
\pmtitle{subring}
\pmrecord{17}{32738}
\pmprivacy{1}
\pmauthor{yark}{2760}
\pmtype{Definition}
\pmcomment{trigger rebuild}
\pmclassification{msc}{20-00}
\pmclassification{msc}{16-00}
\pmclassification{msc}{13-00}
\pmrelated{Ideal}
\pmrelated{Ring}
\pmrelated{Group}
\pmrelated{Subgroup}
\pmdefines{ideal}

\endmetadata

\usepackage{amsfonts}

\newcommand{\Z}{\mathbb{Z}}
\begin{document}
\PMlinkescapeword{closed}
\PMlinkescapeword{equivalent}
\PMlinkescapeword{restricted}
\PMlinkescapeword{properties}
\PMlinkescapeword{property}
\PMlinkescapeword{theory}

Let $(R,+,*)$ a ring. A subring is a subset $S$ of $R$ with the operations $+$ and $*$ of $R$ restricted to $S$ and such that $S$ is a ring by itself.

Notice that the restricted operations inherit the associative and distributive properties of $+$ and $*$, as well as commutativity of $+$.
So for $(S,+,*)$ to be a ring by itself, we need that $(S,+)$ be a subgroup of $(R,+)$ and that $(S,*)$ be closed.
The subgroup condition is equivalent to $S$ being non-empty and having the property that $x-y\in S$ for all $x,y\in S$.

A subring $S$ is called a left ideal if for all $s\in S$ and all $r\in R$ we have $r*s\in S$. Right ideals are defined similarly, with $s*r$ instead of $r*s$.
If $S$ is both a left ideal and a right ideal, then it is called a two-sided ideal. If $R$ is commutative, then all three definitions coincide. In ring theory, ideals are far more important than subrings, as they play a role analogous to normal subgroups in group theory.

Example:

Consider the ring $(\Z,+,\cdot$). Then $(2\Z,+,\cdot)$ is a subring, since the difference and product of two even numbers is again an even number.
%%%%%
%%%%%
\end{document}
