\documentclass[12pt]{article}
\usepackage{pmmeta}
\pmcanonicalname{GroupActionsAndHomomorphisms}
\pmcreated{2013-03-22 13:18:48}
\pmmodified{2013-03-22 13:18:48}
\pmowner{CWoo}{3771}
\pmmodifier{CWoo}{3771}
\pmtitle{group actions and homomorphisms}
\pmrecord{15}{33820}
\pmprivacy{1}
\pmauthor{CWoo}{3771}
\pmtype{Derivation}
\pmcomment{trigger rebuild}
\pmclassification{msc}{20A05}
%\pmkeywords{SymmetricGroup}
%\pmkeywords{GroupHomomorphism}
\pmrelated{GroupHomomorphism}

\endmetadata

% this is the default PlanetMath preamble.  as your knowledge
% of TeX increases, you will probably want to edit this, but
% it should be fine as is for beginners.

% almost certainly you want these
\usepackage{amssymb}
\usepackage{amsmath}
\usepackage{amsfonts}

% used for TeXing text within eps files
%\usepackage{psfrag}
% need this for including graphics (\includegraphics)
%\usepackage{graphicx}
% for neatly defining theorems and propositions
%\usepackage{amsthm}
% making logically defined graphics
%%%\usepackage{xypic}

% there are many more packages, add them here as you need them

% define commands here
\begin{document}
\textbf{Notes on group actions and homomorphisms}

Let $G$ be a group, $X$ a non-empty set and $S_X$ the symmetric group 
of $X$,
i.e. the group of all bijective maps on $X$. $\cdot$ may denote a left 
group
action of $G$ on $X$.
\begin{enumerate}
\item
For each $g \in G$ and $x \in X$ we define
\begin{displaymath}
f_g\colon X \to X, \quad x \mapsto g\cdot x \mbox{.}
\end{displaymath}
Since $f_{g^-1}(f_g(x)) =g^{-1} \cdot (g \cdot x) =x$ for each
$x \in X$,
$f_{g^-1}$ is the inverse of $f_g$. so $f_g$ is bijective and thus element of
$S_X$. We define $F: G \to S_X, F(g) =f_g$ for all $g \in G$. This
mapping is a group homomorphism: Let $g,h \in G, x \in X$. Then
\begin{align*}
F(gh)(x) &=f_{gh}(x) =(gh) \cdot x =g \cdot (h \cdot x) \\
&=(f_g \circ f_h)(x)
=(F(g) \circ F(h))(x)
\end{align*}
for all $x\in X$ implies $F(gh) =F(g) \circ F(h)$. --- The same is obviously true for a right
group action.
\item
Now let $F: G \to S_x$ be a group homomorphism, and let 
$f: G \times X \to X, (g,x) \mapsto F(g)(x)$ satisfy
\begin{enumerate}
\item
$f(1_G, x) =F(1_g)(x) =x$ for all $x \in X$ and
\item
$f(gh, x) =F(gh)(x) =(F(g) \circ F(h)(x) =F(g)(F(h)(x)) =f(g, f(h,x))$,
\end{enumerate}
so $f$ is a group action induced by $F$.
\end{enumerate}

\section*{Characterization of group actions}
Let $G$ be a group acting on a set $X$.
Using the same notation as above, we have for each 
$g \in \operatorname{ker}(F)$
\begin{equation}
F(g) =\operatorname{id}_x =f_g \Leftrightarrow g \cdot x =x, \quad \forall  x \in X \Leftrightarrow
g \in \cup_{x \in X} G_x
\end{equation}
and it follows
\begin{displaymath}
\operatorname{ker}(F) =\bigcap_{x \in X} G_x.
\end{displaymath}
Let $G$ act transitively on $X$. Then for any $x \in X$, $X$ is the 
orbit $G(x)$
of $x$. As shown in ``conjugate stabilizer subgroups', all stabilizer 
subgroups
of elements $y \in G(x)$ are conjugate subgroups to $G_x$ in $G$. From 
the above it follows that
\begin{displaymath}
\operatorname{ker}(F) =\bigcap_{g \in G} gG_xg^{-1}.
\end{displaymath}
For a faithful operation of $G$ the condition $g \cdot x =x, \;
\forall x \in X
\rightarrow g =1_G$ is equivalent to
\begin{displaymath}
\operatorname{ker}(F) =\{1_G\}
\end{displaymath}
and therefore $F\colon G \to S_X$ is a monomorphism.

For the trivial operation of $G$ on $X$ given by $g \cdot x =x, \; \forall g \in G$
the stabilizer subgroup $G_x$ is $G$ for all $x \in X$, and thus
\begin{displaymath}
\operatorname{ker}(F) =G.
\end{displaymath}

If the operation of $G$ on $X$ is free, then
$G_x =\{1_G\}, \; \forall  x \in X$, thus the kernel of $F$ is 
$\{1_G\}$--like for a faithful operation. But:

Let $X=\{1, \ldots, n\}$ and $G=S_n$. Then the operation of $G$ on $X$ 
given by
\begin{displaymath}
\pi \cdot i := \pi(i), \quad \forall i \in X,\; \pi \in S_n
\end{displaymath}
is faithful but not free.
%%%%%
%%%%%
\end{document}
