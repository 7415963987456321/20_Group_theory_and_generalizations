\documentclass[12pt]{article}
\usepackage{pmmeta}
\pmcanonicalname{InverseOfAProduct}
\pmcreated{2015-01-30 21:19:19}
\pmmodified{2015-01-30 21:19:19}
\pmowner{pahio}{2872}
\pmmodifier{pahio}{2872}
\pmtitle{inverse of a product}
\pmrecord{17}{36575}
\pmprivacy{1}
\pmauthor{pahio}{2872}
\pmtype{Theorem}
\pmcomment{trigger rebuild}
\pmclassification{msc}{20A05}
\pmclassification{msc}{20-00}
\pmsynonym{inverse of a product in group}{InverseOfAProduct}
\pmsynonym{inverse of product}{InverseOfAProduct}
%\pmkeywords{inverse}
%\pmkeywords{group operation}
\pmrelated{InverseOfCompositionOfFunctions}
\pmrelated{GeneralAssociativity}
\pmrelated{Division}
\pmrelated{InverseNumber}
\pmrelated{OrderOfProducts}

% this is the default PlanetMath preamble.  as your knowledge
% of TeX increases, you will probably want to edit this, but
% it should be fine as is for beginners.

% almost certainly you want these
\usepackage{amssymb}
\usepackage{amsmath}
\usepackage{amsfonts}

% used for TeXing text within eps files
%\usepackage{psfrag}
% need this for including graphics (\includegraphics)
%\usepackage{graphicx}
% for neatly defining theorems and propositions
 \usepackage{amsthm}
% making logically defined graphics
%%%\usepackage{xypic}

% there are many more packages, add them here as you need them

% define commands here

\theoremstyle{definition}
\newtheorem*{thmplain}{Theorem}
\begin{document}
\begin{thmplain}
 \,If $a$ and $b$ are arbitrary elements of the group \,$(G,*)$, then the inverse of $a*b$ is 
\begin{align}  
      (a*b)^{-1} = b^{-1}*a^{-1}.
\end{align}
\end{thmplain}

{\em Proof.}\, Let the neutral element of the group, which may be proved unique, be\, $e$.\, Using only the group postulates we obtain
   $$(a*b)*(b^{-1}*a^{-1}) = a*(b*(b^{-1}*a^{-1})) = 
a*((b*b^{-1})*a^{-1}) = a*(e*a^{-1}) = a*a^{-1} = e,$$
   $$(b^{-1}*a^{-1})*(a*b) = b^{-1}*(a^{-1}*(a*b)) = b^{-1}*((a^{-1}*a)*b) = 
b^{-1}*(e*b) = b^{-1}*b = e,$$
Q.E.D.

\textbf{Note.}\, The \PMlinkescapetext{formula} (1) may be by induction extended to the form
$$(a_1*\cdots*a_n)^{-1} = a_n^{-1}*\cdots*a_1^{-1}.$$
%%%%%
%%%%%
\end{document}
