\documentclass[12pt]{article}
\usepackage{pmmeta}
\pmcanonicalname{WagnerCongruence}
\pmcreated{2013-03-22 16:11:07}
\pmmodified{2013-03-22 16:11:07}
\pmowner{Mazzu}{14365}
\pmmodifier{Mazzu}{14365}
\pmtitle{Wagner congruence}
\pmrecord{15}{38272}
\pmprivacy{1}
\pmauthor{Mazzu}{14365}
\pmtype{Definition}
\pmcomment{trigger rebuild}
\pmclassification{msc}{20M05}
\pmclassification{msc}{20M18}
%\pmkeywords{Inverse Semigroups}
%\pmkeywords{Free Object}
\pmdefines{Wagner congruence}
\pmdefines{free inverse semigroup}
\pmdefines{free inverse monoid}

\endmetadata

% this is the default PlanetMath preamble. as your knowledge
% of TeX increases, you will probably want to edit this, but
% it should be fine as is for beginners.

% almost certainly you want these
\usepackage{amssymb}
\usepackage{amsmath}
\usepackage{amsfonts}

% used for TeXing text within eps files
%\usepackage{psfrag}
% need this for including graphics (\includegraphics)
%\usepackage{graphicx}
% for neatly defining theorems and propositions
%\usepackage{amsthm}
% making logically defined graphics
%%\usepackage{xypic} 

% there are many more packages, add them here as you need them

% define commands here 
 
 

\begin{document}
\newcommand{\e}{\mathrm{e}}
\newcommand{\co}{\mathrm{c}}

\newcommand{\cbra}[1]{\left( #1 \right)}
\newcommand{\qbra}[1]{\left[ #1 \right]}
\newcommand{\gbra}[1]{\left\{ #1 \right\}}
\newcommand{\abra}[1]{\left\langle #1 \right\rangle}

\newcommand{\mipres}[2]{\mathrm{Inv}^1\abra{#1 | #2}}
\newcommand{\sipres}[2]{\mathrm{Inv}\abra{#1 | #2}}

\newcommand{\double}[1]{\cbra{#1\cup #1^{-1}}}
\newcommand{\doubles}[1]{\cbra{#1\cup #1^{-1}}^\ast}
\newcommand{\doublep}[1]{\cbra{#1\cup #1^{-1}}^+}
\newcommand{\fim}{\mathrm{FIM}}
\newcommand{\fis}{\mathrm{FIS}}

Let $\tilde\rho_X\subseteq\doublep{X}$ be the binary relation on the free semigroup with involution $\doublep X$ defined by $$\tilde\rho_X=\gbra{(ww^{-1}w,w),\ (ww^{-1}vv^{-1},vv^{-1}ww^{-1})\,|\,v,w\in\doublep{X}}.$$
The \emph{Wagner congruence} on $X$ is the congruence $\rho_X$ generated by $\tilde\rho_X$, i.e. $\rho_X=(\tilde\rho_X)^\co$.

A well known result of inverse semigroups theory says that the quotient $$\fis(X)=\doublep{X}/\rho_X$$ is an inverse semigroup. Moreover $\fis(X)$ is the \emph{Free Inverse Semigroup} on $X$, in the sense that it resolve the following universal mapping problem: given an inverse semigroup $S$ and a map $\Phi:X\rightarrow S$, a unique inverse semigroups homomorphism $\overline\Phi:\fis(X)\rightarrow S$ exists such that the following diagram commutes:
$$
\xymatrix{
& X \ar[r]^{\iota} \ar[d]_{\Phi} & \fis(X) \ar[dl]^{\overline{\Phi}} \\
& S & 
}
$$
where $\iota:X\rightarrow\fis(X)$ is the projection to the quotient, i.e. $\iota(x)=[x]_{\rho_X}$. It is well known from universal algebra that $\fis(X)$ is unique up to isomorphisms.

In  analogous way, using the free monoid with involution $\doubles{X}$ instead of the free semigroup with involution $\doublep{X}$, we obtain the inverse monoid $$\fim(X)=\doubles{X}/\rho_X,$$ that is the  \emph{Free Inverse Monoid} on $X$.


\begin{thebibliography}{9}
\bibitem{b:petrich} N. Petrich, \emph{Inverse Semigroups}, Wiley, New York, 1984.
\bibitem{b:wagner} V.V. Wagner, \emph{Generalized Groups}, Dokl. Akad. Nauk SSSR 84 (1952), 1119-1122.
\end{thebibliography}
%%%%%
%%%%%
\end{document}
