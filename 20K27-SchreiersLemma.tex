\documentclass[12pt]{article}
\usepackage{pmmeta}
\pmcanonicalname{SchreiersLemma}
\pmcreated{2013-03-22 15:53:55}
\pmmodified{2013-03-22 15:53:55}
\pmowner{Algeboy}{12884}
\pmmodifier{Algeboy}{12884}
\pmtitle{Schreier's lemma}
\pmrecord{17}{37901}
\pmprivacy{1}
\pmauthor{Algeboy}{12884}
\pmtype{Theorem}
\pmcomment{trigger rebuild}
\pmclassification{msc}{20K27}
\pmsynonym{Schreier-Sims}{SchreiersLemma}
%\pmkeywords{transversal}
%\pmkeywords{generators}
\pmrelated{LiftsTransversals}
\pmrelated{FindingTheOrderOfAGroup}

\endmetadata

\usepackage{latexsym}
\usepackage{amssymb}
\usepackage{amsmath}
\usepackage{amsfonts}
\usepackage{amsthm}

%%\usepackage{xypic}

%-----------------------------------------------------

%       Standard theoremlike environments.

%       Stolen directly from AMSLaTeX sample

%-----------------------------------------------------

%% \theoremstyle{plain} %% This is the default

\newtheorem{thm}{Theorem}

\newtheorem{coro}[thm]{Corollary}

\newtheorem{lem}[thm]{Lemma}

\newtheorem{lemma}[thm]{Lemma}

\newtheorem{prop}[thm]{Proposition}

\newtheorem{conjecture}[thm]{Conjecture}

\newtheorem{conj}[thm]{Conjecture}

\newtheorem{defn}[thm]{Definition}

\newtheorem{remark}[thm]{Remark}

\newtheorem{ex}[thm]{Example}



%\countstyle[equation]{thm}



%--------------------------------------------------

%       Item references.

%--------------------------------------------------


\newcommand{\exref}[1]{Example-\ref{#1}}

\newcommand{\thmref}[1]{Theorem-\ref{#1}}

\newcommand{\defref}[1]{Definition-\ref{#1}}

\newcommand{\eqnref}[1]{(\ref{#1})}

\newcommand{\secref}[1]{Section-\ref{#1}}

\newcommand{\lemref}[1]{Lemma-\ref{#1}}

\newcommand{\propref}[1]{Prop\-o\-si\-tion-\ref{#1}}

\newcommand{\corref}[1]{Cor\-ol\-lary-\ref{#1}}

\newcommand{\figref}[1]{Fig\-ure-\ref{#1}}

\newcommand{\conjref}[1]{Conjecture-\ref{#1}}


% Normal subgroup or equal.

\providecommand{\normaleq}{\unlhd}

% Normal subgroup.

\providecommand{\normal}{\lhd}

\providecommand{\rnormal}{\rhd}
% Divides, does not divide.

\providecommand{\divides}{\mid}

\providecommand{\ndivides}{\nmid}


\providecommand{\union}{\cup}

\providecommand{\bigunion}{\bigcup}

\providecommand{\intersect}{\cap}

\providecommand{\bigintersect}{\bigcap}










\begin{document}
Let $G$ be a group, $H$ a subgroup and $T$ transversal of $H$ in $G$.  For every $g\in G$,
 we denote by $\bar{g}$ the unique element $t\in T$ such that $Hg=Ht$.
We also insist that $1\in T$.  Under these conditions we can state Schreier's lemma.

\begin{lemma}[Schreier]\label{lem:schreier}
Given a group $G=\langle S\rangle$, a subgroup $H$, and transversal
$T$ for $G/H$, then the set 
\[U=H\intersect\{ts(\overline{ts})^{-1}~:~s\in S,t\in T\}\]
generates $H$.
\end{lemma}

\begin{proof}\cite{Seress}
We show that $H$ is generated by $U\union U^{-1}$.  We already know $U$
is contained in $H$ and we know every element $h\in H$ is expressible as
a word over $S$, as $S$ generates $G$.  So $h=s_1\cdots s_k$ where 
$s_i\in S\union S^{-1}$.  Note that $h=1s_1\cdots s_k$ which is of
the form $u_1\cdots u_i t_{i+1}s_{i+1}\cdots s_k$ for $u_j\in U\union U^{-1}$,
$t_{i+1}\in T$ and $s_j\in S\union S^{-1}$, where $i=0$.  We assume for
induction that $h$ has this form for some $i$.  Now we perform the 
sleight-of-hand.
\begin{eqnarray*}
h & = & u_1\cdots u_i t_{i+1}s_{i+1}\cdots s_k\\
	& = & u_1\cdots u_i t_{i+1}s_{i+1}
	(\overline{t_{i+1}s_{i+1}})^{-1}(\overline{t_{i+1}s_{i+1}})s_{i+2}\cdots s_k\\
	& = & u_1\cdots u_i (t_{i+1}s_{i+1}(\overline{t_{i+1}s_{i+1}})^{-1})
		\overline{t_{i+1}s_{i+1}}s_{i+2}\cdots s_k\\
	& = & u_1\cdots u_i u_{i+1}t_{i+2}s_{i+2}\cdots s_k
\end{eqnarray*}
by letting $u_{i+1}=t_{i+1}s_{i+1}(\overline{t_{i+1}s_{i+1}})^{-1}$ which is
in $U\union U^{-1}$, and $t_{i+2}=\overline{t_{i+1}s_{i+1}}\in T$.  At the
end of the iteration we have replaced all the $s_i$'s with elements of
$U\union U^{-1}$ and the last element $t_{k}\in T$.  However, $U$ lies in
$H$ so the product $u_1\cdots u_k\in H$ and as $h\in H$ and $h=u_1\cdots u_kt_k$
we know $t_k\in H$.  As $Ht_k=H$, and $t_k\in T$ we now use the fact
that $1\in T$ to assert that $t_k=1$.  So $h$ is a word in $U\union U^{-1}$
and thus $U$ generates $H$.
\end{proof}


The lemma was discovered in the course of studying free groups as a way to produce generators 
for a subgroup of a free group.  The Reidemeister-Schreier 
theorem strengthened this result to produce a presentation for $H$ given one
for $G$ and a transversal of $G/H$.

The lemma lead to an unexpected use in the 1970's where it was rediscovered by various authors  
to solve certain problems in computational group theory.  Sims used the result to find
generators of the stabilizer of a point in a permutation group.  This method could then be
repeated to find generators for each stabilizer in a stabilizer chain of a base of the group.
A naive approach can result in exponentially many generators, so Sims' algorithm additionally
removes unnecessary generators along the way.  This is now called the Schreier-Sims algorithm.

Independently a version of this algorithm was developed by Frust, Hopcroft, and Luks as a means to 
prove the polynomial-time complexity of various problems in finite group theory and graph 
theory.  Many improvements followed including versions by Knuth, Jerum, and even a probabilistic
nearly linear time method by Seress.

The lemma still finds use in non-finite groups as Schreier had intended.  The associated 
algorithms have been repeatedly improved principally by the work of George Havas and Colin Ramsay.



\bibliographystyle{amsplain}
\providecommand{\bysame}{\leavevmode\hbox to3em{\hrulefill}\thinspace}
\providecommand{\MR}{\relax\ifhmode\unskip\space\fi MR }
% \MRhref is called by the amsart/book/proc definition of \MR.
\providecommand{\MRhref}[2]{%
  \href{http://www.ams.org/mathscinet-getitem?mr=#1}{#2}
}
\providecommand{\href}[2]{#2}
\begin{thebibliography}{10}


\bibitem{Seress}
Seress, {\'A}kos
    \emph{Permutation group algorithms},
    Cambridge Tracts in Mathematics, vol. 152,
    Cambridge University Press, Cambridge, 2003.

\end{thebibliography}

%%%%%
%%%%%
\end{document}
