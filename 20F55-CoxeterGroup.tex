\documentclass[12pt]{article}
\usepackage{pmmeta}
\pmcanonicalname{CoxeterGroup}
\pmcreated{2013-03-22 15:38:20}
\pmmodified{2013-03-22 15:38:20}
\pmowner{Simone}{5904}
\pmmodifier{Simone}{5904}
\pmtitle{Coxeter group}
\pmrecord{12}{37570}
\pmprivacy{1}
\pmauthor{Simone}{5904}
\pmtype{Definition}
\pmcomment{trigger rebuild}
\pmclassification{msc}{20F55}
\pmdefines{Coxeter group}
\pmdefines{Coxeter matrix}
\pmdefines{Cartan matrix}
\pmdefines{Weyl group}
\pmdefines{exchange condition}
\pmdefines{length of Weyl group}

\endmetadata

\usepackage{latexsym}
\usepackage{amssymb}
\usepackage{amsmath}
\usepackage{amsfonts}
\usepackage{amsthm}

%%\usepackage{xypic}

%-----------------------------------------------------

%       Standard theoremlike environments.

%       Stolen directly from AMSLaTeX sample

%-----------------------------------------------------

%% \theoremstyle{plain} %% This is the default

\newtheorem{thm}{Theorem}

\newtheorem{coro}[thm]{Corollary}

\newtheorem{lem}[thm]{Lemma}

\newtheorem{lemma}[thm]{Lemma}

\newtheorem{prop}[thm]{Proposition}

\newtheorem{conjecture}[thm]{Conjecture}

\newtheorem{conj}[thm]{Conjecture}

\newtheorem{defn}[thm]{Definition}

\newtheorem{remark}[thm]{Remark}

\newtheorem{ex}[thm]{Example}



%\countstyle[equation]{thm}



%--------------------------------------------------

%       Item references.

%--------------------------------------------------


\newcommand{\exref}[1]{Example-\ref{#1}}

\newcommand{\thmref}[1]{Theorem-\ref{#1}}

\newcommand{\defref}[1]{Definition-\ref{#1}}

\newcommand{\eqnref}[1]{(\ref{#1})}

\newcommand{\secref}[1]{Section-\ref{#1}}

\newcommand{\lemref}[1]{Lemma-\ref{#1}}

\newcommand{\propref}[1]{Prop\-o\-si\-tion-\ref{#1}}

\newcommand{\corref}[1]{Cor\-ol\-lary-\ref{#1}}

\newcommand{\figref}[1]{Fig\-ure-\ref{#1}}

\newcommand{\conjref}[1]{Conjecture-\ref{#1}}


% Normal subgroup or equal.

\providecommand{\normaleq}{\unlhd}

% Normal subgroup.

\providecommand{\normal}{\lhd}

\providecommand{\rnormal}{\rhd}
% Divides, does not divide.

\providecommand{\divides}{\mid}

\providecommand{\ndivides}{\nmid}


\providecommand{\union}{\cup}

\providecommand{\bigunion}{\bigcup}

\providecommand{\intersect}{\cap}

\providecommand{\bigintersect}{\bigcap}










\begin{document}
A \emph{Coxeter group} $G$ is a finitely generated group, which carries a presentation of the form
$$
W=\langle w_1,\dots,w_n\mid (w_iw_j)^{m_{ij}}=1\rangle 
$$
where the integers $m_{ij}$ satisfy $m_{ii}=1$ for $i=1,\dots,n$ and $m_{ij}=m_{ji}\ge 2$ for $i\ne j$.  The exponents form a matrix 
$[m_{ij}]_{1\leq i,j\leq n}$ often called the \emph{Coxeter matrix}.
This is a cousin of the \emph{Cartan matrix} and both encode the information
of the Dynkin diagrams.  

A Dynkin diagram is the graph with the adjacency matrix given by 
$[m_{ij}-2]_{1\leq i,j\leq n}$ where $[m_{ij}]_{1\leq i,j\leq n}$ is a
Coxeter matrix.

A finite Coxeter group is \emph{irreducible} if it is not the direct product
of smaller coxeter groups.  These groups are classified and labeled
labeled by the Bourbaki types 
\[\mathsf{A}_n, \mathsf{B}_n, \mathsf{C}_n, 
\mathsf{D}_n, \mathsf{E}_6, \mathsf{E}_7, \mathsf{E}_8, \mathsf{F}_4, 
\mathsf{G}_2, \mathsf{H}_2^n, \mathsf{I}_3,\mathsf{I}_4.\]
The classification depends on realizing the groups as
reflections of hyperplanes in a finite dimensional real vector space.
Then observing a condition on an inner product to be integer valued, it
is possible to show these families of symmetry are all that can exist.
The Cartan matrix encodes these integer values of the inner product
of adjacent reflections while the Coxeter matrix encodes the orders of 
adjacent products of generators.  

\begin{remark}
The notation $\mathsf{A}_n$ should not be confused with the natation for
the alternating group on $n$ elements, $A_n$.  This unfortunate overlap is
also a problem with $\mathsf{D}_n$ which is not the same as the dihedral
group on $n$-vertices, $D_n$.
\end{remark}

Alternative methods to study Coxeter groups is through the use of a length
measurement on elements in the group.  As every element in $g$ in a Coxeter
group is the product of the involutions $w_1,\dots,w_n$, the \emph{length} is defined as the shortest word in these $w_i's$ to equal $g$.  We denote this
$\l(g)$.  Then using careful 
analysis and the \emph{exchange condition} it is also possible to specify many of the necessary properties of irreducible Coxeter groups.

Recall that a Weyl group $W$ is a group generated by involutions $S$, that is,
generated by elements of order 2.  The \emph{exchange condition} on a
$W$ with respect to $S$ states that given a reduced word 
$w=w_{i_1}\cdots w_{i_k}$ in $W$, $w_i\in S$, such that for every $s\in S$,
$\l(sw)\leq \l(w)$ then there exists an $j$ such that
\[sw=w_{i_1}\cdots w_{i_{j-1}}w_{i_{j+1}}\cdots w_{i_k}.\]

The insistence that $w_i^2=1$ shows that Coxeter groups are generated by
involutions.  This makes every Coxeter group a Weyl group.  However, 
not every Weyl group is a Coxeter group.  

The remaining condition to make a Weyl group a Coxeter group is the exchange condition.  Thus every finite Weyl group with the exchange condition is a
Coxeter group, and visa-versa.

Coxeter groups arrise as the Weyl groups of Lie algebra, Lie groups, and groups of with a BN-pair.  However many other usese exist.  It should be noted
that the study of Lie theory makes use only of the \emph{crystallographic}
coxeter groups, which are those of type
\[\mathsf{A}_n, \mathsf{B}_n, \mathsf{C}_n, 
\mathsf{D}_n, \mathsf{E}_6, \mathsf{E}_7, \mathsf{E}_8, \mathsf{F}_4, 
\mathsf{G}_2.\]
Thus it omits $\mathsf{H}_2^n$, $\mathsf{I}_3$ and $\mathsf{I}_4$

\section{Coxeter groups as reflections}

Let us see more concretely how a finite Coxeter group can be realized.

Let $V$ be a real Euclidean vector space and $\mathcal O(V)$ the group of all orthogonal transformations of $V$. 

A \emph{reflection} of $V$ is a linear transformation $S$ that carries each vector to its mirror image with respect to a fixed hyperplane $\mathcal P$; it is clear geometrically that a reflection is an orthogonal transformation.

A subgroup $\mathcal G\le\mathcal O(V)$ will be called \emph{effective} if $V_0(\mathcal G)=0$ where $V_0(\mathcal G)=\bigcap_{T\in\mathcal G}\{x\in V\mid Tx=x\}$.

A finite Coxeter group can be realized as (i.e. is always isomorphic to) a finite effective subgroup $\mathcal G$ of $\mathcal O(V)$ that is generated by a set of reflections, for some Euclidean space $V$.

\section{Classification of irreducible finite Coxeter groups}

Type $\mathsf{A}_n$:  This group is isomorphic to the symmetric group on $n$ elements, $S_n$.  The coxeter matrix is encoded by $m_{i,i+1}=3=m_{i+1,i}$ and all other terms are 2.  To observe the isomorphism let
\[w_1=(1,2),(2,3),\dots,w_n=(n-1,n).\]
Then $w_i^2=1$, for instance $(1,2)^2=()$, $(w_i w_j)^2=1$ if $|i-j|>1$, for example $((1,2)(3,4))^2=0$ and $(w_i w_{i+1})^3=1$ as we see with $(1,2)(2,3)=(1,2,3)$ which has order 3.

The Dynkin diagram is:
\[\xymatrix{
\circ\ar@{-}[r] & \circ\ar@{-}[r] & \circ\ar@{--}[r] & \circ .
}\]

Type $\mathsf{B}_n$, $\mathsf{C}_n$: This group is isomorphic to the wreath product $\mathbb{Z}_2\wr S_n$, that is, the semi-direct product of 
$\mathbb{Z}_2^n \rtimes S_n$ where $S_n$ permutes the entries of the vectors in
$\mathbb{Z}_2^n$.

The designation of type $\mathsf{B}_n$ and $\mathsf{C}_n$ relate to the fact that two different methods can be given to construct the same group (as the Weyl group of $O(2n+1,k)$ or as the Weyl group of $Sp(2n,k)$).
It is also common to see $\mathsf{C}_n$ used as the sole label.

Type $\mathsf{H}_2^n$: These groups are the dihedral group $D_{2n}$ for 
$n\geq 5$ and $n\neq 6$.

Type $\mathsf{G}_2$:  This group is isomorphic to $S_3$.

\begin{thebibliography}
{}L. C. Grove, C. T. Benson, \emph{Finite Reflection Groups. Second Edition.}, Springer-Verlag, 1985.
\end{thebibliography}
%%%%%
%%%%%
\end{document}
