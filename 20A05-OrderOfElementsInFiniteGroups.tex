\documentclass[12pt]{article}
\usepackage{pmmeta}
\pmcanonicalname{OrderOfElementsInFiniteGroups}
\pmcreated{2013-03-22 16:34:02}
\pmmodified{2013-03-22 16:34:02}
\pmowner{rm50}{10146}
\pmmodifier{rm50}{10146}
\pmtitle{order of elements in finite groups}
\pmrecord{5}{38757}
\pmprivacy{1}
\pmauthor{rm50}{10146}
\pmtype{Theorem}
\pmcomment{trigger rebuild}
\pmclassification{msc}{20A05}

\endmetadata

% this is the default PlanetMath preamble.  as your knowledge
% of TeX increases, you will probably want to edit this, but
% it should be fine as is for beginners.

% almost certainly you want these
\usepackage{amssymb}
\usepackage{amsmath}
\usepackage{amsfonts}

% used for TeXing text within eps files
%\usepackage{psfrag}
% need this for including graphics (\includegraphics)
%\usepackage{graphicx}
% for neatly defining theorems and propositions
%\usepackage{amsthm}
% making logically defined graphics
%%%\usepackage{xypic}

% there are many more packages, add them here as you need them

% define commands here
\newtheorem{thm}{Theorem}
\newtheorem{cor}[thm]{Corollary}
\newtheorem{lem}[thm]{Lemma}
\newtheorem{prop}[thm]{Proposition}
\newtheorem{ax}{Axiom}

\begin{document}
This article proves two elementary results regarding the orders of group elements in finite groups.

\begin{thm} Let $G$ be a finite group, and let $a\in G$ and $b\in G$ be elements of $G$ that commute with each other. Let $m=\lvert a\rvert$, $n=\lvert b\rvert$. If $\gcd(m,n)=1$, then $mn=\lvert ab\rvert$.
\end{thm}
\textbf{Proof. }
Note first that
\[(ab)^{mn}=a^{mn}b^{mn}=(a^m)^n(b^n)^m=e_G\]
since $a$ and $b$ commute with each other. Thus $\lvert ab\rvert \leq mn$. Now suppose $\lvert ab\rvert=k$. Then
\[e_G=(ab)^k=(ab)^{km}=a^{km}b^{km}=b^{km}\]
and thus $n|km$. But $\gcd(m,n)=1$, so $n|k$. Similarly, $m|k$ and thus $mn|k=\lvert ab\rvert$. These two results together imply that $mn=k$.

\begin{thm} Let $G$ be a finite abelian group. If $G$ contains elements of orders $m$ and $n$, then it contains an element of order $\mathrm{lcm}(m,n)$. 
\end{thm}
\textbf{Proof. } 
Choose $a$ and $b$ of orders $m$ and $n$ respectively, and write
\[\mathrm{lcm}(m,n)=\prod p_i^{k_i}\]
where the $p_i$ are distinct primes. Thus for each $i$, either $p_i^{k_i}\mid m$ or $p_i^{k_i}\mid n$. Thus either $a^{m/p_i^{k_i}}$ or $b^{n/p_i^{k_i}}$ has order $p_i^{k_i}$. Let this element be $c_i$. Now, the orders of the $c_i$ are pairwise coprime by construction, so 
\[\left\lvert \prod c_i\right\rvert=\prod \left\lvert c_i\right\rvert=\mathrm{lcm}(m,n)\]
and thus $\prod c_i$ is the required element.

%%%%%
%%%%%
\end{document}
