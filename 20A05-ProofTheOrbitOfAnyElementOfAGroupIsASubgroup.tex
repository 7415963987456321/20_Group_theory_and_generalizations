\documentclass[12pt]{article}
\usepackage{pmmeta}
\pmcanonicalname{ProofTheOrbitOfAnyElementOfAGroupIsASubgroup}
\pmcreated{2013-03-22 13:30:58}
\pmmodified{2013-03-22 13:30:58}
\pmowner{drini}{3}
\pmmodifier{drini}{3}
\pmtitle{Proof: The orbit of any element of a group is a subgroup}
\pmrecord{6}{34102}
\pmprivacy{1}
\pmauthor{drini}{3}
\pmtype{Proof}
\pmcomment{trigger rebuild}
\pmclassification{msc}{20A05}
%\pmkeywords{group}
%\pmkeywords{subgroup}
\pmrelated{Group}
\pmrelated{Subgroup}
\pmrelated{ProofThatEveryGroupOfPrimeOrderIsCyclic}
\pmrelated{ProofOfTheConverseOfLagrangesTheoremForCyclicGroups}
\pmdefines{orbit}

\endmetadata

\usepackage{graphicx}
%%%\usepackage{xypic} 
\usepackage{bbm}
\newcommand{\Z}{\mathbbmss{Z}}
\newcommand{\C}{\mathbbmss{C}}
\newcommand{\R}{\mathbbmss{R}}
\newcommand{\Q}{\mathbbmss{Q}}
\newcommand{\mathbb}[1]{\mathbbmss{#1}}
\newcommand{\figura}[1]{\begin{center}\includegraphics{#1}\end{center}}
\newcommand{\figuraex}[2]{\begin{center}\includegraphics[#2]{#1}\end{center}}
\begin{document}
Following is a proof that, if $G$ is a group and $g \in G$, then $\langle g \rangle \le G$. Here $\langle g \rangle$ is the orbit of $g$ and is defined as
$$\langle g \rangle=\{g^n : n\in\Z\}$$

Since $g \in \langle g \rangle$, then $\langle g \rangle$ is nonempty.

Let $a,b \in \langle g \rangle$.  Then there exist $x,y \in {\mathbb Z}$ such that $a=g^x$ and $b=g^y$.  Since $ab^{-1}=g^x(g^y)^{-1}=g^xg^{-y}=g^{x-y} \in \langle g \rangle$, it follows that $\langle g \rangle \le G$.
%%%%%
%%%%%
\end{document}
