\documentclass[12pt]{article}
\usepackage{pmmeta}
\pmcanonicalname{SimpleSemigroup}
\pmcreated{2013-03-22 13:05:59}
\pmmodified{2013-03-22 13:05:59}
\pmowner{mclase}{549}
\pmmodifier{mclase}{549}
\pmtitle{simple semigroup}
\pmrecord{7}{33521}
\pmprivacy{1}
\pmauthor{mclase}{549}
\pmtype{Definition}
\pmcomment{trigger rebuild}
\pmclassification{msc}{20M10}
\pmdefines{simple}
\pmdefines{zero simple}
\pmdefines{right simple}
\pmdefines{left simple}

\endmetadata

% this is the default PlanetMath preamble.  as your knowledge
% of TeX increases, you will probably want to edit this, but
% it should be fine as is for beginners.

% almost certainly you want these
\usepackage{amssymb}
\usepackage{amsmath}
\usepackage{amsfonts}

% used for TeXing text within eps files
%\usepackage{psfrag}
% need this for including graphics (\includegraphics)
%\usepackage{graphicx}
% for neatly defining theorems and propositions
%\usepackage{amsthm}
% making logically defined graphics
%%%\usepackage{xypic}

% there are many more packages, add them here as you need them

% define commands here
\begin{document}
\PMlinkescapeword{structure}
\PMlinkescapeword{theory}
Let $S$ be a semigroup.  If $S$ has no ideals other than itself, then $S$ is said to be \emph{simple}.

If $S$ has no left ideals [resp. right ideals] other than itself, then $S$ is said to be \emph{left simple} [resp. \emph{right simple}].

Right simple and left simple are stronger conditions than simple.

A semigroup $S$ is left simple if and only if $Sa = S$ for all $a \in S$.
A semigroup is both left and right simple if and only if it is a group.

If $S$ has a zero element $\theta$, then $0 = \{ \theta \}$ is always an ideal of $S$, so $S$ is not simple (unless it has only one element).  So in studying semigroups with a zero, a slightly weaker definition is required.

Let $S$ be a semigroup with a zero.  Then $S$ is \emph{zero simple}, or $0$-simple, if the following conditions hold:
\begin{itemize}
\item $S^2 \neq 0$
\item $S$ has no ideals except $0$ and $S$ itself
\end{itemize}

The condition $S^2 = 0$ really only eliminates one semigroup: the 2-element null semigroup.  Excluding this semigroup makes parts of the structure theory of semigroups cleaner.
%%%%%
%%%%%
\end{document}
