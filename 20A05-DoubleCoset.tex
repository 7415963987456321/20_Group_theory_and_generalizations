\documentclass[12pt]{article}
\usepackage{pmmeta}
\pmcanonicalname{DoubleCoset}
\pmcreated{2013-03-22 16:17:28}
\pmmodified{2013-03-22 16:17:28}
\pmowner{yark}{2760}
\pmmodifier{yark}{2760}
\pmtitle{double coset}
\pmrecord{8}{38408}
\pmprivacy{1}
\pmauthor{yark}{2760}
\pmtype{Definition}
\pmcomment{trigger rebuild}
\pmclassification{msc}{20A05}

\def\genby#1{\langle#1\rangle}
\begin{document}
\PMlinkescapeword{coset}
\PMlinkescapeword{cosets}
\PMlinkescapeword{obvious}

Let $H$ and $K$ be subgroups of a group $G$.
An \emph{$(H,K)$-double coset} is a set of the form $HxK$ for some $x\in G$.
Here $HxK$ is defined in the obvious way as
\[
  HxK = \{ hxk \mid h\in H \hbox{ and } k\in K \}.
\]

Note that the $(H,\{1\})$-double cosets are just the right cosets of $H$,
and the $(\{1\},K)$-double cosets are just the left cosets of $K$.
In general, every $(H,K)$-double coset is a union of right cosets of $H$,
and also a union of left cosets of $K$.

The set of all $(H,K)$-double cosets is denoted $H\backslash G/K$.
It is straightforward to show that $H\backslash G/K$ is a \PMlinkname{partition}{Partition} of $G$,
that is, every element of $G$ lies in exactly one $(H,K)$-double coset.

In contrast to the situation with ordinary \PMlinkname{cosets}{Coset},
the $(H,K)$-double cosets need not all be of the same cardinality.
For example, if $G$ is the \PMlinkname{symmetric group}{SymmetricGroup} $S_3$,
and $H=\genby{(1,2)}$ and $K=\genby{(1,3)}$,
then the two $(H,K)$-double cosets
are $\{e,(1,2),(1,3),(1,3,2)\}$ and $\{(2,3),(1,2,3)\}$.

%%%%%
%%%%%
\end{document}
