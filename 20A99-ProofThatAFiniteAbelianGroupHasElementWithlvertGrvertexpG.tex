\documentclass[12pt]{article}
\usepackage{pmmeta}
\pmcanonicalname{ProofThatAFiniteAbelianGroupHasElementWithlvertGrvertexpG}
\pmcreated{2013-03-22 16:34:05}
\pmmodified{2013-03-22 16:34:05}
\pmowner{rm50}{10146}
\pmmodifier{rm50}{10146}
\pmtitle{proof that a finite abelian group has element with $\lvert g\rvert=\exp(G)$}
\pmrecord{4}{38758}
\pmprivacy{1}
\pmauthor{rm50}{10146}
\pmtype{Proof}
\pmcomment{trigger rebuild}
\pmclassification{msc}{20A99}

% this is the default PlanetMath preamble.  as your knowledge
% of TeX increases, you will probably want to edit this, but
% it should be fine as is for beginners.

% almost certainly you want these
\usepackage{amssymb}
\usepackage{amsmath}
\usepackage{amsfonts}

% used for TeXing text within eps files
%\usepackage{psfrag}
% need this for including graphics (\includegraphics)
%\usepackage{graphicx}
% for neatly defining theorems and propositions
%\usepackage{amsthm}
% making logically defined graphics
%%%\usepackage{xypic}

% there are many more packages, add them here as you need them

% define commands here
\newtheorem{thm}{Theorem}
\newtheorem{cor}[thm]{Corollary}
\newtheorem{lem}[thm]{Lemma}
\newtheorem{prop}[thm]{Proposition}
\newtheorem{ax}{Axiom}

\begin{document}
\begin{thm} If $G$ is a finite abelian group, then $G$ has an element of order $\exp(G)$.
\end{thm}
\textbf{Proof. } Write $\exp(G)=\prod p_i^{k_i}$. Since $\exp(G)$ is the least common multiple of the orders of each group element, it follows that for each $i$, there is an element whose order is a multiple of $p_i^{k_i}$, say $\lvert c_i\rvert=a_i p_i^{k_i}$. Let $d_i=c_i^{a_i}$. Then $\lvert d_i\rvert=p_i^{k_i}$. The $d_i$ thus have pairwise relatively prime orders, and thus
\[\left\lvert\prod d_i\right\rvert=\prod\left\lvert d_i\right\rvert=\exp(G)\]
so that $\prod d_i$ is the desired element.

%%%%%
%%%%%
\end{document}
