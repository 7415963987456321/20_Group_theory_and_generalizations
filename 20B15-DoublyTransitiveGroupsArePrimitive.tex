\documentclass[12pt]{article}
\usepackage{pmmeta}
\pmcanonicalname{DoublyTransitiveGroupsArePrimitive}
\pmcreated{2013-03-22 17:21:50}
\pmmodified{2013-03-22 17:21:50}
\pmowner{rm50}{10146}
\pmmodifier{rm50}{10146}
\pmtitle{doubly transitive groups are primitive}
\pmrecord{7}{39724}
\pmprivacy{1}
\pmauthor{rm50}{10146}
\pmtype{Theorem}
\pmcomment{trigger rebuild}
\pmclassification{msc}{20B15}

% this is the default PlanetMath preamble.  as your knowledge
% of TeX increases, you will probably want to edit this, but
% it should be fine as is for beginners.

% almost certainly you want these
\usepackage{amssymb}
\usepackage{amsmath}
\usepackage{amsfonts}

% used for TeXing text within eps files
%\usepackage{psfrag}
% need this for including graphics (\includegraphics)
%\usepackage{graphicx}
% for neatly defining theorems and propositions
\usepackage{amsthm}
% making logically defined graphics
%%%\usepackage{xypic}

% there are many more packages, add them here as you need them

% define commands here
\newtheorem*{thm}{Theorem}
\begin{document}
\begin{thm} Every doubly transitive group is \PMlinkname{primitive}{PrimativeTransitivePermutationGroupOnASet}.
\end{thm}
\begin{proof}
Let $G$ acting on $X$ be doubly transitive. To show the action is \PMlinkescapetext{primitive}, we must show that all blocks are trivial blocks; to do this, it suffices to show that any block containing more than one element is all of $X$. So choose a block $Y$ with two distinct elements $y_1, y_2$. Given an arbitrary $x\in X$, since $G$ is doubly transitive, we can choose $\sigma\in G$ such that
\[\sigma\cdot(y_1,y_2)=(y_1,x)\]
But then $\sigma\cdot Y\cap Y\neq\emptyset$, since $y_1$ is in both. Thus $\sigma\cdot Y=Y$, so $x\in Y$ as well. So $Y=X$ and we are done.
\end{proof}
%%%%%
%%%%%
\end{document}
