\documentclass[12pt]{article}
\usepackage{pmmeta}
\pmcanonicalname{ProofOfTheBurnsideBasisTheorem}
\pmcreated{2013-03-22 15:46:25}
\pmmodified{2013-03-22 15:46:25}
\pmowner{Algeboy}{12884}
\pmmodifier{Algeboy}{12884}
\pmtitle{proof of the Burnside basis theorem}
\pmrecord{12}{37731}
\pmprivacy{1}
\pmauthor{Algeboy}{12884}
\pmtype{Proof}
\pmcomment{trigger rebuild}
\pmclassification{msc}{20D15}

\endmetadata

% this is the default PlanetMath preamble.  as your knowledge
% of TeX increases, you will probably want to edit this, but
% it should be fine as is for beginners.

% almost certainly you want these
\usepackage{amssymb}
\usepackage{amsmath}
\usepackage{amsfonts}

% used for TeXing text within eps files
%\usepackage{psfrag}
% need this for including graphics (\includegraphics)
%\usepackage{graphicx}
% for neatly defining theorems and propositions
%\usepackage{amsthm}
% making logically defined graphics
%%\usepackage{xypic}

% there are many more packages, add them here as you need them

% define commands here
\begin{document}
Let $P$ be a $p$-group and $\Phi(P)$ its Frattini subgroup.

Every maximal subgroup $Q$ of $P$ is of index $p$ in $P$ and is therefore
normal in $P$.  Thus $P/Q\cong \mathbb{Z}_p$.  So given 
$g\in P$, $g^p\in Q$
which proves $P^p\leq Q$.  Likewise, $\mathbb{Z}_p$ is abelian so 
$[P,P]\leq Q$.  As $Q$ is any maximal subgroup, it follows $[P,P]$ and
$P^p$ lie in $\Phi(P)$.

Now both $[P,P]$ and $P^p$ are characteristic subgroups of $P$ so in particular
$F =[P,P]P^p$ is normal in $P$.  If we pass to $V=P/F$ we find that $V$ is abelian and every element has order $p$ -- that is, $V$ is a vector space over $\mathbb{Z}_p$.  So the maximal subgroups of $P$ are in a 1-1 correspondence with the hyperplanes of $V$.  As the intersection of all hyperplanes of a vector space is the origin, it follows the intersection of all maximal subgroups of $P$ is $F$.  That is, $[P,P]P^p=\Phi(P)$.

%%%%%
%%%%%
\end{document}
