\documentclass[12pt]{article}
\usepackage{pmmeta}
\pmcanonicalname{SemigroupOfTransformations}
\pmcreated{2013-03-22 13:07:36}
\pmmodified{2013-03-22 13:07:36}
\pmowner{mclase}{549}
\pmmodifier{mclase}{549}
\pmtitle{semigroup of transformations}
\pmrecord{6}{33561}
\pmprivacy{1}
\pmauthor{mclase}{549}
\pmtype{Definition}
\pmcomment{trigger rebuild}
\pmclassification{msc}{20M20}
\pmsynonym{transformation semigroup}{SemigroupOfTransformations}
\pmdefines{full transformation semigroup}

\endmetadata

% this is the default PlanetMath preamble.  as your knowledge
% of TeX increases, you will probably want to edit this, but
% it should be fine as is for beginners.

% almost certainly you want these
\usepackage{amssymb}
\usepackage{amsmath}
\usepackage{amsfonts}

% used for TeXing text within eps files
%\usepackage{psfrag}
% need this for including graphics (\includegraphics)
%\usepackage{graphicx}
% for neatly defining theorems and propositions
%\usepackage{amsthm}
% making logically defined graphics
%%%\usepackage{xypic}

% there are many more packages, add them here as you need them

% define commands here
\begin{document}
Let $X$ be a set.  A transformation of $X$ is a function from $X$ to $X$.

If $\alpha$ and $\beta$ are transformations on $X$, then their product $\alpha \beta$ is defined (writing functions on the right) by $(x)(\alpha \beta) = ((x) \alpha)\beta$.

With this definition, the set of all transformations on $X$ becomes a semigroup, the \emph{full semigroupf of transformations} on $X$, denoted $\mathcal{T}_X$.

More generally, a \emph{semigroup of transformations} is any subsemigroup of a full set of transformations.

When $X$ is finite, say $X = \{x_1, x_2, \dots, x_n\}$, then the transformation $\alpha$ which maps $x_i$ to $y_i$ (with $y_i \in X$, of course) is often written:
$$
\alpha = 
\begin{pmatrix}
x_1 & x_2 & \dots & x_n \\
y_1 & y_2 & \dots & y_n 
\end{pmatrix}
$$

With this notation it is quite easy to \PMlinkescapetext{calculate} products.  For example, if $X = \{1, 2, 3, 4\}$, then
$$
\begin{pmatrix}
1 & 2 & 3 & 4 \\
3 & 2 & 1 & 2 
\end{pmatrix}
\begin{pmatrix}
1 & 2 & 3 & 4 \\
2 & 3 & 3 & 4 
\end{pmatrix}
=
\begin{pmatrix}
1 & 2 & 3 & 4 \\
3 & 3 & 2 & 3 
\end{pmatrix}
$$

When $X$ is infinite, say $X = \{1, 2, 3, \dotsc \}$, then this notation is still useful for illustration in cases where the transformation pattern is apparent.  For example, if $\alpha \in \mathcal{T}_X$ is given by $\alpha \colon n \mapsto n+1$, we can write
$$
\alpha = 
\begin{pmatrix}
1 & 2 & 3 & 4 & \dots \\
2 & 3 & 4 & 5 & \dots 
\end{pmatrix}
$$
%%%%%
%%%%%
\end{document}
