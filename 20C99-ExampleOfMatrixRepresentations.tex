\documentclass[12pt]{article}
\usepackage{pmmeta}
\pmcanonicalname{ExampleOfMatrixRepresentations}
\pmcreated{2013-03-22 14:53:31}
\pmmodified{2013-03-22 14:53:31}
\pmowner{drini}{3}
\pmmodifier{drini}{3}
\pmtitle{example of matrix representations}
\pmrecord{6}{36573}
\pmprivacy{1}
\pmauthor{drini}{3}
\pmtype{Example}
\pmcomment{trigger rebuild}
\pmclassification{msc}{20C99}

\endmetadata

\usepackage{amsmath}
%%%\usepackage{xypic} 
\usepackage{bbm}
\newcommand{\Z}{\mathbbmss{Z}}
\newcommand{\C}{\mathbbmss{C}}
\newcommand{\R}{\mathbbmss{R}}
\newcommand{\Q}{\mathbbmss{Q}}
\newcommand{\mathbb}[1]{\mathbbmss{#1}}
\newcommand{\figura}[1]{\begin{center}\includegraphics{#1}\end{center}}
\newcommand{\figuraex}[2]{\begin{center}\includegraphics[#2]{#1}\end{center}}
\newtheorem{dfn}{Definition}
\begin{document}
\textbf{Sign representation of $S_n$}\\
Let $G=S_n$ the $n$-th symmetric group, and consider $X(\sigma) = \mathrm{sign}(\sigma)$ where $\sigma$ is any permutation in $S_n$. 
That is, $\mathrm{sign}(\sigma)=1$ when $\sigma$ is an even permutation, and  $\mathrm{sign}(\sigma)=-1$ when $\sigma$ is an odd permutation.

The function $X$ is a group homomorphism between $S_n$ and $GL(\C)=\C \setminus\{0\}$ (that is invertible matrices of size $1\times1$, which is the set of non-zero complex numbers). And thus we say that $\C\setminus\{0\}$ carries a representation of the symmetric group. 

\textbf{Defining representation of $S_n$}\\
For each $\sigma \in S_n$, let $X:S_n\to GL_n(\C)$ the function given by $X(\sigma)=(a_{ij})_{n\times n}$ where $(a_{ij})$ is the \emph{permutation matrix} given by
\[
a_{ij}=\begin{cases}
1 & \text{if } \sigma(i)=j\\
0 & \text{if } \sigma(i)\ne j\\
\end{cases}
\]
Such matrices are called permutation matrices because they are obtained permuting the colums of the identity matrix. The function so defined is then a group homomorphism, and thus $GL_n(\C)$ carries a representation of the symmetric group.
%%%%%
%%%%%
\end{document}
