\documentclass[12pt]{article}
\usepackage{pmmeta}
\pmcanonicalname{ScottWiegoldConjecture}
\pmcreated{2013-03-22 18:29:34}
\pmmodified{2013-03-22 18:29:34}
\pmowner{whm22}{2009}
\pmmodifier{whm22}{2009}
\pmtitle{Scott-Wiegold conjecture}
\pmrecord{8}{41174}
\pmprivacy{1}
\pmauthor{whm22}{2009}
\pmtype{Theorem}
\pmcomment{trigger rebuild}
\pmclassification{msc}{20E06}
\pmsynonym{one relator products of cyclic groups}{ScottWiegoldConjecture}
%\pmkeywords{free product}
%\pmkeywords{normal closure}

\endmetadata

% this is the default PlanetMath preamble.  as your knowledge
% of TeX increases, you will probably want to edit this, but
% it should be fine as is for beginners.

% almost certainly you want these
\usepackage{amssymb}
\usepackage{amsmath}
\usepackage{amsfonts}

% used for TeXing text within eps files
%\usepackage{psfrag}
% need this for including graphics (\includegraphics)
%\usepackage{graphicx}
% for neatly defining theorems and propositions
%\usepackage{amsthm}
% making logically defined graphics
%%%\usepackage{xypic}

% there are many more packages, add them here as you need them

% define commands here

\begin{document}
The Scott-Wiegold conjecture (1976) is stated as follows: 

Given distinct prime numbers $p$, $q$ and $r$, the free product of cyclic groups $C_p * C_q * C_r$ is not the normal closure of any single element.

In 1992 this was included as problem 5.53 of The Kourovka Notebook: {\it Unsolved Problems in \PMlinkescapetext{Group Theory}} \cite{Kour}.

The conjecture was proven to be true in 2001 by James Howie \cite{Howi}.  Despite remaining an unsolved problem for 25 years, the proof is both brief and fairly elementary.

Whilst the question is group theoretic and involves only \PMlinkescapetext{discrete objects}, the proof does not use any combinatorial \PMlinkescapetext{arguments} but instead depends on basic notions from topology.

\begin{thebibliography}{}

\bibitem{Kour} V.D.Mazurov, E.I. Khukhro (Eds.), {\it Unsolved Problems in Group Theory}: The Kourovka Notebook, $12^{\it th}$ Edition, Russian Academy of Sciences, Novosibirsk, 1992.


\bibitem{Howi} James Howie, {\it A proof of the Scott-Wiegold conjecture on free products of cyclic groups}, Journal of Pure and Applied Algebra 173, 2002 pp.167--176


\end{thebibliography}
%%%%%
%%%%%
\end{document}
