\documentclass[12pt]{article}
\usepackage{pmmeta}
\pmcanonicalname{AGroupEmbedsIntoItsProfiniteCompletionIfAndOnlyIfItIsResiduallyFinite}
\pmcreated{2013-03-22 15:15:56}
\pmmodified{2013-03-22 15:15:56}
\pmowner{yark}{2760}
\pmmodifier{yark}{2760}
\pmtitle{a group embeds into its profinite completion if and only if it is residually finite}
\pmrecord{13}{37052}
\pmprivacy{1}
\pmauthor{yark}{2760}
\pmtype{Theorem}
\pmcomment{trigger rebuild}
\pmclassification{msc}{20E18}
\pmrelated{ProfiniteCompletion}
\pmrelated{ProfiniteGroup}
\pmrelated{ResiduallyCalP}

\usepackage{amssymb}
\usepackage{amsmath}
\usepackage{amsfonts}
\usepackage{amsthm}
\usepackage[matrix,arrow,curve]{xy}
\usepackage[only,trianglelefteqslant]{stmaryrd}

\newcommand{\sgp}{\leqslant}
\newcommand{\sgpf}{\leqslant_{\mathrm{f}}}
\newcommand{\nsgp}{\trianglelefteqslant}
\newcommand{\nsgpf}{\trianglelefteqslant_{\mathrm{f}}}
\newcommand{\nsgpo}{\trianglelefteqslant_{\mathrm{o}}}
\newcommand{\hir}{\mathrm{h}}
\newcommand{\stab}{\mathrm{Stab}}
\begin{document}
\PMlinkescapeword{clear}
\PMlinkescapeword{open}
\PMlinkescapeword{theorem}

Let $G$ be a group.

First suppose that $G$ is residually finite, that is,
\[
        \mathrm{R}(G) := \bigcap_{N \nsgpf G} N = 1
\]
(where $N \nsgpf G$ denotes that $N$ is a normal subgroup of finite index in $G$).
Consider the natural mapping of $G$ into its profinite completion $\hat{G}$
given by $g \mapsto (Ng)_{N \nsgpf G}$.
It is clear that the kernel of this map is precisely $\mathrm{R}(G)$,
so that it is a monomorphism when $G$ is residually finite.

Now suppose that $G$ embeds into its profinite completion $\hat{G}$
and identify $G$ with a subgroup of $\hat{G}$. Now, a theorem on
profinite groups tells us that
\[
        \bigcap_{N \nsgpo \hat{G}} N = 1,
\]
(where $N \nsgpo G$ denotes that $N$ is an \PMlinkname{open}{TopologicalSpace} normal subgroup of $G$) and since open subgroups of a profinite group have finite index, we
have that
\[
        \mathrm{R}(\hat{G}) = 1,
\]
so $\hat{G}$ is residually finite. Then $G$ is a subgroup of a
residually finite group, so is itself residually finite, as required.

\begin{thebibliography}{99}

\bibitem{ddms}
J.~D. Dixon, M.~P.~F. du~Sautoy, A.~Mann, and D.~Segal, \emph{Analytic
 pro-$p$ groups}, 2nd ed., Cambridge studies in advanced mathematics,
 Cambridge University Press, 1999.

\end{thebibliography}
%%%%%
%%%%%
\end{document}
