\documentclass[12pt]{article}
\usepackage{pmmeta}
\pmcanonicalname{CommensurableSubgroups}
\pmcreated{2013-03-22 18:34:14}
\pmmodified{2013-03-22 18:34:14}
\pmowner{asteroid}{17536}
\pmmodifier{asteroid}{17536}
\pmtitle{commensurable subgroups}
\pmrecord{4}{41294}
\pmprivacy{1}
\pmauthor{asteroid}{17536}
\pmtype{Definition}
\pmcomment{trigger rebuild}
\pmclassification{msc}{20C08}
\pmrelated{CommensurableNumbers}
\pmdefines{commensurable}

% this is the default PlanetMath preamble.  as your knowledge
% of TeX increases, you will probably want to edit this, but
% it should be fine as is for beginners.

% almost certainly you want these
\usepackage{amssymb}
\usepackage{amsmath}
\usepackage{amsfonts}

% used for TeXing text within eps files
%\usepackage{psfrag}
% need this for including graphics (\includegraphics)
%\usepackage{graphicx}
% for neatly defining theorems and propositions
%\usepackage{amsthm}
% making logically defined graphics
%%%\usepackage{xypic}

% there are many more packages, add them here as you need them

% define commands here

\begin{document}
\subsection{Definition}

{\bf Definition -} Let $G$ be a group. Two subgroups $S_1, S_2 \subseteq G$ are said to be {\bf commensurable}, in which case we write $S_1 \sim S_2$, if $S_1 \cap S_2$ has finite index both in $S_1$ and in $S_2$, i.e. if $[S_1 : S_1 \cap S_2]$ and $[S_2 : S_1 \cap S_2]$ are both finite.


This \PMlinkescapetext{property} can be interpreted informally in the following \PMlinkescapetext{way}: $S_1$ and $S_2$ are commensurable if their intersection $S_1 \cap S_2$ is ``big'' in both $S_1$ and $S_2$.

\subsection{Commensurability is an equivalence relation}

{\bf \PMlinkescapetext{Proposition} -} \PMlinkescapetext{Commensurability} of subgroups is an equivalence relation. In particular, if $S_1 \sim S_2$ and $S_2 \sim S_3$, then $S_1 \sim S_3$.

{\bf \emph{\PMlinkescapetext{Proof}:}} Let $S_1$, $S_2$ and $S_3$ be subgroups of a group $G$.
\begin{itemize}
\item Reflexivity: we have that $S_1 \sim S_1$, since $[S_1: S_1] = 1$.
\item Symmetry: is clear from the definition.
\item Transitivity: if $S_1 \sim S_2$ and $S_2 \sim S_3$, then one has
\begin{eqnarray*}
[S_1:S_1 \cap S_3] & \leq & [S_1:S_1 \cap S_2 \cap S_3]\\
& = & [S_1:S_1 \cap S_2][S_1 \cap S_2 : S_1 \cap S_2 \cap S_3]\\
& \leq & [S_1:S_1 \cap S_2][S_2:S_2 \cap S_3]\\
& < & \infty.
\end{eqnarray*}
Similarly, we can prove that $[S_3:S_1 \cap S_3] < \infty$ and therefore $S_1 \sim S_3$. $\square$
\end{itemize}

\subsection{Examples:}
\begin{itemize}
\item All non-zero subgroups of $\mathbb{Z}$ are commensurable with each other.
\item All conjugacy classes of the general linear group $GL(n;\mathbb{Z})$, seen as a subgroup of $GL(n;\mathbb{Q})$, are commensurable with each other.
\end{itemize}

\begin{thebibliography}{9}
\bibitem{krieg} A. Krieg, \emph{\PMlinkescapetext{Hecke algebras}}, Mem. Amer. Math. Soc., no. 435, vol. 87, 1990.
\end{thebibliography}


%%%%%
%%%%%
\end{document}
