\documentclass[12pt]{article}
\usepackage{pmmeta}
\pmcanonicalname{Submodule}
\pmcreated{2013-03-22 15:15:26}
\pmmodified{2013-03-22 15:15:26}
\pmowner{PrimeFan}{13766}
\pmmodifier{PrimeFan}{13766}
\pmtitle{submodule}
\pmrecord{19}{37040}
\pmprivacy{1}
\pmauthor{PrimeFan}{13766}
\pmtype{Definition}
\pmcomment{trigger rebuild}
\pmclassification{msc}{20-00}
\pmclassification{msc}{16-00}
\pmclassification{msc}{13-00}
\pmrelated{SumOfIdeals}
\pmrelated{QuotientOfIdeals}
\pmdefines{R-submodule}
\pmdefines{generated submodule}
\pmdefines{generator}
\pmdefines{sum of submodules}
\pmdefines{product submodule}
\pmdefines{quotient of submodules}

% this is the default PlanetMath preamble.  as your knowledge
% of TeX increases, you will probably want to edit this, but
% it should be fine as is for beginners.

% almost certainly you want these
\usepackage{amssymb}
\usepackage{amsmath}
\usepackage{amsfonts}

% used for TeXing text within eps files
%\usepackage{psfrag}
% need this for including graphics (\includegraphics)
%\usepackage{graphicx}
% for neatly defining theorems and propositions
 \usepackage{amsthm}
% making logically defined graphics
%%%\usepackage{xypic}

% there are many more packages, add them here as you need them

% define commands here

\theoremstyle{definition}
\newtheorem*{thmplain}{Theorem}
\begin{document}
Given a ring $R$ and a left $R$-module $T$, a subset $A$ of $T$ is called a ({\em left}) {\em submodule} of $T$, if\, $(A,\,+)$\, is a subgroup of\, $(M,\,+)$\, and\, $ra \in A$\, for all elements $r$ of $R$ and $a$ of $A$.\\

\textbf{Examples}
\begin{enumerate}
 \item The subsets $\{0\}$ and $T$ are always submodules of the module $T$.
 \item The set \,$\{t\in T:\,\,\,rt = t\,\,\,\forall r\in R\}$\, of all invariant elements of $T$ is a submodule of $T$.
 \item If \,$X \subseteq T$\, and $\mathfrak{a}$ is a left ideal of $R$, then the set 
$$\mathfrak{a}X := \{\mbox{finite}\sum_\nu a_\nu x_\nu:
 \,\,\,a_\nu\in\mathfrak{a},\,\,x_\nu\in X\,\,\forall\nu\}$$
is a submodule of $T$.\, Especially, $RX$ is called the submodule {\em generated} by the subset $X$; then the elements of $X$ are \emph{generators} of this submodule.
\end{enumerate}

There are some operations on submodules.\, Given the submodules $A$ and $B$ of $T$, the {\em sum}\, $A + B := \{a + b\in T:\,\,a\in A \,\land\, b\in B\}$\, and the intersection $A\cap B$ are submodules of $T$.

The notion of sum may be extended for any family \,$\{A_j:\,\,j\in J\}$\, of submodules:\, the sum $\sum_{j\in J}A_j$ of submodules consists of all finite sums $\sum_j a_j$ where every $a_j$ belongs to one $A_j$ of those submodules.\, The sum of submodules as well as the intersection $\bigcap_{j\in J}A_j$ are submodules of $T$.\, The submodule $RX$ is the intersection of all submodules containing the subset $X$.

If $T$ is a ring and $R$ is a subring of $T$, then $T$ is an $R$-module; then one can consider the {\em product} and the {\em quotient} of the left $R$-submodules $A$ and $B$ of $T$:
\begin{itemize}
\item $AB := \{\mbox{finite}\sum_\nu a_\nu b_\nu:
 \,\,\,a_\nu\in A,\,\,b_\nu\in B\,\,\forall\nu\}$
\item $[A:B] := \{t\in T:\,\, tB\subseteq A\}$
\end{itemize}
Also these are left $R$-submodules of $T$.
%%%%%
%%%%%
\end{document}
