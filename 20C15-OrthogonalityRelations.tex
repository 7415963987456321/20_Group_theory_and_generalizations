\documentclass[12pt]{article}
\usepackage{pmmeta}
\pmcanonicalname{OrthogonalityRelations}
\pmcreated{2013-03-22 13:21:27}
\pmmodified{2013-03-22 13:21:27}
\pmowner{mhale}{572}
\pmmodifier{mhale}{572}
\pmtitle{orthogonality relations}
\pmrecord{16}{33878}
\pmprivacy{1}
\pmauthor{mhale}{572}
\pmtype{Theorem}
\pmcomment{trigger rebuild}
\pmclassification{msc}{20C15}

\endmetadata

% this is the default PlanetMath preamble.  as your knowledge
% of TeX increases, you will probably want to edit this, but
% it should be fine as is for beginners.

% almost certainly you want these
\usepackage{amssymb}
\usepackage{amsmath}
\usepackage{amsfonts}

% used for TeXing text within eps files
%\usepackage{psfrag}
% need this for including graphics (\includegraphics)
%\usepackage{graphicx}
% for neatly defining theorems and propositions
\usepackage{amsthm}
% making logically defined graphics
%%%\usepackage{xypic}

% there are many more packages, add them here as you need them

% define commands here
\newtheorem{thm}{Theorem}
\newtheorem{prop}{Proposition}

\newcommand{\ab}[1]{{#1}_{\mathrm{ab}}}
\newcommand{\Ad}{\mathrm{Ad}}
\newcommand{\ad}{\mathrm{ad}}
\newcommand{\Aut}{\mathrm{Aut}\,}
\newcommand{\Aff}[2]{\mathrm{Aff}_{#1} #2}
\newcommand{\aff}[2]{\mathfrak{aff}_{#1} #2}
\newcommand{\mcB}{\mathcal{B}}
\newcommand{\bb}[1]{\mathbb{#1}}
\newcommand{\bfrac}[2]{\left[\frac{#1}{#2}\\right]}
\newcommand{\bkh}{\backslash}
\newcommand{\Cyc}[2]{\mathcal{C}^{#1}_{#2}}
\newcommand{\Cbar}[2]{\overline{\C{#1}{#2}}}
%\newcommand{\CD}{\R[\Delta]}
\newcommand{\C}{\mathbb{C}}
\newcommand{\CF}[2]{\ensuremath{\mathfrak{C}(#1,#2)}}
\newcommand{\Cinf}{\EuScript{C}^{\infty}}
\newcommand{\cmp}{cyclic mod $p$\xspace}
\newcommand{\cp}{\mathrm{c.p.}}
\newcommand{\CS}{\EuScript{CS}}
\newcommand{\deck}{\EuScript{D}}
\newcommand{\defl}[1]{\mathfrak{def}_{#1}}
\newcommand{\Der}{\mathrm{Der}\,}
\newcommand{\eH}{[X_H]-[Y_H]}
\newcommand{\EL}{\mathcal{EL}}
\newcommand{\End}{\mathrm{End}}
\newcommand{\ES}[1]{\EuScript{#1}}
\newcommand{\Ext}{\mathrm{Ext}}
\newcommand{\Fix}{\mathrm{Fix}}
\newcommand{\fr}[1]{\mathfrak{#1}}
\newcommand{\Frat}{\mathrm{Frat}\,}
\newcommand{\Gal}[1]{\Gamma(#1 |\Q)}
\newcommand{\GL}[2]{\mathrm{GL}_{#1} #2}
\newcommand{\gl}[2]{\mathfrak{gl}_{#1} #2}
\newcommand{\GrR}[1]{a(#1 G)}
\newcommand{\Gr}{\mathrm{Gr}\,}
\newcommand{\mcH}{\mathcal{H}}
\renewcommand{\H}{\mathbb{H}}
\newcommand{\Hom}[2]{\mathrm{Hom}(#1,#2)}
\newcommand{\id}{\mathrm{id}}
\newcommand{\im}{\mathrm{im}}
\newcommand{\ind}[2]{\mathrm{ind}^{#1}_{#2}}
\newcommand{\indp}[2]{\mathfrak{ind}^{#1}_{#2}}
\renewcommand{\inf}[1]{\mathfrak{inf}_{#1}}
\newcommand{\inn}[1]{\langle #1\rangle}
\renewcommand{\int}{\mathrm{int}}
\newcommand{\Iso}{\mathrm{Iso}}
\newcommand{\K}{\mathcal{K}}
\renewcommand{\ker}{\mathrm{ker}\,}
\renewcommand{\L}[1]{\mathfrak{L}(#1)}
\newcommand{\lap}[1]{\Delta_{#1}}
\newcommand{\lapM}{\Delta_M}
\newcommand{\Lie}{\mathrm{Lie}}
\newcommand{\lineq}{linearly equivalent\xspace}
\newcommand{\mc}[1]{\mathcal{#1}}
\newcommand{\mG}{m_G}
\newcommand{\mK}{m_{\K}}
\newcommand{\mindeg}[1]{\fr{md}(#1)}
\newcommand{\N}{\mathbb{N}}
\renewcommand{\O}{\mathcal{O}}
\newcommand{\Om}{\Omega}
\newcommand{\om}{\omega}
\newcommand{\Orb}{\mathrm{Orb}}
\newcommand{\pad}{\hat{\Z}_p}
\newcommand{\pder}[2]{\frac{\partial #1}{\partial #2}}
\newcommand{\pderw}[1]{\frac{\partial}{\partial #1}}
\newcommand{\pdersec}[2]{\frac{\partial^2 #1}{\partial {#2}^2}} 
\newcommand{\perm}[1]{\pi_{#1}}
\newcommand{\Q}{\mathbb{Q}}
\newcommand{\R}{\mathbb{R}}
\newcommand{\rad}{\mathrm{rad}\,}
\newcommand{\res}[2]{\mathrm{res}^{#1}_{#2}}
\newcommand{\resp}[2]{\mathfrak{res}^{#1}_{#2}}
\newcommand{\RG}{\EuScript{R}_G}
\newcommand{\rk}{\mathrm{rk}\,}
\newcommand{\V}[1]{\mathbf{#1}}
\newcommand{\vp}{\varphi}
\newcommand{\Stab}{\mathrm{Stab}}
\newcommand{\SL}[2]{\mathrm{SL}_{#1} #2}
\renewcommand{\sl}[2]{\fr{sl}_{#1} #2}
\newcommand{\SO}[2]{\mathrm{SO}_{#1} #2}
\newcommand{\Sp}[2]{\mathrm{Sp}_{#1} #2}
\renewcommand{\sp}[2]{\fr{sp}_{#1} #2}
\newcommand{\SU}[1]{\mathrm{SU}( #1)}
\newcommand{\su}[1]{\fr{su}_{#1}}
\newcommand{\Sym}{\mathrm{Sym}}
\newcommand{\sym}{\mathrm{sym}}
\newcommand{\Tg}{\mc{T}(\fr g)}
\newcommand{\tom}{\tilde{\omega}}
\newcommand{\ghtghp}{\fr g/\fr h\oplus(\fr g/\fr h^\perp)^*}
\newcommand{\ghps}{(\fr g/\fr h^\perp)^*}
\newcommand{\Tr}{\mathrm{Tr}}
\newcommand{\tr}{\mathrm{tr}}
%\renewcommand{\thechapter}{\Roman{chapter}}
%\renewcommand{\thesection}{\thechapter.\arabic{section}}
%\renewcommand{\thethm}{\thechapter.\arabic{thm}}
\newcommand{\Ug}{\mc{U}(\fr g)}
\newcommand{\Uh}{\mc{U}(\fr h)}
\renewcommand{\V}[1]{\mathbf{#1}}
\newcommand{\Z}{\mathbb{Z}}
\newcommand{\Zp}{\Z/p}
\begin{document}
\textbf{First orthogonality relations}:
Let $\rho_\alpha\colon G \to V_\alpha$ and $\rho_\beta\colon G \to V_\beta$ be irreducible representations of a finite group $G$ over the field $\C$.  Then
\[
\frac{1}{|G|}\sum_{g\in G}\overline{\rho^{(\alpha)}_{ij}(g)}\rho^{(\beta)}_{kl}(g)=\frac{\delta_{\alpha\beta}\delta_{ik}\delta_{jl}}{\dim V_\alpha}.
\]

We have the following useful corollary.
Let $\chi_1$, $\chi_2$ be characters of representations $V_1$, $V_2$ of a finite group $G$ over a field $k$ of characteristic $0$.  Then

$$(\chi_1,\chi_2)=\frac{1}{|G|}\sum_{g\in G}\overline{\chi_1(g)}\chi_2(g)=\dim(\Hom{V_1}{V_2}).$$

\begin{proof} 
First of all, consider the special case where $V=k$ with the trivial action
of the group.  Then $\mathrm{Hom}_G(k,V_2)\cong V_2^G$, the fixed points.
On the other hand, consider the map 
$$\phi=\frac{1}{|G|}\sum_{g\in G} g\colon V_2\to V_2$$ (with the sum in $\mathrm{End}(V_2)$).  Clearly,
the image of this map is contained in $V_2^G$, and it is the identity restricted
to $V_2^G$.  Thus, it is a projection with image $V_2^G$.  Now, the rank of
a projection (over a field of characteristic 0) is its trace.  Thus,
$$\dim_k \mathrm{Hom}_G(k,V_2)=\dim V_2^G=\mathrm{tr}(\phi)=
\frac{1}{|G|}\sum\chi_2(g)$$ which is exactly the orthogonality formula
for $V_1=k$.  

Now, in general, $\mathrm{Hom}(V_1,V_2)\cong V_1^*\otimes V_2$ is
a representation, and $\mathrm{Hom}_G(V_1,v_2)=(\mathrm{Hom}(V_1,V_2))^G$.  Since $\chi_{V^*_1\otimes V_2}=\overline{\chi_1}\chi_2$,

$$\dim_k\mathrm{Hom}_G(V_1,V_2)=\dim_k (\mathrm{Hom}(V_1,V_2))^G=
\sum_{g\in G}\overline{\chi_1}\chi_2$$
which is exactly the relation we desired.
\end{proof}

In particular, if $V_1, V_2$ irreducible, by Schur's Lemma

$$\Hom{V_1}{V_2}=\begin{cases} D &V_1\cong V_2\\ 0&V_1\ncong V_2\end{cases}$$

where $D$ is a division algebra.  In particular, non-isomorphic irreducible
representations have orthogonal characters.  Thus, for any representation $V$,
the multiplicities $n_i$ in the unique decomposition of $V$ into the \PMlinkname{direct sum}{DirectSum}
of irreducibles

$$V\cong V_1^{\oplus n_1}\oplus\cdots\oplus V_m^{\oplus n_m}$$ 

where $V_i$ ranges over irreducible representations of $G$ over $k$, can be 
determined in terms of the character inner product: 

$$n_i=\frac{(\psi,\chi_i)}{(\chi_i,\chi_i)}$$

where $\psi$ is the character of $V$ and $\chi_i$ the character of $V_i$.
In particular, representations over a field of characteristic zero are determined by their character.  Note: This is not true over fields of positive 
characteristic.

If the field $k$ is algebraically closed, 
the only finite division algebra over $k$ is $k$ itself, so 
the characters of irreducible representations form an orthonormal basis for 
the vector space of class functions with respect to this inner product.
Since $(\chi_i,\chi_i)=1$ for all irreducibles, the multiplicity formula
above reduces to $n_i=(\psi,\chi_i)$.


\textbf{Second orthogonality relations}:
We assume now that $k$ is algebraically closed.
Let $g,g'$ be elements of a finite group $G$.
Then $$\sum_{\chi}\chi(g)\overline{\chi(g')}=\begin{cases}|C_G(g_1)|&g\sim g'\\ 0&g\nsim g'\end{cases}$$
where the sum is over the characters of irreducible representations, and $C_G(g)$ is the centralizer of $g$.

\begin{proof}
Let $\chi_1,\ldots,\chi_n$ be the characters of the irreducible representations,
and let $g_1,\ldots,g_n$ be representatives of the conjugacy classes.

Let $A$ be the matrix whose $ij$th entry is $\sqrt{|G:C_G(g_j)|}(\overline{\chi_i(g_j)})$.
By first orthogonality, $AA^{*}=|G|I$ (here $*$ denotes conjugate transpose),
where $I$ is the identity matrix.  Since left \PMlinkname{inverses}{MatrixInverse} are right \PMlinkescapetext{inverses},
$A^{*}A=|G|I$.  Thus, 
$$\sqrt{|G:C_G(g_i)||G:C_G(g_k)|}\sum_{j=1}^n\chi_j(g_i)\overline{\chi_j(g_k)}=|G|\delta_{ik}.$$
Replacing $g_i$ or $g_k$ with any conjuagate will not change the expression above.
thus, if our two elements are not conjugate, we obtain that $\sum_\chi\chi(g)\overline{\chi(g')}=0$.
On the other hand, if $g\sim g'$, then $i=k$ in the sum above, which reduced to the expression
we desired.
\end{proof}

A special case of this result, applied to $1$ is that $|G|=\sum_{\chi}\chi(1)^2$, that is, the sum of the squares of the \PMlinkname{dimensions}{Dimension} of the irreducible representations of any finite group is the order of the group.
%%%%%
%%%%%
\end{document}
