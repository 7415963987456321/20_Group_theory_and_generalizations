\documentclass[12pt]{article}
\usepackage{pmmeta}
\pmcanonicalname{ProofOfFrobeniusReciprocity}
\pmcreated{2013-03-22 18:36:23}
\pmmodified{2013-03-22 18:36:23}
\pmowner{rm50}{10146}
\pmmodifier{rm50}{10146}
\pmtitle{proof of Frobenius reciprocity}
\pmrecord{5}{41339}
\pmprivacy{1}
\pmauthor{rm50}{10146}
\pmtype{Proof}
\pmcomment{trigger rebuild}
\pmclassification{msc}{20C99}

% this is the default PlanetMath preamble.  as your knowledge
% of TeX increases, you will probably want to edit this, but
% it should be fine as is for beginners.

% almost certainly you want these
\usepackage{amssymb}
\usepackage{amsmath}
\usepackage{amsfonts}

% used for TeXing text within eps files
%\usepackage{psfrag}
% need this for including graphics (\includegraphics)
%\usepackage{graphicx}
% for neatly defining theorems and propositions
\usepackage{amsthm}
% making logically defined graphics
%%%\usepackage{xypic}

% there are many more packages, add them here as you need them

% define commands here
\newcommand{\Abs}[1]{\left\lvert #1\right\rvert}
\newcommand{\ind}{\negmedspace\uparrow}
\newcommand{\res}{\negmedspace\downarrow}
%% \theoremstyle{plain} %% This is the default
\newtheorem{thm}{Theorem}[section]
\newtheorem{cor}[thm]{Corollary}
\newtheorem{lem}[thm]{Lemma}
\newtheorem{prop}[thm]{Proposition}

\theoremstyle{definition}
\newtheorem{defn}[thm]{Definition}
\begin{document}
We prove the slightly more general result
\begin{thm} If $G$ is a finite group with subgroup $H$, $\alpha$ a class function on $H$ and $\beta$ a class function on $G$, then
\[
  \langle \alpha\ind_H^G,\beta \rangle_G = \langle \alpha,\beta\res_H^G\rangle_H
\]
\end{thm}
Here we use $\ind_H^G$ to refer to the \PMlinkname{induction}{InducedRepresentation} to $G$ of a class function on $H$, and $\res_H^G$ to refer to the \PMlinkname{restriction}{RestrictionRepresentation} of a class function on $G$ to one on $H$.
\begin{proof}
\[
  \langle\alpha\ind_H^G,\beta\rangle_G 
  = \frac{1}{\Abs{G}}\sum_{g\in G}\left(\frac{1}{\Abs{H}}\sum_{\substack{t\in G\\t^{-1}gt\in H}}
      \alpha(t^{-1}gt)\right)\overline{\beta(g)} 
  = \frac{1}{\Abs{G}\Abs{H}}\sum_{t\in G}\left(\sum_{\substack{g\in G\\t^{-1}gt\in H}} 
      \alpha(t^{-1}gt)\right)\overline{\beta(g)}
\]
Since $\beta$ is a class function, this is the same as
\[
  \frac{1}{\Abs{G}\Abs{H}} \sum_{\substack{t\in G\\g\in G\\t^{-1}gt\in H}} 
    \alpha(t^{-1}gt)\overline{\beta(t^{-1}gt)}
  =\frac{1}{\Abs{G}\Abs{H}}\sum_{h\in H}\sum_{\substack{t\in G\\g\in G\\t^{-1}gt=h}} 
    \alpha(h)\overline{\beta(h)}
\]
Clearly for every $h\in H, t\in G$ there is a unique $g\in G$ with $t^{-1}gt=h$, so every element of $H$ is counted $\Abs{G}$ times by the \PMlinkescapetext{inner} sum. Thus the sum is equal to
\[
  \frac{\Abs{G}}{\Abs{G}\Abs{H}}\sum_{h\in H} \alpha(h)\overline{\beta(h)} 
  = \frac{1}{\Abs{H}}\sum_{h\in H} \alpha(h)\overline{\beta(h)} 
  = \langle \alpha,\beta\res_H^G\rangle_H
\]\qedhere

\end{proof}
%%%%%
%%%%%
\end{document}
