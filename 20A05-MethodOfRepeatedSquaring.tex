\documentclass[12pt]{article}
\usepackage{pmmeta}
\pmcanonicalname{MethodOfRepeatedSquaring}
\pmcreated{2013-03-22 16:16:14}
\pmmodified{2013-03-22 16:16:14}
\pmowner{Algeboy}{12884}
\pmmodifier{Algeboy}{12884}
\pmtitle{method of repeated squaring}
\pmrecord{9}{38381}
\pmprivacy{1}
\pmauthor{Algeboy}{12884}
\pmtype{Theorem}
\pmcomment{trigger rebuild}
\pmclassification{msc}{20A05}
\pmclassification{msc}{08A99}
\pmclassification{msc}{20-00}
\pmsynonym{successive squaring}{MethodOfRepeatedSquaring}

\usepackage{latexsym}
\usepackage{amssymb}
\usepackage{amsmath}
\usepackage{amsfonts}
\usepackage{amsthm}

%%\usepackage{xypic}

%-----------------------------------------------------

%       Standard theoremlike environments.

%       Stolen directly from AMSLaTeX sample

%-----------------------------------------------------

%% \theoremstyle{plain} %% This is the default

\newtheorem{thm}{Theorem}

\newtheorem{coro}[thm]{Corollary}

\newtheorem{lem}[thm]{Lemma}

\newtheorem{lemma}[thm]{Lemma}

\newtheorem{prop}[thm]{Proposition}

\newtheorem{conjecture}[thm]{Conjecture}

\newtheorem{conj}[thm]{Conjecture}

\newtheorem{defn}[thm]{Definition}

\newtheorem{remark}[thm]{Remark}

\newtheorem{ex}[thm]{Example}



%\countstyle[equation]{thm}



%--------------------------------------------------

%       Item references.

%--------------------------------------------------


\newcommand{\exref}[1]{Example-\ref{#1}}

\newcommand{\thmref}[1]{Theorem-\ref{#1}}

\newcommand{\defref}[1]{Definition-\ref{#1}}

\newcommand{\eqnref}[1]{(\ref{#1})}

\newcommand{\secref}[1]{Section-\ref{#1}}

\newcommand{\lemref}[1]{Lemma-\ref{#1}}

\newcommand{\propref}[1]{Prop\-o\-si\-tion-\ref{#1}}

\newcommand{\corref}[1]{Cor\-ol\-lary-\ref{#1}}

\newcommand{\figref}[1]{Fig\-ure-\ref{#1}}

\newcommand{\conjref}[1]{Conjecture-\ref{#1}}


% Normal subgroup or equal.

\providecommand{\normaleq}{\unlhd}

% Normal subgroup.

\providecommand{\normal}{\lhd}

\providecommand{\rnormal}{\rhd}
% Divides, does not divide.

\providecommand{\divides}{\mid}

\providecommand{\ndivides}{\nmid}


\providecommand{\union}{\cup}

\providecommand{\bigunion}{\bigcup}

\providecommand{\intersect}{\cap}

\providecommand{\bigintersect}{\bigcap}










\begin{document}
\begin{thm}
Given a semigroup $S$ and an $n\in \mathbb{Z}^+$ then the function
$f:S\rightarrow S$ defined by $f(a)=a^n$ has an SLP representations of
computational length $O(\log_2 n)$.  Inparticular, if we can compute $a^n$
using only $O(\log_2 n)$ multiplications.
\end{thm}
\begin{proof}
Let $f_{0}:S\rightarrow S$ be the map $f(x)=x$ and 
$f_1:S\rightarrow S$ be defined by $f_1(x)=xx$.  So far we
recognize that $f_0(x)$ evaluates to $x^1=x^{2^0}$ and 
$f_1(x)$ evaluates to $x^2=x^{2^1}$.  \emph{But we caution that 
we do not define these functions with exponents because we want
to demonstrate that from an algorithm to multiply we can create
an efficient algorithm to exponentiate.}

For $1<i\leq k$ define (recalling that $f_{i+1}$ may use the output
of any previous $f_i$ evaluation) 
\[f_{i+1}(x):=f_1(f_i(x)).\]
Evidently, $f_{i+1}(x)$ evaluates to $f_i(x)^2$.  By induction,
$f_i(x)$ evaluates to $x^{2^i}$ and thus $f_{i+1}(x)$ evaluates to
$(x^{2^i})^2=x^{2^{i+1}}$.  Therefore we establish that $f_j(x)=x^{2^j}$ 
for all nonnegative integers $j$.  Furthermore, to evaluate $f_j(x)$ uses
$j$ multiplications.

Now write $n$ in binary: $n=a_0 2^0+a_1 2^1 +\cdots +a_k 2^k$ with
$a_i=0,1$ for $0\leq i< k\leq \log_2 n$.  Let $0\leq i_1<i_2<\cdots<i_s\leq k$ be
the indices for which $a_{i_t}\neq 0$, $1\leq t\leq s$.  Then define
\[g_{n}(x)=f_{i_1}(x) f_{i_2}(x)\cdots f_{i_s}(x)\]
Thus, $g_n(x)$ evaluates to:
\begin{align*}
(x^{2^{i_1}}) (x^{2^{i_2}}) \cdots (x^{2^{i_s}}) & =
 (x^{2^0})^{a_0} (x^{2^1})^{a_1} \cdots (x^{2^k})^{a_k}\\
& = x^{a_0 2^0+a_1 2^1+\cdots +a_k 2^k}\\
& = x^n.
\end{align*}
Because each $f_{i+1}(x)$ uses the output of $f_{i}(x)$, the cost of
evaluating $g_n$ is the cost of evaluating $f_{i_s}(x)$ plus the final
cost of multiplying $f_{i_1}(x)\cdots f_{i_s}(x)$.  Thus, evaluating $g_n$
uses $i_s+(s-1)$ multiplications.  As $s\leq \log_2 n$ it follows that
evaluating $g_n$ uses no more than $2\log_2 n$ multiplications.  
\end{proof}


%%%%%
%%%%%
\end{document}
