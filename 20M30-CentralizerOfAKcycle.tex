\documentclass[12pt]{article}
\usepackage{pmmeta}
\pmcanonicalname{CentralizerOfAKcycle}
\pmcreated{2013-03-22 17:18:00}
\pmmodified{2013-03-22 17:18:00}
\pmowner{rm50}{10146}
\pmmodifier{rm50}{10146}
\pmtitle{centralizer of a k-cycle}
\pmrecord{6}{39647}
\pmprivacy{1}
\pmauthor{rm50}{10146}
\pmtype{Theorem}
\pmcomment{trigger rebuild}
\pmclassification{msc}{20M30}

\endmetadata

% this is the default PlanetMath preamble.  as your knowledge
% of TeX increases, you will probably want to edit this, but
% it should be fine as is for beginners.

% almost certainly you want these
\usepackage{amssymb}
\usepackage{amsmath}
\usepackage{amsfonts}

% used for TeXing text within eps files
%\usepackage{psfrag}
% need this for including graphics (\includegraphics)
%\usepackage{graphicx}
% for neatly defining theorems and propositions
\usepackage{amsthm}
% making logically defined graphics
%%%\usepackage{xypic}

% there are many more packages, add them here as you need them

% define commands here
\newtheorem{thm}{Theorem}
\begin{document}
\begin{thm} Let $\sigma$ be a $k$-cycle in $S_n$. Then the centralizer of $\sigma$ is
\[C_{S_n}(\sigma)=\{\sigma^i \tau \mid 0\leq i\leq k-1, \tau\in S_{n-k}\}\]
where $S_{n-k}$ is the subgroup of $S_n$ consisting of those permutations that fix all elements appearing in $\sigma$.
\end{thm}
\begin{proof}
This is fundamentally a counting argument. It is clear that $\sigma$ commutes with each element in the set given, since $\sigma$ commutes with powers of itself and also commutes with disjoint permutations. The size of the given set is $k\cdot (n-k)!$. However, the number of conjugates of $\sigma$ is the index of $C_{S_n}(\sigma)$ in $S_n$ by the orbit-stabilizer theorem, so to determine $\lvert C_{S_n}(\sigma)\rvert$ we need only count the number of conjugates of $\sigma$, i.e. the number of $k$-cycles.

In a $k$-cycle $(a_1~\ldots~a_k)$, there are $n$ choices for $a_1$, $n-1$ choices for $a_2$, and so on. So there are $n(n-1)\cdots(n-k+1)$ choices for the elements of the cycle. But this counts each cycle $k$ times, depending on which element appears as $a_1$. So the number of $k$-cycles is
\[\frac{n(n-1)\cdots(n-k+1)}{k}\]

Finally,
\[n!=\lvert S_n\rvert = \frac{n(n-1)\cdots(n-k+1)}{k}\lvert C_{S_n}(\sigma)\rvert\]
so that
\[\lvert C_{S_n}(\sigma)\rvert=\frac{k\cdot n!}{n(n-1)\cdots(n-k+1)}=k\cdot (n-k)!\]
and we see that the elements in the list above must exhaust $C_{S_n}(\sigma)$.
\end{proof}
%%%%%
%%%%%
\end{document}
