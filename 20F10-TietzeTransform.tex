\documentclass[12pt]{article}
\usepackage{pmmeta}
\pmcanonicalname{TietzeTransform}
\pmcreated{2013-03-22 15:42:40}
\pmmodified{2013-03-22 15:42:40}
\pmowner{rspuzio}{6075}
\pmmodifier{rspuzio}{6075}
\pmtitle{Tietze transform}
\pmrecord{5}{37658}
\pmprivacy{1}
\pmauthor{rspuzio}{6075}
\pmtype{Definition}
\pmcomment{trigger rebuild}
\pmclassification{msc}{20F10}
\pmdefines{elementary Tietze transformation}
\pmdefines{general Tietze transform}

\endmetadata

% this is the default PlanetMath preamble.  as your knowledge
% of TeX increases, you will probably want to edit this, but
% it should be fine as is for beginners.

% almost certainly you want these
\usepackage{amssymb}
\usepackage{amsmath}
\usepackage{amsfonts}

% used for TeXing text within eps files
%\usepackage{psfrag}
% need this for including graphics (\includegraphics)
%\usepackage{graphicx}
% for neatly defining theorems and propositions
%\usepackage{amsthm}
% making logically defined graphics
%%%\usepackage{xypic}

% there are many more packages, add them here as you need them

% define commands here
\begin{document}
\emph{Tietze transforms} are the following four transformations whereby one
can transform a presentation of a group into another presentation of
the same group:

\begin{enumerate}
\item If a relation $W=V$, where $W$ and $V$ are some word in the
generators of the group, can be derived from the defining relations of
a group, add $W=V$ to the list of relations.
\item If a relation $W=V$ can be derived from the remaining
generators, remove $W=V$ fronm the list of relations.
\item If $W$ is a word in the generators and $W=x$, then add $x$ to
the list of generators and $W=x$ to the list of relations.
\item If a relation takes the form $W=x$, where $x$ is a generator and
$W$ is a word in generators other than $x$, then remove $W=x$ from the
list of relations, replace all occurences of $x$ in the remaining
relations by $W$ and remove $x$ from the list of generators.
\end{enumerate}

Note that transforms 1 and 2 are inverse to each other and likewise 3
and 4 are inverses.  More generally, the term ``Tietze transform''
referes to a transform which can be expressed as the compositon of a
finite number of the four transforms listed above.  By way of
contrast, the term ``\emph{elementary Tietze transformation}'' is used
to denote the four transformations given above and the term
``\emph{general Tietze transform}'' could be used to indicate a member
of the larger class.

Tieze showed that any two presentations of the same finitely presented
group differ by a general Tietze transform.
%%%%%
%%%%%
\end{document}
