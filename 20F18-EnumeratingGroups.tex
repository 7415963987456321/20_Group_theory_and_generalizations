\documentclass[12pt]{article}
\usepackage{pmmeta}
\pmcanonicalname{EnumeratingGroups}
\pmcreated{2013-03-22 15:51:35}
\pmmodified{2013-03-22 15:51:35}
\pmowner{Algeboy}{12884}
\pmmodifier{Algeboy}{12884}
\pmtitle{enumerating groups}
\pmrecord{10}{37849}
\pmprivacy{1}
\pmauthor{Algeboy}{12884}
\pmtype{Definition}
\pmcomment{trigger rebuild}
\pmclassification{msc}{20F18}
%\pmkeywords{enumeration}
%\pmkeywords{p-group}
%\pmkeywords{nilpotent group}
\pmrelated{EnumeratingGraphs}
\pmrelated{EnumeratingAlgebras}

\endmetadata

\usepackage{latexsym}
\usepackage{amssymb}
\usepackage{amsmath}
\usepackage{amsfonts}
\usepackage{amsthm}

%%\usepackage{xypic}

%-----------------------------------------------------

%       Standard theoremlike environments.

%       Stolen directly from AMSLaTeX sample

%-----------------------------------------------------

%% \theoremstyle{plain} %% This is the default

\newtheorem{thm}{Theorem}

\newtheorem{coro}[thm]{Corollary}

\newtheorem{lem}[thm]{Lemma}

\newtheorem{lemma}[thm]{Lemma}

\newtheorem{prop}[thm]{Proposition}

\newtheorem{conjecture}[thm]{Conjecture}

\newtheorem{conj}[thm]{Conjecture}

\newtheorem{defn}[thm]{Definition}

\newtheorem{remark}[thm]{Remark}

\newtheorem{ex}[thm]{Example}



%\countstyle[equation]{thm}



%--------------------------------------------------

%       Item references.

%--------------------------------------------------


\newcommand{\exref}[1]{Example-\ref{#1}}

\newcommand{\thmref}[1]{Theorem-\ref{#1}}

\newcommand{\defref}[1]{Definition-\ref{#1}}

\newcommand{\eqnref}[1]{(\ref{#1})}

\newcommand{\secref}[1]{Section-\ref{#1}}

\newcommand{\lemref}[1]{Lemma-\ref{#1}}

\newcommand{\propref}[1]{Prop\-o\-si\-tion-\ref{#1}}

\newcommand{\corref}[1]{Cor\-ol\-lary-\ref{#1}}

\newcommand{\figref}[1]{Fig\-ure-\ref{#1}}

\newcommand{\conjref}[1]{Conjecture-\ref{#1}}


% Normal subgroup or equal.

\providecommand{\normaleq}{\unlhd}

% Normal subgroup.

\providecommand{\normal}{\lhd}

\providecommand{\rnormal}{\rhd}
% Divides, does not divide.

\providecommand{\divides}{\mid}

\providecommand{\ndivides}{\nmid}


\providecommand{\union}{\cup}

\providecommand{\bigunion}{\bigcup}

\providecommand{\intersect}{\cap}

\providecommand{\bigintersect}{\bigcap}
\begin{document}
\section{How many finite groups are there?}

The current tables list the number of groups up to order 2000 [Besche, Eick, O'Brien] (2000).

\begin{center}
\input{scatter-plot-logy}
\end{center}


The graph is chaotic -- both figuratively and mathematically.
Most groups are distributed along the interval at values $2^i m$ where $m$ is odd and $i$ large, for instance $i>5$.  Indeed most groups are actually of order $2^{10}=1024$.  We see this by connecting the dots of certain families of groups.
\begin{center}
\input{line-plots}
\end{center}

Most integers are square-free, most groups are not [Mays 1980; Miller 1930; 
Balas 1966].

An explanation for this distribution is offered by considering nilpotent groups.
Nilpotent groups are the product of their Sylow subgroups.  So enumerating
nilpotent groups asks to enumerating $p$-groups.  

\section{How many nilpotent groups are there?}

\begin{center}
\input{Nilpotent-plot-logy}
\end{center}

\begin{thm}[Pyber, 1993]
If $g_{nil}(N)$ is the number of nilpotent groups of order $\leq N$ and $g(N)$ the number
of groups of order $\leq N$ then 
\[\lim_{N\rightarrow \infty} \frac{\log g_{nil}(N)}{\log g(N)}=1.\]
\end{thm}
The proof bounds the number of groups with a given set of Sylow subgroups and involves
the Classification of Finite Simple Groups.


\begin{conj}[Pyber, 1993]
\[\lim_{N\rightarrow \infty} \frac{g_{nil}(N)}{g(N)}=1.\]
\end{conj}

If the conjecture is true, then most groups are 2-groups.

\section{The Higman and Sims bounds}

\begin{thm}[Higman 1960, Sims 1964]
The number of $p$-groups of order $p^n$, denoted, $f(p^n)$, satisfies
	\[\frac{2}{27}n^3 + C_1 n^2 \leq \log_p f(p^n)\leq \frac{2}{27}n^3 + C_2 n^{8/3}\]
for constants $C_1$ and $C_2$.
\end{thm}

\begin{center}
\input{Higman-Sims-plots}
\end{center}

This result should be compared to the later work of Neretin on enumerating algebras.  The lower bound is the work of Higman and is achieved by constructing a large family of class 2 $p$-groups (called $\Phi$-class 2 groups as $\Phi(\Phi(P))=1$ where $\Phi$ is the Frattini subgroup of $P$).

The $n^{8/3}$ factor has been improved to $o(n^{5/2})$ by M. Newman and
Seeley.  Sims' suggests that it should be possible to show
	\[\log_p f(p,n)\in \frac{2}{27}n^3 + O(n^2)\]
(with a positive leading coefficient) which would prove Pyber's conjecture [Shalev].

S. R. Blackburn's work (1992) on the number of class 3 p-groups 
provides strong evidence that this claim is true as he demonstrates that class 3 groups also attain this lower bound.  Since class 3 groups involve the Jacobi identity (Hall-Witt identity) it is plausible to expect class c, for c less than some fixed bound, will asymptotically achieve the lower bound as well.
%%%%%
%%%%%
\end{document}
