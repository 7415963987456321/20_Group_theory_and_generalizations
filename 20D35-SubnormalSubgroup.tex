\documentclass[12pt]{article}
\usepackage{pmmeta}
\pmcanonicalname{SubnormalSubgroup}
\pmcreated{2013-03-22 13:16:27}
\pmmodified{2013-03-22 13:16:27}
\pmowner{yark}{2760}
\pmmodifier{yark}{2760}
\pmtitle{subnormal subgroup}
\pmrecord{21}{33756}
\pmprivacy{1}
\pmauthor{yark}{2760}
\pmtype{Definition}
\pmcomment{trigger rebuild}
\pmclassification{msc}{20D35}
\pmclassification{msc}{20E15}
\pmsynonym{subinvariant subgroup}{SubnormalSubgroup}
\pmsynonym{attainable subgroup}{SubnormalSubgroup}
\pmrelated{SubnormalSeries}
\pmrelated{ClassificationOfFiniteNilpotentGroups}
\pmrelated{NormalSubgroup}
\pmrelated{CharacteristicSubgroup}
\pmrelated{FullyInvariantSubgroup}
\pmdefines{subnormal}
\pmdefines{subnormality}

\usepackage{amssymb}
\usepackage{amsmath}
\usepackage{amsfonts}

\def\normal{\triangleleft}
\def\normaleq{\trianglelefteq}
\def\sn{\operatorname{sn}}

\begin{document}
Let $G$ be a group, and $H$ a subgroup of $G$.
Then $H$ is a \emph{subnormal subgroup} of $G$ 
if there is a natural number $n$ and subgroups $H_0,\dots,H_n$ of $G$ 
such that $$H=H_0\normal H_1\normal\cdots\normal H_n=G,$$
where $H_i$ is a normal subgroup of $H_{i+1}$ for $i=0,\dots,n-1$.

Subnormality is a \PMlinkescapetext{strictly weaker condition than normality},
as normality of subgroups is not transitive.

We may write $H\sn G$ or $H\normal\normal\, G$ or $H\!\normaleq\normaleq G$ to indicate that $H$ is a subnormal subgroup of $G$.

In a nilpotent group, all subgroups are subnormal.

Subnormal subgroups are ascendant and descendant.
%%%%%
%%%%%
\end{document}
