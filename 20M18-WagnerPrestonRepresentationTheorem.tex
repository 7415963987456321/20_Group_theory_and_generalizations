\documentclass[12pt]{article}
\usepackage{pmmeta}
\pmcanonicalname{WagnerPrestonRepresentationTheorem}
\pmcreated{2013-03-22 16:11:16}
\pmmodified{2013-03-22 16:11:16}
\pmowner{Mazzu}{14365}
\pmmodifier{Mazzu}{14365}
\pmtitle{Wagner-Preston representation theorem}
\pmrecord{10}{38275}
\pmprivacy{1}
\pmauthor{Mazzu}{14365}
\pmtype{Theorem}
\pmcomment{trigger rebuild}
\pmclassification{msc}{20M18}
%\pmkeywords{Inverse Semigroups}
\pmdefines{representation by bijective partial maps}
\pmdefines{faithful representation}
\pmdefines{Wagner-Preston representation}

\endmetadata

% this is the default PlanetMath preamble.  as your knowledge
% of TeX increases, you will probably want to edit this, but
% it should be fine as is for beginners.

% almost certainly you want these
\usepackage{amssymb}
\usepackage{amsmath}
\usepackage{amsfonts}

% used for TeXing text within eps files
%\usepackage{psfrag}
% need this for including graphics (\includegraphics)
%\usepackage{graphicx}
% for neatly defining theorems and propositions
%\usepackage{amsthm}
% making logically defined graphics
%%%\usepackage{xypic}

% there are many more packages, add them here as you need them

% THEOREM Environments --------------------------------------------------

\newtheorem{thm}{Theorem}

% define commands here

\begin{document}
\PMlinkescapeword{representation}


\newcommand{\domi}{\mathrm{dom}}
\newcommand{\rang}{\mathrm{ran}}
\newcommand{\FFF}{\mathfrak{F}}
\newcommand{\III}{\mathfrak{I}}
\newcommand{\cbra}[1]{\left( #1 \right)}
\newcommand{\qbra}[1]{\left[ #1 \right]}
\newcommand{\gbra}[1]{\left\{ #1 \right\}}
\newcommand{\abra}[1]{\left\langle #1 \right\rangle}


Let $S$ be an inverse semigroup and $X$ a set. An inverse semigroup homomorphism $\phi:S\rightarrow\III(X)$, where $\III(X)$ denotes the symmetric inverse semigroup, is called a \emph{representation} of $S$ by bijective partial maps on $X$. The representation is said to be \emph{faithful} if $\phi$ is a monomorphism, i.e. it is injective.

Given $s\in S$, we define $\rho_s\in\III(S)$ as the bijective partial map with domain 
$$\domi(\rho_s)=Ss^{-1}=\gbra{ts^{-1}\,|\,t\in S}$$ 
and defined by
$$\rho_s(t)=ts,\ \ \forall t\in \domi(\rho_s).$$
Then the map $s\mapsto\rho_s$ is a representation  called the \emph{Wagner-Preston representation} of $S$. The following result, due to Wagner and Preston, is analogous to the Cayley representation theorem for groups.\\




\begin{thm}[\textbf{Wagner-Preston representation theorem}]
The Wagner-Preston representation of an inverse semigroup is faithful.
\end{thm}

\begin{thebibliography}{9}
\bibitem{b:petrich} N. Petrich, \emph{Inverse Semigroups}, Wiley, New York, 1984.
\bibitem{b:pres} G.B. Preston, \emph{Representation of inverse semi-groups}, J. London Math. Soc. 29 (1954), 411-419.
\end{thebibliography}
%%%%%
%%%%%
\end{document}
