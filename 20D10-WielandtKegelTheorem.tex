\documentclass[12pt]{article}
\usepackage{pmmeta}
\pmcanonicalname{WielandtKegelTheorem}
\pmcreated{2013-03-22 16:17:37}
\pmmodified{2013-03-22 16:17:37}
\pmowner{yark}{2760}
\pmmodifier{yark}{2760}
\pmtitle{Wielandt-Kegel theorem}
\pmrecord{11}{38412}
\pmprivacy{1}
\pmauthor{yark}{2760}
\pmtype{Theorem}
\pmcomment{trigger rebuild}
\pmclassification{msc}{20D10}
\pmsynonym{Kegel-Wielandt theorem}{WielandtKegelTheorem}

\endmetadata

\usepackage{amsthm}

\newtheorem*{thm*}{Theorem}
\begin{document}
\PMlinkescapeword{product}

\begin{thm*}
If a finite group is the product of two nilpotent subgroups,
then it is solvable.
\end{thm*}

That is, if $H$ and $K$ are nilpotent subgroups of a finite group $G$,
and $G=HK$, then $G$ is solvable.

This result can be considered as
a generalization of \PMlinkname{Burnside's $p$-$q$ Theorem}{BurnsidePQTheorem},
because if a group $G$ is of order $p^m q^n$, where $p$ and $q$ are distinct primes, then $G$ is the product of a \PMlinkname{Sylow $p$-subgroup}{SylowPSubgroup} and Sylow $q$-subgroup, both of which are nilpotent.
%%%%%
%%%%%
\end{document}
