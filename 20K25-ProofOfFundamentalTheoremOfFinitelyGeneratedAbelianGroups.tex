\documentclass[12pt]{article}
\usepackage{pmmeta}
\pmcanonicalname{ProofOfFundamentalTheoremOfFinitelyGeneratedAbelianGroups}
\pmcreated{2013-03-22 18:24:58}
\pmmodified{2013-03-22 18:24:58}
\pmowner{puuhikki}{9774}
\pmmodifier{puuhikki}{9774}
\pmtitle{proof of fundamental theorem of finitely generated abelian groups}
\pmrecord{7}{41066}
\pmprivacy{1}
\pmauthor{puuhikki}{9774}
\pmtype{Proof}
\pmcomment{trigger rebuild}
\pmclassification{msc}{20K25}

% this is the default PlanetMath preamble.  as your knowledge
% of TeX increases, you will probably want to edit this, but
% it should be fine as is for beginners.

% almost certainly you want these
\usepackage{amssymb}
\usepackage{amsmath}
\usepackage{amsfonts}

% used for TeXing text within eps files
%\usepackage{psfrag}
% need this for including graphics (\includegraphics)
%\usepackage{graphicx}
% for neatly defining theorems and propositions
%\usepackage{amsthm}
% making logically defined graphics
%%%\usepackage{xypic}

% there are many more packages, add them here as you need them

% define commands here

\begin{document}
Every finitely generated abelian group $A$ is a direct sum of its cyclic subgroups, i.e. 
\[A=C_{m_1}\oplus C_{m_2}\oplus\ldots\oplus C_{m_k}\oplus \mathbb{Z}\oplus\ldots\oplus\mathbb{Z},\]
where\, $1<m_1 \mid m_2 \mid \ldots \mid m_k$.  The numbers $m_i$ are uniquely determined as well as the number of $\mathbb{Z}$'s, which is the rank of an abelian group.

Proof. Let $G$ be an abelian group with $n$ generators. Then for a free group $F_n$, $G$ is isomorphic to the quotient group $F_n/A$.  Now $F_n$ and $A$ contain a basis $f_1,\ldots,f_n$ and $a_1,\ldots,a_k$ satisfying $a_i=m_if_i$ for all $1\leq i\leq k$.  As $G\cong F_n/A$, it suffices to show that $F_n/A$ is a direct sum of its cyclic subgroups $\langle f_1+A\rangle$.

It is clear that $F_n/A$ is generated by its subgroups $\langle f_i+A\rangle$. Assume that the zero element of $F_n/A$ can be written as a form $A=l_1f_1+\ldots+l_nf_n+A$.  It follows that $l_1f_1+\ldots+l_nf_n=a\in A$. As we write $a$ as a linear combination of that basis and using $a_i=m_if_i$ we get the equations
\[l_1f_1+\ldots+l_nf_n=s_1a_1+\ldots s_ka_k=s_1m_1f_1+\ldots +s_km_kf_k.\]
As every element can be represented uniquely as a linear combination of its free generators $f_1$, we have $l_i=s_im_i$ for every $1\leq i\leq k$ and $l_j=0$ for every $k<j\leq n$.

This means that every element $l_if_i$ belongs to $A$, so\, $l_if_i+A=A$.  Therefore the zero element has a unique representation as a sum of the elements of the subgroup $\langle f_i+\!A\rangle$.

\begin{thebibliography}{9}
\bibitem{P.P.}{\sc P. Paajanen:} {\em Ryhm\"ateoria}.\, Lecture notes, Helsinki university, Finland (fall 2008)
\end{thebibliography}
%%%%%
%%%%%
\end{document}
