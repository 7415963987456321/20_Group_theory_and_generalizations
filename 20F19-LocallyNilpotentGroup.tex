\documentclass[12pt]{article}
\usepackage{pmmeta}
\pmcanonicalname{LocallyNilpotentGroup}
\pmcreated{2013-03-22 15:40:42}
\pmmodified{2013-03-22 15:40:42}
\pmowner{yark}{2760}
\pmmodifier{yark}{2760}
\pmtitle{locally nilpotent group}
\pmrecord{7}{37619}
\pmprivacy{1}
\pmauthor{yark}{2760}
\pmtype{Definition}
\pmcomment{trigger rebuild}
\pmclassification{msc}{20F19}
\pmrelated{LocallyCalP}
\pmrelated{NilpotentGroup}
\pmrelated{NormalizerCondition}
\pmdefines{locally nilpotent}
\pmdefines{Hirsch-Plotkin radical}
\pmdefines{locally nilpotent radical}

\endmetadata

\usepackage{amsmath}
\usepackage{amsfonts}

\DeclareMathOperator{\Dih}{Dih}
\DeclareMathOperator{\HP}{HP}
\def\Z{\mathbb{Z}}
\begin{document}
\PMlinkescapeword{finite}
\PMlinkescapeword{satisfies}

\section*{Definition}

A \emph{locally nilpotent group} is 
a group in which every finitely generated subgroup is nilpotent.

\section*{Examples}

All nilpotent groups are locally nilpotent, 
because subgroups of nilpotent groups are nilpotent.

An example of a locally nilpotent group that is not nilpotent 
is $\Dih(\Z(2^\infty))$, the generalized dihedral group 
formed from the quasicyclic \PMlinkname{$2$-group}{PGroup4} $\Z(2^\infty)$.

The Fitting subgroup of any group is locally nilpotent.

All N-groups are locally nilpotent. More generally, all Gruenberg groups are locally nilpotent.

\section*{Properties}

Any subgroup or \PMlinkname{quotient}{QuotientGroup} of a locally nilpotent group is locally nilpotent.
Restricted direct products of locally nilpotent groups are locally nilpotent.

For each prime $p$, 
the elements of $p$-power order in a locally nilpotent group 
form a fully invariant subgroup
(the maximal \PMlinkname{$p$-subgroup}{PGroup4}).
The elements of finite order in a locally nilpotent group 
also form a fully invariant subgroup (the torsion subgroup), 
which is the restricted direct product of the maximal $p$-subgroups.
(This generalizes the fact that a finite nilpotent group 
is the direct product of its Sylow subgroups.)

Every group $G$ has a unique maximal locally nilpotent normal subgroup.
This subgroup is called the \emph{Hirsch-Plotkin radical}, 
or \emph{locally nilpotent radical}, and is often denoted $\HP(G)$.
If $G$ is finite (or, more generally, satisfies the maximal condition),
then the Hirsch-Plotkin radical is the same as the Fitting subgroup,
and is nilpotent.
%%%%%
%%%%%
\end{document}
