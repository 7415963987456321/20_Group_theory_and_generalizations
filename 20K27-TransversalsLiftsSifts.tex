\documentclass[12pt]{article}
\usepackage{pmmeta}
\pmcanonicalname{TransversalsLiftsSifts}
\pmcreated{2013-03-22 15:53:52}
\pmmodified{2013-03-22 15:53:52}
\pmowner{Algeboy}{12884}
\pmmodifier{Algeboy}{12884}
\pmtitle{transversals / lifts / sifts}
\pmrecord{11}{37900}
\pmprivacy{1}
\pmauthor{Algeboy}{12884}
\pmtype{Definition}
\pmcomment{trigger rebuild}
\pmclassification{msc}{20K27}
%\pmkeywords{transversal}
%\pmkeywords{lift}
%\pmkeywords{sift}
\pmrelated{SchreiersLemma}
\pmrelated{ExampleOfSchreiersLemma}
\pmdefines{transversal}
\pmdefines{lift}
\pmdefines{sift}

\usepackage{latexsym}
\usepackage{amssymb}
\usepackage{amsmath}
\usepackage{amsfonts}
\usepackage{amsthm}

%%\usepackage{xypic}

%-----------------------------------------------------

%       Standard theoremlike environments.

%       Stolen directly from AMSLaTeX sample

%-----------------------------------------------------

%% \theoremstyle{plain} %% This is the default

\newtheorem{thm}{Theorem}

\newtheorem{coro}[thm]{Corollary}

\newtheorem{lem}[thm]{Lemma}

\newtheorem{lemma}[thm]{Lemma}

\newtheorem{prop}[thm]{Proposition}

\newtheorem{conjecture}[thm]{Conjecture}

\newtheorem{conj}[thm]{Conjecture}

\newtheorem{defn}[thm]{Definition}

\newtheorem{remark}[thm]{Remark}

\newtheorem{ex}[thm]{Example}



%\countstyle[equation]{thm}



%--------------------------------------------------

%       Item references.

%--------------------------------------------------


\newcommand{\exref}[1]{Example-\ref{#1}}

\newcommand{\thmref}[1]{Theorem-\ref{#1}}

\newcommand{\defref}[1]{Definition-\ref{#1}}

\newcommand{\eqnref}[1]{(\ref{#1})}

\newcommand{\secref}[1]{Section-\ref{#1}}

\newcommand{\lemref}[1]{Lemma-\ref{#1}}

\newcommand{\propref}[1]{Prop\-o\-si\-tion-\ref{#1}}

\newcommand{\corref}[1]{Cor\-ol\-lary-\ref{#1}}

\newcommand{\figref}[1]{Fig\-ure-\ref{#1}}

\newcommand{\conjref}[1]{Conjecture-\ref{#1}}


% Normal subgroup or equal.

\providecommand{\normaleq}{\unlhd}

% Normal subgroup.

\providecommand{\normal}{\lhd}

\providecommand{\rnormal}{\rhd}
% Divides, does not divide.

\providecommand{\divides}{\mid}

\providecommand{\ndivides}{\nmid}


\providecommand{\union}{\cup}

\providecommand{\bigunion}{\bigcup}

\providecommand{\intersect}{\cap}

\providecommand{\bigintersect}{\bigcap}










\begin{document}
\begin{defn}
Given a group $G$ and a subgroup $H$ of $G$, a \emph{transversal} of $H$ in $G$
is a subset $T\subseteq G$ such that for every $g\in G$ there exists a unique
$t\in T$ such that $Hg=Ht$.
\end{defn}

Typically one insists $1\in T$ so that the coset $H$ is described uniquely
by $H1$.  However no standard terminology has emerged for transversals of this sort.  

An alternative definition for a transversal is to use functions and homomorphisms in a method more conducive to a categorical setting.  Here one replaces the notion of a transversal as a subset of $G$ and instead treats it as a certain type of map $T:G/H\rightarrow G$.  Since $H$ is generally not normal in $G$, $G/H$ simply means the set of cosets, and $T$ is therefore a function not a homomorphism.  We only require that $T$ satisfy the following property:  Given the canonical projection map $\pi:G\rightarrow G/H$ given by $g\mapsto Hg$ (this is generally not a homomorphism either, and so both $\pi$ and $T$ are simply functions between sets) then $\pi T=1_{G/H}$.  It follows immediately that the image of $T$ in $G$ is a transversal in the original sense of the term.

\begin{remark}
Because it is customary in group theory to write actions to the right of elements many times it is preferable to write $T\pi=1_{G/H}$ to match the right side notation.
\end{remark}

When $H$ is a normal subgroup of $G$ our terminology adjusts from transversals to \emph{lifts}.

\begin{defn}
Given a group $G$ and a homomorphism $\pi:G\rightarrow Q$, a \emph{lift} of $Q$ to $G$ is a function $f:Q\rightarrow G$ such that $\pi f=1_Q$.
\end{defn}

It follows that $\pi$ must be an epimorphism if it has a lift.  Once again it is nearly always requested that $f(1)=1$ but this restriction is generally not part of the definition.

Because both lifts and transversals are injective mappings it is common to use the word lift/transversal for the image and the map with the context of the use providing any necessary clarification.

\begin{defn}
Given a group $G$ and a homomorphism $\pi:G\rightarrow Q$, a \emph{splitting map} of $Q$ to $G$ is a \emph{homomorphism} $f:Q\rightarrow G$ such that $\pi f=1_Q$.
\end{defn}

So we see a gradual progression in the definitions:  We always have a group $G$ and a set $Q$, and the maps $\pi:G\rightarrow Q$, $f:Q\rightarrow G$ satisfying
\[ \pi f=1_{Q}.\]
It follows, $f$ is injective and $\pi$ is surjective.
\begin{itemize}
\item $f$ is a transversal if $Q=G/H$ for some subgroup $H$. Here
$\pi$ and $f$ are simply functions.
\item $f$ is a lift if $Q$ is a group.  Here $\pi$ is a homomorphism and
$f$ a function.
\item $f$ is a splitting map if $Q$ is group and both $\pi$ and $f$ are
homomorphisms.
\end{itemize}

Finally we arrive at a stronger requirement for transversals and lifts which makes greater use of the group structure involved.

\begin{defn}
Given a group $G=\langle S\rangle$, there is a natural map
$\pi:F(S)\rightarrow G$ from the free group on $S$ onto $G$.  A \emph{lift}
is a map $l:G\rightarrow F(S)$ such that $\pi l =1_G$.  Furthermore a
\emph{sift} is a lift $s:G\rightarrow F(S)$ with the added condition that
$sg=g$ for all $g\in S$.
\end{defn}

Although a general sift is no more than a map that writes the 
elements of $G$ as reduced words in $S$, in many cases the sifts
have the added property of providing the words in a canonical form.
This occurs when $G=T_0\cdots T_{n-1}$ where $T_i$ is a transversal
of $G^i/G^{i+1}$.  In such a case every element in $G$ has a unique 
decomposition as a word $t_0t_1\cdots t_{n-1}$ for unique $t_i\in T_i$.

%%%%%
%%%%%
\end{document}
