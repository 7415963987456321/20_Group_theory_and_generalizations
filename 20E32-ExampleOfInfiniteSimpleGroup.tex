\documentclass[12pt]{article}
\usepackage{pmmeta}
\pmcanonicalname{ExampleOfInfiniteSimpleGroup}
\pmcreated{2013-03-22 16:53:31}
\pmmodified{2013-03-22 16:53:31}
\pmowner{rspuzio}{6075}
\pmmodifier{rspuzio}{6075}
\pmtitle{example of infinite simple group}
\pmrecord{8}{39148}
\pmprivacy{1}
\pmauthor{rspuzio}{6075}
\pmtype{Example}
\pmcomment{trigger rebuild}
\pmclassification{msc}{20E32}
\pmclassification{msc}{20D06}

\endmetadata

% this is the default PlanetMath preamble.  as your knowledge
% of TeX increases, you will probably want to edit this, but
% it should be fine as is for beginners.

% almost certainly you want these
\usepackage{amssymb}
\usepackage{amsmath}
\usepackage{amsfonts}

% used for TeXing text within eps files
%\usepackage{psfrag}
% need this for including graphics (\includegraphics)
%\usepackage{graphicx}
% for neatly defining theorems and propositions
%\usepackage{amsthm}
% making logically defined graphics
%%%\usepackage{xypic}

% there are many more packages, add them here as you need them

% define commands here

\begin{document}
This fact that finite alternating groups are simple can be extended to a result about an infinite group.  Let $G$ be the subgroup of the group of permutations on a countably infinite set $M$ (which we may take to be the set of natural numbers for concreteness) which is generated by cycles of length $3$.  Note that any since every element of this group is a product of a finite number of cycles, the permutations of $G$ are such that only a finite number of elements of our set are not mapped to themselves by a given permutation.

We will now show that $G$ is simple.  Suppose that $\pi$ is an element of $G$ other than the identity.  Let $m$ be the set of all $x$ such that $\pi (x) \neq x$.  By our previous comment, $m$ is finite.  Consider the restriction $\pi_m$ of $\pi$ to $m$.  By the theorem of the \PMlinkname{parent entry}{SimplicityOfA_n}, the subgroup of $A_m$ generated by the conjugates of $\pi_m$ is the whole of $A_m$.  In particular, this means that there exists a cycle of order $3$ in $A_m$ which can be expressed as a product of $\pi_m$ and its conjugates.  Hence the subgroup of $G$ generated by conjugates of $\pi$ contains a cycle of length three as well.  However, every cycle of order $3$ is conjugate to every other cycle of order $3$ so, in fact, the subgroup of $G$ generated by the conjugates of $\pi$ is the whole of $G$.  Hence, the only normal subgroups of $G$ are the group consisting of solely the identity element and the whole of $G$, so $G$ is a simple group.
%%%%%
%%%%%
\end{document}
