\documentclass[12pt]{article}
\usepackage{pmmeta}
\pmcanonicalname{AutomaticGroup}
\pmcreated{2013-03-22 14:16:54}
\pmmodified{2013-03-22 14:16:54}
\pmowner{mathcam}{2727}
\pmmodifier{mathcam}{2727}
\pmtitle{automatic group}
\pmrecord{7}{35735}
\pmprivacy{1}
\pmauthor{mathcam}{2727}
\pmtype{Definition}
\pmcomment{trigger rebuild}
\pmclassification{msc}{20F10}
\pmrelated{AutomaticPresentation}
\pmdefines{automatic semigroup}
\pmdefines{automatic structure}

% this is the default PlanetMath preamble.  as your knowledge
% of TeX increases, you will probably want to edit this, but
% it should be fine as is for beginners.

% almost certainly you want these
\usepackage{amssymb}
\usepackage{amsmath}
\usepackage{amsfonts}

% used for TeXing text within eps files
%\usepackage{psfrag}
% need this for including graphics (\includegraphics)
%\usepackage{graphicx}
% for neatly defining theorems and propositions
%\usepackage{amsthm}
% making logically defined graphics
%%%\usepackage{xypic}

% there are many more packages, add them here as you need them

% define commands here
\begin{document}
Let $G$ be a finitely generated group.  Let $A$ be a finite generating set for $G$ \PMlinkescapetext{closed} under inverses.

$G$ is an \emph{automatic group} if there is a language $L\subseteq A^*$ and a surjective map $f:L\rightarrow G$ such that
\begin{itemize}
\item $L$ can be checked by a \PMlinkname{finite automaton}{DeterministicFiniteAutomaton}
\item The language of all convolutions of $x,y$ where $f(x)=f(y)$ can be checked by a \PMlinkescapetext{finite automaton}
\item For each $a\in A$, the language of all convolutions of $x,y$ where $f(x).a=f(y)$ can be checked by a \PMlinkescapetext{finite automaton}
\end{itemize}

$(A, L)$ is said to be an \emph{automatic structure} for $G$.

Note that by taking a finitely generated semigroup $S$, and some finite generating set $A$, these conditions define an \emph{automatic semigroup}.
%%%%%
%%%%%
\end{document}
