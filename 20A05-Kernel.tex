\documentclass[12pt]{article}
\usepackage{pmmeta}
\pmcanonicalname{Kernel}
\pmcreated{2013-03-22 11:58:24}
\pmmodified{2013-03-22 11:58:24}
\pmowner{rmilson}{146}
\pmmodifier{rmilson}{146}
\pmtitle{kernel}
\pmrecord{14}{30812}
\pmprivacy{1}
\pmauthor{rmilson}{146}
\pmtype{Definition}
\pmcomment{trigger rebuild}
\pmclassification{msc}{20A05}
\pmsynonym{kernel of a group homomorphism}{Kernel}
\pmrelated{GroupHomomorphism}
\pmrelated{Kernel}
\pmrelated{AHomomorphismIsInjectiveIffTheKernelIsTrivial}

\endmetadata

\usepackage{amssymb}
\usepackage{amsmath}
\usepackage{amsfonts}
\begin{document}
Let $\rho :G\to K$ be a group homomorphism. The preimage of the
codomain identity element $e_K\in K$ forms a subgroup of the domain
$G$, called the \emph{kernel} of the homomorphism;
$$\operatorname{ker}(\rho)= \{ s \in G\mid\rho   (s)=e_K\} $$

The kernel is a normal subgroup.  It is the trivial subgroup if and
only if $\rho$ is a monomorphism.
%%%%%
%%%%%
%%%%%
\end{document}
