\documentclass[12pt]{article}
\usepackage{pmmeta}
\pmcanonicalname{Unity}
\pmcreated{2013-03-22 14:47:17}
\pmmodified{2013-03-22 14:47:17}
\pmowner{pahio}{2872}
\pmmodifier{pahio}{2872}
\pmtitle{unity}
\pmrecord{15}{36439}
\pmprivacy{1}
\pmauthor{pahio}{2872}
\pmtype{Definition}
\pmcomment{trigger rebuild}
\pmclassification{msc}{20-00}
\pmclassification{msc}{16-00}
\pmclassification{msc}{13-00}
\pmsynonym{multiplicative identity}{Unity}
\pmsynonym{characterization of unity}{Unity}
\pmrelated{ZeroDivisor}
\pmrelated{RootOfUnity}
\pmrelated{ZeroRing}
\pmrelated{NonZeroDivisorsOfFiniteRing}
\pmrelated{OppositePolynomial}
\pmdefines{non-zero unity}
\pmdefines{nonzero unity}

\endmetadata

% this is the default PlanetMath preamble.  as your knowledge
% of TeX increases, you will probably want to edit this, but
% it should be fine as is for beginners.

% almost certainly you want these
\usepackage{amssymb}
\usepackage{amsmath}
\usepackage{amsfonts}

% used for TeXing text within eps files
%\usepackage{psfrag}
% need this for including graphics (\includegraphics)
%\usepackage{graphicx}
% for neatly defining theorems and propositions
 \usepackage{amsthm}
% making logically defined graphics
%%%\usepackage{xypic}

% there are many more packages, add them here as you need them

% define commands here
\theoremstyle{definition}
\newtheorem*{thmplain}{Theorem}
\begin{document}
The {\em unity} of a ring \,$(R,\,+,\,\cdot)$\, is the multiplicative identity of the ring, if it has such. \,The unity is often denoted by $e$, $u$ or 1. \,Thus, the unity satisfies
   $$e\cdot a \;=\; a\cdot e \;=\; a\quad\forall a\in R.$$

If $R$ consists of the mere 0, then $0$ is its unity, since in every ring, \,$0\cdot a = a\cdot 0 = 0$. \,Conversely, if 0 is the unity in some ring $R$, then \,$R = \{0\}$\, (because \,$a = 0\cdot a = 0\,\,\forall a\in R$).

\textbf{Note.} \,When considering a ring $R$ it is often mentioned that ``...having $1 \neq 0$'' or that ``...with non-zero unity'', sometimes only ``...with unity'' or ``...with \PMlinkescapetext{identity element}''; all these exclude the case \,$R = \{0\}$.

\begin{thmplain}
\, An element $u$ of a ring $R$ is the unity iff $u$ is an idempotent and regular element.
\end{thmplain}

{\em Proof.} \,Let $u$ be an idempotent and regular element. \,For any element $x$ of $R$ we have
               $$ux \;=\; u^2x \;=\; u(ux),$$
and because $u$ is no left zero divisor, it may be cancelled from the equation; thus we get \,$x = ux$. \,Similarly,\, $x = xu$. \,So $u$ is the unity of the ring.\, The other half of the \PMlinkescapetext{theorem} is apparent.
%%%%%
%%%%%
\end{document}
