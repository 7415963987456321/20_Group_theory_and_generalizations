\documentclass[12pt]{article}
\usepackage{pmmeta}
\pmcanonicalname{Equation}
\pmcreated{2013-03-22 15:28:33}
\pmmodified{2013-03-22 15:28:33}
\pmowner{pahio}{2872}
\pmmodifier{pahio}{2872}
\pmtitle{equation}
\pmrecord{30}{37330}
\pmprivacy{1}
\pmauthor{pahio}{2872}
\pmtype{Definition}
\pmcomment{trigger rebuild}
\pmclassification{msc}{20N02}
%\pmkeywords{equation}
%\pmkeywords{root}
%\pmkeywords{solution}
\pmrelated{Equality2}
\pmrelated{AlgebraicEquation}
\pmrelated{DiophantineEquation}
\pmrelated{TrigonometricEquation}
\pmrelated{DifferenceEquation}
\pmrelated{DifferentialEquation}
\pmrelated{IntegralEquation}
\pmrelated{FunctionalEquation}
\pmrelated{HomogeneousEquation}
\pmrelated{ProportionEquation}
\pmrelated{FiniteDifference}
\pmrelated{RecurrenceRelation}
\pmrelated{CharacteristicEquation}
\pmdefines{equate}
\pmdefines{side}
\pmdefines{root}
\pmdefines{solution}
\pmdefines{root of an equation}
\pmdefines{left hand side}
\pmdefines{right hand side}
\pmdefines{multiplicity of the root}
\pmdefines{order of the root}
\pmdefines{multiple root}

% this is the default PlanetMath preamble.  as your knowledge
% of TeX increases, you will probably want to edit this, but
% it should be fine as is for beginners.

% almost certainly you want these
\usepackage{amssymb}
\usepackage{amsmath}
\usepackage{amsfonts}

% used for TeXing text within eps files
%\usepackage{psfrag}
% need this for including graphics (\includegraphics)
%\usepackage{graphicx}
% for neatly defining theorems and propositions
 \usepackage{amsthm}
% making logically defined graphics
%%%\usepackage{xypic}

% there are many more packages, add them here as you need them

% define commands here

\theoremstyle{definition}
\newtheorem*{thmplain}{Theorem}

\begin{document}
\textbf{Equation}

An {\em equation} concerns usually elements of a certain set $M$, where one can say if two elements are equal.\, In the simplest case, $M$ has one binary operation ``$*$'' producing as result some elements of $M$, and these can be compared.\, Then, an equation in\, $(M,\,*)$\, is a proposition of the form
\begin{align}
                             E_1 = E_2,
\end{align}
where one has {\em equated} two expressions $E_1$ and $E_2$ formed with ``$*$'' of the elements or indeterminates of $M$.\, We call the expressions $E_1$ and $E_2$ respectively the {\em left hand side} and the {\em right hand side} of the equation (1).\\

\textbf{Example.}\, Let $S$ be a set and $2^S$ the set of its subsets.\, In the groupoid\, $(2^S,\,\smallsetminus)$,\, where ``$\smallsetminus$'' is the set difference, we can write the equation
                $$(A\!\smallsetminus\!B)\!\smallsetminus\!B = A\!\smallsetminus\!B$$
(which is always true).\\

Of course, $M$ may be equipped with more operations or be a module with some ring of multipliers --- then an equation (1) may \PMlinkescapetext{contain} them.

But one need not assume any algebraic structure for the set $M$ where the expressions $E_1$ and $E_2$ are values or where they \PMlinkescapetext{represent generic} elements.\, Such a situation would occur e.g. if one has a continuous mapping $f$ from a topological space $L$ to another $M$; then one can consider an equation
$$f(x) = y.$$
A somewhat \PMlinkescapetext{comparable} case is the equation
$$\dim{V} = 2$$ 
where $V$ is a certain or a \PMlinkescapetext{generic} vector space; both \PMlinkescapetext{sides represent} elements of the extended real number system.\\

\textbf{Root of equation}

If an equation (1) in $M$ \PMlinkescapetext{contains} one indeterminate, say $x$, then a value of $x$ which satisfies (1), i.e. makes it true, is called a {\em root} or a {\em solution} of the equation.
Especially, if we have a polynomial equation\, $f(x) = 0$,\, we may speak of the \PMlinkescapetext{{\em multiplicity}} or the {\em \PMlinkescapetext{order of a root}} $x_0$; it is the multiplicity of the zero $x_0$ of the polynomial $f(x)$.  A {\em multiple root} has multiplicity greater than 1.\\

\textbf{Example.}\, The equation
                           $$x^2\!+\!1 = x$$
in the system $\mathbb{C}$ of the complex numbers has as its roots the numbers
      $$x := \frac{1\!\pm\!i\sqrt{3}}{2},$$
which, by the way, are the primitive sixth roots of unity.\, Their multiplicities are 1.
%%%%%
%%%%%
\end{document}
