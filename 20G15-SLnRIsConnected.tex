\documentclass[12pt]{article}
\usepackage{pmmeta}
\pmcanonicalname{SLnRIsConnected}
\pmcreated{2013-03-22 18:52:05}
\pmmodified{2013-03-22 18:52:05}
\pmowner{Stephaninos}{23208}
\pmmodifier{Stephaninos}{23208}
\pmtitle{SL(n;R) is connected}
\pmrecord{5}{41708}
\pmprivacy{1}
\pmauthor{Stephaninos}{23208}
\pmtype{Result}
\pmcomment{trigger rebuild}
\pmclassification{msc}{20G15}

\endmetadata

% this is the default PlanetMath preamble.  as your knowledge
% of TeX increases, you will probably want to edit this, but
% it should be fine as is for beginners.

% almost certainly you want these
\usepackage{amssymb}
\usepackage{amsmath}
\usepackage{amsfonts}
\usepackage{amsthm, graphicx}



% used for TeXing text within eps files
%\usepackage{psfrag}
% need this for including graphics (\includegraphics)
%\usepackage{graphicx}
% for neatly defining theorems and propositions
%\usepackage{amsthm}
% making logically defined graphics
%%%\usepackage{xypic}

% there are many more packages, add them here as you need them

% define commands here

\newtheorem{thm}{Theorem}[section]
\newtheorem{lem}[thm]{Lemma}
\newtheorem{prop}[thm]{Proposition}
\newtheorem{cor}[thm]{Corollary}

\theoremstyle{definition}
\newtheorem{defn}{Definition}[section]
\newtheorem{conj}{Conjecture}[section]
\newtheorem{exmp}{Example}[section]

\newcommand{\tr}{\ensuremath{\operatorname{tr}}}
\newcommand{\id}{\ensuremath{\operatorname{id}}}
\begin{document}
The special feature is that although not every element of $SL(n,\mathbb{R})$ is in the image of the exponential map of $\mathfrak{sl}(n,\mathbb{R})$, $SL(n,\mathbb{R})$ is still a connected Lie group. The proof below is a guideline and should be clarified a bit more at some points, but this was done intentionally.\\

To illustrate the point, first we show
\begin{prop} 
$\begin{pmatrix}
	-1 & 1 \\
	0 & -1
\end{pmatrix}
\notin \exp{\mathfrak{sl}(2,\mathbb{R})}$, but it is in $SL(2,\mathbb{R})$.
\end{prop}

\begin{proof}
$\det x =: \det \begin{pmatrix}
	-1 & 1 \\
	0 & -1
\end{pmatrix} = 1$, so $x\in SL(2,\mathbb{R})$. We see that $x$ is not diagonalizable, it already is in Jordan normal form. Moreover, it has a double eigenvalue, $-1$. Suppose that $x=\exp X, X \in \mathfrak{sl}(2,\mathbb{R})$, then $\tr{X}=0$. Since $x$ had a double eigenvalue, so does $X$, hence the eigenvalues of $X$ both are $0$. But this implies the eigenvalues of $x$ are $1$. This is a contradiction.
\end{proof}

\begin{lem}
We have $\forall x\in SL(n,\mathbb{R}): x = \exp(X_a)\exp(X_s)$ with $X_a^t=-X_a, X_s^t=X_s\in \mathfrak{sl}(n,\mathbb{R})$.
\end{lem}

\begin{proof}
The keyword here is \textit{polar decomposition}.
We notice that $x^tx$ is symmetric and positive definite, since $\forall \psi\in\mathbb{R}^n: \langle \psi,x^tx\psi\rangle >0$, with the standard inner product on $\mathbb{R}^n$. Hence, we can write $x=RP$, with $P=(x^tx)^\frac{1}{2}$ and $R=xP^{-1}$. $P$ is well defined, since any real symmetric, positive definite matrix is diagonalizable. It's easy to check that $RR^t=\id_n$, hence $R\in O(n)$. We had $\det{P}>0$ and $\det{x}=1$, hence $\det(R)>0 \Rightarrow \det{R}=1 \Rightarrow R\in SO(n\mathbb{n})$ and so $\det{P}=1$. Since the choice of positive root is unique, $R$ and $P$ are unique. Moreover, $SO(n)$ is exactly generated by the set $\{ X\in GL(n,\mathbb{R}) | X^t=-X \}$ and $\Omega$, the set of real symmetric matrices of determinant $1$, by $\{ X\in GL(n,\mathbb{R}) | X^t=X, \tr{X}=0 \}$, we have the wanted statement: $SL(n,\mathbb{R} \subset SO(n)\times \exp{\Omega}$.
\end{proof}

The reverse inclusion is simply shown: any such combination is trivially in $SL(n,\mathbb{R})$. 
\begin{cor}
$SL(n,\mathbb{R})$ is connected.
\end{cor}
\begin{proof}
This is now clear from the fact that both $SO(n)$ and $\Omega$ are connected and so $\forall s,t\in [0,1]: \exp{sX}\exp{tY}\in SL(n,\mathbb{R})$, a fact easily checked by taking the determinant. So $SL(n,\mathbb{R})$ is path-connected, hence connected.
\end{proof}
%%%%%
%%%%%
\end{document}
