\documentclass[12pt]{article}
\usepackage{pmmeta}
\pmcanonicalname{PropertiesOfConjugacy}
\pmcreated{2013-03-22 18:56:35}
\pmmodified{2013-03-22 18:56:35}
\pmowner{pahio}{2872}
\pmmodifier{pahio}{2872}
\pmtitle{properties of conjugacy}
\pmrecord{5}{41799}
\pmprivacy{1}
\pmauthor{pahio}{2872}
\pmtype{Topic}
\pmcomment{trigger rebuild}
\pmclassification{msc}{20A05}
%\pmkeywords{conjugacy}
\pmrelated{NormalClosure2}
\pmrelated{NonIsomorphicGroupsOfGivenOrder}
\pmdefines{normal closure}

\endmetadata

% this is the default PlanetMath preamble.  as your knowledge
% of TeX increases, you will probably want to edit this, but
% it should be fine as is for beginners.

% almost certainly you want these
\usepackage{amssymb}
\usepackage{amsmath}
\usepackage{amsfonts}

% used for TeXing text within eps files
%\usepackage{psfrag}
% need this for including graphics (\includegraphics)
%\usepackage{graphicx}
% for neatly defining theorems and propositions
 \usepackage{amsthm}
% making logically defined graphics
%%%\usepackage{xypic}

% there are many more packages, add them here as you need them

% define commands here

\theoremstyle{definition}
\newtheorem*{thmplain}{Theorem}

\begin{document}
Let $S$ be a nonempty subset of a group $G$.\, When $g$ is an element of $G$, a conjugate of $S$ is the subset
$$gSg^{-1} \;=\; \{gsg^{-1}\,\vdots\;\; s \in S\}.$$
We denote here
\begin{align}
gSg^{-1} \;:=\; S^g.
\end{align}

If $T$ is another nonempty subset and $h$ another element of $G$, then it's easily verified the formulae
\begin{itemize}
\item $(ST)^g \;=\; S^gT^g$
\item $(S^g)^h \;=\; S^{gh}$
\end{itemize}

The conjugates $H^g$ of a subgroup $H$ of $G$ are subgroups of $G$, since any mapping
$$x \mapsto gxg^{-1}$$
is an automorphism (an inner automorphism) of $G$ and the homomorphic image of group is always a group.\\

The notation (1) can be extended to
\begin{align}
\langle S^g\,\vdots\;\; g \in G\rangle \;:=\; S^G
\end{align}
where the angle parentheses express a generated subgroup.\, $S^G$ is the least normal subgroup of $G$ containing the subset $S$, and it is called the \emph{normal closure} of $S$.\\


\PMlinkexternal{Wiki}{http://en.wikipedia.org/wiki/Conjugacy}


%%%%%
%%%%%
\end{document}
