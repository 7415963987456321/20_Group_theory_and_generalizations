\documentclass[12pt]{article}
\usepackage{pmmeta}
\pmcanonicalname{GeneralizedDihedralGroup}
\pmcreated{2013-03-22 14:53:28}
\pmmodified{2013-03-22 14:53:28}
\pmowner{yark}{2760}
\pmmodifier{yark}{2760}
\pmtitle{generalized dihedral group}
\pmrecord{9}{36572}
\pmprivacy{1}
\pmauthor{yark}{2760}
\pmtype{Definition}
\pmcomment{trigger rebuild}
\pmclassification{msc}{20E22}
\pmsynonym{generalised dihedral group}{GeneralizedDihedralGroup}
\pmrelated{DihedralGroup}
\pmdefines{infinite dihedral group}
\pmdefines{infinite dihedral}

\endmetadata

\usepackage{amssymb}
\usepackage{amsmath}

\DeclareMathOperator{\Dih}{Dih}
\def\semidirect{\rtimes}
\begin{document}
\PMlinkescapeword{finite}
\PMlinkescapeword{property}
\PMlinkescapeword{source}

Let $A$ be an abelian group.
The \emph{generalized dihedral group} $\Dih(A)$
is the semidirect product $A\semidirect C_2$,
where $C_2$ is the cyclic group of order $2$,
and the \PMlinkname{generator}{Generator} of $C_2$ maps elements of $A$ to their inverses.

If $A$ is cyclic, then $\Dih(A)$ is called a dihedral group.
The finite dihedral group $\Dih(C_n)$ is commonly denoted by $D_n$ or $D_{2n}$
(the differing conventions being a source of confusion).
The infinite dihedral group $\Dih(C_\infty)$ is denoted by $D_\infty$,
and is isomorphic to 
the free product $C_2*C_2$ of two cyclic groups of order $2$.

If $A$ is an elementary abelian $2$-group, then so is $\Dih(A)$.
If $A$ is not an elementary abelian $2$-group, then $\Dih(A)$ is non-abelian.

The subgroup $A\times\{1\}$ of $\Dih(A)$ is of index $2$,
and every element of $\Dih(A)$ that is not in this subgroup has order $2$.
This property in fact characterizes generalized dihedral groups,
in the sense that if a group $G$ has a subgroup $N$ of index $2$ such that all elements of the complement $G\setminus N$ are of order $2$,
then $N$ is abelian and $G\cong \Dih(N)$.
%%%%%
%%%%%
\end{document}
