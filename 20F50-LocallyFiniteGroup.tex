\documentclass[12pt]{article}
\usepackage{pmmeta}
\pmcanonicalname{LocallyFiniteGroup}
\pmcreated{2013-03-22 14:18:44}
\pmmodified{2013-03-22 14:18:44}
\pmowner{CWoo}{3771}
\pmmodifier{CWoo}{3771}
\pmtitle{locally finite group}
\pmrecord{6}{35776}
\pmprivacy{1}
\pmauthor{CWoo}{3771}
\pmtype{Definition}
\pmcomment{trigger rebuild}
\pmclassification{msc}{20F50}
\pmrelated{LocallyCalP}
\pmrelated{PeriodicGroup}
\pmrelated{ProofThatLocalFinitenessIsClosedUnderExtension}
\pmdefines{locally finite}

% this is the default PlanetMath preamble.  as your knowledge
% of TeX increases, you will probably want to edit this, but
% it should be fine as is for beginners.

% almost certainly you want these
\usepackage{amssymb}
\usepackage{amsmath}
\usepackage{amsfonts}

% used for TeXing text within eps files
%\usepackage{psfrag}
% need this for including graphics (\includegraphics)
%\usepackage{graphicx}
% for neatly defining theorems and propositions
%\usepackage{amsthm}
% making logically defined graphics
%%%\usepackage{xypic}

% there are many more packages, add them here as you need them

% define commands here
\begin{document}
A group $G$ is \emph{locally finite} if any finitely generated subgroup of $G$ is finite.

A locally finite group is a torsion group.  The converse, also known as the Burnside Problem, is not true.  Burnside, however, did show that if a matrix group is torsion, then it is locally finite.

(Kaplansky) If $G$ is a group such that for a normal subgroup $N$ of $G$, $N$ and $G/N$ are locally finite, then $G$ is locally finite.

A solvable torsion group is locally finite.  To see this, let $G = G_0 \supset G_1 \supset \cdots \supset G_n = (1)$ be a composition series for $G$.  We have that each $G_{i+1}$ is normal in $G_i$ and the factor group $G_i/G_{i+1}$ is abelian.  Because $G$ is a torsion group, so is the factor group $G_i/G_{i+1}$.  Clearly an abelian torsion group is locally finite.  By applying the fact in the previous paragraph for each step in the composition series, we see that $G$ must be locally finite.

\begin{thebibliography}{9}
\bibitem{golod} E. S. Gold and I. R. Shafarevitch, {\em On towers of class fields}, Izv. Akad. Nauk SSR, 28 (1964) 261-272.
\bibitem{herstein} I. N. Herstein, {\em Noncommutative Rings}, The Carus Mathematical Monographs, Number 15, (1968).
\bibitem{kaplansky} I. Kaplansky, {\em Notes on Ring Theory}, University of Chicago, Math Lecture Notes, (1965).
\bibitem{procesi} C. Procesi, {\em On the Burnside problem}, Journal of Algebra, 4 (1966) 421-426.
\end{thebibliography}
%%%%%
%%%%%
\end{document}
