\documentclass[12pt]{article}
\usepackage{pmmeta}
\pmcanonicalname{LeftIdentityAndRightIdentity}
\pmcreated{2013-03-22 13:02:05}
\pmmodified{2013-03-22 13:02:05}
\pmowner{mclase}{549}
\pmmodifier{mclase}{549}
\pmtitle{left identity and right identity}
\pmrecord{5}{33435}
\pmprivacy{1}
\pmauthor{mclase}{549}
\pmtype{Definition}
\pmcomment{trigger rebuild}
\pmclassification{msc}{20N02}
\pmclassification{msc}{20M99}
\pmrelated{IdentityElement}
\pmrelated{Unity}
\pmdefines{left identity}
\pmdefines{right identity}

% this is the default PlanetMath preamble.  as your knowledge
% of TeX increases, you will probably want to edit this, but
% it should be fine as is for beginners.

% almost certainly you want these
\usepackage{amssymb}
\usepackage{amsmath}
\usepackage{amsfonts}

% used for TeXing text within eps files
%\usepackage{psfrag}
% need this for including graphics (\includegraphics)
%\usepackage{graphicx}
% for neatly defining theorems and propositions
%\usepackage{amsthm}
% making logically defined graphics
%%%\usepackage{xypic}

% there are many more packages, add them here as you need them

% define commands here
\begin{document}
Let $G$ be a groupoid.  An element $e \in G$ is called a \emph{left identity element} if $ex = x$ for all $x \in G$.  Similarly, $e$ is a \emph{right identity element} if $xe = x$ for all $x \in G$.

An element which is both a left and a right identity is an identity element.

A groupoid may have more than one left identify element: in fact the operation 
defined by $x y = y$ for all $x, y \in G$ defines a groupoid (in fact, a semigroup) on any set $G$, and every element is a left identity.

But as soon as a groupoid has both a left and a right identity, they are necessarily unique and equal.  For if $e$ is a left identity and $f$ is a right identity, then $f = ef = e$.
%%%%%
%%%%%
\end{document}
