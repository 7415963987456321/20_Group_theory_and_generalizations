\documentclass[12pt]{article}
\usepackage{pmmeta}
\pmcanonicalname{ProofThatDimensionOfComplexIrreducibleRepresentationDividesOrderOfGroup}
\pmcreated{2013-03-22 17:09:04}
\pmmodified{2013-03-22 17:09:04}
\pmowner{whm22}{2009}
\pmmodifier{whm22}{2009}
\pmtitle{proof that dimension of complex irreducible representation divides order of group}
\pmrecord{8}{39460}
\pmprivacy{1}
\pmauthor{whm22}{2009}
\pmtype{Proof}
\pmcomment{trigger rebuild}
\pmclassification{msc}{20C99}

\endmetadata

% this is the default PlanetMath preamble.  as your knowledge
% of TeX increases, you will probably want to edit this, but
% it should be fine as is for beginners.

% almost certainly you want these
\usepackage{amssymb}
\usepackage{amsmath}
\usepackage{amsfonts}

% used for TeXing text within eps files
%\usepackage{psfrag}
% need this for including graphics (\includegraphics)
%\usepackage{graphicx}
% for neatly defining theorems and propositions
%\usepackage{amsthm}
% making logically defined graphics
%%%\usepackage{xypic}

% there are many more packages, add them here as you need them

% define commands here

\begin{document}
{\bf Theorem} Let $G$ be a finite group and $V$ an irreducible
complex representation of finite dimension $d$. Then $d$ divides
$|G|$.

Proof: Given any $\alpha$ in the group ring of $G$ (denoted
$\mathbb{Z}G$) we may define a sequence of submodules of
$\mathbb{Z}G$ (regarded as a module over $\mathbb{Z}$)
by $A_i$ equals the $\mathbb{Z}$ linear span of
$\{1,\alpha, \alpha^2, \cdots, \alpha^i\}$.

$\mathbb{Z}G$ is Noetherian as a module over $\mathbb{Z}$ so we
must have $A_i =A_{i-1}$ for some $i$.  Hence $\alpha^i$ may be
expressed as a $\mathbb{Z}$ linear combination of lower powers of
$\alpha$. In other \PMlinkescapetext{words} $\alpha$ solves a monic polynomial of
degree $i$ with coefficients in $\mathbb{Z}$.

Given a conjugacy class $C$ in $G$, we may set $\phi_C = \sum_{g
\in C} g$.  Then $\phi_C$ is central in $\mathbb{Z}G$, as given $h
\in G$, we have:

$$\phi_C h =h\sum_{g \in C} h^{-1}gh = h\sum_{g \in C} g=h\phi_C$$

Hence applying $\phi_C$ to $V$ induces a $\mathbb{C}G$ linear map
$V \to V$.  By Schur's lemma this must be 
multiplication by some
complex number $\lambda_C$.  Then $\lambda_C$ is an algebraic
integer as it solves the same monic polynomial as $\phi_C$.

Also any $g \in G$ has finite order so the map it induces on $V$
must have eigenvalues which are roots of unity and hence algebraic
integers.  Hence the sum of the eigenvalues, $\chi_V (g)$, must
also be an algebraic integer.

Now $V$ is irreducible so,

$$
|G|= \sum_{g \in G} \chi_V(g) {\chi_V(g)}^* = \sum_{C \subset G}
{\rm tr}(\phi_C){\chi_V(C)}^* = d\sum_{C \subset  G} \lambda_C
{\chi_V(C)}^*
$$

Therefore $|G|/d$ is both rational and an algebraic integer. Hence
it is an integer and $d$ divides $|G|$.

%%%%%
%%%%%
\end{document}
