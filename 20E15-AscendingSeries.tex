\documentclass[12pt]{article}
\usepackage{pmmeta}
\pmcanonicalname{AscendingSeries}
\pmcreated{2013-03-22 16:14:55}
\pmmodified{2013-03-22 16:14:55}
\pmowner{yark}{2760}
\pmmodifier{yark}{2760}
\pmtitle{ascending series}
\pmrecord{12}{38353}
\pmprivacy{1}
\pmauthor{yark}{2760}
\pmtype{Definition}
\pmcomment{trigger rebuild}
\pmclassification{msc}{20E15}
\pmclassification{msc}{20F22}
\pmrelated{DescendingSeries}
\pmrelated{SubnormalSeries}
\pmrelated{SubnormalSubgroup}
\pmdefines{ascending normal series}
\pmdefines{ascendant subgroup}
\pmdefines{ascendant}
\pmdefines{hyperabelian group}
\pmdefines{hyperabelian}
\pmdefines{Gruenberg group}

\usepackage{amsmath}
\usepackage{amsfonts}
\def\normal{\trianglelefteq}
\def\asc{\operatorname{asc}}

\begin{document}
\PMlinkescapeword{factor}
\PMlinkescapeword{factors}
\PMlinkescapeword{property}
\PMlinkescapeword{quotient}
\PMlinkescapeword{quotients}
\PMlinkescapeword{satisfy}
\PMlinkescapeword{series}
\PMlinkescapeword{term}
\PMlinkescapeword{terms}

Let $G$ be a group.

An \emph{ascending series} of $G$
is a family $(H_\alpha)_{\alpha\le\beta}$ of subgroups of $G$,
where $\beta$ is an ordinal,
such that $H_0=\{1\}$ and $H_\beta=G$,
and $H_\alpha\normal H_{\alpha+1}$ for all $\alpha<\beta$,
and $$\bigcup_{\alpha<\delta}H_\alpha=H_\delta$$
whenever $\delta\le\beta$ is a limit ordinal.

Note that this is a generalization of the concept of a subnormal series.
Compare also the dual concept of a descending series.

Given an ascending series $(H_\alpha)_{\alpha\le\beta}$,
the subgroups $H_\alpha$ are called the \emph{terms} of the series
and the \PMlinkname{quotients}{QuotientGroup} $H_{\alpha+1}/H_\alpha$
are called the \emph{factors} of the series.

A subgroup of $G$ that is a term of some ascending series of $G$
is called an \emph{ascendant subgroup} of $G$.
The notation $H\asc G$ is sometimes used
to indicate that $H$ is an ascendant subgroup of $G$.

The groups in which every subgroup is ascendant
are precisely the groups that satisfy the normalizer condition.
Groups in which every cyclic subgroup is ascendant
are called \emph{Gruenberg groups}.
It can be shown that in a Gruenberg group,
every finitely generated subgroup is ascendant and nilpotent
(and so, in particular, Gruenberg groups are locally nilpotent).

An ascending series of $G$
in which all terms are normal in $G$
is called an \emph{ascending normal series}.

Let $\mathfrak{X}$ be a property of groups.
A group is said to be \emph{hyper-$\mathfrak{X}$}
if it has an ascending normal series
whose factors all have property $\mathfrak{X}$.
So, for example, a \emph{hyperabelian group}
is a group that has an ascending normal series with abelian factors.
Hyperabelian groups are sometimes called \emph{$SI^*$-groups}.
%%%%%
%%%%%
\end{document}
