\documentclass[12pt]{article}
\usepackage{pmmeta}
\pmcanonicalname{Groupoid}
\pmcreated{2013-03-22 12:16:03}
\pmmodified{2013-03-22 12:16:03}
\pmowner{akrowne}{2}
\pmmodifier{akrowne}{2}
\pmtitle{groupoid}
\pmrecord{11}{31691}
\pmprivacy{1}
\pmauthor{akrowne}{2}
\pmtype{Definition}
\pmcomment{trigger rebuild}
\pmclassification{msc}{20N02}
\pmsynonym{magma}{Groupoid}
\pmrelated{Semigroup}
\pmrelated{Group}
\pmrelated{LoopAndQuasigroup}

\endmetadata

\usepackage{amssymb}
\usepackage{amsmath}
\usepackage{amsfonts}

%\usepackage{psfrag}
%\usepackage{graphicx}
%%%%\usepackage{xypic}
\begin{document}
A groupoid $G$ is a set together with a binary operation $\cdot : G \times G \longrightarrow G$.  The groupoid (or ``magma'') is closed under the operation.

There is also a separate, \PMlinkname{category-theoretic}{GroupoidCategoryTheoretic} definition of ``groupoid.''
%%%%%
%%%%%
%%%%%
\end{document}
