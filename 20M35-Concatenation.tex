\documentclass[12pt]{article}
\usepackage{pmmeta}
\pmcanonicalname{Concatenation}
\pmcreated{2013-03-22 17:16:38}
\pmmodified{2013-03-22 17:16:38}
\pmowner{CWoo}{3771}
\pmmodifier{CWoo}{3771}
\pmtitle{concatenation}
\pmrecord{17}{39617}
\pmprivacy{1}
\pmauthor{CWoo}{3771}
\pmtype{Definition}
\pmcomment{trigger rebuild}
\pmclassification{msc}{20M35}
\pmclassification{msc}{68Q70}
\pmsynonym{juxtaposition}{Concatenation}
\pmsynonym{monoidal}{Concatenation}
\pmrelated{Word}
\pmdefines{length}
\pmdefines{length of a word}
\pmdefines{monoidal language}

\endmetadata

\usepackage{amssymb,amscd}
\usepackage{amsmath}
\usepackage{amsfonts}
\usepackage{mathrsfs}

% used for TeXing text within eps files
%\usepackage{psfrag}
% need this for including graphics (\includegraphics)
%\usepackage{graphicx}
% for neatly defining theorems and propositions
\usepackage{amsthm}
% making logically defined graphics
%%\usepackage{xypic}
\usepackage{pst-plot}
\usepackage{psfrag}

% define commands here
\newtheorem{prop}{Proposition}
\newtheorem{thm}{Theorem}
\newtheorem{ex}{Example}
\newcommand{\real}{\mathbb{R}}
\newcommand{\pdiff}[2]{\frac{\partial #1}{\partial #2}}
\newcommand{\mpdiff}[3]{\frac{\partial^#1 #2}{\partial #3^#1}}
\begin{document}
\PMlinkescapeword{order}
\PMlinkescapeword{simple}

\subsubsection*{Concatenation on Words}

Let $a,b$ be two words.  Loosely speaking, the \emph{concatenation}, or \emph{juxtaposition} of $a$ and $b$ is the word of the form $ab$.  In order to define this rigorously, let us first do a little review of what words are.  

Let $\Sigma$ be a set whose elements we call \emph{letters} (we also call $\Sigma$ an \emph{alphabet}).  A (finite) word or a string on $\Sigma$ is a partial function $w:\mathbb{N} \to \Sigma$, (where $\mathbb{N}$ is the set of natural numbers), such that, if $\operatorname{dom}(w)\ne \varnothing$, then there is an $n\in \mathbb{N}$ such that 
\begin{displaymath}
w \textrm{ is } \left\{
\begin{array}{ll}
\textrm{defined for every }m\le n,\\
\textrm{undefined otherwise.}
\end{array}
\right.
\end{displaymath}
This $n$ is necessarily unique, and is called the \emph{length} of the word $w$.  The length of a word $w$ is usually denoted by $|w|$.  The word whose domain is $\varnothing$, the empty set, is called the empty word, and is denoted by $\lambda$.  It is easy to see that $|\lambda|=0$.  Any element in the range of $w$ has the form $w(i)$, but it is more commonly written $w_i$.  If a word $w$ is not the empty word, then we may write it as $w_1w_2\cdots w_n$, where $n=|w|$.  The collection of all words on $\Sigma$ is denoted $\Sigma^*$ (the asterisk $^*$ is commonly known as the Kleene star operation of a set).  Using the definition above, we see that $\lambda\in \Sigma^*$.

Now we define a binary operation $\circ$ on $\Sigma^*$, called the \emph{concatenation} on the alphabet $\Sigma$, as follows: let $v,w\in \Sigma^*$ with $m=|v|$ and $n=|w|$.  Then $\circ(v,w)$ is the partial function whose domain is the set $\lbrace 1, \ldots, m, m+1, \ldots, m+n\rbrace$, such that
\begin{displaymath}
\circ(v,w)(i) = \left\{ 
\begin{array}{ll}
v(i) & \textrm{if }i\le m\\
w(i-m) & \textrm{otherwise.}
\end{array} 
\right.
\end{displaymath}
The partial function $\circ(v,w)$ is written $v\circ w$, or simply $vw$, when it does not cause any confusion.  Therefore, if $v=v_1\cdots v_m$ and $w=w_1\cdots w_n$, then $vw=v_1\cdots v_mw_1\cdots w_n$.

Below are some simple properties of $\circ$ on words:
\begin{itemize}
\item $\circ$ is associative: $(uv)w=u(vw)$.
\item $\lambda w=w\lambda = w$.
\item As a result, $\Sigma^*$ together with $\circ$ is a monoid.
\item $vw=\lambda$ iff $v=w=\lambda$.
\item As a result, $\Sigma^*$ is never a group unless $\Sigma^*=\lbrace \lambda\rbrace$.
\item If $a=bc$ where $a$ is a letter, then one of $b,c$ is $a$, and the other the empty word $\lambda$.
\item If $ab=cd$ where $a,b,c,d$ are letters, then $a=c$ and $b=d$.
\end{itemize}

\subsubsection*{Concatenation on Languages}

The concatenation operation $\circ$ over an alphabet $\Sigma$ can be extended to operations on languages over $\Sigma$.  Suppose $A,B$ are two languages over $\Sigma$, we define $$A\circ B:= \lbrace u\circ v \mid u\in A, v\in B\rbrace.$$
When there is no confusion, we write $AB$ for $A\circ B$.

Below are some simple properties of $\circ$ on languages:
\begin{itemize}
\item $(AB)C=A(BC)$; \PMlinkname{i.e.}{Ie}, concatenation of sets of letters is associative.
\item Because of the associativity of $\circ$, we can inductively define $A^n$ for any positive integer $n$, as $A^1=A$, and $A^{n+1}=A^nA$.
\item It is not hard to see that $\Sigma^*=\lbrace \lambda \rbrace \cup \Sigma \cup \Sigma^2 \cup \cdots \cup \Sigma^n \cup \cdots$.
\end{itemize}

\textbf{Remark}.  A formal language containing the empty word, and is closed under concatenation is said to be \emph{monoidal}, since it has the structure of a monoid.

\begin{thebibliography}{9}
\bibitem{hlcp} H.R. Lewis, C.H. Papadimitriou {\em Elements of the Theory of Computation}. Prentice-Hall, Englewood Cliffs, New Jersey (1981).
\end{thebibliography}
%%%%%
%%%%%
\end{document}
