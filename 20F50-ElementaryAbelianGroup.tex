\documentclass[12pt]{article}
\usepackage{pmmeta}
\pmcanonicalname{ElementaryAbelianGroup}
\pmcreated{2013-03-22 14:53:11}
\pmmodified{2013-03-22 14:53:11}
\pmowner{yark}{2760}
\pmmodifier{yark}{2760}
\pmtitle{elementary abelian group}
\pmrecord{12}{36566}
\pmprivacy{1}
\pmauthor{yark}{2760}
\pmtype{Definition}
\pmcomment{trigger rebuild}
\pmclassification{msc}{20F50}
\pmclassification{msc}{20K10}
\pmdefines{elementary abelian}
\pmdefines{Boolean group}


\begin{document}
\PMlinkescapeword{boolean}
\PMlinkescapeword{odd}

An \emph{elementary abelian group} is an abelian group in which every non-trivial element has the same finite order. It is easy to see that the non-trivial elements must in fact be of prime order, so every elementary abelian group is a \PMlinkname{$p$-group}{PGroup4} for some prime $p$.

Elementary abelian $2$-groups are sometimes called \emph{Boolean groups}.
A group in which every non-trivial element has order $2$ is necessarily Boolean, because abelianness is automatic: $xy=(xy)^{-1}=y^{-1}x^{-1}=yx$.
There is no analogous result for odd primes, because for every odd prime $p$ there is a non-abelian group of order $p^3$ and exponent $p$.

Let $p$ be a prime number.
Any elementary abelian $p$-group can be considered as a vector space over the field of order $p$, and is therefore isomorphic to the direct sum of $\kappa$ copies of the cyclic group of order $p$, for some cardinal number $\kappa$. Conversely, any such direct sum is obviously an elementary abelian $p$-group.
So, in particular, for every infinite cardinal $\kappa$ there is, up to isomorphism, exactly one elementary abelian $p$-group of order $\kappa$.
%%%%%
%%%%%
\end{document}
