\documentclass[12pt]{article}
\usepackage{pmmeta}
\pmcanonicalname{CycleNotation}
\pmcreated{2013-03-22 12:33:41}
\pmmodified{2013-03-22 12:33:41}
\pmowner{rmilson}{146}
\pmmodifier{rmilson}{146}
\pmtitle{cycle notation}
\pmrecord{6}{32808}
\pmprivacy{1}
\pmauthor{rmilson}{146}
\pmtype{Definition}
\pmcomment{trigger rebuild}
\pmclassification{msc}{20B05}
\pmclassification{msc}{05A05}
\pmrelated{Cycle2}
\pmrelated{Permutation}
\pmrelated{OneLineNotationForPermutations}

\endmetadata

\usepackage{amsmath}
\usepackage{amsfonts}
\usepackage{amssymb}

\newcommand{\reals}{\mathbb{R}}
\newcommand{\natnums}{\mathbb{N}}
\newcommand{\cnums}{\mathbb{C}}
\newcommand{\znums}{\mathbb{Z}}

\newcommand{\lp}{\left(}
\newcommand{\rp}{\right)}
\newcommand{\lb}{\left[}
\newcommand{\rb}{\right]}

\newcommand{\supth}{^{\text{th}}}


\newtheorem{proposition}{Proposition}
\begin{document}
The cycle notation is a useful convention for writing down a
permutations in terms of its constituent cycles.  Let $S$ be a finite
set, and
$$a_1,\ldots,a_k,\quad k\geq 2$$
distinct elements of $S$.  The
expression $(a_1,\ldots,a_k)$ denotes the cycle whose action is
$$a_1\mapsto a_2\mapsto a_3\ldots a_k \mapsto a_1.$$
Note there are $k$ different expressions for the same cycle; the
following all represent the same cycle:
$$(a_1,a_2,a_3,\ldots,a_k) = (a_2,a_3,\ldots,a_k,a_1),=\ldots =
(a_k,a_1,a_2,\ldots,a_{k-1}).$$
Also note that a 1-element cycle is
the same thing as the identity permutation, and thus there is not
much point in writing down such things.  Rather, it is customary to
express the identity permutation simply as $()$ or $(1)$.

Let $\pi$ be a permutation of $S$, and let 
$$S_1,\ldots, S_k\subset S,\quad k\in\natnums$$
be the orbits of $\pi$
with more than 1 element.  For each $j=1,\ldots,k$ let $n_j$ denote
the cardinality of $S_j$.  Also, choose an $a_{1,j}\in S_j$, and
define
$$a_{i+1,j} = \pi(a_{i,j}),\quad i\in\natnums.$$
We can now express $\pi$ as a product of disjoint cycles, namely
$$\pi = (a_{1,1},\ldots a_{n_1,1}) (a_{2,1},\ldots,a_{n_2,2}) \ldots
(a_{k,1},\ldots,a_{n_k,k}).$$

By way of illustration, here are the 24 elements of the symmetric
group on $\{1,2,3,4\}$ expressed using the cycle notation, and grouped
according to their conjugacy classes:
\begin{align*}
&(),\\
&(12), \;(13),\; (14),\; (23),\; (24),\; (34)\\
&(123),\; (213),\; (124),\; (214),\; (134),\; (143),\; (234),\;
(243)\\
&(12)(34),\;(13)(24),\; (14)(23)\\
&(1234),\; (1243),\; (1324),\; (1342),\; (1423),\; (1432)
\end{align*}
%%%%%
%%%%%
\end{document}
