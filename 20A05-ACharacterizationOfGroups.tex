\documentclass[12pt]{article}
\usepackage{pmmeta}
\pmcanonicalname{ACharacterizationOfGroups}
\pmcreated{2013-03-22 14:45:08}
\pmmodified{2013-03-22 14:45:08}
\pmowner{yark}{2760}
\pmmodifier{yark}{2760}
\pmtitle{a characterization of groups}
\pmrecord{10}{36391}
\pmprivacy{1}
\pmauthor{yark}{2760}
\pmtype{Theorem}
\pmcomment{trigger rebuild}
\pmclassification{msc}{20A05}
%\pmkeywords{group}
\pmrelated{Group}
\pmrelated{RegularSemigroup}
\pmrelated{AlternativeDefinitionOfGroup}

\endmetadata

\usepackage{amssymb}
\usepackage{amsmath}
\usepackage{amsfonts}
\usepackage{amsthm}

\newtheorem*{thm*}{Theorem}
\begin{document}
\PMlinkescapeword{argument}
\PMlinkescapeword{idempotent}
\PMlinkescapephrase{idempotent element}
\PMlinkescapeword{identity}
\PMlinkescapeword{solution}
\PMlinkescapeword{symmetric}

\begin{thm*}
A non-empty semigroup $S$ is a group
if and only if
for every $x\in S$ there is a unique $y\in S$ such that $xyx=x$.
\end{thm*}

\begin{proof}
Suppose that $S$ is a non-empty semigroup,
and for every $x\in S$ there is a unique $y\in S$ such that $xyx=x$.
For each $x\in S$,
let $x'$ denote the unique element of $S$ such that $xx'x=x$.
Note that $x(x'xx')x=(xx'x)x'x=xx'x=x$,
so, by uniqueness, $x'xx'=x'$,
and therefore $x''=x$.

For any $x\in S$, the element $xx'$ is \PMlinkname{idempotent}{Idempotency},
because $(xx')^2=(xx'x)x'=xx'$.
As $S$ is nonempty, this means that $S$ has at least one idempotent element.
If $i\in S$ is idempotent,
then $ix=ix(ix)'ix=ix(ix)'iix$, and so $(ix)'i=(ix)'$,
and therefore $(ix)'=(ix)'(ix)''(ix)'=(ix)'ix(ix)'=(ix)'x(ix)'$,
which means that $ix=(ix)''=x$.
So every idempotent is a left identity,
and, by a symmetric argument, a right identity.
Therefore, $S$ has at most one idempotent element.
Combined with the previous result,
this means that $S$ has exactly one idempotent element,
which we will denote by $e$.
We have shown that $e$ is an identity,
and that $xx'=e$ for each $x\in S$, so $S$ is a group.

Conversely, if $S$ is a group
then $xyx=x$ clearly has a unique solution, namely $y=x^{-1}$.
\end{proof}

{\bf Note.} Note that inverse semigroups do not in general
satisfy the hypothesis of this theorem:
in an inverse semigroup there is for each $x$ a unique $y$ such that $xyx=x$ and $yxy=y$,
but this $y$ need not be unique as a solution of $xyx=x$ alone.
%%%%%
%%%%%
\end{document}
