\documentclass[12pt]{article}
\usepackage{pmmeta}
\pmcanonicalname{StrongerHilbertTheorem90}
\pmcreated{2013-03-22 13:50:27}
\pmmodified{2013-03-22 13:50:27}
\pmowner{alozano}{2414}
\pmmodifier{alozano}{2414}
\pmtitle{stronger Hilbert theorem 90}
\pmrecord{6}{34577}
\pmprivacy{1}
\pmauthor{alozano}{2414}
\pmtype{Theorem}
\pmcomment{trigger rebuild}
\pmclassification{msc}{20J06}
\pmsynonym{Hilbert 90}{StrongerHilbertTheorem90}
%\pmkeywords{cohomology}
\pmrelated{HilbertTheorem90}

\endmetadata

% this is the default PlanetMath preamble.  as your knowledge
% of TeX increases, you will probably want to edit this, but
% it should be fine as is for beginners.

% almost certainly you want these
\usepackage{amssymb}
\usepackage{amsmath}
\usepackage{amsthm}
\usepackage{amsfonts}

% used for TeXing text within eps files
%\usepackage{psfrag}
% need this for including graphics (\includegraphics)
%\usepackage{graphicx}
% for neatly defining theorems and propositions
%\usepackage{amsthm}
% making logically defined graphics
%%%\usepackage{xypic}

% there are many more packages, add them here as you need them

% define commands here

\newtheorem{thm}{Theorem}
\newtheorem{defn}{Definition}
\newtheorem{prop}{Proposition}
\newtheorem{lemma}{Lemma}
\newtheorem{cor}{Corollary}
\begin{document}
Let $K$ be a field and let $\bar{K}$ be an algebraic closure of
$K$. By $\bar{K}^+$ we denote the abelian group $(\bar{K},+)$ and
similarly $\bar{K}^{\ast}=(\bar{K},\ast)$ (here the operation is
multiplication). Also we let
$$G_{\bar{K}/K}=\operatorname{Gal}(\bar{K}/K)$$ be the absolute
Galois group of $K$.\\


\begin{thm}[Hilbert 90]
Let $K$ be a field.
\begin{enumerate}
\item $$ H^1(G_{\bar{K}/K},\bar{K}^+)=0$$

\item $$ H^1(G_{\bar{K}/K},\bar{K}^{\ast})=0$$

\item If $\operatorname{char}(K)$, the characteristic of $K$, does
not divide $m$ (or $\operatorname{char}(K)=0$) then
$$ H^1(G_{\bar{K}/K},{\mu}_m)\cong K^{\ast}/K^{\ast m}$$
where ${\mu}_m$ denotes the set of all $m^{th}$-roots of unity.
\end{enumerate}
\end{thm}

\begin{thebibliography}{9}
\bibitem{serre} J.P. Serre, {\em Galois Cohomology},
Springer-Verlag, New York.
\bibitem{serre} J.P. Serre, {\em Local Fields},
Springer-Verlag, New York.
\end{thebibliography}
%%%%%
%%%%%
\end{document}
