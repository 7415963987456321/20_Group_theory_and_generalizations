\documentclass[12pt]{article}
\usepackage{pmmeta}
\pmcanonicalname{PermutableSubgroup}
\pmcreated{2013-03-22 16:15:47}
\pmmodified{2013-03-22 16:15:47}
\pmowner{yark}{2760}
\pmmodifier{yark}{2760}
\pmtitle{permutable subgroup}
\pmrecord{9}{38372}
\pmprivacy{1}
\pmauthor{yark}{2760}
\pmtype{Definition}
\pmcomment{trigger rebuild}
\pmclassification{msc}{20E07}
\pmsynonym{quasinormal subgroup}{PermutableSubgroup}
\pmsynonym{quasi-normal subgroup}{PermutableSubgroup}
\pmdefines{permutable}
\pmdefines{quasinormal}
\pmdefines{quasi-normal}

\usepackage{amssymb}
\usepackage{amsmath}

\def\per{\operatorname{per}}

% The following lines should work as the command
% \renewcommand{\bibname}{References}
% without creating havoc when rendering an entry in
% the page-image mode.
\makeatletter
\@ifundefined{bibname}{}{\renewcommand{\bibname}{References}}
\makeatother
\begin{document}
\PMlinkescapephrase{permutable subgroups}

Let $G$ be a group.
A subgroup $H$ of $G$ is said to be \emph{permutable}
if it permutes with all subgroups of $G$,
that is, $KH=HK$ for all $K\leq G$.
We sometimes write $H\per G$
to indicate that $H$ is a permutable subgroup of $G$.

Permutable subgroups were introduced by \PMlinkname{{\O}ystein Ore}{OysteinOre},
who called them \emph{quasinormal} subgroups.

Normal subgroups are clearly permutable.

Permutable subgroups are ascendant.
This is a result of Stonehewer\cite{stonehewer}, 
who also showed that in a finitely generated group,
all permutable subgroups are subnormal.

\begin{thebibliography}{9}
\bibitem{stonehewer}
 Stewart E.\ Stonehewer,
 {\it Permutable subgroups of infinite groups},
 Math.\ Z.\ 125 (1972), 1--16. (This paper is \PMlinkexternal{available from GDZ}{http://gdz.sub.uni-goettingen.de/dms/resolveppn/?GDZPPN002410435}.)
\end{thebibliography}






%%%%%
%%%%%
\end{document}
