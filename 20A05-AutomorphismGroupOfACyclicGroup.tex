\documentclass[12pt]{article}
\usepackage{pmmeta}
\pmcanonicalname{AutomorphismGroupOfACyclicGroup}
\pmcreated{2013-03-22 18:42:35}
\pmmodified{2013-03-22 18:42:35}
\pmowner{rm50}{10146}
\pmmodifier{rm50}{10146}
\pmtitle{automorphism group of a cyclic group}
\pmrecord{6}{41475}
\pmprivacy{1}
\pmauthor{rm50}{10146}
\pmtype{Theorem}
\pmcomment{trigger rebuild}
\pmclassification{msc}{20A05}
\pmclassification{msc}{20F28}

\endmetadata

% this is the default PlanetMath preamble.  as your knowledge
% of TeX increases, you will probably want to edit this, but
% it should be fine as is for beginners.

% almost certainly you want these
\usepackage{amssymb}
\usepackage{amsmath}
\usepackage{amsfonts}

% used for TeXing text within eps files
%\usepackage{psfrag}
% need this for including graphics (\includegraphics)
%\usepackage{graphicx}
% for neatly defining theorems and propositions
\usepackage{amsthm}
% making logically defined graphics
%%%\usepackage{xypic}

% there are many more packages, add them here as you need them

% define commands here
\newcommand{\Ints}{\mathbb{Z}}
\DeclareMathOperator{\Aut}{Aut}
\newcommand{\UI}[1]{(\Ints/{#1}\Ints)^{\times}}
\newtheorem{thm}{Theorem}
\begin{document}
\begin{thm} The automorphism group of the cyclic group $\Ints/n\Ints$ is $\UI{n}$, which is of order $\phi(n)$ (here $\phi$ is the Euler totient function).
\end{thm}
\begin{proof} Choose a generator $x$ for $\Ints/n\Ints$. If $\rho\in \Aut(\Ints/n\Ints)$, then $\rho(x) = x^a$ for some integer $a$ (defined up to multiples of $n$); further, since $x$ generates $\Ints/n\Ints$, it is clear that $a$ uniquely determines $\rho$. Write $\rho_a$ for this automorphism. Since $\rho_a$ is an automorphism, $x^a$ is also a generator, and thus $a$ and $n$ are relatively prime\footnote{
If they were not, say $(a,n)=d$, then $(x^a)^{n/d} = (x^{a/d})^n=1$ so that $x^a$ would not generate.}. Clearly, then, every $a$ relatively prime to $n$ induces an automorphism. We can therefore define a surjective map
\[
  \Phi : \Aut(\Ints/n\Ints) \to \UI{n}: \rho_a\mapsto a\pmod n
\]
$\Phi$ is also obviously injective, so all that remains is to show that it is a group homomorphism. But for every $a,b\in\UI{n}$, we have
\[
  (\rho_a\circ\rho_b)(x) = \rho_a(x^b) = (x^b)^a = x^{ab} = \rho_{ab}(x)
\]
and thus
\[
  \Phi(\rho_a\circ\rho_b) = \Phi(\rho_{ab}) = ab\pmod n = \Phi(\rho_a)\Phi(\rho_b)
\]
\end{proof}
\begin{thebibliography}{10}
\bibitem{bib:df}
Dummit,~D.,~Foote,~R.M., \emph{Abstract Algebra, Third Edition}, Wiley, 2004.
\end{thebibliography}
%%%%%
%%%%%
\end{document}
