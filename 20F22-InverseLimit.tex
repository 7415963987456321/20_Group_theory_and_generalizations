\documentclass[12pt]{article}
\usepackage{pmmeta}
\pmcanonicalname{InverseLimit}
\pmcreated{2013-03-22 13:54:20}
\pmmodified{2013-03-22 13:54:20}
\pmowner{alozano}{2414}
\pmmodifier{alozano}{2414}
\pmtitle{inverse limit}
\pmrecord{10}{34655}
\pmprivacy{1}
\pmauthor{alozano}{2414}
\pmtype{Definition}
\pmcomment{trigger rebuild}
\pmclassification{msc}{20F22}
\pmsynonym{inverse system}{InverseLimit}
\pmsynonym{projective limit}{InverseLimit}
%\pmkeywords{inverse limit}
\pmrelated{PAdicIntegers}
\pmrelated{GaloisRepresentation}
\pmrelated{InfiniteGaloisTheory}
\pmrelated{ProfiniteGroup}
\pmrelated{CategoryAssociatedToAPartialOrder}
\pmrelated{DirectLimit}
\pmrelated{CohomologyOfSmallCategories}
\pmdefines{inverse limit}

% this is the default PlanetMath preamble.  as your knowledge
% of TeX increases, you will probably want to edit this, but
% it should be fine as is for beginners.

% almost certainly you want these
\usepackage{amssymb}
\usepackage{amsmath}
\usepackage{amsthm}
\usepackage{amsfonts}

% used for TeXing text within eps files
%\usepackage{psfrag}
% need this for including graphics (\includegraphics)
%\usepackage{graphicx}
% for neatly defining theorems and propositions
%\usepackage{amsthm}
% making logically defined graphics
%%\usepackage{xypic}

% there are many more packages, add them here as you need them

% define commands here

\newtheorem{thm}{Theorem}
\newtheorem{defn}{Definition}
\newtheorem{prop}{Proposition}
\newtheorem{lemma}{Lemma}
\newtheorem{cor}{Corollary}

% Some sets
\newcommand{\Nats}{\mathbb{N}}
\newcommand{\Ints}{\mathbb{Z}}
\newcommand{\Reals}{\mathbb{R}}
\newcommand{\Complex}{\mathbb{C}}
\newcommand{\Rats}{\mathbb{Q}}
\begin{document}
Let $\{ G_i \}_{i=0}^{\infty}$ be a sequence of groups which are
related by a chain of surjective homomorphisms $f_i\colon G_i \to
G_{i-1}$ such that
$$\xymatrix{
{G_0} & {G_1} \ar@{->}[l]_{f_1} & {G_2} \ar@{->}[l]_{f_2} & {G_3}
\ar@{->}[l]_{f_3} & {\ldots} \ar@{->}[l]_{f_4}}$$

\begin{defn} The \emph{inverse limit} of $(G_i,f_i)$, denoted by
$$\varprojlim(G_i,f_i),\quad \text{ or }\quad \varprojlim
G_i$$ is the subset of $\prod_{i=0}^{\infty} G_i$ formed by
elements satisfying
$$(\ g_0,\ g_1,\ g_2,\ g_3,\ \ldots),\ \text{with}\quad g_i\in G_i, \quad
f_i(g_i)=g_{i-1}$$
\end{defn}

Note: The inverse limit of $G_i$ can be checked to be a subgroup
of the product $\prod_{i=0}^{\infty} G_i$. See below for a more general definition.

{\bf Examples}:
\begin{enumerate}
\item Let $p\in\Nats$ be a prime. Let $G_0=\{0\}$ and
$G_i=\Ints/p^i\Ints$. Define the connecting homomorphisms $f_i$, for $i\geq 2$, to
be ``reduction modulo $p^{i-1}$'' i.e.
$$f_i\colon \Ints/p^i\Ints \to \Ints/p^{i-1}\Ints$$
$$f_i(x \operatorname{mod}p^i)= x \operatorname{mod} p^{i-1}$$
which are obviously surjective homomorphisms. The inverse limit of
$(\Ints/p^i\Ints, f_i)$ is called the $p$-adic integers and
denoted by
$$\Ints_p=\varprojlim\Ints/p^i\Ints$$

\item Let $E$ be an elliptic curve defined over $\Complex$. Let
$p$ be a prime and for any natural number $n$ write $E[n]$ for the
$n$-torsion group, i.e.
$$E[n]=\{ Q\in E \mid n\cdot Q=O\}$$
In this case we define $G_i=E[p^i]$, and
$$f_i\colon E[p^i] \to E[p^{i-1}], \quad f_i(Q)=p\cdot Q$$
The inverse limit of $(E[p^i],f_i)$ is called the Tate module of
$E$ and denoted
$$T_p(E)=\varprojlim E[p^i]$$

\end{enumerate}

The concept of inverse limit can be defined in far more
generality. Let $(S,\leq)$ be a directed set and let $\mathcal{C}$
be a category. Let $\{ G_{\alpha}\}_{\alpha\in S}$ be a collection
of objects in the category $\mathcal{C}$ and let
$$\{f_{\alpha,\beta}\colon G_{\beta} \to G_{\alpha}\mid \alpha,\beta \in S,\quad \alpha\leq\beta \}$$ be a collection of
morphisms satisfying:
\begin{enumerate}
\item For all $\alpha\in S$,
$f_{\alpha,\alpha}=\operatorname{Id}_{G_{\alpha}}$, the identity
morphism.

\item For all $\alpha,\beta,\gamma\in S$ such that $\alpha \leq
\beta \leq \gamma$, we have $f_{\alpha,\gamma} = f_{\alpha,\beta}
\circ f_{\beta,\gamma}$ (composition of morphisms).
\end{enumerate}

\begin{defn}
The \emph{inverse limit} of $(\{G_{\alpha}\}_{\alpha\in
S},\{f_{\alpha,\beta}\})$, denoted by
$$\varprojlim (G_{\alpha},f_{\alpha,\beta}),\quad \text{ or }\quad \varprojlim
G_{\alpha}$$ is defined to be the set of all $(g_{\alpha})\in
\prod_{\alpha\in S} G_{\alpha}$ such that for all $\alpha,\beta
\in S$
$$ \alpha\leq\beta \Rightarrow
f_{\alpha,\beta}(g_{\beta})=g_{\alpha}$$
\end{defn}

For a good example of this more general construction, see infinite
Galois theory.
%%%%%
%%%%%
\end{document}
