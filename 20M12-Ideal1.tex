\documentclass[12pt]{article}
\usepackage{pmmeta}
\pmcanonicalname{Ideal1}
\pmcreated{2013-03-22 13:05:43}
\pmmodified{2013-03-22 13:05:43}
\pmowner{mclase}{549}
\pmmodifier{mclase}{549}
\pmtitle{ideal}
\pmrecord{8}{33516}
\pmprivacy{1}
\pmauthor{mclase}{549}
\pmtype{Definition}
\pmcomment{trigger rebuild}
\pmclassification{msc}{20M12}
\pmclassification{msc}{20M10}
\pmrelated{ReesFactor}
\pmdefines{left ideal}
\pmdefines{right ideal}
\pmdefines{principal ideal}
\pmdefines{principal left ideal}
\pmdefines{principal right ideal}

\endmetadata

% this is the default PlanetMath preamble.  as your knowledge
% of TeX increases, you will probably want to edit this, but
% it should be fine as is for beginners.

% almost certainly you want these
\usepackage{amssymb}
\usepackage{amsmath}
\usepackage{amsfonts}

% used for TeXing text within eps files
%\usepackage{psfrag}
% need this for including graphics (\includegraphics)
%\usepackage{graphicx}
% for neatly defining theorems and propositions
%\usepackage{amsthm}
% making logically defined graphics
%%%\usepackage{xypic}

% there are many more packages, add them here as you need them

% define commands here
\begin{document}
\PMlinkescapeword{principal}
\PMlinkescapeword{closed}

Let $S$ be a semigroup.  An \emph{ideal} of $S$ is a non-empty subset of $S$ which is closed under multiplication on either side by elements of $S$.  Formally, $I$ is an ideal of $S$ if $I$ is non-empty, and for all $x \in I$ and $s \in S$, we have $sx \in I$ and $xs \in I$.

One-sided ideals are defined similarly.  A non-empty subset $A$ of $S$ is a \emph{left ideal} (resp. \emph{right ideal}) of $S$ if for all $a \in A$ and $s \in S$, we have $sa \in A$ (resp. $as \in A$).

A \emph{principal left ideal} of $S$ is a left ideal generated by a single element.  If $a \in S$, then the principal left ideal of $S$ generated by $a$ is $S^1a = Sa \cup \{a\}$.  (The notation $S^1$ is explained \PMlinkname{here}{AdjoiningAnIdentityToASemigroup3}.)

Similarly, the \emph{principal right ideal} generated by $a$ is $aS^1 = aS \cup \{a\}$.

The notation $L(a)$ and $R(a)$ are also common for the principal left and right ideals generated by $a$ respectively.

A \emph{principal ideal} of $S$ is an ideal generated by a single element.  The ideal generated by $a$ is $$S^1aS^1 = SaS \cup Sa \cup aS \cup \{a\}.$$  The notation $J(a) = S^1aS^1$ is also common.
%%%%%
%%%%%
\end{document}
