\documentclass[12pt]{article}
\usepackage{pmmeta}
\pmcanonicalname{IndecomposableGroup}
\pmcreated{2013-03-22 15:23:46}
\pmmodified{2013-03-22 15:23:46}
\pmowner{CWoo}{3771}
\pmmodifier{CWoo}{3771}
\pmtitle{indecomposable group}
\pmrecord{8}{37232}
\pmprivacy{1}
\pmauthor{CWoo}{3771}
\pmtype{Definition}
\pmcomment{trigger rebuild}
\pmclassification{msc}{20-00}
\pmsynonym{indecomposable}{IndecomposableGroup}
%\pmkeywords{indecomposable}
%\pmkeywords{decomposable}
\pmrelated{KrullSchmidtTheorem}
\pmdefines{decomposable}
\pmdefines{indecomposable module}

\endmetadata

% this is the default PlanetMath preamble.  as your knowledge
% of TeX increases, you will probably want to edit this, but
% it should be fine as is for beginners.

% almost certainly you want these
\usepackage{amssymb}
\usepackage{amsmath}
\usepackage{amsfonts}

% used for TeXing text within eps files
%\usepackage{psfrag}
% need this for including graphics (\includegraphics)
%\usepackage{graphicx}
% for neatly defining theorems and propositions
%\usepackage{amsthm}
% making logically defined graphics
%%%\usepackage{xypic}

% there are many more packages, add them here as you need them

% define commands here
\begin{document}
By definition, an \emph{indecomposable group} is a nontrivial group that cannot be expressed as the internal direct product of two proper normal subgroups. A group that is not indecomposable is called, predictably enough, \emph{decomposable}.

The analogous concept exists in module theory. An indecomposable module is a nonzero module that cannot be expressed as the direct sum of two nonzero submodules.

The following examples are left as exercises for the reader.
\begin{enumerate}
\item Every simple group is indecomposable.
\item If $p$ is prime and $n$ is any positive integer, then the additive group $\mathbb{Z}/p^n\mathbb{Z}$ is indecomposable. Hence, not every indecomposable group is simple.
\item The additive groups $\mathbb{Z}$ and $\mathbb{Q}$ are indecomposable, but the additive group $\mathbb{R}$ is decomposable.
\item If $m$ and $n$ are relatively prime integers (and both greater than one), then the additive group $\mathbb{Z}/mn\mathbb{Z}$ is decomposable.
\item Every finitely generated abelian group can be expressed as the direct sum of finitely many indecomposable groups. These summands are uniquely determined up to isomorphism.
\end{enumerate}

{\bf References}.
\begin{itemize}
\item Dummit, D. and R. Foote, \emph{Abstract Algebra}. (2d ed.), New York: John Wiley and Sons, Inc., 1999.
\item Goldhaber, J. and G. Ehrlich, \emph{Algebra}. London: The Macmillan Company, 1970.
\item Hungerford, T., \emph{Algebra}. New York: Springer, 1974.
\end{itemize}
%%%%%
%%%%%
\end{document}
