\documentclass[12pt]{article}
\usepackage{pmmeta}
\pmcanonicalname{AutomatonOverAMonoid}
\pmcreated{2013-03-22 19:01:31}
\pmmodified{2013-03-22 19:01:31}
\pmowner{CWoo}{3771}
\pmmodifier{CWoo}{3771}
\pmtitle{automaton over a monoid}
\pmrecord{7}{41897}
\pmprivacy{1}
\pmauthor{CWoo}{3771}
\pmtype{Definition}
\pmcomment{trigger rebuild}
\pmclassification{msc}{20M35}
\pmclassification{msc}{68Q70}

\usepackage{amssymb,amscd}
\usepackage{amsmath}
\usepackage{amsfonts}
\usepackage{mathrsfs}

% used for TeXing text within eps files
%\usepackage{psfrag}
% need this for including graphics (\includegraphics)
%\usepackage{graphicx}
% for neatly defining theorems and propositions
\usepackage{amsthm}
% making logically defined graphics
%%\usepackage{xypic}
\usepackage{pst-plot}

% define commands here
\newcommand*{\abs}[1]{\left\lvert #1\right\rvert}
\newtheorem{prop}{Proposition}
\newtheorem{thm}{Theorem}
\newtheorem{ex}{Example}
\newcommand{\real}{\mathbb{R}}
\newcommand{\pdiff}[2]{\frac{\partial #1}{\partial #2}}
\newcommand{\mpdiff}[3]{\frac{\partial^#1 #2}{\partial #3^#1}}
\begin{document}
Recall that a semiautomaton $A$ can be visually represented by a directed graph $G_A$, whose vertices (or nodes) are states of $A$, and whose edges are labeled by elements from the input alphabet $\Sigma$ of $A$.  Labeling can be extended to paths of $G_A$ in a natural way: if $(e_1,\ldots, e_n)$ is a path $p$ and each edge $e_i$ is labeled $a_i$, then the label of $p$ is $a_1\cdots a_n$.  Thus, the labels of paths of $G_A$ are just elements of the monoid $\Sigma^*$.  The concept can be readily generalized to arbitrary monoids.

\textbf{Definition}.  Let $M$ be a monoid.  A \emph{semiautomaton} over $M$ is a directed graph $G$ whose edges are labeled by elements of $M$.  An \emph{automaton} over $M$ is a semiautomaton over $M$, where the vertex set has two designated subsets $I$ and $F$ (not necessarily disjoint), where elements of $I$ are called the start nodes, and elements of $F$ the final nodes.

Note that if $M=\Sigma^*$ for some alphabet $\Sigma$, then a semiautomaton $G$ over $M$ according to the definition given above is not necessarily a semiautomaton over $\Sigma$ under the standard definition of a semiautomaton, since labels of the edges are words over $\Sigma$, not elements of $\Sigma$.  However, $G$ can be ``transformed'' into a ``standard'' semiautomaton (over $\Sigma$).

\textbf{Definition}.  Let $A$ be a finite automaton over a monoid $M$.  An element in $M$ is said to be \emph{accepted by} $A$ if it is the label of a path that begins at an initial node and end at a final node.  The set of all elements of $M$ accepted by $A$ is denoted by $L(A)$.

The following is a generalization of Kleene's theorem.

\begin{thm}.  A subset $R$ of a monoid $M$ is a rational set iff $R=L(A)$ for some finite automaton $A$ over $M$. \end{thm}

\begin{thebibliography}{8}
\bibitem{se} S. Eilenberg, {\em Automata, Languages, and Machines, Vol. A}, Academic Press (1974).
\end{thebibliography}
%%%%%
%%%%%
\end{document}
