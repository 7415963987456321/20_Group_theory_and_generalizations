\documentclass[12pt]{article}
\usepackage{pmmeta}
\pmcanonicalname{ExamplesOfOuterAutomorphismGroup}
\pmcreated{2013-03-22 16:30:25}
\pmmodified{2013-03-22 16:30:25}
\pmowner{juanman}{12619}
\pmmodifier{juanman}{12619}
\pmtitle{examples of outer automorphism group}
\pmrecord{8}{38682}
\pmprivacy{1}
\pmauthor{juanman}{12619}
\pmtype{Example}
\pmcomment{trigger rebuild}
\pmclassification{msc}{20F28}

% this is the default PlanetMath preamble.  as your knowledge
% of TeX increases, you will probably want to edit this, but
% it should be fine as is for beginners.

% almost certainly you want these
\usepackage{amssymb}
\usepackage{amsmath}
\usepackage{amsfonts}

% used for TeXing text within eps files
%\usepackage{psfrag}
% need this for including graphics (\includegraphics)
%\usepackage{graphicx}
% for neatly defining theorems and propositions
%\usepackage{amsthm}
% making logically defined graphics
%%%\usepackage{xypic}

% there are many more packages, add them here as you need them

% define commands here

\begin{document}
It is easy to understand that ${\rm Out}\mathbb{Z}={\rm Aut}\mathbb{Z}=\mathbb{Z}/2\mathbb{Z}$, since $\mathbb{Z}$ is abelian and there are no inner-automorphisms, save the trivial one.

Also, it is known that ${\rm Out} SL(2,\mathbb{Z})=\mathbb{Z}/2\mathbb{Z}$

Another example is that, at least for orientable surfaces, the extended mapping class group (or the zeroth homeotopy group) of a surface $F$ is related to its fundamental group via ${\cal{M}^*}(F)={\rm Out}(\pi_1(F))$.


References

1. L. K. Hua, I Reiner {\it Automorphisms of unimodular group}, Trans Amer. Math. Soc. 71 (1951), 331-348.

2. H. Zieschang, E. Vogt, H. D. Coldewey, {\it Surfaces and planar discontinuous groups}, L.N.M. 875 (1981) Springer-Verlag. 
%%%%%
%%%%%
\end{document}
