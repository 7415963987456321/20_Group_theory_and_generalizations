\documentclass[12pt]{article}
\usepackage{pmmeta}
\pmcanonicalname{YoungsProjectionOperators}
\pmcreated{2013-03-22 16:48:25}
\pmmodified{2013-03-22 16:48:25}
\pmowner{rspuzio}{6075}
\pmmodifier{rspuzio}{6075}
\pmtitle{Young's projection operators}
\pmrecord{16}{39041}
\pmprivacy{1}
\pmauthor{rspuzio}{6075}
\pmtype{Definition}
\pmcomment{trigger rebuild}
\pmclassification{msc}{20C30}
\pmclassification{msc}{11P99}
\pmclassification{msc}{05A17}

\endmetadata

% this is the default PlanetMath preamble.  as your knowledge
% of TeX increases, you will probably want to edit this, but
% it should be fine as is for beginners.

% almost certainly you want these
\usepackage{amssymb}
\usepackage{amsmath}
\usepackage{amsfonts}

% used for TeXing text within eps files
%\usepackage{psfrag}
% need this for including graphics (\includegraphics)
%\usepackage{graphicx}
% for neatly defining theorems and propositions
%\usepackage{amsthm}
% making logically defined graphics
%%%\usepackage{xypic}

% there are many more packages, add them here as you need them

% define commands here

\begin{document}
Associated to a Young tableau with $n$ boxes, we have two elements 
of the group ring of the permutation group on $n$ symbols.  To
construct the operators, we first construct the antisymmetrizing
operators corresponding to the columns and the symmetrizing
operators corresponding to the rows.  Then one operator corresponding
to the tableau consists of the product of the symmetrizing operators
corresponding to the rows multiplied by the product of the 
antisymmetrizing operators corresponding to the columns and the
other consists of the product of the antisymmetrizing operators 
corresponding to the columns multiplied by the symmetrizing operators
corresponding to the rows.

How this works may be illustrated with a simple example.  Consider
the tableau
\[
\begin{matrix}
1 & 2\\
3 & 4
\end{matrix}.
\]

Corresponding to the first row, we have the symmetrization operator
\[
1 + (1\,2).
\]
Corresponding to the second row, we have the symmetrization operator
\[
1 + (3\,4)
\]
Multiplying these two symmetrization operators (the order does not
matter because they involve permutations of different elements)
produces
\[
1 + (1\,2) + (3\,4) + (1\,2) (3\,4)
\]

Corresponding to the first column, we have the antisymmetrization
operator
\[
1 - (1\,3).
\]
Corresponding to the second column, we have the antisymmetrization
operator
\[
1 - (2\,4).
\]
Multiplying these two antisymmetrization operators (the order does not
matter because they involve permutations of different elements)
produces
\[
1 - (1\,3) - (2\,4) + (1\,3) (2\,4).
\]

To obtain one Young projector, we multiply the product of
the symmetrization operators by the product of the 
antisymmetrization 
operators.
\[
\left( 1 + (1\,2) + (3\,4) + (1\,2) (3\,4) \right)
\left( 1 - (1\,3) - (2\,4) + (1\,3) (2\,4) \right) =
\]
\begin{eqnarray*}
1 &+& (1\,2) + (3\,4) + (1\,2) (3\,4) -
(1\,3) - (1\,2\,3) - (1\,3\,4) - (1\,2\,3\,4) - \\
(2\,4) &-& (1\,4\,2) - (2\,4\,3) - (1\,4\,3\,2) +
(1\,3) (2\,4) + (1\,4\,2\,3) + (1\,3\,2\,4) + (1\,4) (2\,3)
\end{eqnarray*}

To obtain the other projector, we multiply in the other order.
\[
\left( 1 - (1\,3) - (2\,4) + (1\,3) (2\,4) \right)
\left( 1 + (1\,2) + (3\,4) + (1\,2) (3\,4) \right) =
\]
\begin{eqnarray*}
1 &+& (1\,2) + (3\,4) + (1\,2) (3\,4) -
(1\,3) - (1\,3\,2) - (1\,4\,3) - (1\,4\,3\,2) - \\
(2\,4) &-& (1\,2\,4) - (2\,3\,4) - (1\,2\,3\,4) +
(1\,3) (2\,4) + (1\,3\,2\,4) + (1\,4\,2\,3) +
(1\,4)(2\,3)
\end{eqnarray*}
%%%%%
%%%%%
\end{document}
