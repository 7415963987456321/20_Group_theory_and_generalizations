\documentclass[12pt]{article}
\usepackage{pmmeta}
\pmcanonicalname{TransfiniteDerivedSeries}
\pmcreated{2013-03-22 14:16:33}
\pmmodified{2013-03-22 14:16:33}
\pmowner{yark}{2760}
\pmmodifier{yark}{2760}
\pmtitle{transfinite derived series}
\pmrecord{14}{35727}
\pmprivacy{1}
\pmauthor{yark}{2760}
\pmtype{Definition}
\pmcomment{trigger rebuild}
\pmclassification{msc}{20F19}
\pmclassification{msc}{20F14}
\pmrelated{DerivedSubgroup}
\pmdefines{perfect radical}
\pmdefines{maximum perfect subgroup}
\pmdefines{hypoabelianization}
\pmdefines{hypoabelianisation}

\usepackage{amssymb}
\usepackage{amsmath}
\usepackage{amsfonts}

\def\P#1{\mathcal{P}{#1}}
\begin{document}
\PMlinkescapeword{extension}
\PMlinkescapeword{name}
\PMlinkescapeword{rank}
\PMlinkescapeword{series}
\PMlinkescapeword{subgroup}
\PMlinkescapeword{subgroups}
\PMlinkescapeword{terms}

The \emph{transfinite derived series} of a group is 
an extension of its derived series, defined as follows.
Let $G$ be a group and let $G^{(0)}=G$.
For each ordinal $\alpha$ 
let $G^{(\alpha+1)}$ be the derived subgroup of $G^{(\alpha)}$.
For each limit ordinal $\delta$ 
let $G^{(\delta)}=\bigcap_{\alpha\in\delta}G^{(\alpha)}$.

Every member of the transfinite derived series of $G$ 
is a fully invariant subgroup of $G$.

The transfinite derived series eventually terminates, that is, 
there is some ordinal $\alpha$ such that $G^{(\alpha+1)}=G^{(\alpha)}$.
All remaining terms of the series are then equal to $G^{(\alpha)}$,
which is called the \emph{perfect radical} or \emph{maximum perfect subgroup}
of $G$, and is denoted $\P{G}$.
As the name suggests, $\P{G}$ is perfect, 
and every perfect \PMlinkname{subgroup}{Subgroup} of $G$ is a subgroup of $\P{G}$.
A group in which the perfect radical is trivial 
(that is, a group without any non-trivial perfect subgroups)
is called a hypoabelian group.
For any group $G$, the \PMlinkname{quotient}{QuotientGroup} $G/\P{G}$
is hypoabelian, and is sometimes called the \emph{hypoabelianization} of $G$
(by analogy with the abelianization).

A group $G$ for which $G^{(n)}$ is trivial for some finite $n$ 
is called a solvable group.
A group $G$ for which $G^{(\omega)}$ (the intersection of the derived series)
is trivial is called a residually solvable group.
\PMlinkname{Free groups}{FreeGroup} of rank greater than $1$
are examples of residually solvable groups that are not solvable.
%%%%%
%%%%%
\end{document}
