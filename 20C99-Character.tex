\documentclass[12pt]{article}
\usepackage{pmmeta}
\pmcanonicalname{Character}
\pmcreated{2013-03-22 12:17:54}
\pmmodified{2013-03-22 12:17:54}
\pmowner{djao}{24}
\pmmodifier{djao}{24}
\pmtitle{character}
\pmrecord{7}{31843}
\pmprivacy{1}
\pmauthor{djao}{24}
\pmtype{Definition}
\pmcomment{trigger rebuild}
\pmclassification{msc}{20C99}

% this is the default PlanetMath preamble.  as your knowledge
% of TeX increases, you will probably want to edit this, but
% it should be fine as is for beginners.

% almost certainly you want these
\usepackage{amssymb}
\usepackage{amsmath}
\usepackage{amsfonts}

% used for TeXing text within eps files
%\usepackage{psfrag}
% need this for including graphics (\includegraphics)
%\usepackage{graphicx}
% for neatly defining theorems and propositions
%\usepackage{amsthm}
% making logically defined graphics
%%%\usepackage{xypic} 

% there are many more packages, add them here as you need them

% define commands here
\begin{document}
Let $\rho: G \longrightarrow \operatorname{GL}(V)$ be a finite dimensional representation of a group $G$ (i.e., $V$ is a finite dimensional vector space over its scalar field $K$). The {\em character} of $\rho$ is the function $\chi_V: G \longrightarrow K$ defined by
$$
\chi_V(g) := \operatorname{Tr}(\rho(g))
$$
where $\operatorname{Tr}$ is the trace function.

Properties:
\begin{itemize}
\item $\chi_V(g) = \chi_V(h)$ if $g$ is conjugate to $h$ in $G$. (Equivalently, a character is a class function on $G$.)
\item If $G$ is finite, the characters of the irreducible representations of $G$ over the complex numbers form a basis of the vector space of all class functions on $G$ (with pointwise addition and scalar multiplication).
\item Over the complex numbers, the characters of the irreducible representations of $G$ are orthonormal under the inner product
$$
(\chi_1, \chi_2) := \frac{1}{|G|} \sum_{g \in G} \overline{\chi_1(g)} \chi_2(g)
$$
\end{itemize}
%%%%%
%%%%%
\end{document}
