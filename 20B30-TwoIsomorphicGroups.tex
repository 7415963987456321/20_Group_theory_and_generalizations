\documentclass[12pt]{article}
\usepackage{pmmeta}
\pmcanonicalname{TwoIsomorphicGroups}
\pmcreated{2013-03-22 15:52:28}
\pmmodified{2013-03-22 15:52:28}
\pmowner{Wkbj79}{1863}
\pmmodifier{Wkbj79}{1863}
\pmtitle{two isomorphic groups}
\pmrecord{10}{37871}
\pmprivacy{1}
\pmauthor{Wkbj79}{1863}
\pmtype{Example}
\pmcomment{trigger rebuild}
\pmclassification{msc}{20B30}

\endmetadata

% this is the default PlanetMath preamble.  as your knowledge
% of TeX increases, you will probably want to edit this, but
% it should be fine as is for beginners.

% almost certainly you want these
\usepackage{amssymb}
\usepackage{amsmath}
\usepackage{amsfonts}

% used for TeXing text within eps files
%\usepackage{psfrag}
% need this for including graphics (\includegraphics)
%\usepackage{graphicx}
% for neatly defining theorems and propositions
%\usepackage{amsthm}
% making logically defined graphics
%%%\usepackage{xypic}

% there are many more packages, add them here as you need them

% define commands here
\begin{document}
\PMlinkescapeword{block}
\PMlinkescapeword{calculate}
\PMlinkescapeword{complete}
\PMlinkescapeword{completes}
\PMlinkescapeword{right}
\PMlinkescapeword{structure}

The set of $3\operatorname{x}3$ permutation matrices form a group under matrix multiplication.  This example demonstrates that fact and develops the multiplication table and compares it to $S_3$.  Although there are alternative ways to fill in the table, this example serves to help the beginner.  Here we will see that the two groups have the
same structure.  We begin by defining the elements of our group.

$\begin{array}{cc}
\hspace{45mm}
I=\left( \begin{array}{ccc}
1 & 0 & 0 \\
0 & 1 & 0 \\
0 & 0 & 1
\end{array} \right )

&

R=\left( \begin{array}{ccc}
1 & 0 & 0 \\
0 & 0 & 1 \\
0 & 1 & 0
\end{array} \right ) \\ \\

\hspace{45mm}

A=\left( \begin{array}{ccc}
0 & 1 & 0 \\
0 & 0 & 1 \\
1 & 0 & 0
\end{array} \right )

&

S=\left( \begin{array}{ccc}
0 & 0 & 1 \\
0 & 1 & 0 \\
1 & 0 & 0
\end{array} \right )
\\
\\

\hspace{45mm}

 B=\left( \begin{array}{ccc}
0 & 0 & 1 \\
1 & 0 & 0 \\
0 & 1 & 0
\end{array} \right )

&

T=\left( \begin{array}{ccc}
0 & 1 & 0 \\
1 & 0 & 0 \\
0 & 0 & 1
\end{array} \right )

\end{array}$

Here, our group is just $P_3=\{I,A,B,R,S,T\}$.  Now, we can start to multiply and then fill in the table.
First, we calculate the square of each elements.

\begin{displaymath}
A^2=\left( \begin{array}{ccc}
0 & 1 & 0 \\
0 & 0 & 1 \\
1 & 0 & 0
\end{array} \right )
\left( \begin{array}{ccc}
0 & 1 & 0 \\
0 & 0 & 1 \\
1 & 0 & 0
\end{array} \right )=
\left( \begin{array}{ccc}
0 & 0 & 1 \\
1 & 0 & 0 \\
0 & 1 & 0
\end{array} \right )=B
\end{displaymath}

\begin{displaymath}
B^2=\left( \begin{array}{ccc}
0 & 0 & 1 \\
1 & 0 & 0 \\
0 & 1 & 0
\end{array} \right )
\left( \begin{array}{ccc}
0 & 0 & 1 \\
1 & 0 & 0\\
0 & 1 & 0
\end{array} \right )=
\left( \begin{array}{ccc}
0 & 1 & 0 \\
0 & 0 & 1 \\
1 & 0 & 0
\end{array} \right )=A
\end{displaymath}

\begin{displaymath}
R^2=\left( \begin{array}{ccc}
1 & 0 & 0 \\
0 & 0 & 1\\
0 & 1 & 0
\end{array} \right )
\left( \begin{array}{ccc}
1 & 0 & 0 \\
0 & 0 & 1\\
0 & 1 & 0
\end{array} \right )=
\left( \begin{array}{ccc}
1 & 0 & 0 \\
0 & 1 & 0\\
0 & 0 & 1
\end{array} \right )=I
\end{displaymath}

\begin{displaymath}
S^2=\left( \begin{array}{ccc}
0 & 0 & 1 \\
0 & 1 & 0\\
1 & 0 & 0
\end{array} \right )
\left( \begin{array}{ccc}
0 & 0 & 1 \\
0 & 1 & 0\\
1 & 0 & 0
\end{array} \right )=
\left( \begin{array}{ccc}
1 & 0 & 0 \\
0 & 1 & 0\\
0 & 0 & 1
\end{array} \right )=I
\end{displaymath}

\begin{displaymath}
T^2=\left( \begin{array}{ccc}
0 & 1 & 0 \\
1 & 0 & 0\\
0 & 0 & 1
\end{array} \right )
\left( \begin{array}{ccc}
0 & 1 & 0 \\
1 & 0 & 0\\
0 & 0 & 1
\end{array} \right )=
\left( \begin{array}{ccc}
1 & 0 & 0 \\
0 & 1 & 0\\
0 & 0 & 1
\end{array} \right )=I
\end{displaymath}

Now starting with the upper left $3\operatorname{x}3$ block, we go through the table.

\begin{displaymath}
AB=\left( \begin{array}{ccc}
0 & 1 & 0 \\
0 & 0 & 1\\
1 & 0 & 0
\end{array} \right )
\left( \begin{array}{ccc}
0 & 0 & 1 \\
1 & 0 & 0\\
0 & 1 & 0
\end{array} \right )=
\left( \begin{array}{ccc}
1 & 0 & 0 \\
0 & 1 & 0\\
0 & 0 & 1
\end{array} \right )=I
\end{displaymath}

\begin{displaymath}
BA= \left( \begin{array}{ccc}
0 & 0 & 1 \\
1 & 0 & 0\\
0 & 1 & 0
\end{array} \right )
\left( \begin{array}{ccc}
0 & 1 & 0 \\
0 & 0 & 1\\
1 & 0 & 0
\end{array} \right )=
\left( \begin{array}{ccc}
1 & 0 & 0 \\
0 & 1 & 0\\
0 & 0 & 1
\end{array} \right )=I
\end{displaymath}

We can complete the upper left $3\operatorname{x}3$ block of the table and complete diagonal using the above values.  We note that no row or column can have a repeated elements which follows from the \PMlinkescapetext{algebraic structure} of a group.  Next, we work on the upper right $3\operatorname{x}3$ block of the table.

\begin{displaymath}
AR=\left( \begin{array}{ccc}
0 & 1 & 0 \\
0 & 0 & 1 \\
1 & 0 & 0
\end{array} \right )
\left( \begin{array}{ccc}
1 & 0 & 0 \\
0 & 0 & 1 \\
0 & 1 & 0
\end{array} \right )=
\left( \begin{array}{ccc}
0 & 0 & 1 \\
0 & 1 & 0 \\
1 & 0 & 0
\end{array} \right )=S
\end{displaymath}

\begin{displaymath}
AS=\left( \begin{array}{ccc}
0 & 1 & 0 \\
0 & 0 & 1 \\
1 & 0 & 0
\end{array} \right )
\left( \begin{array}{ccc}
0 & 0 & 1 \\
0 & 1 & 0 \\
1 & 0 & 0
\end{array} \right )=
\left( \begin{array}{ccc}
0 & 1 & 0 \\
1 & 0 & 0 \\
0 & 0 & 1
\end{array} \right )=T
\end{displaymath}

Now we can complete the upper right 3 x 3 block of the table.  Next, we work on the lower left $3\operatorname{x}3$ block of the table.

\begin{displaymath}
RA= \left( \begin{array}{ccc}
1 & 0 & 0 \\
0 & 0 & 1 \\
0 & 1 & 0
\end{array} \right )
\left( \begin{array}{ccc}
0 & 1 & 0 \\
0 & 0 & 1 \\
1 & 0 & 0
\end{array} \right )=
\left( \begin{array}{ccc}
0 & 1 & 0 \\
1 & 0 & 0 \\
0 & 0 & 1
\end{array} \right )=T
\end{displaymath}

\begin{displaymath}
RB= \left( \begin{array}{ccc}
1 & 0 & 0 \\
0 & 0 & 1 \\
0 & 1 & 0
\end{array} \right )
\left( \begin{array}{ccc}
0 & 0 & 1 \\
1 & 0 & 0 \\
0 & 1 & 0
\end{array} \right )=
\left( \begin{array}{ccc}
0 & 0 & 1 \\
0 & 1 & 0 \\
1 & 0 & 0
\end{array} \right )=S
\end{displaymath}

\begin{displaymath}
SA= \left( \begin{array}{ccc}
0 & 0 & 1 \\
0 & 1 & 0 \\
1 & 0 & 0
\end{array} \right )
\left( \begin{array}{ccc}
0 & 1 & 0 \\
0 & 0 & 1 \\
1 & 0 & 0
\end{array} \right )=
\left( \begin{array}{ccc}
1 & 0 & 0 \\
0 & 0 & 1 \\
0 & 1 & 0
\end{array} \right )=R
\end{displaymath}

Now we can complete the lower left $3\operatorname{x}3$ block of the table.  Finally, we work on the lower right $3\operatorname{x}3$ block of the table.

\begin{displaymath}
RS=\left( \begin{array}{ccc}
1 & 0 & 0 \\
0 & 0 & 1 \\
0 & 1 & 0
\end{array} \right )
\left( \begin{array}{ccc}
0 & 0 & 1 \\
0 & 1 & 0 \\
1 & 0 & 0
\end{array} \right )
= \left( \begin{array}{ccc}
0 & 0 & 1 \\
1 & 0 & 0 \\
0 & 1 & 0
\end{array} \right )=B
\end{displaymath}

\begin{displaymath}
SR=\left( \begin{array}{ccc}
0 & 0 & 1 \\
0 & 1 & 0 \\
1 & 0 & 0
\end{array} \right )
\left( \begin{array}{ccc}
1 & 0 & 0 \\
0 & 0 & 1 \\
0 & 1 & 0
\end{array} \right )
= \left( \begin{array}{ccc}
0 & 1 & 0 \\
1 & 0 & 0 \\
0 & 0 & 1
\end{array} \right )=A
\end{displaymath}

This completes the multiplication and the table is given below.

\begin{center}
$\begin{array}{|c||c|c|c|c|c|c|}
\hline
 & I & A & B & R & S & T \\
\hline \hline
I & I & A & B & R & S & T \\
\hline
A & A & B & I & S & T & R \\
\hline
B & B & I & A & T & R & S\\
\hline
R & R & T & S & I & B & A\\
\hline
S & S & R & T & A & I & B\\
\hline
T & T & S & R & B & A & I\\
\hline
\end{array}$
\end{center}

Next, we want to compare this table to the symmetric group $S_3$.  We begin as before by defining the elements
as follows.

$$e={ 1\ 2\ 3 \choose 1\ 2\ 3} \hspace{20mm} r={ 1\ 2\ 3 \choose 2\ 1\ 3}$$
$$a={ 1\ 2\ 3 \choose 2\ 3\ 1} \hspace{20mm} s={ 1\ 2\ 3 \choose 3\ 2\ 1}$$
$$b={ 1\ 2\ 3 \choose 3\ 1\ 2} \hspace{20mm} t={ 1\ 2\ 3 \choose 1\ 3\ 2}$$

The multiplication table for this group is obtained within the entry symmetric group on three letters.  The table is:

\begin{center}
$\begin{array}{|c||c|c|c|c|c|c|}
\hline
 & e & a & b & r & s & t \\
\hline
e & e & a & b & r & s & t \\
\hline
a & a & b & e & s & t & r \\
\hline
b & b & e & a & t & r & s\\
\hline
r & r & t & s & e & b & a\\
\hline
s & s & r & t & a & e & b\\
\hline
t & t & s & r & b & a & e\\
\hline
\end{array}$
\end{center}

Define the following homomorphism $\varphi \colon S_3 \to P_3$ by the following:

$\varphi(e)=I$;

$\varphi(a)=A$;

$\varphi(b)=B$;

$\varphi(r)=R$;

$\varphi(s)=S$;

$\varphi(t)=T$.

Since $\varphi$ is a bijection, we conclude that $P_3$ and $S_3$ are isomorphic.
%%%%%
%%%%%
\end{document}
