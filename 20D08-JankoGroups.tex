\documentclass[12pt]{article}
\usepackage{pmmeta}
\pmcanonicalname{JankoGroups}
\pmcreated{2013-03-22 13:59:17}
\pmmodified{2013-03-22 13:59:17}
\pmowner{mathcam}{2727}
\pmmodifier{mathcam}{2727}
\pmtitle{Janko groups}
\pmrecord{12}{34762}
\pmprivacy{1}
\pmauthor{mathcam}{2727}
\pmtype{Definition}
\pmcomment{trigger rebuild}
\pmclassification{msc}{20D08}
%\pmkeywords{sporadic groups}
\pmrelated{ExamplesOfFiniteSimpleGroups}
\pmrelated{Solvable}

% this is the default PlanetMath preamble.  as your knowledge
% of TeX increases, you will probably want to edit this, but
% it should be fine as is for beginners.

% almost certainly you want these
\usepackage{amssymb}
\usepackage{amsmath}
\usepackage{amsfonts}

% used for TeXing text within eps files
%\usepackage{psfrag}
% need this for including graphics (\includegraphics)
%\usepackage{graphicx}
% for neatly defining theorems and propositions
%\usepackage{amsthm}
% making logically defined graphics
%%%\usepackage{xypic}

% there are many more packages, add them here as you need them

% define commands here
\begin{document}
\PMlinkescapeword{group}
\PMlinkescapeword{algebra}
\PMlinkescapeword{simple}
\PMlinkescapeword{finite}
\PMlinkescapeword{solvable}

The Janko groups denoted by $J_1, J_2, J_3$, and $J_4$ are four of the 26 sporadic groups.  They were discovered by \PMlinkescapetext{Z}. Janko in 1966 and published  in the article "A new finite simple group with abelian Sylow $2$-subgroups and its characterization.''  (Journal of Algebra, \textbf{3}, 1966, 32: 147-186).

Each of these groups have very intricate matrix representations as maps into large general linear groups.  For example, the matrix $K$ corresponding to $J_4$ gives a representation of $J_4$ in $GL_{112}(2)$.
%%%%%
%%%%%
\end{document}
