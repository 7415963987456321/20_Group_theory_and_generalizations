\documentclass[12pt]{article}
\usepackage{pmmeta}
\pmcanonicalname{Subcommutative}
\pmcreated{2013-03-22 19:13:45}
\pmmodified{2013-03-22 19:13:45}
\pmowner{pahio}{2872}
\pmmodifier{pahio}{2872}
\pmtitle{subcommutative}
\pmrecord{12}{42152}
\pmprivacy{1}
\pmauthor{pahio}{2872}
\pmtype{Definition}
\pmcomment{trigger rebuild}
\pmclassification{msc}{20M25}
\pmclassification{msc}{20M99}
%\pmkeywords{subcommutative semigroup}
%\pmkeywords{subcommutative ring}
\pmrelated{Commutative}
\pmrelated{Klein4Ring}
\pmrelated{Anticommutative}
\pmdefines{left subcommutative}
\pmdefines{right subcommutative}

% this is the default PlanetMath preamble.  as your knowledge
% of TeX increases, you will probably want to edit this, but
% it should be fine as is for beginners.

% almost certainly you want these
\usepackage{amssymb}
\usepackage{amsmath}
\usepackage{amsfonts}

% used for TeXing text within eps files
%\usepackage{psfrag}
% need this for including graphics (\includegraphics)
%\usepackage{graphicx}
% for neatly defining theorems and propositions
 \usepackage{amsthm}
% making logically defined graphics
%%%\usepackage{xypic}

% there are many more packages, add them here as you need them

% define commands here

\theoremstyle{definition}
\newtheorem*{thmplain}{Theorem}

\begin{document}
A semigroup \,$(S,\,\cdot)$\, is said to be \emph{left subcommutative} if for any two of its elements $a$ and $b$, there exists its element $c$ such that
\begin{align}
ab = ca.
\end{align}
A semigroup \,$(S,\,\cdot)$\, is said to be \emph{right subcommutative} if for any two of its elements $a$ and $b$, there exists its element $d$ such that
\begin{align}
ab = bd.
\end{align}
If $S$ is both left subcommutative and right subcommutative, it is \emph{subcommutative}.

The commutativity is a special case of all the three kinds of subcommutativity.\\

\textbf{Example 1.}\, The following operation table defines a right subcommutative semigroup\, $\{0,\,1,\,2,\,3\}$\, which is not left subcommutative (e.g. $0\!\cdot\!3 = 2 \neq c\!\cdot\!0$):
$$\begin{array}{c|cccc}
\cdot & 0 & 1 & 2 & 3 \\
\hline
\;  0 & 0 & 0 & 2 & 2 \\
\;  1 & 0 & 1 & 2 & 3 \\
\;  2 & 0 & 0 & 2 & 2 \\
\;  3 & 0 & 1 & 2 & 3
\end{array}$$\\

\textbf{Example 2.}\, The \PMlinkescapetext{multiplicative} group of the square matrices over a field is both left and right subcommutative (but not commutative), since the equations (1) and (2) are satisfied by 
$$c \;=\; aba^{-1} \quad \mbox{and} \quad d\;=\; b^{-1}ab.$$\\



\textbf{Remark.}\, One uses the above \PMlinkescapetext{attributes} also for a ring \,$(S,\,+,\,\cdot)$\, if its multiplicative semigroup \,$(S,\,\cdot)$\, satisfies the corresponding requirements.

\begin{thebibliography}{8}
\bibitem{LS}{\sc S. Lajos}: ``On $(m,\,n)$-ideals in subcommutative semigroups''.\, -- \emph{Elemente der Mathematik} \textbf{24} (1969).
\bibitem{VE}{\sc V. P. Elizarov}: ``Subcommutative Q-rings''.\, -- \emph{Mathematical notes} \textbf{2} (1967).
\end{thebibliography}

%%%%%
%%%%%
\end{document}
