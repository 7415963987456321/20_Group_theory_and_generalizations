\documentclass[12pt]{article}
\usepackage{pmmeta}
\pmcanonicalname{SubgroupsOfFiniteCyclicGroup}
\pmcreated{2013-03-22 18:57:13}
\pmmodified{2013-03-22 18:57:13}
\pmowner{pahio}{2872}
\pmmodifier{pahio}{2872}
\pmtitle{subgroups of finite cyclic group}
\pmrecord{5}{41810}
\pmprivacy{1}
\pmauthor{pahio}{2872}
\pmtype{Theorem}
\pmcomment{trigger rebuild}
\pmclassification{msc}{20A05}

% this is the default PlanetMath preamble.  as your knowledge
% of TeX increases, you will probably want to edit this, but
% it should be fine as is for beginners.

% almost certainly you want these
\usepackage{amssymb}
\usepackage{amsmath}
\usepackage{amsfonts}

% used for TeXing text within eps files
%\usepackage{psfrag}
% need this for including graphics (\includegraphics)
%\usepackage{graphicx}
% for neatly defining theorems and propositions
 \usepackage{amsthm}
% making logically defined graphics
%%%\usepackage{xypic}

% there are many more packages, add them here as you need them

% define commands here

\theoremstyle{definition}
\newtheorem*{thmplain}{Theorem}

\begin{document}
Let $n$ be the order of a finite cyclic group $G$.\, For every positive \PMlinkname{divisor}{Divisibility} $m$ of $n$, there exists one and only one subgroup of order $m$ of $G$.\, The group $G$ has no other subgroups.\\

\emph{Proof.}\, If $g$ is a generator of $G$ and\, $n = mk$,\, then $g^k$ generates the subgroup $\langle g^k\rangle$, the order of which is equal to the order of $g^k$, i.e. equal to $m$.\, Any subgroup $H$ of $G$ is cyclic (see \PMlinkid{this entry}{4097}).\, If\, $|H| = m$,\, then $H$ must have a generator of order $m$; thus apparently\, $H = \langle g^{\pm k}\rangle = \langle g^k\rangle$.
%%%%%
%%%%%
\end{document}
