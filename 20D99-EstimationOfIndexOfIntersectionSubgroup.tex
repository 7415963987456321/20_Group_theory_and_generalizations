\documentclass[12pt]{article}
\usepackage{pmmeta}
\pmcanonicalname{EstimationOfIndexOfIntersectionSubgroup}
\pmcreated{2013-03-22 18:56:46}
\pmmodified{2013-03-22 18:56:46}
\pmowner{pahio}{2872}
\pmmodifier{pahio}{2872}
\pmtitle{estimation of index of intersection subgroup}
\pmrecord{5}{41803}
\pmprivacy{1}
\pmauthor{pahio}{2872}
\pmtype{Theorem}
\pmcomment{trigger rebuild}
\pmclassification{msc}{20D99}
\pmsynonym{index of intersection subgroup}{EstimationOfIndexOfIntersectionSubgroup}
%\pmkeywords{index of subgroup}
%\pmkeywords{intersection of subgroups}
\pmrelated{LogicalAnd}
\pmrelated{Cardinality}

\endmetadata

% this is the default PlanetMath preamble.  as your knowledge
% of TeX increases, you will probably want to edit this, but
% it should be fine as is for beginners.

% almost certainly you want these
\usepackage{amssymb}
\usepackage{amsmath}
\usepackage{amsfonts}

% used for TeXing text within eps files
%\usepackage{psfrag}
% need this for including graphics (\includegraphics)
%\usepackage{graphicx}
% for neatly defining theorems and propositions
 \usepackage{amsthm}
% making logically defined graphics
%%%\usepackage{xypic}

% there are many more packages, add them here as you need them

% define commands here
\DeclareMathOperator{\card}{card}
\theoremstyle{definition}
\newtheorem*{thmplain}{Theorem}

\begin{document}
\textbf{Theorem.}\, If $H_1,\,H_2,\,\ldots,\,H_n$ are subgroups of $G$, then
$$\left[G:\bigcap_{i=1}^nH_i\right] \leqq \prod_{i=1}^n[G:H_i].$$\\

\emph{Proof.}\, We prove here only the case \,$n = 2$;\, the general case may be handled by the induction.

Let\, $H_1\!\cap\!H_2 := K$.\, Let $R$ be the set of the right cosets of $K$ and $R_i$ the set of the right cosets of 
$H_i$\, ($i = 1,\,2$).\, Define the relation $\varrho$ from $R$ to $R_1\!\times\!R_2$ as
$$\varrho \;:=\; \{\left(Kx,\,(H_1x,\,H_2x)\right)\vdots\;\; x \in G \}.$$
We then have the \PMlinkname{equivalent}{Equivalent3} conditions
$$Kx \;=\; Ky,$$
$$xy^{-1} \in K,$$
$$xy^{-1} \in H_1 \quad\land\quad xy^{-1} \in H_2,$$
$$H_1x \;=\; H_1y \quad\land\quad H_2x \;=\; H_2y,$$
$$(H_1x,\,H_2x) \;=\; (H_1y,\,H_2y),$$
whence $\varrho$ is a mapping and \PMlinkescapetext{even} injective,\; $\varrho:\, R \to R_1\!\times\!R_2$.\, i.e. it is a bijection from $R$ onto the subset \,$\{\varrho(x)\vdots\;\; x \in R\}$\, of $R_1\!\times\!R_2$.\, Therefore,
$$\card(R) \;\leqq\; \card(R_1\!\times\!R_2) \;=\; \card(R_1)\cdot\card(R_2).$$\\


As a consequence one obtains the

\textbf{Theorem (Poincar\'e}).\, The index of the intersection of finitely many subgroups with finite \PMlinkname{indices}{Coset} is finite.


%%%%%
%%%%%
\end{document}
