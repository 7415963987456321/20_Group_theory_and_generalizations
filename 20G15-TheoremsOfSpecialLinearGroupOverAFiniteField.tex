\documentclass[12pt]{article}
\usepackage{pmmeta}
\pmcanonicalname{TheoremsOfSpecialLinearGroupOverAFiniteField}
\pmcreated{2013-03-22 14:55:54}
\pmmodified{2013-03-22 14:55:54}
\pmowner{Daume}{40}
\pmmodifier{Daume}{40}
\pmtitle{theorems of special linear group over a finite field}
\pmrecord{6}{36620}
\pmprivacy{1}
\pmauthor{Daume}{40}
\pmtype{Theorem}
\pmcomment{trigger rebuild}
\pmclassification{msc}{20G15}
\pmrelated{ProjectiveSpecialLinearGroup}

% this is the default PlanetMath preamble.  as your knowledge
% of TeX increases, you will probably want to edit this, but
% it should be fine as is for beginners.

% almost certainly you want these
\usepackage{amssymb}
\usepackage{amsmath}
\usepackage{amsfonts}

% used for TeXing text within eps files
%\usepackage{psfrag}
% need this for including graphics (\includegraphics)
%\usepackage{graphicx}
% for neatly defining theorems and propositions
%\usepackage{amsthm}
% making logically defined graphics
%%%\usepackage{xypic} 

% there are many more packages, add them here as you need them

% define commands here

% The below lines should work as the command
% \renewcommand{\bibname}{References}
% without creating havoc when rendering an entry in
% the page-image mode.
\makeatletter
\@ifundefined{bibname}{}{\renewcommand{\bibname}{References}}
\makeatother
\begin{document}
Let $\mathbb{F}_q$ be the finite field with $q$ elements, and consider the special linear group $\operatorname{SL}(n, \mathbb{F}_q)$ over the field $\mathbb{F}_q$.
\begin{enumerate}
\item $\operatorname{SL}(n, \mathbb F_q)$ is finite. Furthermore, $\lvert \operatorname{SL}(n, \mathbb{F}_q) \rvert = \frac{1}{q-1}\prod_{i=0}^{n-1}(q^n-q^i)$.
\item $\operatorname{SL}(n, \mathbb F_q)$ is a perfect group, meaning that $[\operatorname{SL}(n,\mathbb{F}_q),\operatorname{SL}(n,\mathbb{F}_q)] = \operatorname{SL}(n,\mathbb{F}_q)$, where $[,]$ is the commutator bracket with two  exceptions: $\operatorname{SL}(2,\mathbb{F}_2)$ and $\operatorname{SL}(2,\mathbb{F}_3)$.
\end{enumerate}
%%%%%
%%%%%
\end{document}
