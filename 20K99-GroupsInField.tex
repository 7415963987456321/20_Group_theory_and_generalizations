\documentclass[12pt]{article}
\usepackage{pmmeta}
\pmcanonicalname{GroupsInField}
\pmcreated{2013-03-22 14:41:58}
\pmmodified{2013-03-22 14:41:58}
\pmowner{pahio}{2872}
\pmmodifier{pahio}{2872}
\pmtitle{groups in field}
\pmrecord{24}{36311}
\pmprivacy{1}
\pmauthor{pahio}{2872}
\pmtype{Topic}
\pmcomment{trigger rebuild}
\pmclassification{msc}{20K99}
\pmclassification{msc}{20F99}
\pmclassification{msc}{20A05}
\pmclassification{msc}{12E99}
\pmrelated{Klein4Group}
\pmrelated{Klein4Ring}
\pmrelated{GroupsOfRealNumbers}
\pmdefines{additive group of the field}
\pmdefines{multiplicative group of the field}
\pmdefines{additive group}
\pmdefines{multiplicative group}

% this is the default PlanetMath preamble.  as your knowledge
% of TeX increases, you will probably want to edit this, but
% it should be fine as is for beginners.

% almost certainly you want these
\usepackage{amssymb}
\usepackage{amsmath}
\usepackage{amsfonts}

% used for TeXing text within eps files
%\usepackage{psfrag}
% need this for including graphics (\includegraphics)
%\usepackage{graphicx}
% for neatly defining theorems and propositions
%\usepackage{amsthm}
% making logically defined graphics
%%%\usepackage{xypic}

% there are many more packages, add them here as you need them

% define commands here
\begin{document}
If \,$(K,\,+,\,\cdot)$\, is a field, then
\begin{itemize}
\item $(K,\,+)$ \,is the {\em additive group of the field},
\item $(K\!\smallsetminus\!\{0\},\,\cdot)$ \,is the {\em multiplicative group of the field}.
\end{itemize}
Both of these groups are Abelian.

The former has always as a subgroup 
          $$\{n\!\cdot\!1\vdots \,\,\,n\in\mathbb{Z}\},$$
the group of the multiples of unity.\, This is, apparently, isomorphic to 
the additive group $\mathbb{Z}$ or $\mathbb{Z}_p$ depending on whether the \PMlinkname{characteristic}{Characteristic} of the field is 0 or a prime number $p$.

The multiplicative group of any field has as its subgroup the set $E$ consisting of all roots of unity in the field.\, The group $E$ has the subgroup\, $\{1,\,-1\}$\, which reduces to $\{1\}$ if the \PMlinkescapetext{characteristic} of the field is two.\, 


\textbf{Example 1.}\, The additive group\, $(\mathbb{R},\,+)$\, of the reals is isomorphic to the multiplicative group\, $(\mathbb{R}_+,\,\cdot)$\, of the positive reals; the isomorphy is implemented e.g. by the isomorphism mapping \,$x\mapsto 2^x$.

\textbf{Example 2.}\, Suppose that the \PMlinkescapetext{characteristic} of $K$ is not 2 and denote the multiplicative group of $K$ by $K^*$.\, We can consider the four functions \, $f_i\!:K^*\!\to\!K^*$\, defined by\, $f_0(x) := x$,\, 
$f_1(x) := -x$,\, $f_2(x) := x^{-1}$,\, $f_3(x) := -x^{-1}$.\, The composition of functions is a binary operation of the set\, $G = \{f_0,\,f_1,\,f_2,\,f_3\}$,\, and we see that $G$ is isomorphic to Klein's 4-group.

\textbf{Note.}\, One may also speak of the {\em additive group} of any {\em ring}.\, Every ring contains also its group of units.
%%%%%
%%%%%
\end{document}
