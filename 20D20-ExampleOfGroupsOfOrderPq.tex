\documentclass[12pt]{article}
\usepackage{pmmeta}
\pmcanonicalname{ExampleOfGroupsOfOrderPq}
\pmcreated{2013-03-22 14:51:15}
\pmmodified{2013-03-22 14:51:15}
\pmowner{jh}{7326}
\pmmodifier{jh}{7326}
\pmtitle{example of groups of order pq}
\pmrecord{4}{36526}
\pmprivacy{1}
\pmauthor{jh}{7326}
\pmtype{Example}
\pmcomment{trigger rebuild}
\pmclassification{msc}{20D20}

% this is the default PlanetMath preamble.  as your knowledge
% of TeX increases, you will probably want to edit this, but
% it should be fine as is for beginners.

% almost certainly you want these
\usepackage{amssymb}
\usepackage{amsmath}
\usepackage{amsfonts}

% used for TeXing text within eps files
%\usepackage{psfrag}
% need this for including graphics (\includegraphics)
%\usepackage{graphicx}
% for neatly defining theorems and propositions
%\usepackage{amsthm}
% making logically defined graphics
%%%\usepackage{xypic}

% there are many more packages, add them here as you need them

% define commands here
\newcommand{\norm}[1]{\left\Vert#1\right\Vert}
\newcommand{\abs}[1]{\left\vert#1\right\vert}
\newcommand{\inn}[1]{\langle#1\rangle}
\newcommand{\set}[1]{\{#1\}}
\newcommand{\Integer}{\mathbb{Z}}
\newcommand{\Rational}{\mathbb{Q}}
\newcommand{\Real}{\mathbb{R}}
\newcommand{\Complex}{\mathbb{C}}
\newcommand{\Field}{\mathbb{F}}
\newcommand{\eps}{\varepsilon}
\newcommand{\Basis}{\mathscr{B}}
\newcommand{\CG}{\Complex G}
\newcommand{\Mn}{\mathcal{M}}
\newcommand{\GL}{\mathrm{GL}}
\newcommand{\SL}{\mathrm{SL}}
\newcommand{\Sn}{\mathrm{S}}
\newcommand{\Orbit}{\mathcal{O}}
\DeclareMathOperator{\Aut}{Aut} \DeclareMathOperator{\Inn}{Inn}
\DeclareMathOperator{\Sym}{Sym} \DeclareMathOperator{\Alt}{Alt}
\DeclareMathOperator{\tr}{tr} \DeclareMathOperator{\Stab}{Stab}
\DeclareMathOperator{\sgn}{sgn} \DeclareMathOperator{\Hom}{Hom}
\DeclareMathOperator{\End}{End} \DeclareMathOperator{\Res}{Res}
\DeclareMathOperator{\Ind}{Ind}
\begin{document}
As a specific example, let us classify groups of order 21. Let $G$ be a group of order 21. There is only one Sylow 7-subgroup $K$ so it is normal. The possibility of there being conjugate Sylow 3-subgroups is not ruled out. Let $x$ denote a generator for $K$, and $y$ a generator for one of the Sylow 3-subgroups $H$. Then $x^7=y^3=1$, and $yxy^{-1}=x^i$ for some $i<7$ since $K$ is normal. Now $x=y^3 x y^{-3}=y^2 x^i y^{-2}= y x^{i^2} y^{-1}=x^{i^3}$, or $i^3=1 \mod 7$. This implies $i=1,2$, or 4.

Case 1: $yxy^{-1}=x$, so $G$ is abelian and isomorphic to
$C_{21}=C_3 \times C_7$.

Case 2: $yxy^{-1}=x^2$, then every product of the elements $x,y$
can be reduced to one in the form $x^i y^j$, $0 \leq i <7$, $0
\leq j <3$. These 21 elements are clearly distinct, so $G=\inn{x,y
\mid x^7=y^3=1, yx=x^2y}$.

Case 3: $yxy^{-1}=x^4$, then since $y^2$ is also a generator of
$H$ and $y^2 x y^{-2}=y x^4 y^{-1}=x^{16}=x^2$, we have recovered
case 2 above.
%%%%%
%%%%%
\end{document}
