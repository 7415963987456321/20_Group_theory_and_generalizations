\documentclass[12pt]{article}
\usepackage{pmmeta}
\pmcanonicalname{AlternativeDefinitionOfGroup}
\pmcreated{2013-03-22 15:07:58}
\pmmodified{2013-03-22 15:07:58}
\pmowner{pahio}{2872}
\pmmodifier{pahio}{2872}
\pmtitle{alternative definition of group}
\pmrecord{13}{36876}
\pmprivacy{1}
\pmauthor{pahio}{2872}
\pmtype{Theorem}
\pmcomment{trigger rebuild}
\pmclassification{msc}{20A05}
\pmclassification{msc}{20-00}
\pmclassification{msc}{08A99}
%\pmkeywords{definition}
\pmrelated{Characterization}
\pmrelated{ACharacterizationOfGroups}
\pmrelated{DivisionInGroup}
\pmrelated{MoreOnDivisionInGroups}
\pmrelated{LoopAndQuasigroup}

\endmetadata

% this is the default PlanetMath preamble.  as your knowledge
% of TeX increases, you will probably want to edit this, but
% it should be fine as is for beginners.

% almost certainly you want these
\usepackage{amssymb}
\usepackage{amsmath}
\usepackage{amsfonts}

% used for TeXing text within eps files
%\usepackage{psfrag}
% need this for including graphics (\includegraphics)
%\usepackage{graphicx}
% for neatly defining theorems and propositions
 \usepackage{amsthm}
% making logically defined graphics
%%%\usepackage{xypic}

% there are many more packages, add them here as you need them

% define commands here
\theoremstyle{definition}
\newtheorem*{thmplain}{Theorem}
\begin{document}
The below theorem gives three conditions that form alternative \PMlinkescapetext{group} postulates. \,It is not hard to show that they hold in the group defined ordinarily.

\begin{thmplain}
 \,Let the non-empty set $G$ satisfy the following three conditions:\\
I. \,\,\,\,For every two elements $a$, $b$ of $G$ there is a unique element $ab$ of $G$.\\
II. \,\,For every three elements $a$, $b$, $c$ of $G$ the equation\, $(ab)c = a(bc)$\, holds.\\
III. For every two elements $a$ and $b$ of $G$ there exists at least one such element $x$ and at least one such element $y$ of $G$ that\, $xa = ay = b$.\\
Then the set $G$ forms a group.
\end{thmplain}

{\em Proof.} \,If $a$ and $b$ are arbitrary elements, then there are at least one such $e_a$ and such $e_b$ that\, $e_aa = a$\, and\, $be_b = b$.\, There are also such $x$ and $y$ that\, $xb = e_a$\, and\, $ay = e_b$.\, Thus we have
   $$e_a = xb = x(be_b) = (xb)e_b = e_ae_b = e_a(ay) = (e_aa)y = ay = e_b,$$
i.e. there is a unique neutral element $e$ in $G$.\, Moreover, for any element $a$ there is at least one couple $a'$, $a''$ such that\,  $a'a = aa'' = e$.\, We then see that
  $$a' = a'e = a'(aa'') = (a'a)a'' = ea'' = a'',$$
i.e. $a$ has a unique neutralizing element $a'$.
%%%%%
%%%%%
\end{document}
