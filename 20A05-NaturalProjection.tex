\documentclass[12pt]{article}
\usepackage{pmmeta}
\pmcanonicalname{NaturalProjection}
\pmcreated{2013-03-22 19:10:16}
\pmmodified{2013-03-22 19:10:16}
\pmowner{pahio}{2872}
\pmmodifier{pahio}{2872}
\pmtitle{natural projection}
\pmrecord{4}{42078}
\pmprivacy{1}
\pmauthor{pahio}{2872}
\pmtype{Definition}
\pmcomment{trigger rebuild}
\pmclassification{msc}{20A05}
\pmsynonym{canonical homomorphism}{NaturalProjection}
\pmsynonym{natural homomorphism}{NaturalProjection}
\pmrelated{QuotientGroup}
\pmrelated{KernelOfAGroupHomomorphismIsANormalSubgroup}

\endmetadata

% this is the default PlanetMath preamble.  as your knowledge
% of TeX increases, you will probably want to edit this, but
% it should be fine as is for beginners.

% almost certainly you want these
\usepackage{amssymb}
\usepackage{amsmath}
\usepackage{amsfonts}

% used for TeXing text within eps files
%\usepackage{psfrag}
% need this for including graphics (\includegraphics)
%\usepackage{graphicx}
% for neatly defining theorems and propositions
 \usepackage{amsthm}
% making logically defined graphics
%%%\usepackage{xypic}

% there are many more packages, add them here as you need them

% define commands here

\theoremstyle{definition}
\newtheorem*{thmplain}{Theorem}

\begin{document}
\textbf{Proposition.}\, If $H$ is a normal subgroup of a group $G$, then the mapping
$$\varphi\!:\, G \to G/H \quad \mbox{where} \quad \varphi(x) \;=\; xH \;\; \forall x \in G$$
is a surjective homomorphism whose kernel is $H$.\\


\emph{Proof.}\, Because every coset appears as image, the mapping $\varphi$ is surjective.\, It is also homomorphic, since for all elements $x,\,y$ of $G$, one has
$$\varphi(xy) \;=\; (xy)H \;=\; xH\!\cdot\!yH \;=\; \varphi(x)\varphi(y).$$
The identity element of the factor group $G/H$ is the coset \,$eH = H$,\, whence
$$\operatorname{ker}(\varphi) \;=\; \{x \in G\,\vdots\;\; \varphi(x) \,=\, H\} \;=\; \{x \in G\,\vdots\;\; xH \,=\, H\}
 \;=\; H.$$\\

The mapping $\varphi$ in the proposition is called \emph{natural projection} or \emph{canonical homomorphism}.



%%%%%
%%%%%
\end{document}
