\documentclass[12pt]{article}
\usepackage{pmmeta}
\pmcanonicalname{SimplyTransitive}
\pmcreated{2013-03-22 14:37:41}
\pmmodified{2013-03-22 14:37:41}
\pmowner{benjaminfjones}{879}
\pmmodifier{benjaminfjones}{879}
\pmtitle{simply transitive}
\pmrecord{7}{36208}
\pmprivacy{1}
\pmauthor{benjaminfjones}{879}
\pmtype{Definition}
\pmcomment{trigger rebuild}
\pmclassification{msc}{20M30}
\pmrelated{GroupAction}

% this is the default PlanetMath preamble.  as your knowledge
% of TeX increases, you will probably want to edit this, but
% it should be fine as is for beginners.

% almost certainly you want these
\usepackage{amssymb}
\usepackage{amsmath}
\usepackage{amsfonts}
\usepackage{amsthm}

% used for TeXing text within eps files
%\usepackage{psfrag}
% need this for including graphics (\includegraphics)
%\usepackage{graphicx}
% for neatly defining theorems and propositions
%\usepackage{amsthm}
% making logically defined graphics
%%%\usepackage{xypic}

% there are many more packages, add them here as you need them

% define commands here
\theoremstyle{plain}
\newtheorem*{theorem}{Theorem}
\begin{document}
Let $G$ be a group acting on a set $X$. The action is said to be \textbf{simply transitive} if it is transitive and $\forall x,y \in X$ there is a \emph{unique} $g \in G$ such that $g.x = y$.


\begin{theorem}
A group action is simply transitive if and only if it is free and transitive
\end{theorem}

\begin{proof}
Necessity follows since $g.x = x$ implies that $g = 1_G$ because $1_G.x = x$ also. Now assume the action is free and transitive and we have elements $g_1, g_2 \in G$ and $x,y \in X$ such that $g_1.x = y$ and $g_2.x = y$. Then $g_1.x = g_2.x \implies g_2^{-1}.g_1.x = (g_2^{-1} g_1).x = x$ hence $g_2^{-1} g_1 = 1_G$ because the action is free. Thus $g_1 = g_2$ and so the action is simply transitive.
\end{proof}
%%%%%
%%%%%
\end{document}
