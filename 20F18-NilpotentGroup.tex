\documentclass[12pt]{article}
\usepackage{pmmeta}
\pmcanonicalname{NilpotentGroup}
\pmcreated{2013-03-22 12:47:50}
\pmmodified{2013-03-22 12:47:50}
\pmowner{djao}{24}
\pmmodifier{djao}{24}
\pmtitle{nilpotent group}
\pmrecord{8}{33113}
\pmprivacy{1}
\pmauthor{djao}{24}
\pmtype{Definition}
\pmcomment{trigger rebuild}
\pmclassification{msc}{20F18}
\pmdefines{nilpotent}
\pmdefines{upper central series}
\pmdefines{lower central series}
\pmdefines{nilpotency class}
\pmdefines{nilpotent class}

% this is the default PlanetMath preamble.  as your knowledge
% of TeX increases, you will probably want to edit this, but
% it should be fine as is for beginners.

% almost certainly you want these
\usepackage{amssymb}
\usepackage{amsmath}
\usepackage{amsfonts}

% used for TeXing text within eps files
%\usepackage{psfrag}
% need this for including graphics (\includegraphics)
%\usepackage{graphicx}
% for neatly defining theorems and propositions
%\usepackage{amsthm}
% making logically defined graphics
%%%\usepackage{xypic} 

% there are many more packages, add them here as you need them

% define commands here
\begin{document}
We define the {\em lower central series} of a group $G$ to be the filtration of subgroups
$$
G = G^1 \supset G^2 \supset \cdots
$$
defined inductively by:
\begin{eqnarray*}
G^1 & := & G, \\
G^i & := & [G^{i-1},G],\ \ i>1,
\end{eqnarray*}
where $[G^{i-1},G]$ denotes the subgroup of $G$ generated by all commutators of the form $hkh^{-1}k^{-1}$ where $h \in G^{i-1}$ and $k \in G$. The group $G$ is said to be {\em nilpotent} if $G^i = 1$ for some $i$.

Nilpotent groups can also be equivalently defined by means of upper central series. For a group $G$, the {\em upper central series} of $G$ is the filtration of subgroups
$$
C_0 \subset C_1 \subset C_2 \subset \cdots
$$
defined by setting $C_0$ to be the trivial subgroup of $G$, and inductively taking $C_i$ to be the unique subgroup of $G$ such that $C_i/C_{i-1}$ is the center of $G/C_{i-1}$, for each $i > 1$. The group $G$ is nilpotent if and only if $G = C_i$ for some $i$. Moreover, if $G$ is nilpotent, then the length of the upper central series (i.e., the smallest $i$ for which $G=C_i$) equals the length of the lower central series (i.e., the smallest $i$ for which $G^{i+1}=1$).

The \emph{nilpotency class} or \emph{nilpotent class} of a nilpotent group is the length of the lower central series (equivalently, the length of the upper central series).

Nilpotent groups are related to nilpotent Lie algebras in that a Lie group is nilpotent as a group if and only if its corresponding Lie algebra is nilpotent. The analogy extends to solvable groups as well: every nilpotent group is solvable, because the upper central series is a filtration with abelian quotients.
%%%%%
%%%%%
\end{document}
