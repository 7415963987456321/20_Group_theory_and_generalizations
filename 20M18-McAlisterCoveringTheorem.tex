\documentclass[12pt]{article}
\usepackage{pmmeta}
\pmcanonicalname{McAlisterCoveringTheorem}
\pmcreated{2013-03-22 14:37:19}
\pmmodified{2013-03-22 14:37:19}
\pmowner{mathcam}{2727}
\pmmodifier{mathcam}{2727}
\pmtitle{McAlister covering theorem}
\pmrecord{5}{36201}
\pmprivacy{1}
\pmauthor{mathcam}{2727}
\pmtype{Theorem}
\pmcomment{trigger rebuild}
\pmclassification{msc}{20M18}
\pmdefines{unitary}
\pmdefines{E-unitary}
\pmdefines{idempotent-separating}

\endmetadata

\usepackage{amssymb}
\usepackage{amsmath}
\usepackage{amsfonts}

% used for TeXing text within eps files
%\usepackage{psfrag}
% need this for including graphics (\includegraphics)
%\usepackage{graphicx}
% for neatly defining theorems and propositions
\usepackage{amsthm}
% making logically defined graphics
%%%\usepackage{xypic}

% there are many more packages, add them here as you need them

% define commands here

\newtheorem*{defn}{Definition}
\newtheorem*{thm}{Theorem}
\newtheorem*{cor}{Corollary}
\newtheorem*{lemma}{Lemma}
\newtheorem*{conj}{Conjecture}
\newtheorem*{prop}{Proposition}
\newtheorem*{open}{Open Question}
\newtheorem*{cond}{Condition}
\newenvironment{skproof}{\noindent\emph{Sketch Proof.}}{\qed\\}
\newtheorem*{rem}{Remark}

\newcommand{\Z}{\ensuremath{\mathbb{Z}}}
\newcommand{\N}{\ensuremath{\mathbb{N}}}
\newcommand{\Q}{\ensuremath{\mathbb{Q}}}
\newcommand{\R}{\ensuremath{\mathbb{R}}}

\newcommand{\eq}{\ensuremath{\thicksim}}
\newcommand{\iso}{\cong}
\newcommand{\normal}{\lhd}
\newcommand{\Aut}[1]{{\ensuremath{\mathrm{Aut}({#1})}}}
\newcommand{\GL}[2]{{\ensuremath{\mathrm{GL}({#1},{#2})}}}
\newcommand{\pad}{\ensuremath{\Box}}
\begin{document}
A subset $X$ in an inverse semigroup $S$ is called \emph{unitary} if for any elements $x\in X$ and $s\in S$, $xs\in X$ or $sx\in X$ implies $s\in X$.

An inverse semigroup is \emph{E-unitary} if its semigroup of idempotents is unitary.

\begin{thm}
Let $S$ be an inverse semigroup; then, there exists an E-unitary inverse semigroup $P$ and a surjective, idempotent-separating homomorphism $\theta:P\rightarrow S$.

Also, if $S$ is finite, then $P$ may be chosen to be finite as well.
\end{thm}

Note that a homomorphism is \emph{idempotent-separating} if it is injective on idempotents.

\begin{thebibliography}{}
\bibitem{} M. Lawson, \emph{Inverse Semigroups: The Theory of Partial Symmetries}, World Scientific, 1998
\end{thebibliography}
%%%%%
%%%%%
\end{document}
