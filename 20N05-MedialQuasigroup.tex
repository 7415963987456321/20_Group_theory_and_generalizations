\documentclass[12pt]{article}
\usepackage{pmmeta}
\pmcanonicalname{MedialQuasigroup}
\pmcreated{2013-03-22 16:27:33}
\pmmodified{2013-03-22 16:27:33}
\pmowner{rspuzio}{6075}
\pmmodifier{rspuzio}{6075}
\pmtitle{medial quasigroup}
\pmrecord{5}{38617}
\pmprivacy{1}
\pmauthor{rspuzio}{6075}
\pmtype{Definition}
\pmcomment{trigger rebuild}
\pmclassification{msc}{20N05}

% this is the default PlanetMath preamble.  as your knowledge
% of TeX increases, you will probably want to edit this, but
% it should be fine as is for beginners.

% almost certainly you want these
\usepackage{amssymb}
\usepackage{amsmath}
\usepackage{amsfonts}

% used for TeXing text within eps files
%\usepackage{psfrag}
% need this for including graphics (\includegraphics)
%\usepackage{graphicx}
% for neatly defining theorems and propositions
%\usepackage{amsthm}
% making logically defined graphics
%%%\usepackage{xypic}

% there are many more packages, add them here as you need them

% define commands here

\begin{document}
A \emph{medial quasigroup} is a quasigroup such that, for any choice of four elements $a,b,c,d$, one has
 \[ (a \cdot b) \cdot (c \cdot d) = (a \cdot c) \cdot (b\cdot d) .\]

Any commutative quasigroup is trivially a medial quasigroup.  A nontrivial class of examples may be constructed as follows.  Take a commutative group $(G,+)$ and two automorphisms $f, g \colon G \to G$ which commute with each other, and an element $c$ of $G$.  Then, if we define an operation $\cdot \colon G \times G \to G$ as 
 \[ x \cdot y = f(a) + g(b) + c ,\]
$(G,\cdot)$ is a medial quasigroup.

Reference:

 V. D. Belousov, Fundamentals of the theory of quasigroups and loops (in Russian)
%%%%%
%%%%%
\end{document}
