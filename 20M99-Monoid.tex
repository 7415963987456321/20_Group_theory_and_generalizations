\documentclass[12pt]{article}
\usepackage{pmmeta}
\pmcanonicalname{Monoid}
\pmcreated{2013-03-22 11:50:15}
\pmmodified{2013-03-22 11:50:15}
\pmowner{djao}{24}
\pmmodifier{djao}{24}
\pmtitle{monoid}
\pmrecord{9}{30389}
\pmprivacy{1}
\pmauthor{djao}{24}
\pmtype{Definition}
\pmcomment{trigger rebuild}
\pmclassification{msc}{20M99}
\pmclassification{msc}{34-01}
\pmsynonym{homomorphism}{Monoid}
\pmrelated{Semigroup}
\pmdefines{monoid homomorphism}

%\usepackage{amssymb}
%\usepackage{amsmath}
%\usepackage{amsfonts}
%\usepackage{graphicx}
%%%%%\usepackage{xypic}

\begin{document}
A monoid is a semigroup $G$ which contains an identity element; that is, there exists an element $e \in G$ such that $e \cdot a = a \cdot e = a$ for all $a \in G$.

If $e$ and $f$ are identity elements of a monoid $G$, then $e=e\cdot f=f\cdot e=f$, so we may speak of ``the'' identity element of $G$.

A \emph{monoid homomorphism} from monoids $G$ to $H$ is a semigroup homomorphism $f:G\to H$ such that $f(e_G)=e_H$, where $e_G,e_H$ are identity elements of $G$ and $H$ respectively.

%%%%%
%%%%%
%%%%%
%%%%%
\end{document}
