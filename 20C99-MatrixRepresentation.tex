\documentclass[12pt]{article}
\usepackage{pmmeta}
\pmcanonicalname{MatrixRepresentation}
\pmcreated{2013-03-22 14:53:56}
\pmmodified{2013-03-22 14:53:56}
\pmowner{drini}{3}
\pmmodifier{drini}{3}
\pmtitle{matrix representation}
\pmrecord{9}{36581}
\pmprivacy{1}
\pmauthor{drini}{3}
\pmtype{Definition}
\pmcomment{trigger rebuild}
\pmclassification{msc}{20C99}
\pmrelated{PermutationRepresentation}

\usepackage{graphicx}
%%%\usepackage{xypic} 
\usepackage{bbm}
\newcommand{\Z}{\mathbbmss{Z}}
\newcommand{\C}{\mathbbmss{C}}
\newcommand{\R}{\mathbbmss{R}}
\newcommand{\Q}{\mathbbmss{Q}}
\newcommand{\mathbb}[1]{\mathbbmss{#1}}
\newcommand{\figura}[1]{\begin{center}\includegraphics{#1}\end{center}}
\newcommand{\figuraex}[2]{\begin{center}\includegraphics[#2]{#1}\end{center}}
\newtheorem{dfn}{Definition}
\begin{document}
A matrix representation of a group $G$ is a group homomorphism between $G$ and $GL_n(\C)$, that is, a function
\[ X:G\to GL_n(\C)\]
such that 
\begin{itemize}
\item $X(gh)=X(g)X(h)$,
\item $X(e)=I$
\end{itemize}

Notice that this definition is equivalent to the group representation definition when the vector space $V$ is finite dimensional over $\C$. The parameter $n$ (or in the case of a group representation, the dimension of $V$) is called the \emph{degree} of the representation.


\begin{thebibliography}{9}
\bibitem{sagan} Bruce E. Sagan. \emph{The Symmetric Group: Representations, Combinatorial Algorithms and Symmetric Functions.}  2a Ed. 2000. Graduate Texts in Mathematics. Springer.
\end{thebibliography}
%%%%%
%%%%%
\end{document}
