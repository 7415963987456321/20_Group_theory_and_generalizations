\documentclass[12pt]{article}
\usepackage{pmmeta}
\pmcanonicalname{AbnormalSubgroup}
\pmcreated{2013-03-22 16:28:15}
\pmmodified{2013-03-22 16:28:15}
\pmowner{yark}{2760}
\pmmodifier{yark}{2760}
\pmtitle{abnormal subgroup}
\pmrecord{4}{38635}
\pmprivacy{1}
\pmauthor{yark}{2760}
\pmtype{Definition}
\pmcomment{trigger rebuild}
\pmclassification{msc}{20E99}
\pmdefines{abnormal}
\pmdefines{abnormality}

\def\genby#1{{\left\langle #1\right\rangle}}
\begin{document}
A subgroup $H$ of a group $G$ is called an \emph{abnormal subgroup}
if $x\in\genby{H,xHx^{-1}}$ for all $x\in G$.

Some facts about abnormal subgroups:
\begin{itemize}
\item Abnormal subgroups are pronormal and self-normalizing.
\item Non-\PMlinkname{normal}{NormalSubgroup} maximal subgroups are abnormal.
\item The normalizer of a pronormal subgroup is abnormal.
\end{itemize}

%%%%%
%%%%%
\end{document}
