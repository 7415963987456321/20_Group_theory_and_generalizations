\documentclass[12pt]{article}
\usepackage{pmmeta}
\pmcanonicalname{GeneralCommutativity}
\pmcreated{2014-05-10 21:59:41}
\pmmodified{2014-05-10 21:59:41}
\pmowner{pahio}{2872}
\pmmodifier{pahio}{2872}
\pmtitle{general commutativity}
\pmrecord{10}{41807}
\pmprivacy{1}
\pmauthor{pahio}{2872}
\pmtype{Theorem}
\pmcomment{trigger rebuild}
\pmclassification{msc}{20-00}
\pmrelated{CommutativeLanguage}
\pmrelated{GeneralAssociativity}
\pmrelated{AbelianGroup2}

\endmetadata

% this is the default PlanetMath preamble.  as your knowledge
% of TeX increases, you will probably want to edit this, but
% it should be fine as is for beginners.

% almost certainly you want these
\usepackage{amssymb}
\usepackage{amsmath}
\usepackage{amsfonts}

% used for TeXing text within eps files
%\usepackage{psfrag}
% need this for including graphics (\includegraphics)
%\usepackage{graphicx}
% for neatly defining theorems and propositions
 \usepackage{amsthm}
% making logically defined graphics
%%%\usepackage{xypic}

% there are many more packages, add them here as you need them

% define commands here

\theoremstyle{definition}
\newtheorem*{thmplain}{Theorem}

\begin{document}
\textbf{Theorem.}\, If the binary operation ``$\cdot$'' on the set $S$ is commutative, then for each\, 
$a_1, a_2,\ldots,a_n$ in $S$ and for each permutation $\pi$ on\, $\{1,\,2,\,\ldots,\,n\}$,\, one has
\begin{align}
\prod_{i=1}^na_{\pi(i)} \;=\; \prod_{i=1}^na_i.
\end{align}

\emph{Proof.}\, If\, $n = 1$,\, we have nothing to prove.\, 
Make the induction hypothesis, that (1) is true for\, 
$n = m\!-\!1$.\, Denote
$$\pi^{-1}(m) \;=\; k, \quad \mbox{i.e.} \quad \pi(k) = m.$$
Then
$$\prod_{i=1}^ma_{\pi(i)} \;=\; 
\prod_{i=1}^{k-1}a_{\pi(i)}\cdot a_{\pi(k)}\cdot\prod_{i=1}^{m-k}a_{\pi(k+i)} \;=\;
\left(\prod_{i=1}^{k-1}a_{\pi(i)}\cdot\prod_{i=1}^{m-k}a_{\pi(k+i)}\right)\cdot a_m,$$
where $a_m$ has been moved to the end by the induction 
hypothesis.\, But the product in the parenthesis, which 
\PMlinkescapetext{contains} exactly the factors 
$a_1, a_2, \ldots, a_{m-1}$ in a certain \PMlinkescapetext{order}, is also by the induction hypothesis equal to $\prod_{i=1}^{m-1}a_i$.\, Thus we obtain
$$\prod_{i=1}^ma_{\pi(i)} \;=\; \prod_{i=1}^{m-1}a_i\cdot a_m \;=\; \prod_{i=1}^ma_i,$$ 
whence (1) is true for\, $n = m$.\\

\textbf{Note.}\, There is mentionned in the Remark of the entry ``\PMlinkid{commutativity}{2148}'' a more general notion of commutativity.

%%%%%
%%%%%
\end{document}
