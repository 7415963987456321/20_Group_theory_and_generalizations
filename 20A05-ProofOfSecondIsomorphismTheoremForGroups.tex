\documentclass[12pt]{article}
\usepackage{pmmeta}
\pmcanonicalname{ProofOfSecondIsomorphismTheoremForGroups}
\pmcreated{2013-03-22 12:49:47}
\pmmodified{2013-03-22 12:49:47}
\pmowner{yark}{2760}
\pmmodifier{yark}{2760}
\pmtitle{proof of second isomorphism theorem for groups}
\pmrecord{17}{33153}
\pmprivacy{1}
\pmauthor{yark}{2760}
\pmtype{Proof}
\pmcomment{trigger rebuild}
\pmclassification{msc}{20A05}
\pmrelated{ProofOfSecondIsomorphismTheoremForRings}

\endmetadata

\usepackage{amssymb}
\usepackage{amsmath}
\usepackage{amsfonts}
\begin{document}
\PMlinkescapeword{between}
\PMlinkescapeword{normality}

First, we shall prove that $HK$ is a subgroup of $G$:
Since $e \in H$ and $e \in K$, clearly $e=e^2 \in HK$.
Take $h_1,h_2 \in H, k_1, k_2 \in K$.
Clearly $h_1k_1, h_2k_2 \in HK$.
Further,
\[
  h_1k_1h_2k_2 = h_1(h_2h_2^{-1})k_1h_2k_2 = h_1h_2(h_2^{-1}k_1h_2)k_2
\]
Since $K$ is a normal subgroup of $G$ and $h_2 \in G$,
then $h_2^{-1}k_1h_2 \in K$.
Therefore $h_1h_2(h_2^{-1}k_1h_2)k_2 \in HK$,
so $HK$ is closed under multiplication.

Also, $(hk)^{-1} \in HK$ for $h \in H$, $k \in K$, since
\[
  (hk)^{-1} = k^{-1}h^{-1}=h^{-1}hk^{-1}h^{-1}
\]
and $hk^{-1}h^{-1} \in K$ since $K$ is a normal subgroup of $G$.
So $HK$ is closed under inverses, and is thus a subgroup of $G$.

Since $HK$ is a subgroup of $G$,
the normality of $K$ in $HK$ follows immediately from
the normality of $K$ in $G$.

Clearly $H \cap K$ is a subgroup of $G$,
since it is the intersection of two subgroups of $G$.

Finally, define $\phi\colon H \rightarrow HK/K$ by $\phi(h)=hK$.
We claim that $\phi$ is a surjective homomorphism from $H$ to $HK/K$.
Let $h_0k_0K$ be some element of $HK/K$;
since $k_0 \in K$, then $h_0k_0K=h_0K$, and $\phi(h_0)=h_0K$.
Now
\[
  \ker(\phi) = \{ h \in H \mid \phi(h)=K \} = \{ h \in H \mid hK=K \}
\]
and if $hK=K$, then we must have $h \in K$.  So
\[
  \ker(\phi) = \{ h \in H \mid h \in K \} = H \cap K
\]

Thus, since $\phi(H)=HK/K$ and $\ker{\phi}=H \cap K$,
by the First Isomorphism Theorem we see that
$H \cap K$ is normal in $H$
and that there is a canonical isomorphism between $H/(H \cap K)$ and $HK/K$.
%%%%%
%%%%%
\end{document}
