\documentclass[12pt]{article}
\usepackage{pmmeta}
\pmcanonicalname{ProofOfScottWiegoldConjecture}
\pmcreated{2013-03-22 18:30:31}
\pmmodified{2013-03-22 18:30:31}
\pmowner{whm22}{2009}
\pmmodifier{whm22}{2009}
\pmtitle{proof of Scott-Wiegold conjecture}
\pmrecord{5}{41193}
\pmprivacy{1}
\pmauthor{whm22}{2009}
\pmtype{Proof}
\pmcomment{trigger rebuild}
\pmclassification{msc}{20E06}
%\pmkeywords{spin}

% this is the default PlanetMath preamble.  as your knowledge
% of TeX increases, you will probably want to edit this, but
% it should be fine as is for beginners.

% almost certainly you want these
\usepackage{amssymb}
\usepackage{amsmath}
\usepackage{amsfonts}

% used for TeXing text within eps files
%\usepackage{psfrag}
% need this for including graphics (\includegraphics)
%\usepackage{graphicx}
% for neatly defining theorems and propositions
%\usepackage{amsthm}
% making logically defined graphics
%%%\usepackage{xypic}

% there are many more packages, add them here as you need them

% define commands here

\begin{document}
Suppose the conjecture were false.  Then we have some $w\in C_p *
C_q * C_r$ with $N(w)= C_p * C_q * C_r$.  Let $a$, $b$, $c$ denote
the \PMlinkescapetext{projections} of $w$ onto $C_p$, $C_q$, $C_r$ respectively. Then
$a$, $b$, $c$ are all non-trivial as otherwise $N(w)$ would be
contained in the kernel of one of the \PMlinkescapetext{projections}.

For $0^\circ < \theta <360^\circ$ we say that a spin through
$\theta$ consists of a unit vector, $\vec{u} \in \mathbb{R}^3$ together with the
rotation of $\mathbb{R}^3$ through the angle $\theta$ anticlockwise about $\vec{u}$.
In \PMlinkescapetext{addition} we have a single spin through the angle $0^\circ$ and
a single spin through $360^\circ$.  Thus the set of spins
(usually denoted Spin(3)) naturally has the topology of a
3-sphere.


The spin through $\theta$ about a unit vector $\vec{u}$ has the
same underlying rotation as the spin through $360^\circ-\theta$
about $-\vec{u}$.  Hence there are precisely two spins
corresponding to each rotation of $\mathbb{R}^3$ about the origin.

\PMlinkescapetext{Multiplication} is well defined on spins as you can compose the
underlying rotations and continuity determines which of the two
spins is the correct result.  For example a $350^\circ$ spin about
$\vec{u}$ composed with a $20^\circ$ spin about $\vec{u}$ is a
$350^\circ$ spin about $-\vec{u}$ (not a $10^\circ$ spin about
$\vec{u}$ which would be at the other end of the 3-sphere).


Let $\vec{n}$ denote the unit vector $(0,0,1)$.  Fix an arc, $I$,
on the unit sphere connecting $\vec{n}$ and $-\vec{n}$.  Let
$\vec{t}$ be a vector on this arc. Let $\vec{u}$ be an arbitrary
unit vector.  We may define a homomorphism
$\phi_{\vec{t},\vec{u}}\colon  F_{\{a,b,c\}} \to {\rm Spin}(3)$
by:

\noindent$\phi_{\vec{t},\vec{u}}\colon a \mapsto$ the spin through
$(\frac{p-1}{2}) \frac{360^\circ}{p}$ (or $180^\circ$ if $p=2$)
about $\vec{n}$


\noindent $\phi_{\vec{t},\vec{u}}\colon b \mapsto$ the spin
through $(\frac{q-1}{2}) \frac{360^\circ}{q}$ (or $180^\circ$ if
$q=2$) about $\vec{t}$

\noindent $\phi_{\vec{t},\vec{u}}\colon c \mapsto$ the spin
through $(\frac{r-1}{2}) \frac{360^\circ}{r}$ (or $180^\circ$ if
$r=2$) about $\vec{u}$

(Here $F_{\{a,b,c\}}$ denotes the free group on $a,b,c$).


So $\phi_{\vec{t},\vec{u}}(a)$, $\phi_{\vec{t},\vec{u}}(b)$ and
$\phi_{\vec{t},\vec{u}}(c)$ are spins of between $120^\circ$ and
$180^\circ$, all having non-trivial underlying rotations.  

Let $\tilde{w}$ be a word in $F_{\{a,b,c\}}$ representing $w$, such that $a,b,c$ occur in it $1$ Mod $(2p)$ times, $1$ Mod $2q$ times and $1$ Mod $(2r)$ times, respectively.  

We have a homomorphism $\phi':C_p*C_q*C_r \to SO(3)$ induced by $\phi$.  If
$\phi_{\vec{t},\vec{u}}(\tilde{w})$ has a trivial underlying rotation for
some $\vec{t}$ and $\vec{u}$, then $N(w)$ will only contain
elements in the kernel of $\phi'$. In particular, we would have $a, b, c \notin N(w)$.  So
we may assume we have a map:


$$h\colon I \times S^2 \to S^2$$

\noindent which maps $(\vec{t}, \vec{u})$ to the unit vector corresponding to
$\phi_{\vec{t},\vec{u}}(\tilde{w})$.

By \PMlinkescapetext{symmetry} we have $h(\vec{n}, R\vec{u}) = Rh(\vec{n},\vec{u})$
for any rotation $R$ about $\vec{n}$.  Thus $h(\vec{n}, \_ )\colon
S^2 \to S^2$ maps latitudes to latitudes (possibly rotating them
and / or moving them up or down).



Also $h(\vec{n},\vec{n}) = -\vec{n}$, as
$\phi_{\vec{n},\vec{n}}(a)$, $\phi_{\vec{n},\vec{n}}(b)$ and
$\phi_{\vec{n},\vec{n}}(c)$ are spins of between $120^\circ$ and
$180^\circ$ anticlockwise about $\vec{n}$, so the sum of the
angles will be greater than $360^\circ$.  Similarly one may
\PMlinkescapetext{calculate} that $h(\vec{n},-\vec{n}) = \vec{n}$. Thus, as $h(n,
\_)$ maps latitudes to latitudes, it must be homotopic to a
reflection of $S^2$.

Again by \PMlinkescapetext{symmetry} we have $h(-\vec{n}, R\vec{u}) =
Rh(-\vec{n},\vec{u})$ for all rotations $R$ about $\vec{n}$.
Hence $h(-\vec{n}, \_)\colon S^2 \to S^2$ also maps latitudes to
latitudes.

Further, $h(-\vec{n},\vec{n}) = \vec{n}$ and $h(-\vec{n},-\vec{n})
= -\vec{n}$.  Thus $h(-\vec{n}, \_)$ is homotopic to the \PMlinkescapetext{identity}.



But $h$ gives a homotopy from $h(\vec{n}, \_)$ to $h(-\vec{n},
\_)$, yielding the desired contradiction.
%%%%%
%%%%%
\end{document}
