\documentclass[12pt]{article}
\usepackage{pmmeta}
\pmcanonicalname{GeneratedSubring}
\pmcreated{2013-03-22 16:57:27}
\pmmodified{2013-03-22 16:57:27}
\pmowner{polarbear}{3475}
\pmmodifier{polarbear}{3475}
\pmtitle{generated subring}
\pmrecord{9}{39227}
\pmprivacy{1}
\pmauthor{polarbear}{3475}
\pmtype{Definition}
\pmcomment{trigger rebuild}
\pmclassification{msc}{20-00}
\pmclassification{msc}{13-00}
\pmclassification{msc}{16-00}
\pmrelated{RingAdjunction}
\pmdefines{subring generated by}
\pmdefines{monomials in rings}

\endmetadata

% this is the default PlanetMath preamble.  as your knowledge
% of TeX increases, you will probably want to edit this, but
% it should be fine as is for beginners.

% almost certainly you want these
\usepackage{amssymb}
\usepackage{amsmath}
\usepackage{amsfonts}

% used for TeXing text within eps files
%\usepackage{psfrag}
% need this for including graphics (\includegraphics)
%\usepackage{graphicx}
% for neatly defining theorems and propositions
%\usepackage{amsthm}
% making logically defined graphics
%%%\usepackage{xypic}

% there are many more packages, add them here as you need them

% define commands here
\newtheorem{defn*}{Definition}
\def\genby#1{{\left\langle #1\right\rangle}}

\begin{document}
\begin{defn*} Let $M$ be a nonempty subset of a ring $A$. The intersection of all subrings of $A$ that include $M$ is the smallest subring of $A$ that includes $M$. It is called the \em{subring generated by} $M$ and is denoted by $\genby{M}$.\end{defn*}
 The subring generated by $M$ is formed by finite sums of monomials of the form :\begin{equation*}
a_1a_2 \cdots a_n, \mbox{where} \;\;\displaystyle a_1,\ldots , a_n \in M.\end{equation*}
 Of particular interest is the subring generated by a family of subrings $E = \{A_i|\;\; i\in I\}$. It is the ring $R$ formed by finite sums of monomials of the form:\begin{equation*}
\displaystyle a_{i_1}a_{i_2} \ldots a_{i_n}, \mbox{where}\;\; a_{i_k} \in A_{i_k}. \end{equation*}
  If $A,B$ are rings, the subring generated by $A \cup B$ is also denoted by $AB$.\newline
 In the case when $A_i$ are fields included in a larger field $A$ then the set of all quotients of elements of $R$ ( the quotient field of $R$) is the composite field $\bigvee_{i\in I}A_i$ of the family $E$. In other words, it is the subfield generated by $\bigcup_{i\in I}A_i$. The notation $\bigvee_{i\in I}A_i$  comes from the fact that the family of all subfields of a field forms a complete lattice.\newline
 The \PMlinkescapetext{composite} of fields is defined only when the respective fields are all included in a larger field. 
 

%%%%%
%%%%%
\end{document}
