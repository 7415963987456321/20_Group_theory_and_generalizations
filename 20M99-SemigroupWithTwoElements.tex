\documentclass[12pt]{article}
\usepackage{pmmeta}
\pmcanonicalname{SemigroupWithTwoElements}
\pmcreated{2013-03-22 16:21:42}
\pmmodified{2013-03-22 16:21:42}
\pmowner{rspuzio}{6075}
\pmmodifier{rspuzio}{6075}
\pmtitle{semigroup with two elements}
\pmrecord{11}{38498}
\pmprivacy{1}
\pmauthor{rspuzio}{6075}
\pmtype{Example}
\pmcomment{trigger rebuild}
\pmclassification{msc}{20M99}

% this is the default PlanetMath preamble.  as your knowledge
% of TeX increases, you will probably want to edit this, but
% it should be fine as is for beginners.

% almost certainly you want these
\usepackage{amssymb}
\usepackage{amsmath}
\usepackage{amsfonts}

% used for TeXing text within eps files
%\usepackage{psfrag}
% need this for including graphics (\includegraphics)
%\usepackage{graphicx}
% for neatly defining theorems and propositions
%\usepackage{amsthm}
% making logically defined graphics
%%%\usepackage{xypic}

% there are many more packages, add them here as you need them

% define commands here

\begin{document}
Perhaps the simplest non-trivial example of a semigroup which is not a group is a particular semigroup with two elements. The underlying set of this 
semigroup is $\{a,b\}$ and the operation is defined as follows:
\begin{eqnarray*}
a \cdot a &=& a \\
a \cdot b &=& b \\
b \cdot a &=& b \\
b \cdot b &=& b
\end{eqnarray*}
It is rather easy to check that this operation is associative, as it
should be: 
\begin{eqnarray*}
a \cdot (a \cdot a) = a \cdot a =& a &= a \cdot a = (a \cdot a) \cdot a \\
a \cdot (a \cdot b) = a \cdot b =& b &= a \cdot b = (a \cdot a) \cdot b \\
a \cdot (b \cdot b) = a \cdot b =& b &= b \cdot b = (a \cdot b) \cdot b \\
b \cdot (a \cdot a) = b \cdot a =& b &= a \cdot a = (a \cdot a) \cdot a \\
a \cdot (b \cdot b) = a \cdot b =& b &= b \cdot b = (a \cdot b) \cdot b \\
b \cdot (a \cdot b) = b \cdot b =& b &= b \cdot b = (b \cdot a) \cdot b \\
b \cdot (b \cdot a) = b \cdot b =& b &= b \cdot a = (b \cdot b) \cdot a \\
b \cdot (b \cdot b) = b \cdot b =& b &= b \cdot b = (b \cdot b) \cdot b
\end{eqnarray*}

It is worth noting that this semigroup is commutative and has an identity
element, which is $a$.  It is not a group because the element $b$ does
not have an inverse.  In fact, it is not even a cancellative semigroup
because we cannot cancel the $b$ in the equation $a \cdot b = b \cdot b$.

This semigroup also arises in various contexts.  For instance,
if we choose $a$ to be the truth value "true" and $b$ to be the truth
value "false" and the operation $\cdot$ to be the logical connective "and",
we obtain this semigroup in logic.  We may also represent it by matrices
like so:
\[a = \left( \begin{matrix} 1 & 0 \\ 0 & 1 \end{matrix} \right) \qquad  
b = \left( \begin{matrix} 1 & 0 \\ 0 & 0 \end{matrix} \right) \]
%%%%%
%%%%%
\end{document}
