\documentclass[12pt]{article}
\usepackage{pmmeta}
\pmcanonicalname{SubsemigroupSubmonoidAndSubgroup}
\pmcreated{2013-03-22 13:02:03}
\pmmodified{2013-03-22 13:02:03}
\pmowner{mclase}{549}
\pmmodifier{mclase}{549}
\pmtitle{subsemigroup,, submonoid,, and subgroup}
\pmrecord{5}{33434}
\pmprivacy{1}
\pmauthor{mclase}{549}
\pmtype{Definition}
\pmcomment{trigger rebuild}
\pmclassification{msc}{20M99}
\pmrelated{Semigroup}
\pmrelated{Subgroup}
\pmdefines{subsemigroup}
\pmdefines{submonoid}
\pmdefines{subgroup}

% this is the default PlanetMath preamble.  as your knowledge
% of TeX increases, you will probably want to edit this, but
% it should be fine as is for beginners.

% almost certainly you want these
\usepackage{amssymb}
\usepackage{amsmath}
\usepackage{amsfonts}

% used for TeXing text within eps files
%\usepackage{psfrag}
% need this for including graphics (\includegraphics)
%\usepackage{graphicx}
% for neatly defining theorems and propositions
%\usepackage{amsthm}
% making logically defined graphics
%%%\usepackage{xypic}

% there are many more packages, add them here as you need them

% define commands here
\begin{document}
Let $S$ be a semigroup, and let $T$ be a subset of $S$.

$T$ is a \emph{subsemigroup} of $S$ if $T$ is closed under the operation of $S$; that it if $xy \in T$ for all $x, y \in T$.

$T$ is a \emph{submonoid} of $S$ if $T$ is a subsemigroup, and $T$ has an identity element.

$T$ is a \emph{subgroup} of $S$ if $T$ is a submonoid which is a group.

Note that submonoids and subgroups do not have to have the same identity element as $S$ itself (indeed, $S$ may not have an identity element).  The identity element may be any idempotent element of $S$.

Let $e \in S$ be an idempotent element.  Then there is a maximal subsemigroup of $S$ for which $e$ is the identity:
$$eSe = \{ exe \mid x \in S \}.$$
In addition, there is a maximal subgroup for which $e$ is the identity:
$$\mathcal{U}(eSe) = \{x \in eSe \mid \exists y \in eSe \;\text{st}\; xy=yx=e \}.$$

Subgroups with different identity elements are disjoint.  To see this, suppose that $G$ and $H$ are subgroups of a semigroup $S$ with identity elements $e$ and $f$ respectively, and suppose $x \in G \cap H$.
Then $x$ has an inverse $y \in G$, and an inverse $z \in H$.  We have:
$$e = xy = fxy = fe = zxe = zx = f.$$
Thus intersecting subgroups have the same identity element.
%%%%%
%%%%%
\end{document}
