\documentclass[12pt]{article}
\usepackage{pmmeta}
\pmcanonicalname{InverseOfInverseInAGroup}
\pmcreated{2013-03-22 15:43:36}
\pmmodified{2013-03-22 15:43:36}
\pmowner{cvalente}{11260}
\pmmodifier{cvalente}{11260}
\pmtitle{inverse of inverse in a group}
\pmrecord{7}{37676}
\pmprivacy{1}
\pmauthor{cvalente}{11260}
\pmtype{Proof}
\pmcomment{trigger rebuild}
\pmclassification{msc}{20-00}
\pmclassification{msc}{20A05}
\pmclassification{msc}{08A99}
\pmrelated{AdditiveInverseOfAnInverseElement}

% this is the default PlanetMath preamble.  as your knowledge
% of TeX increases, you will probably want to edit this, but
% it should be fine as is for beginners.

% almost certainly you want these
\usepackage{amssymb}
\usepackage{amsmath}
\usepackage{amsfonts}

% used for TeXing text within eps files
%\usepackage{psfrag}
% need this for including graphics (\includegraphics)
%\usepackage{graphicx}
% for neatly defining theorems and propositions
%\usepackage{amsthm}
% making logically defined graphics
%%%\usepackage{xypic}

% there are many more packages, add them here as you need them

% define commands here
\begin{document}
Let $(G,*)$ be a group.
We aim to prove that  ${(a^{-1})}^{-1}=a$ for every $a\in G$.
That is, the inverse of the inverse of a group element is the element itself.

By definition $a*a^{-1}=a^{-1}*a=e$, where $e$ is the identity in $G$. Reinterpreting this equation we can read it as saying that $a$ is the inverse of $a^{-1}$.

In fact, consider $b=a^{-1}$, the equation can be written $a*b=b*a=e$ and thus $a$ is the inverse of $b=a^{-1}$.

%${(a^{-1})}^{-1}=a$
%%%%%
%%%%%
\end{document}
