\documentclass[12pt]{article}
\usepackage{pmmeta}
\pmcanonicalname{DihedralGroup}
\pmcreated{2013-03-22 12:22:53}
\pmmodified{2013-03-22 12:22:53}
\pmowner{rmilson}{146}
\pmmodifier{rmilson}{146}
\pmtitle{dihedral group}
\pmrecord{15}{32159}
\pmprivacy{1}
\pmauthor{rmilson}{146}
\pmtype{Definition}
\pmcomment{trigger rebuild}
\pmclassification{msc}{20F55}
\pmrelated{Symmetry2}

\endmetadata

\usepackage{amsmath}
\usepackage{amsfonts}
\usepackage{amssymb}

\newcommand{\reals}{\mathbb{R}}
\newcommand{\natnums}{\mathbb{N}}
\newcommand{\cnums}{\mathbb{C}}

\newcommand{\lp}{\left(}
\newcommand{\rp}{\right)}
\newcommand{\lb}{\left[}
\newcommand{\rb}{\right]}

\newtheorem{proposition}{Proposition}
\newcommand{\supth}{^{\text{th}}}
\newcommand{\cD}{\mathcal{D}}
\begin{document}
The $n^{\text{th}}$ dihedral group is the symmetry group of
the regular $n$-sided polygon.  The group consists of $n$ reflections,
$n-1$ rotations, and the identity transformation.  In this entry we will denote the group in question by
$\cD_n$.
An alternate notation is $\cD_{2n}$; in this approach, the subscript indicates the order of the group.  

Letting
$\omega=\exp(2\pi i/n)$ denote a primitive $n^{\text{th}}$ root of
unity, and assuming the polygon is centered at the origin, the
rotations $R_k,\; k=0,\ldots,n-1$ (Note: $R_0$ denotes the identity)
are given by
$$R_k:z \mapsto \omega^k z,\quad z\in\cnums,$$
and the reflections $M_k,\; k=0,\ldots,n-1$ by
$$M_k: z\mapsto \omega^k \bar{z},\quad z\in\cnums$$
The abstract group structure is given by
\begin{align*}
R_k R_l &= R_{k+l}, &  R_k M_l &= M_{k+l}\\
M_k M_l &= R_{k-l}, &  M_k R_l &= M_{k-l},
\end{align*}
where the addition and subtraction is carried out modulo $n$. 

The group can also be described in terms of generators and relations as
$$\lp M_0\rp^2 =\lp M_1\rp^2 = (M_1 M_0)^n = \mathrm{id}.$$
This means that $\cD_n$ is a rank-1 Coxeter group.

Since the group acts by linear transformations
$$(x,y)\to(\hat{x},\hat{y}),\quad (x,y)\in \reals^2$$
there is a
corresponding action on polynomials $p\to\hat{p}$, defined by
$$\hat{p}(\hat{x},\hat{y}) = p(x,y),\quad p\in \reals[x,y].$$
The polynomials
left invariant by all the group transformations form an algebra.  This
algebra is freely generated by the following two basic invariants:
$$x^2+y^2,\quad x^n-\binom{n}{2} x^{n-2}y^2 + \cdots,$$
the latter
polynomial being the real part of $(x+iy)^n$.  It is easy to check
that these two polynomials are invariant.  The first polynomial
describes the distance of a point from the origin, and this is
unaltered by Euclidean reflections through the origin.  The second
polynomial is unaltered by a rotation through $2\pi/n$ radians, and is
also invariant with respect to complex conjugation.  These two
transformations generate the $n\supth$ dihedral group.  Showing that
these two invariants polynomially generate the full algebra of
invariants is somewhat trickier, and is best done as an application of
Chevalley's theorem regarding the invariants of a finite reflection
group.
%%%%%
%%%%%
\end{document}
