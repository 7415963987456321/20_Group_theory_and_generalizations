\documentclass[12pt]{article}
\usepackage{pmmeta}
\pmcanonicalname{FrattiniSubgroupOfAFiniteGroupIsNilpotentThe}
\pmcreated{2013-03-22 13:16:44}
\pmmodified{2013-03-22 13:16:44}
\pmowner{yark}{2760}
\pmmodifier{yark}{2760}
\pmtitle{Frattini subgroup of a finite group is nilpotent, the}
\pmrecord{16}{33762}
\pmprivacy{1}
\pmauthor{yark}{2760}
\pmtype{Theorem}
\pmcomment{trigger rebuild}
\pmclassification{msc}{20D25}

\usepackage{amssymb}
\usepackage{amsmath}
\usepackage{amsfonts}
\usepackage{amsthm}

\def\Frat#1{\Phi(#1)}
\def\genby#1{{\langle #1\rangle}}
\begin{document}
\PMlinkescapeword{nilpotent}

The Frattini subgroup of a finite group is \PMlinkname{nilpotent}{NilpotentGroup}.

\begin{proof}
Let $\Frat G$ denote the Frattini subgroup of a finite group $G$.
Let $S$ be a Sylow subgroup of $\Frat G$.
Then by the Frattini argument, $G=\Frat{G}N_G(S)=\genby{\Frat G\cup N_G(S)}$.
But the Frattini subgroup is finite and formed of non-generators,
so it follows that $G=\genby{N_G(S)}=N_G(S)$.
Thus $S$ is normal in $G$, and therefore normal in $\Frat G$.
The result now follows, as \PMlinkname{any finite group whose Sylow subgroups are all normal is nilpotent}{ClassificationOfFiniteNilpotentGroups}.
\end{proof}

In fact, the same proof shows that for any group $G$,
if $\Frat G$ is finite then $\Frat G$ is nilpotent.

%%%%%
%%%%%
\end{document}
