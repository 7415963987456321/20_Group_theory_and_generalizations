\documentclass[12pt]{article}
\usepackage{pmmeta}
\pmcanonicalname{TriangleGroups}
\pmcreated{2013-03-22 14:25:07}
\pmmodified{2013-03-22 14:25:07}
\pmowner{rmilson}{146}
\pmmodifier{rmilson}{146}
\pmtitle{triangle groups}
\pmrecord{10}{35925}
\pmprivacy{1}
\pmauthor{rmilson}{146}
\pmtype{Definition}
\pmcomment{trigger rebuild}
\pmclassification{msc}{20F05}
\pmrelated{ExamplesOfGroups}
\pmdefines{von Dyck groups}

% this is the default PlanetMath preamble.  as your knowledge
% of TeX increases, you will probably want to edit this, but
% it should be fine as is for beginners.

% almost certainly you want these
\usepackage{amssymb}
\usepackage{amsmath}
\usepackage{amsfonts}

% used for TeXing text within eps files
%\usepackage{psfrag}
% need this for including graphics (\includegraphics)
%\usepackage{graphicx}
% for neatly defining theorems and propositions
\usepackage{amsthm}
% making logically defined graphics
%%%\usepackage{xypic}

% there are many more packages, add them here as you need them

% define commands here
\begin{document}
Consider the following group presentation:
$$\Delta(l,m,n)=\langle a,b,c:a^2,b^2,c^2,(ab)^l,(bc)^n,(ca)^m\rangle$$
where $l,m,n\in\mathbb{N}$.

A group with this presentation corresponds to a triangle; roughly, the generators are reflections in its sides and its angles are $\pi/l,\pi/m,\pi/n$.

Denote by $D(l,m,n)$ the subgroup of \PMlinkname{index}{Coset} 2 in $\Delta(l,m,n)$, corresponding to preservation of \PMlinkescapetext{orientation} of the triangle.

The $D(l,m,n)$ are defined by the following presentation:
$$D(l,m,n)=\langle x,y:x^l,y^m,(xy)^n\rangle$$

Note that $D(l,m,n)\cong D(m,l,n)\cong D(n,m,l)$, so $D(l,m,n)$ is \PMlinkescapetext{independent of the order} of the $l,m,n$.

Arising from the geometrical nature of these groups, $$1/l+1/m+1/n>1$$is called the \emph{spherical case},$$1/l+1/m+1/n=1$$is called the \emph{Euclidean case}, and$$1/l+1/m+1/n<1$$is called the \emph{hyperbolic case}

Groups either of the form $\Delta(l,m,n)$ or $D(l,m,n)$ are referred to as \emph{triangle groups}; groups of the form $D(l,m,n)$ are sometimes refered to as \emph{von Dyck groups}.
%%%%%
%%%%%
\end{document}
