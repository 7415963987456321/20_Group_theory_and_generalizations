\documentclass[12pt]{article}
\usepackage{pmmeta}
\pmcanonicalname{ResiduallymathfrakX}
\pmcreated{2013-03-22 14:53:22}
\pmmodified{2013-03-22 14:53:22}
\pmowner{yark}{2760}
\pmmodifier{yark}{2760}
\pmtitle{residually $\mathfrak{X}$}
\pmrecord{15}{36570}
\pmprivacy{1}
\pmauthor{yark}{2760}
\pmtype{Definition}
\pmcomment{trigger rebuild}
\pmclassification{msc}{20E26}
\pmrelated{AGroupsEmbedsIntoItsProfiniteCompletionIfAndOnlyIfItIsResiduallyFinite}
\pmdefines{residually finite}
\pmdefines{residually nilpotent}
\pmdefines{residually solvable}
\pmdefines{residually soluble}

\usepackage{amsfonts}
\usepackage{amssymb}

\def\normal{\trianglelefteq}
\def\X{\mathfrak{X}}
\begin{document}
\PMlinkescapeword{hereditary}
\PMlinkescapeword{invariant}
\PMlinkescapeword{isomorphic}
\PMlinkescapeword{property}
\PMlinkescapeword{subgroup}

Let $\X$ be a property of groups,
assumed to be an isomorphic invariant
(that is, if a group $G$ has property $\X$,
then every group isomorphic to $G$ also has property $\X$).
We shall sometimes refer to groups with property $\X$ as $\X$-groups.

A group $G$ is said to be \emph{residually $\X$}
if for every $x\in G\backslash\{1\}$ there is a normal subgroup $N$ of $G$
such that $x\notin N$ and $G/N$ has property $\X$.
Equivalently, $G$ is residually $\X$ if and only if
\[
  \bigcap_{N\normal_\X G}\!\!N=\{1\},
\]
where $N\normal_\X G$ means that
$N$ is normal in $G$ and $G/N$ has property $\X$.

It can be shown that a group is residually $\X$
if and only if it is isomorphic to a subdirect product of $\X$-groups.
If $\X$ is a hereditary property 
(that is, every \PMlinkname{subgroup}{Subgroup} of an $\X$-group is an $\X$-group),
then a group is residually $\X$ if and only if
it can be embedded in an unrestricted direct product of $\X$-groups.

It can be shown that a group $G$ is residually solvable if and only if
the intersection of the derived series of $G$ is trivial
(see transfinite derived series).
Similarly, a group $G$ is residually nilpotent if and only if
the intersection of the lower central series of $G$ is trivial.
%%%%%
%%%%%
\end{document}
