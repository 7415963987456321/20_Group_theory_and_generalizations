\documentclass[12pt]{article}
\usepackage{pmmeta}
\pmcanonicalname{SymmetricGroup}
\pmcreated{2013-03-22 12:01:53}
\pmmodified{2013-03-22 12:01:53}
\pmowner{bwebste}{988}
\pmmodifier{bwebste}{988}
\pmtitle{symmetric group}
\pmrecord{11}{31040}
\pmprivacy{1}
\pmauthor{bwebste}{988}
\pmtype{Definition}
\pmcomment{trigger rebuild}
\pmclassification{msc}{20B30}
\pmrelated{Group}
\pmrelated{Cycle2}
\pmrelated{CayleyGraphOfS_3}
\pmrelated{Symmetry2}

\usepackage{amssymb}
\usepackage{amsmath}
\usepackage{amsfonts}
\usepackage{graphicx}
%%%\usepackage{xypic}
\begin{document}
Let $X$ be a set.  
Let ${\rm Sym}(X)$ be the set of permutations of $X$ 
(i.e. the set of bijective functions from $X$ to itself).  
Then the act of taking the composition of two permutations 
induces a group structure on ${\rm Sym}(X)$.  
We call this group the {\it symmetric group}.

The group ${\rm Sym}(\{1,2,\ldots, n\})$ is often denoted $S_n$ or $\mathfrak{S}_n$.

$S_n$ is generated by the transpositions $\{(1,2),(2,3),\ldots,(n-1,n)\}$,
and by any pair of a 2-cycle and $n$-cycle.

$S_n$ is the Weyl group of the $A_{n-1}$ root system (and hence of the special linear group $SL_{n-1}$).
%%%%%
%%%%%
%%%%%
\end{document}
