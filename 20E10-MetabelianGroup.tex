\documentclass[12pt]{article}
\usepackage{pmmeta}
\pmcanonicalname{MetabelianGroup}
\pmcreated{2013-03-22 15:36:42}
\pmmodified{2013-03-22 15:36:42}
\pmowner{yark}{2760}
\pmmodifier{yark}{2760}
\pmtitle{metabelian group}
\pmrecord{8}{37532}
\pmprivacy{1}
\pmauthor{yark}{2760}
\pmtype{Definition}
\pmcomment{trigger rebuild}
\pmclassification{msc}{20E10}
\pmclassification{msc}{20F16}
\pmsynonym{meta-abelian group}{MetabelianGroup}
\pmrelated{AbelianGroup2}
\pmdefines{metabelian}
\pmdefines{meta-abelian}

\usepackage{amssymb}
\usepackage{amsmath}
\usepackage{amsfonts}
\begin{document}
\PMlinkescapeword{class}
\PMlinkescapeword{equivalent}
\PMlinkescapeword{length}
\PMlinkescapeword{subgroup}
\PMlinkescapeword{subgroups}
\PMlinkescapeword{term}
\PMlinkescapeword{words}

\section*{Definition}

A \emph{metabelian group} is a group $G$ that possesses a normal subgroup $N$ such that $N$ and $G/N$ are both abelian. 
Equivalently, $G$ is metabelian if and only if the commutator subgroup $[G,G]$ is abelian. 
Equivalently again, $G$ is metabelian if and only if it is solvable of length at most $2$.

(Note that in older literature the term tends to be used in the stronger sense that the central quotient $G/Z(G)$ is abelian. This is equivalent to being nilpotent of class at most $2$. We shall not use this sense here.)

\section*{Examples}

\begin{itemize}
\item All abelian groups.
\item All generalized dihedral groups.
\item All groups of order less than $24$.
\item All metacyclic groups.
\end{itemize}

\section*{Properties}

\PMlinkname{Subgroups}{Subgroup}, \PMlinkname{quotients}{QuotientGroup} and (unrestricted) direct products of metabelian groups are also metabelian.
In other words, metabelian groups form a \PMlinkname{variety}{VarietyOfGroups};
they are, in fact, the groups in which $(w^{-1}x^{-1}wx)(y^{-1}z^{-1}yz)=(y^{-1}z^{-1}yz)(w^{-1}x^{-1}wx)$ for all elements $w$, $x$, $y$ and $z$.
%%%%%
%%%%%
\end{document}
