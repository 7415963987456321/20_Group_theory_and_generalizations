\documentclass[12pt]{article}
\usepackage{pmmeta}
\pmcanonicalname{PresentationOfAGroup}
\pmcreated{2013-03-22 12:23:23}
\pmmodified{2013-03-22 12:23:23}
\pmowner{rmilson}{146}
\pmmodifier{rmilson}{146}
\pmtitle{presentation of a group}
\pmrecord{20}{32182}
\pmprivacy{1}
\pmauthor{rmilson}{146}
\pmtype{Definition}
\pmcomment{trigger rebuild}
\pmclassification{msc}{20A05}
\pmclassification{msc}{20F05}
\pmsynonym{presentation}{PresentationOfAGroup}
\pmsynonym{finite presentation}{PresentationOfAGroup}
\pmsynonym{finitely presented}{PresentationOfAGroup}
%\pmkeywords{finitely presented}
\pmrelated{GeneratingSetOfAGroup}
\pmrelated{CayleyGraph}
\pmdefines{generator}
\pmdefines{relation}
\pmdefines{generators and relations}
\pmdefines{relator}

\usepackage{amsmath}
\usepackage{amsfonts}
\usepackage{amssymb}

\newcommand{\reals}{\mathbb{R}}
\newcommand{\natnums}{\mathbb{N}}
\newcommand{\cnums}{\mathbb{C}}

\newcommand{\lp}{\left(}
\newcommand{\rp}{\right)}
\newcommand{\lb}{\left[}
\newcommand{\rb}{\right]}

\newtheorem{proposition}{Proposition}
\begin{document}
A \emph{presentation} of a group $G$ is a description of $G$ in terms of
generators and relations (sometimes also known as relators).
We say that the group is finitely
presented, if it can be described in terms of a finite number of
generators and a finite number of defining relations. A collection of
group elements $g_i\in G,\;i\in I$ is said to generate $G$ if every
element of $G$ can be specified as a product of the $g_i$, and of their
inverses.  A relation is a word over the alphabet consisting of the
generators $g_i$ and their inverses, with the property that it
multiplies out to the identity in $G$.  A set of relations $r_j,\;
j\in J$ is said to be defining, if all relations in $G$ can be given
as a product of the $r_j$, their inverses, and the $G$-conjugates of
these.

The standard notation for the presentation of a group is
$$G= \langle g_i \mid r_j \rangle,$$
meaning that $G$ is generated by generators $g_i$, subject to
relations $r_j$.  Equivalently, one has a short exact sequence of
groups 
$$1 \to N \to F[I] \to G\to 1,$$
where $F[I]$ denotes the free group
generated by the $g_i$, and where $N$ is the smallest normal subgroup
containing all the $r_j$. By the Nielsen-Schreier Theorem, the kernel $N$
is itself a free group, and hence we assume without loss of generality
that there are no relations among the relations. 

{\bf Example.}  The symmetric group on $n$ elements $1,\ldots, n$
admits the following finite presentation (Note: this presentation is
not canonical.  Other presentations are known.)  As generators take
$$g_i=(i, i+1),\quad i=1,\ldots, n-1,$$
the transpositions of adjacent elements. As defining relations take
$$(g_i g_j)^{n_{i,j}} = \mathrm{id},\quad i,j=1,\ldots n,$$
where
\begin{align*}
  n_{i,i} &= 1 \\
  n_{i,i+1} &= 3 \\
  n_{i,j} &=2,\quad \vert j-i \vert > 1.
\end{align*}
This means that a finite symmetric group is a Coxeter group.
%%%%%
%%%%%
\end{document}
