\documentclass[12pt]{article}
\usepackage{pmmeta}
\pmcanonicalname{CharacteristicSubgroup}
\pmcreated{2013-03-22 12:50:56}
\pmmodified{2013-03-22 12:50:56}
\pmowner{yark}{2760}
\pmmodifier{yark}{2760}
\pmtitle{characteristic subgroup}
\pmrecord{13}{33180}
\pmprivacy{1}
\pmauthor{yark}{2760}
\pmtype{Definition}
\pmcomment{trigger rebuild}
\pmclassification{msc}{20A05}
%\pmkeywords{characteristic subgroup}
%\pmkeywords{group}
\pmrelated{FullyInvariantSubgroup}
\pmrelated{NormalSubgroup}
\pmrelated{SubnormalSubgroup}
\pmdefines{characteristic}

\usepackage{amssymb}
\usepackage{amsmath}
\usepackage{amsfonts}

\DeclareMathOperator{\Char}{\,char\,}
\begin{document}
\PMlinkescapeword{homomorphism}
\PMlinkescapeword{properties}

If $(G,*)$ is a group, then $H$ is a \emph{characteristic subgroup} of $G$ (written $H \Char G$) if every automorphism of $G$ maps $H$ to itself.  That is, if $f\in{\rm Aut}(G)$ and $h\in H$ then $f(h)\in H$.

A few properties of characteristic subgroups:
\begin{itemize}
\item If $H\Char G$ then $H$ is a normal subgroup of $G$.
\item If $G$ has only one subgroup of a given cardinality then that subgroup is characteristic.
\item If $K\Char H$ and $H\trianglelefteq G$ then $K\trianglelefteq G$. (Contrast with normality of subgroups is not transitive.)
\item If $K\Char H$ and $H\Char G$ then $K\Char G$.
\end{itemize}

Proofs of these properties:
\begin{itemize}
\item Consider $H\Char G$ under the inner automorphisms of $G$.  Since every automorphism preserves $H$, in particular every inner automorphism preserves $H$, and therefore $g*h*g^{-1}\in H$ for any $g\in G$ and $h\in H$.  This is precisely the definition of a normal subgroup.
\item Suppose $H$ is the only subgroup of $G$ of order $n$.  In general, \PMlinkname{homomorphisms}{GroupHomomorphism} take subgroups to subgroups, and of course isomorphisms take subgroups to subgroups of the same order.  But since there is only one subgroup of $G$ of order $n$, any automorphism must take $H$ to $H$, and so $H\Char G$.
\item Take $K\Char H$ and $H\trianglelefteq G$, and consider the inner automorphisms of $G$ (automorphisms of the form $h\mapsto g*h*g^{-1}$ for some $g\in G$).  These all preserve $H$, and so are automorphisms of $H$.  But any automorphism of $H$ preserves $K$, so for any $g\in G$ and $k\in K$, $g*k*g^{-1}\in K$.
\item Let $K\Char H$ and $H\Char G$, and let $\phi$ be an automorphism of $G$.  Since $H\Char G$, $\phi[H]=H$, so $\phi_H$, the restriction of $\phi$ to $H$ is an automorphism of $H$.  Since $K\Char H$, so $\phi_H[K]=K$.  But $\phi_H$ is just a restriction of $\phi$, so $\phi[K]=K$.  Hence $K\Char G$.
\end{itemize}
%%%%%
%%%%%
\end{document}
