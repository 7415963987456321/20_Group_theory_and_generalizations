\documentclass[12pt]{article}
\usepackage{pmmeta}
\pmcanonicalname{StarHeight}
\pmcreated{2013-03-22 18:57:55}
\pmmodified{2013-03-22 18:57:55}
\pmowner{CWoo}{3771}
\pmmodifier{CWoo}{3771}
\pmtitle{star height}
\pmrecord{9}{41824}
\pmprivacy{1}
\pmauthor{CWoo}{3771}
\pmtype{Definition}
\pmcomment{trigger rebuild}
\pmclassification{msc}{20M35}
\pmclassification{msc}{68Q70}
\pmrelated{StarFree}

\usepackage{amssymb,amscd}
\usepackage{amsmath}
\usepackage{amsfonts}
\usepackage{mathrsfs}

% used for TeXing text within eps files
%\usepackage{psfrag}
% need this for including graphics (\includegraphics)
%\usepackage{graphicx}
% for neatly defining theorems and propositions
\usepackage{amsthm}
% making logically defined graphics
%%\usepackage{xypic}
\usepackage{pst-plot}

% define commands here
\newcommand*{\abs}[1]{\left\lvert #1\right\rvert}
\newtheorem{prop}{Proposition}
\newtheorem{thm}{Theorem}
\newtheorem{ex}{Example}
\newcommand{\real}{\mathbb{R}}
\newcommand{\pdiff}[2]{\frac{\partial #1}{\partial #2}}
\newcommand{\mpdiff}[3]{\frac{\partial^#1 #2}{\partial #3^#1}}
\newcommand{\h}{\operatorname{ht}}
\begin{document}
The \emph{star height} of a regular expression $p$ over an alphabet $\Sigma$, denoted by $\h(p)$, is defined recursively as follows:
\begin{enumerate}
\item $\h(\varnothing)=\h(a)=0$ for any $a\in \Sigma$;
\item $\h(p\cup q)=\h(pq)=\max(\h(p),\h(q))$;
\item $\h(p^*)=\h(p)+1$.
\end{enumerate}

Let $\Sigma=\lbrace a,b,c\rbrace$.  The following expressions have star height $0,1,2,3$:
$$ (a\cup b)c  \qquad a^*b^* \qquad (a^*b)^*c \qquad ((ab^*a \cup c)^* b)^*  $$

Since any regular expression $p$ describes a regular language $L(p)$, we may extend the definition of star height to regular languages.  However, since a regular language may be described by more than one regular expressions, we need to be a little careful:

The \emph{star height} of a regular language $L$ is the least integer $i$ such that $L$ may be described by a regular expression of star height $i$.  In other words: $$\h(L)=\min \lbrace h(p) \mid L=L(p)\mbox{, }p\in R(\Sigma) \rbrace,$$
where $R(\Sigma)$ is the set of all regular expressions over $\Sigma$.

\textbf{Remarks}.
\begin{itemize}
\item Any regular language over a singleton alphabet has star height at most 1, which can be proved by the pumping lemma for regular languages.
\item If the alphabet $\Sigma$ consists of at least two letters, then for every positive integer $n$, there is a regular language whose star height is $n$.  In fact, it can be shown that if $|\Sigma|=2$, then for every $n$, there is a regular language $L$ over $\Sigma$ such that $\h(L)=n$.
\item It was an open question whether an algorithm exists for determining $\h(L)$ for an arbitrary regular language $L$.  In 2005, the question was resolved by D. Kirsten in the positive, and the algorithm is that of a non-deterministic finite automaton.
\item We may also define star height on generalized regular expressions.  For a regular language $L$, define $\h_g(L)=\min \lbrace h(p) \mid L=L(p)\mbox{, }p\in R_g(\Sigma) \rbrace$, where $R_g(\Sigma)$ is the set of all generalized regular expressions.  It is clear that $\h_g(L)\le \h(L)$.  However, it is still an open question whether, for every integer $n$, there is a regular language $L$ with $\h_g(L)=n$.

A star-free language has star height $0$ with respect to representations by generalized regular expressions, but may be positive with respect to representations by regular expressions, for example $L=\lbrace ab\rbrace^*$.
\end{itemize}

\begin{thebibliography}{9}
\bibitem{AS} A. Salomaa, {\em Formal Languages}, Academic Press, New York (1973).
\bibitem{AS1} A. Salomaa, {\em Jewels of Formal Language Theory}, Computer Science Press (1981).
\end{thebibliography}
%%%%%
%%%%%
\end{document}
