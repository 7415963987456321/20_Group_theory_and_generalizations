\documentclass[12pt]{article}
\usepackage{pmmeta}
\pmcanonicalname{SymmetricGroup1}
\pmcreated{2013-03-22 14:03:53}
\pmmodified{2013-03-22 14:03:53}
\pmowner{antizeus}{11}
\pmmodifier{antizeus}{11}
\pmtitle{symmetric group}
\pmrecord{5}{35421}
\pmprivacy{1}
\pmauthor{antizeus}{11}
\pmtype{Definition}
\pmcomment{trigger rebuild}
\pmclassification{msc}{20B30}
\pmrelated{Symmetry2}

\endmetadata

% this is the default PlanetMath preamble.  as your knowledge
% of TeX increases, you will probably want to edit this, but
% it should be fine as is for beginners.

% almost certainly you want these
\usepackage{amssymb}
\usepackage{amsmath}
\usepackage{amsfonts}

% used for TeXing text within eps files
%\usepackage{psfrag}
% need this for including graphics (\includegraphics)
%\usepackage{graphicx}
% for neatly defining theorems and propositions
%\usepackage{amsthm}
% making logically defined graphics
%%%\usepackage{xypic} 

% there are many more packages, add them here as you need them

% define commands here
\begin{document}
Let $X$ be a set.
Let $S(X)$ be the set of permutations of $X$
(i.e. the set of bijective functions on $X$).
Then the act of taking the composition of two permutations
induces a group structure on $S(X)$.
We call this group the {\it symmetric group}
and it is often denoted ${\rm Sym}(X)$.

When $X$ has a finite number $n$ of elements,
we often refer to the symmetric group as $S_n$,
and describe the elements by using cycle notation.
%%%%%
%%%%%
\end{document}
