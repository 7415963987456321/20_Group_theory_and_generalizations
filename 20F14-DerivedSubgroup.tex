\documentclass[12pt]{article}
\usepackage{pmmeta}
\pmcanonicalname{DerivedSubgroup}
\pmcreated{2013-03-22 12:33:53}
\pmmodified{2013-03-22 12:33:53}
\pmowner{yark}{2760}
\pmmodifier{yark}{2760}
\pmtitle{derived subgroup}
\pmrecord{22}{32812}
\pmprivacy{1}
\pmauthor{yark}{2760}
\pmtype{Definition}
\pmcomment{trigger rebuild}
\pmclassification{msc}{20F14}
\pmclassification{msc}{20E15}
\pmclassification{msc}{20A05}
\pmsynonym{commutator subgroup}{DerivedSubgroup}
\pmrelated{JordanHolderDecomposition}
\pmrelated{Solvable}
\pmrelated{TransfiniteDerivedSeries}
\pmrelated{Abelianization}
\pmdefines{commutator}
\pmdefines{derived series}
\pmdefines{second derived subgroup}

\usepackage{amsmath}
\usepackage{amsfonts}
\usepackage{amssymb}

\newcommand{\reals}{\mathbb{R}}
\newcommand{\natnums}{\mathbb{N}}
\newcommand{\cnums}{\mathbb{C}}
\newcommand{\znums}{\mathbb{Z}}

\newcommand{\lp}{\left(}
\newcommand{\rp}{\right)}
\newcommand{\lb}{\left[}
\newcommand{\rb}{\right]}

\newcommand{\supth}{^{\text{th}}}

\newtheorem{proposition}{Proposition}
\begin{document}
\PMlinkescapeword{degree}
\PMlinkescapeword{measures}
\PMlinkescapeword{proposition}
\PMlinkescapeword{represent}

Let $G$ be a group.
For any $a,b\in G$, the element $a^{-1}b^{-1}ab$ is called the \emph{commutator of $a$ and $b$}.

The commutator $a^{-1}b^{-1}ab$ is sometimes written $[a,b]$.
(Usage varies, however, and some authors instead use $[a,b]$ to represent the commutator $aba^{-1}b^{-1}$.)
If $A$ and $B$ are subsets of $G$, then $[A,B]$ denotes the subgroup of $G$ generated by $\{[a,b]\mid a\in A\hbox{ and }b\in B\}$.
This notation can be further extended by recursively defining
$[X_1,\dots,X_{n+1}]=[[X_1,\dots,X_n],X_{n+1}]$
for subsets $X_1,\dots,X_{n+1}$ of $G$.

The subgroup of $G$ generated by all the commutators in $G$ 
(that is, the smallest subgroup of $G$ containing all the commutators)
is called the \emph{derived subgroup}, 
or the \emph{commutator subgroup}, of $G$.
Using the notation of the previous paragraph, the derived subgroup is denoted by $[G,G]$.
Alternatively, it is often denoted by $G'$, or sometimes $G^{(1)}$.

Note that $a$ and $b$ commute if and only if the commutator of $a,b\in G$ is trivial, i.e.,
\[ a^{-1} b^{-1}a b = 1. \]
Thus, in a fashion, the derived subgroup measures the degree to which a group fails to be abelian.

\begin{proposition}
The derived subgroup $[G,G]$ is normal (in fact, fully invariant) in $G$,
and the factor group $G/[G,G]$ is abelian.  
Moreover, $G$ is abelian if and only if $[G,G]$ is the trivial subgroup.
\end{proposition}

The factor group $G/[G,G]$ is the largest abelian \PMlinkname{quotient}{QuotientGroup} of $G$,
and is called the abelianization of $G$.

One can of course form the derived subgroup of the derived subgroup;
this is called the \emph{second derived subgroup}, and denoted by $G''$ or $G^{(2)}$. Proceeding inductively one defines the $n\supth$ derived
subgroup $G^{(n)}$ as the derived subgroup of $G^{(n-1)}$. In this fashion one
obtains a sequence of subgroups, called the \emph{derived series} of $G$:
$$G=G^{(0)} \supseteq G^{(1)} \supseteq G^{(2)} \supseteq \cdots$$

\begin{proposition}
The group $G$ is solvable if and only if the derived series
terminates in the trivial group $\{ 1 \}$ after a \PMlinkname{finite}{Finite} number of steps.
\end{proposition}

The derived series can also be continued transfinitely---see the article on the transfinite derived series.
%%%%%
%%%%%
\end{document}
