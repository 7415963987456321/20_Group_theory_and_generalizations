\documentclass[12pt]{article}
\usepackage{pmmeta}
\pmcanonicalname{NonabelianGroup}
\pmcreated{2013-03-22 14:02:04}
\pmmodified{2013-03-22 14:02:04}
\pmowner{drini}{3}
\pmmodifier{drini}{3}
\pmtitle{nonabelian group}
\pmrecord{10}{35138}
\pmprivacy{1}
\pmauthor{drini}{3}
\pmtype{Definition}
\pmcomment{trigger rebuild}
\pmclassification{msc}{20A05}
\pmsynonym{non-abelian group}{NonabelianGroup}
\pmsynonym{noncommutative group}{NonabelianGroup}
\pmsynonym{non-commutative group}{NonabelianGroup}
\pmrelated{AbelianGroup2}

\endmetadata

%fancy typeface/symbols
%\usepackage{beton}
%\usepackage{concmath}
%\usepackage{stmaryrd}
%margins
%\usepackage{simplemargins}
%\usepackage{multicol}
%\setleftmargin{1in}
%\setrightmargin{1in}
%\settopmargin{1in}
%\setbottommargin{1in}
% symbols
\usepackage{amssymb}
\usepackage{amsfonts}
\usepackage[mathscr]{euscript}
\usepackage[T1]{fontenc}
% graphics
%\usepackage{pstricks}
% math
\usepackage{amsmath}
\usepackage{amsopn}
\usepackage{amstext}
\usepackage{amsthm}
% only one theoremstyle 
%\theoremstyle{definition}   		
%\newtheorem{exercise}{}[subsection]
% math operators
\DeclareMathOperator{\pipe}{{\big |} \hspace{-2.85pt} {\big |}} 
\DeclareMathOperator{\et}{\&}
% misc math stuff
\newcommand{\vol}{\mathrm{vol}}
\newcommand{\id}{\mathrm{id}}
\newcommand{\domain}{\mathrm{domain}}
\newcommand{\supp}{\mathrm{supp}}
\newcommand{\diam}{\mathrm{diameter}}
\newcommand{\incl}{\mathrm{incl}}
\newcommand{\interior}{\mathrm{interior}}
\newcommand{\triv}{\mathrm{triv}}
\newcommand{\image}{\mathrm{image}}
\newcommand{\closure}{\mathrm{closure}}
\newcommand{\degree}{\mathrm{degree}}
\newcommand{\im}{\mathrm{im}}
\newcommand{\esssup}{\mathrm{ess\ sup}}
\newcommand{\openset}{\mathrel{{\mathchoice{\rlap{$\subset$}{\;\circ}}%
   {\rlap{$\subset$}{\;\circ}}%
   {\rlap{$\scriptstyle\subset$}{\;\circ}}%
   {\rlap{$\scriptscriptstyle\subset$}{\;\circ}}}}}

% foreignisms
\newcommand{\ie}{\emph{i.e.},}
\newcommand{\Ie}{\emph{I.e.},}
\newcommand{\cf}{\emph{cf.}}
\newcommand{\eg}{\emph{e.g.},}
\newcommand{\Eg}{\emph{E.g.},}
\newcommand{\ala}{\emph{\'a la}}
% labels for spaces
\newcommand{\N}{\mathbf{N}}
\newcommand{\Z}{\mathbf{Z}}
\newcommand{\Q}{\mathbf{Q}}
\newcommand{\C}{\mathbf{C}}
\newcommand{\CP}{\mathbf{CP}}
\newcommand{\RP}{\mathbf{RP}}
\newcommand{\R}{\mathbf{R}}
\newcommand{\Rtw}{\mathbf{R}^2}
\newcommand{\Rth}{\mathbf{R}^3}
\newcommand{\Rfo}{\mathbf{R}^4}
\newcommand{\Rfi}{\mathbf{R}^5}
\newcommand{\Rp}{\mathbf{R}^p}
\newcommand{\Rn}{\mathbf{R}^n}
\newcommand{\Rnmon}{\mathbf{R}^{n-1}}
\newcommand{\Rnpon}{\mathbf{R}^{n+1}}
\newcommand{\Rmpon}{\mathbf{R}^{m+1}}
\newcommand{\RNpon}{\mathbf{R}^{N+1}}
\newcommand{\Rtwn}{\mathbf{R}^{2n}}
\newcommand{\RN}{\mathbf{R}^N}
\newcommand{\Rm}{\mathbf{R}^m}
\newcommand{\Hon}{\mathbf{H}^1}
\newcommand{\Htw}{\mathbf{H}^2}
\newcommand{\Hth}{\mathbf{H}^3}
\newcommand{\Hn}{\mathbf{H}^n}
\newcommand{\Torus}{\mathbf{T}}
\newcommand{\Ttw}{\mathbf{T}^2}
\newcommand{\Tth}{\mathbf{T}^3}
\newcommand{\Tn}{\mathbf{T}^n}
\newcommand{\Son}{\mathbf{S}^1}
\newcommand{\Stw}{\mathbf{S}^2}
\newcommand{\Sth}{\mathbf{S}^3}
\newcommand{\Sfo}{\mathbf{S}^4}
\newcommand{\Sfi}{\mathbf{S}^5}
\newcommand{\SN}{\mathbf{S}^N}
\newcommand{\Sn}{\mathbf{S}^n}
\newcommand{\Sm}{\mathbf{S}^m}
\newcommand{\Smmon}{\mathbf{S}^{m-1}}
\newcommand{\Snmon}{\mathbf{S}^{n-1}}
\newcommand{\Snmtw}{\mathbf{S}^{n-2}}
\newcommand{\Hnmon}{\mathbf{H}^{n-1}}
\newcommand{\Ton}{\mathbf{T}^1}
\newcommand{\T}{\mathbf{T}}
% differential operators
\newcommand{\pd}[2]{\frac{\partial #1}{\partial #2}}
\newcommand{\pkd}[3]{\frac{\partial^{#3} #1}{\partial #2^{#3}}}
\newcommand{\dkd}[3]{\frac{d^{#3} #1}{d#2^{#3}}}
\newcommand{\td}[2]{\frac{d #1}{d #2}}
\newcommand{\pdat}[3]{\left . \frac{\partial #1}{\partial #2}\right|_{#3}}
\newcommand{\dbyd}[1]{ \frac{d}{d #1}}

% Fri Oct  3 11:00:53 2003 -- should check at some point to see whether the replaced versions are actually used at all, I don't think so
\newcommand{\ddk}[3]{\frac{d^{#3} #1}{d#2^{#3}}}
% replaces...
\newcommand{\dbydk}[2]{ \frac{d^{#2}}{d #1^{#2}}}

\newcommand{\ppk}[3]{\frac{\partial^{#3} #1}{\partial #2^{#3}}}
% replaces...
\newcommand{\pbypk}[2]{ \frac{\partial^{#2}}{\partial #1^{#2}}}

\newcommand{\pp}[2]{\frac{\partial #1}{\partial #2}}
% replaces...
\newcommand{\pbyp}[1]{ \frac{\partial}{\partial #1}}

\newcommand{\ddat}[3]{\left .\frac{d #1}{d #2}\right|_{#3}}
% replaces...
\newcommand{\dbydat}[2]{\left . \frac{d}{d #1} \right|_{#2}}

\newcommand{\ppkat}[4]{\left .\frac{\partial^{#3} #1}{\partial #2^{#3}}\right|_{#4}}

\newcommand{\ddkat}[4]{\left .\frac{d^{#3} #1}{d#2^{#3}}\right|_{#4}}

% counters for new lists
%\newcounter{alistctr}
%\newcounter{Alistctr}
\newcounter{rlistctr}
\newcounter{Rlistctr}
\newcounter{123listctr}
\newcounter{123listcolonstylectr}

% a,b,c    - small latin letter list
\newenvironment{alist}{
\indent \begin{list}{(\alph{alistctr})}{\usecounter{alistctr}}}
                      {\end{list}\setcounter{alistctr}{0}} 
% A,B,C    - LARGE LATIN LETTER LIST
\newenvironment{Alist}{
\indent \begin{list}{(\Alph{Alistctr})}{\usecounter{Alistctr}}
                      }
                      {\end{list}\setcounter{Alistctr}{0}}
% i,ii,iii - small roman numeral list
\newenvironment{rlist}{
\indent \begin{list}{(\roman{rlistctr})}{\usecounter{rlistctr}}
                      }
                      {\end{list}\setcounter{rlistctr}{0}} 
% I,II,III - large roman numeral list
\newenvironment{Rlist}{
\indent \begin{list}{(\Roman{Rlistctr})}{\usecounter{Rlistctr}}
                      }
                      {\end{list}\setcounter{Rlistctr}{0}}
%1,2,3 - arabic numeral list
\newenvironment{123list}{
\indent \begin{list}{(\arabic{123listctr})}{\usecounter{123listctr}}
                      }
                      {\end{list}\setcounter{123listctr}{0}} 

%1:,2:,3: - arabic numeral list with colon decoration
\newenvironment{123listcolonstyle}{\indent \begin{list}{\arabic{123listcolonstylectr}:}{\usecounter{123listcolonstylectr}}}
                      {\end{list}\setcounter{123listcolonstylectr}{0}}

% environment for definitions
\def\defn#1{\addcontentsline{toc}{subsection}{$\ast$} {\footnotesize \noindent \begin{123listcolonstyle} \setlength{\itemsep}{0em} \setlength{\topsep}{0em} \setlength{\parsep}{0em} #1 \end{123listcolonstyle}}}

%environment for proofs
\def\proof#1{\par {\footnotesize \indent \begin{tabular}{ll} #1  \end{tabular}}}

%% \include{packages}
%% \include{renewed_commands}
%% \include{simple_theorems}
%% \include{abbreviations}
%% \include{spaces}
%% \include{new_environments}
%% \include{margins}
%% \include{mathoperators}
%% \include{differentiation}
%% \include{limited_things}
%% \include{argawarga}
\begin{document}
A group is said to be \emph{nonabelian}, or \emph{noncommutative}, if
has elements which do not commute, that is, if there exist $a$ and $b$
in the group such that $ab \neq ba$.  Equivalently, a group is
nonabelian if there exist $a$ and $b$ in the group such that the
commutator $[a,b]$ is not equal to the identity of the group.  There
exist many natural nonabelian groups, with order as small as $6$.
While any group for which the square map is a homomorphism is abelian,
there exist nonabelian groups of order as small as $27$ for which the
cube map is a homomorphism.

In the first section we give a way to visualize the group of rotations
of a sphere and prove that it is nonabelian.  This should be readable
by an undergraduate student in algebra.  In the second section, we
discuss groups admitting a cube map and show that there are small 
nonabelian examples.  The second section is somewhat more technical 
than the first and will require more facility with group theory, 
especially working with finitely presented groups and the commutator 
calculus.

\section{Concrete examples of nonabelian groups}

Although most number systems we use are abelian by design, there exist
quite natural nonabelian groups.  Perhaps the simplest example to
visualize is given by the group of rotations of a sphere.\footnote{The
treatment we give here is informal.  For a more formal treatment of
the group of rotations, consult the entries ``\PMlinkname{Rotation
matrix}{RotationMatrix}'' and ``\PMlinkname{Dimension of the special
orthogonal group}{DimensionOfTheSpecialOrthogonalGroup}''.}  We can
compose two rotations by performing them in sequence, and we can
invert a rotation by rotating in the opposite direction, so rotations
do form a group.  To follow what rotation does to the sphere, imagine
that inside is suspended a copy of Marshall Hall's classic text {\it
The Theory of Groups}.  We will keep track of three pieces of
information, namely, the directions that the front cover, the spine, and
the bottom of the book face.  When the sphere is in the identity position,
the front cover faces the reader, the spine faces the left, and the
bottom of the book is oriented downward.

In preparation for verifying that the group is not abelian, we define
two rotations, $F$ and $R$.  First, let $F$ (for ``flip'') be the
rotation which takes the point at the very top of the sphere and moves
it forward through an angle of $\pi$.  For example, if we start with
the sphere in the identity position and then perform $F$, the front
cover will face away from the reader, the spine will remain to the
left, and the bottom of the book will be oriented upward.  Second, let
$R$ (for ``rotate'') be the rotation which takes the point at the
very top of the sphere and moves it left through an angle of
$\frac{\pi}{2}$.  If we start with the sphere in the identity position
and then perform $R$, the front cover will continue to face the
reader, the spine will face downward, and the bottom of the book will
be oriented to the right.

We now verify that the group of rotations is not abelian.  If we start
with the sphere in the identity position and perform $FR$, that is,
first $F$, then $R$, then the front cover will face away from the
reader, the spine will face downward, and the bottom will be oriented
to the left.  On the other hand, if we start with the sphere in the
identity position and perform $RF$, then while the front cover will
face away from the reader, the spine will face \emph{upward}, and the
bottom will be oriented to the \emph{right}.  So it matters in which
order we perform $F$ and $R$, that is, $FR \ne RF$, proving that the
group is not abelian.

Since every rotation in three-dimensional Euclidean space can be
decomposed as a finite sequence of reflections and rotations in the
Euclidean plane, one might hope that we can find finite nonabelian
groups arising from objects in the plane, and in fact we can.  For
each regular polygon, there is an associated group, the dihedral group
$D_{2n}$, which is the group of symmetries of the polygon.  (Here $n$
denotes the number of the sides of the polygon, and $2n$ gives the
number of elements of the group of symmetries.)  It is generated by
two elements, $F$ (for ``flip'') and $R$ (for ``rotate'').  These
elements can be defined by analogy with the $F$ and $R$ above; for
full details, consult the entry ``\PMlinkname{Dihedral
group}{DihedralGroup}'', where flips are labelled instead by $M$ (for
``mirror'').  If $n\ge 3$ (so we are dealing with an actual polygon
here), it is possible to show that $FR \ne RF$.  Moreover, every group
with order $1$, $p$, or $p^2$, where $p$ is a prime, is abelian.  Thus
the smallest possible order for a nonabelian group is $6$.  But
$D_{2\cdot 3}$ has $6$ elements and is nonabelian, so it is the
smallest possible nonabelian group.

\section{Small nonabelian groups admitting a cube map}

When we say that a group admits $x\mapsto x^n$, we mean that the
function $\varphi$ defined on the group by the formula $\varphi(x) =
x^n$ is a homomorphism, that is, that is, that for any $x$ and $y$ in
the group,
\[
  (xy)^n = \varphi(xy) = \varphi(x)\varphi(y) = x^ny^n.  
\]
If a group admits $x\mapsto x^2$, then for any $x$ and $y$ we have
that $(xy)^2 = x^2y^2$.  Multiplying on the left by $x^{-1}$ and on
the right by $y^{-1}$ yields the identity $yx = xy$.  Thus all such
groups are abelian.  Moreover, the generalized commutativity and
associativity laws for abelian groups imply that an abelian group
admits all maps $x\mapsto x^n$.  It is therefore reasonable to wonder
whether the converse holds.  In fact it is possible for a nonabelian
group to admit $x\mapsto x^3$.  The smallest order for such a group is
$27$.  It is %currently
beyond the scope of this entry to prove that $27$ is the smallest 
possible order, but we will give an explicit example.  

Let $G$ be the group with presentation
\[
  G = \langle a, b, c \mid a^3, b^3, c^3, [a,c], [b,c], [a,b]c \rangle.
\]
This can be realized concretely as the group of upper-triangular
matrices over $\mathbb{Z}/3\mathbb{Z}$ with $1$s on the diagonal, but for
simplicity we shall work directly with the presentation.

The first three relators tell us that each generator of the group has
order $3$.  The next two tell us that $c$ is central --- since it
commutes with the other two generators and commutes with itself, it
must therefore commute with everything.  The final relator is perhaps
the most interesting.  We can interpret it as the rewrite rule
\[
  ba \mapsto abc,
\]
that is,
\[
  \text{``when $b$ moves past $a$ it turns into $bc$.''}
\]
Thus given an element of $G$ we can always write it in the normal form
$a^j b^k c^{\ell}$, where $0\le j, k, \ell<3$, and all such elements
are distinct.  This proves that the cardinality of $G$ is $27$.
Moreover, we also observe that
\[
  ba = abc \ne ab,
\]
so $G$ is not abelian.

It remains to check that $G$ admits the cube map.  We will prove the
simpler statement that $G$ has exponent $3$.  Given $x$ in $G$, we
first normalize it, so $x = a^j b^k c^{\ell}$.  Since $c$ is in the
center of $G$,
\[
  x^3 = (a^j b^k)^3 c^{3\ell} = (a^j b^k)^3 = a^j(b^k a^j)^2b^k.
\]

To normalize the word $b^k a^j$, we push each $b$ past all of the
$a$s.  Since pushing $b$ past a \emph{single} $a$ turns it into $bc$, pushing
it past $a^j$ turns it into $bc^j$, that is,
\[
  b^k a^j = b^{k-1} a^j b c^j.
\]
By induction it follows that
\[
  b^k a^j = a^j b^k c^{jk}.
\]
Applying this result to $x^3$, we get that
\[
  x^3 = a^j(a^j b^k c^{jk})^2 b^k = a^{2j} (b^ka^j) b^{2k} c^{2jk} = a^{3j} b^{3k} c^{3jk} = 1.
\]
Since $x^3$ is trivial for any $x$, it follows that $G$ admits the
cube map.

The other nonabelian group of order $27$ has exponent $9$ and also admits the 
cube map.  This will be % fix this when done
described in an attached entry.

\PMlinkescapeword{algebra}
\PMlinkescapeword{calculus}
\PMlinkescapeword{closed}
\PMlinkescapeword{cover}
\PMlinkescapeword{cube}
\PMlinkescapeword{design}
\PMlinkescapeword{face}
\PMlinkescapeword{faces}
\PMlinkescapeword{information}
\PMlinkescapeword{mean}
\PMlinkescapeword{normalize}
\PMlinkescapeword{object}
\PMlinkescapeword{objects}
\PMlinkescapeword{opposite}
\PMlinkescapeword{orientation}
\PMlinkescapeword{push}
\PMlinkescapeword{right}
\PMlinkescapeword{scope}
\PMlinkescapeword{section}
\PMlinkescapeword{sequence}
\PMlinkescapeword{square}
\PMlinkescapeword{theory}
%%%%%
%%%%%
\end{document}
