\documentclass[12pt]{article}
\usepackage{pmmeta}
\pmcanonicalname{FrattiniSubset}
\pmcreated{2013-03-22 16:30:56}
\pmmodified{2013-03-22 16:30:56}
\pmowner{Algeboy}{12884}
\pmmodifier{Algeboy}{12884}
\pmtitle{Frattini subset}
\pmrecord{7}{38693}
\pmprivacy{1}
\pmauthor{Algeboy}{12884}
\pmtype{Definition}
\pmcomment{trigger rebuild}
\pmclassification{msc}{20D15}
\pmrelated{FrattiniSubgroup}

\usepackage{latexsym}
\usepackage{amssymb}
\usepackage{amsmath}
\usepackage{amsfonts}
\usepackage{amsthm}

%%\usepackage{xypic}

%-----------------------------------------------------

%       Standard theoremlike environments.

%       Stolen directly from AMSLaTeX sample

%-----------------------------------------------------

%% \theoremstyle{plain} %% This is the default

\newtheorem{thm}{Theorem}

\newtheorem{coro}[thm]{Corollary}

\newtheorem{lem}[thm]{Lemma}

\newtheorem{lemma}[thm]{Lemma}

\newtheorem{prop}[thm]{Proposition}

\newtheorem{conjecture}[thm]{Conjecture}

\newtheorem{conj}[thm]{Conjecture}

\newtheorem{defn}[thm]{Definition}

\newtheorem{remark}[thm]{Remark}

\newtheorem{ex}[thm]{Example}



%\countstyle[equation]{thm}



%--------------------------------------------------

%       Item references.

%--------------------------------------------------


\newcommand{\exref}[1]{Example-\ref{#1}}

\newcommand{\thmref}[1]{Theorem-\ref{#1}}

\newcommand{\defref}[1]{Definition-\ref{#1}}

\newcommand{\eqnref}[1]{(\ref{#1})}

\newcommand{\secref}[1]{Section-\ref{#1}}

\newcommand{\lemref}[1]{Lemma-\ref{#1}}

\newcommand{\propref}[1]{Prop\-o\-si\-tion-\ref{#1}}

\newcommand{\corref}[1]{Cor\-ol\-lary-\ref{#1}}

\newcommand{\figref}[1]{Fig\-ure-\ref{#1}}

\newcommand{\conjref}[1]{Conjecture-\ref{#1}}


% Normal subgroup or equal.

\providecommand{\normaleq}{\unlhd}

% Normal subgroup.

\providecommand{\normal}{\lhd}

\providecommand{\rnormal}{\rhd}
% Divides, does not divide.

\providecommand{\divides}{\mid}

\providecommand{\ndivides}{\nmid}


\providecommand{\union}{\cup}

\providecommand{\bigunion}{\bigcup}

\providecommand{\intersect}{\cap}

\providecommand{\bigintersect}{\bigcap}










\begin{document}
Suppose $A$ is a set with a binary operation.  Then we say a subset $S$ of $A$ 
\emph{generates} $A$ if finite iterations of products (commonly called words over $S$) from the set $S$ eventually produce every element of $A$.  We write
this property as $A=\langle S\rangle$.

\begin{defn}
An element $g\in A$ is said to be a \emph{non-generator} if given a subset $S$
of $A$ such that $A=\langle S\union\{g\}\rangle$ then in fact
$A=\langle S\rangle$.  The set of all non-generators is called the
\emph{Frattini subset} of $A$.
\end{defn}

\begin{ex}\label{ex:int}
The only non-generator of $\mathbb{Z}$, the group of integers under addition, 
is 0.
\end{ex}
\begin{proof}
Take $n\neq 0$.  Without loss of generality, $n$ is positive.  Then take 
$1<m$ some integer relatively prime to $n$.  Thus by the 
Euclidean algorithm we know there are integers $a,b$ such that $1=am+bn$.  
This shows $1\in \langle m,n\rangle$ so in fact $\{m,n\}$ generates 
$\mathbb{Z}$.

However, $\mathbb{Z}$ is not generated by $m$, as $m>1$.  Therefore $n$ cannot be removed form the generating set $\{m,n\}$ and so indeed the only non-generator of $\mathbb{Z}$ is $0$.
\end{proof}

\begin{ex}
In the ring $\mathbb{Z}_{4}$, the element $2$ is a non-generator.
\end{ex}
\begin{proof}
Check the possible generating sets directly.
\end{proof}

\begin{ex}
The set of positive integers under addition, $\mathbb{N}$, has no non-generators.
\end{ex}
\begin{proof} Apply the same proof as done to in Example \ref{ex:int}.
\end{proof}

So we see not all sets with binary operations have non-generators.  In the case that a binary operation has an identity then the identity always serves as a non-generator due to the convention that the empty word be defined as the identity.  However, without further assumptions on the product, such as associativity, it is not always possible to treat the Frattini subset as a subobject in the category of the orignal object.  For example, we have just shown that the Frattini subset of a semi-group need not be a semi-group.

\begin{prop}
In the category of groups, the Frattini subset is a fully invariant subgroup.
\end{prop}

To prove this, we prove the following strong re-characterize the Frattini subset.
\begin{thm}
In a group $G$, the intersection of all maximal subgroups is the 
Frattini subset.
\end{thm}
\begin{proof}
For a group $G$, given a non-generator $a$, and $M$ any maximal subgroup 
of $G$.  If $a$ is not in $M$ then $\langle M,a\rangle$ is a larger subgroup
than $M$.  Thus $G=\langle M,a\rangle$.  But $a$ is a non-generator so
$G=\langle M\rangle=M$.  This contradicts the assumption that $M$ is a maximal
subgroup and therefore $a\in M$.  So the Frattini subset lies in every
maximal subgroup.

In contrast, if $a$ is in all maximal subgroups of $G$, then given any 
subset $S$ of $G$ for which $G=\langle S,a\rangle$, then set $M=\langle S\rangle$.  If $M=G$, then $a$ is a non-generator.  If not, then $M$ lies
in some maximal subgroup $H$ of $G$.  Since $a$ lies in all maximal subgroup,
$a$ lies in $H$, and thus $H$ contains $\langle S,a\rangle=G$.  As $H$ is
maximal, this is impossible.  Hence $G=M$ and $a$ is a non-generator.
\end{proof}

%%%%%
%%%%%
\end{document}
