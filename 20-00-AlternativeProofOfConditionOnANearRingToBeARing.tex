\documentclass[12pt]{article}
\usepackage{pmmeta}
\pmcanonicalname{AlternativeProofOfConditionOnANearRingToBeARing}
\pmcreated{2013-03-22 17:20:06}
\pmmodified{2013-03-22 17:20:06}
\pmowner{Wkbj79}{1863}
\pmmodifier{Wkbj79}{1863}
\pmtitle{alternative proof of condition on a near ring to be a ring}
\pmrecord{9}{39689}
\pmprivacy{1}
\pmauthor{Wkbj79}{1863}
\pmtype{Proof}
\pmcomment{trigger rebuild}
\pmclassification{msc}{20-00}
\pmclassification{msc}{16-00}
\pmclassification{msc}{13-00}

\endmetadata

\usepackage{amssymb}
\usepackage{amsmath}
\usepackage{amsfonts}
\usepackage{pstricks}
\usepackage{psfrag}
\usepackage{graphicx}
\usepackage{amsthm}
%%\usepackage{xypic}
\newtheorem{thm*}{Theorem}

\begin{document}
\PMlinkescapeword{near}
\PMlinkescapeword{combination}

\begin{thm*}
Let $(R,+,\cdot)$ be a near ring with a multiplicative identity $1$ such that the $\cdot$ also left distributes over $+$; that is, $c\cdot (a+b)=c\cdot a+c\cdot b$. Then $R$ is a ring.
\end{thm*}

\begin{proof}
All that needs to be verified is commutativity of $+$.

Let $a,b\in R$.  Consider the expression $(1+1)(a+b)$.

We have:

\begin{alignat*}{2}
(1+1)(a+b) &=(1+1)a+(1+1)b & \quad \qquad \text{by left distributivity} \\
&=1a+1a+1b+1b & \quad \qquad \text{by right distributivity} \\
&=a+a+b+b & \quad \qquad \text{since } 1 \text{ is a multiplicative identity} \end{alignat*}

On the other hand, we have:

\begin{alignat*}{2}
(1+1)(a+b) &=1(a+b)+1(a+b) & \quad \qquad \text{by right distributivity} \\
&=a+b+a+b & \quad \qquad \text{since } 1 \text{ is a multiplicative identity} \end{alignat*}

Thus, $a+a+b+b=a+b+a+b$.  Hence:

\begin{alignat*}{2}
a+b &=0+(a+b)+0 & \quad \qquad \text{since } 0 \text{ is an \PMlinkname{additive identity}{AdditiveIdentity}} \\
&=(-a+a)+(a+b)+(b+-b) & \quad \qquad \text{by definition of \PMlinkname{additive inverse}{AdditiveInverse}} \\
&=-a+(a+a+b+b)+-b & \quad \qquad \text{by associativity of } + \\
&=-a+(a+b+a+b)+-b & \quad \qquad \text{since } a+a+b+b=a+b+a+b \\
&=(-a+a)+(b+a)+(b+-b) & \quad \qquad \text{by associativity of } + \\
&=0+(b+a)+0 & \quad \qquad \text{by definition of \PMlinkescapetext{additive inverse}} \\
&=b+a & \quad \qquad \text{since } 0 \text{ is an \PMlinkescapetext{additive identity}} \end{alignat*}
\end{proof}
%%%%%
%%%%%
\end{document}
