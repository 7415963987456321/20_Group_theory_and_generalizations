\documentclass[12pt]{article}
\usepackage{pmmeta}
\pmcanonicalname{ExternalDirectProductOfGroups}
\pmcreated{2013-03-22 12:23:17}
\pmmodified{2013-03-22 12:23:17}
\pmowner{mathcam}{2727}
\pmmodifier{mathcam}{2727}
\pmtitle{external direct product of groups}
\pmrecord{8}{32180}
\pmprivacy{1}
\pmauthor{mathcam}{2727}
\pmtype{Definition}
\pmcomment{trigger rebuild}
\pmclassification{msc}{20K25}
\pmsynonym{direct product}{ExternalDirectProductOfGroups}
\pmrelated{CategoricalDirectProduct}
\pmrelated{DirectProductAndRestrictedDirectProductOfGroups}

% this is the default PlanetMath preamble.  as your knowledge
% of TeX increases, you will probably want to edit this, but
% it should be fine as is for beginners.

% almost certainly you want these
\usepackage{amssymb}
\usepackage{amsmath}
\usepackage{amsfonts}

% used for TeXing text within eps files
%\usepackage{psfrag}
% need this for including graphics (\includegraphics)
%\usepackage{graphicx}
% for neatly defining theorems and propositions
%\usepackage{amsthm}
% making logically defined graphics
%%%\usepackage{xypic}

% there are many more packages, add them here as you need them

% define commands here
\begin{document}
The \emph{external direct product} $G \times H$ of two groups $G$ and $H$ is defined to be the set of ordered pairs $(g,h)$, with $g\in G$ and $h\in H$. The group operation is defined by

$(g,h)(g',h') = (gg', hh')$

It can be shown that $G \times H$ obeys the group axioms. More generally, we can define the external direct product of $n$ groups, in the obvious way. Let $G = G_1 \times \ldots \times G_n$ be the set of all ordered n-tuples $\{(g_1, g_2 \ldots ,g_n) \mid g_i \in G_i\}$ and define the group operation by componentwise multiplication as before.
%%%%%
%%%%%
\end{document}
