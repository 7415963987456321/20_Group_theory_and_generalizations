\documentclass[12pt]{article}
\usepackage{pmmeta}
\pmcanonicalname{SubnormalSeries}
\pmcreated{2013-03-22 13:58:42}
\pmmodified{2013-03-22 13:58:42}
\pmowner{mclase}{549}
\pmmodifier{mclase}{549}
\pmtitle{subnormal series}
\pmrecord{8}{34750}
\pmprivacy{1}
\pmauthor{mclase}{549}
\pmtype{Definition}
\pmcomment{trigger rebuild}
\pmclassification{msc}{20D30}
\pmsynonym{subinvariant series}{SubnormalSeries}
\pmrelated{SubnormalSubgroup}
\pmrelated{JordanHolderDecompositionTheorem}
\pmrelated{Solvable}
\pmrelated{DescendingSeries}
\pmrelated{AscendingSeries}
\pmdefines{composition series}
\pmdefines{normal series}
\pmdefines{principal series}
\pmdefines{chief series}

\endmetadata

% this is the default PlanetMath preamble.  as your knowledge
% of TeX increases, you will probably want to edit this, but
% it should be fine as is for beginners.

% almost certainly you want these
\usepackage{amssymb}
\usepackage{amsmath}
\usepackage{amsfonts}

% used for TeXing text within eps files
%\usepackage{psfrag}
% need this for including graphics (\includegraphics)
%\usepackage{graphicx}
% for neatly defining theorems and propositions
%\usepackage{amsthm}
% making logically defined graphics
%%%\usepackage{xypic}

% there are many more packages, add them here as you need them

% define commands here
\begin{document}
\PMlinkescapeword{term}
\PMlinkescapeword{compatible}
\PMlinkescapeword{series}
\PMlinkescapeword{addition}

Let $G$ be a group with a subgroup $H$, and let
\begin{equation}
G = G_0 \rhd G_1 \rhd \cdots \rhd G_n = H
\end{equation}
be a series of subgroups with each $G_i$ a normal subgroup of $G_{i-1}$.
Such a series is called a \emph{subnormal series} or a \emph{subinvariant series}.

If in addition, each $G_i$ is a normal subgroup of $G$,
then the series is called a \emph{normal series}.

A subnormal series in which each $G_i$ is a maximal normal subgroup
of $G_{i-1}$ is called a \emph{composition series}.

A normal series in which $G_i$ is a maximal normal subgroup of $G$ contained in $G_{i-1}$
is called a \emph{principal series} or a \emph{chief series}.

Note that a composition series need not end in the trivial group $1$.
One speaks of a composition series (1) as a \emph{composition series from $G$ to $H$}.
But the term \emph{composition series for $G$}
generally means a composition series from $G$ to $1$.

Similar remarks apply to principal series.

Some authors use normal series as a synonym for subnormal series.  This usage is, of course, not compatible with the stronger definition of normal series given above.
%%%%%
%%%%%
\end{document}
