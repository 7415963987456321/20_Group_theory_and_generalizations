\documentclass[12pt]{article}
\usepackage{pmmeta}
\pmcanonicalname{NullSemigroup}
\pmcreated{2013-03-22 13:02:22}
\pmmodified{2013-03-22 13:02:22}
\pmowner{mclase}{549}
\pmmodifier{mclase}{549}
\pmtitle{null semigroup}
\pmrecord{4}{33441}
\pmprivacy{1}
\pmauthor{mclase}{549}
\pmtype{Definition}
\pmcomment{trigger rebuild}
\pmclassification{msc}{20M99}
\pmrelated{Semigroup}
\pmrelated{ZeroElements}
\pmdefines{null semigroup}
\pmdefines{left zero semigroup}
\pmdefines{right zero semigroup}

\endmetadata

% this is the default PlanetMath preamble.  as your knowledge
% of TeX increases, you will probably want to edit this, but
% it should be fine as is for beginners.

% almost certainly you want these
\usepackage{amssymb}
\usepackage{amsmath}
\usepackage{amsfonts}

% used for TeXing text within eps files
%\usepackage{psfrag}
% need this for including graphics (\includegraphics)
%\usepackage{graphicx}
% for neatly defining theorems and propositions
%\usepackage{amsthm}
% making logically defined graphics
%%%\usepackage{xypic}

% there are many more packages, add them here as you need them

% define commands here
\begin{document}
A \emph{left zero semigroup} is a semigroup in which every element is a left zero element.  In other words, it is a set $S$ with a product defined as $xy = x$ for all $x, y \in S$.

A \emph{right zero semigroup} is defined similarly.

Let $S$ be a semigroup.  Then $S$ is a \emph{null semigroup} if it has a zero element and if the product of any two elements is zero.  In other words, there is an element $\theta \in S$ such that $xy = \theta$ for all $x, y \in S$.
%%%%%
%%%%%
\end{document}
