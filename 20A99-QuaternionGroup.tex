\documentclass[12pt]{article}
\usepackage{pmmeta}
\pmcanonicalname{QuaternionGroup}
\pmcreated{2013-03-22 12:35:35}
\pmmodified{2013-03-22 12:35:35}
\pmowner{mathcam}{2727}
\pmmodifier{mathcam}{2727}
\pmtitle{quaternion group}
\pmrecord{12}{32844}
\pmprivacy{1}
\pmauthor{mathcam}{2727}
\pmtype{Definition}
\pmcomment{trigger rebuild}
\pmclassification{msc}{20A99}
\pmsynonym{quaternionic group}{QuaternionGroup}
\pmrelated{Quaternions}
\pmdefines{quaternion group}

\endmetadata

\usepackage{amssymb}
\usepackage{amsmath}
\usepackage{amsfonts}
\begin{document}
\PMlinkescapeword{order}
The quaternion group, or quaternionic group, is a noncommutative
group with eight elements. It is traditionally denoted by $Q$ (not to be
confused with $\mathbb{Q}$) or by $Q_8$. This group is defined by the
presentation
$$\{i,j;i^4,i^2j^2,iji^{-1}j\}$$
or, equivalently, defined by the multiplication table

\begin{center}
\begin{tabular}{rrrrrrrrr}\hline
$\cdot$& $1$  & $i$  & $j$  & $k$  & $-i$ & $-j$ & $-k$ & $-1$ \\
%\hline
$1$    & $1$  & $i$  & $j$  & $k$  & $-i$ & $-j$ & $-k$ & $-1$ \\\hline
$i$    & $i$  & $-1$ & $k$  & $-j$ & $1$  & $-k$ & $j$  & $-i$ \\\hline
$j$    & $j$  & $-k$ & $-1$ & $i$  & $k$  & $1$  & $-i$ & $-j$ \\\hline
$k$    & $k$  & $j$  & $-i$ & $-1$ & $-j$ & $i$  & $1$  & $-k$ \\\hline
$-i$   & $-i$ & $1$  & $-k$ & $j$  & $-1$ & $k$  & $-j$ & $i$  \\\hline
$-j$   & $-j$ & $k$  & $1$  & $-i$ & $-k$ & $-1$ & $i$  & $j$  \\\hline
$-k$   & $-k$ & $-j$ & $i$  & $1$  & $j$  & $-i$ & $-1$ & $k$  \\\hline
$-1$   & $-1$ & $-i$ & $-j$ & $-k$ & $i$  & $j$  & $k$  & $1$\\\hline
\end{tabular}
\end{center}

\noindent
where we have put each product $xy$ into row $x$ and column $y$.
The minus signs are justified by the fact that $\{1,-1\}$ is subgroup
contained in the center of $Q$.
Every subgroup of $Q$ is normal and, except for
the trivial subgroup $\{1\}$, contains $\{1,-1\}$.
The dihedral group $D_4$ (the group of symmetries of a square) is the
only other noncommutative group of order 8.

Since $i^2 = j^2 = k^2 = -1$,
the elements $i$, $j$, and $k$ are known as the imaginary units, by
analogy with $i\in\mathbb{C}$. Any pair of the imaginary units generate
the group. Better, given $x,y\in\{i,j,k\}$, any element of $Q$
is expressible in the form $x^my^n$.

$Q$ is identified with the group of units (invertible elements) of the
ring of quaternions over $\mathbb{Z}$. That ring
is not identical to the group ring $\mathbb{Z}[Q]$, which has dimension 8
(not 4) over $\mathbb{Z}$. Likewise the usual quaternion algebra
is not quite the same thing as the group algebra $\mathbb{R}[Q]$.

Quaternions were known to Gauss in 1819 or 1820, but he did not
publicize this discovery, and quaternions weren't rediscovered until
1843, with Hamilton. 
%%%%%
%%%%%
\end{document}
