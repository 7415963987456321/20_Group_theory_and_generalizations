\documentclass[12pt]{article}
\usepackage{pmmeta}
\pmcanonicalname{Commutative}
\pmcreated{2013-03-22 12:22:45}
\pmmodified{2013-03-22 12:22:45}
\pmowner{CWoo}{3771}
\pmmodifier{CWoo}{3771}
\pmtitle{commutative}
\pmrecord{11}{32148}
\pmprivacy{1}
\pmauthor{CWoo}{3771}
\pmtype{Definition}
\pmcomment{trigger rebuild}
\pmclassification{msc}{20-00}
\pmsynonym{commutativity}{Commutative}
\pmsynonym{commutative law}{Commutative}
\pmrelated{Associative}
\pmrelated{AbelianGroup2}
\pmrelated{QuantumTopos}
\pmrelated{NonCommutativeStructureAndOperation}
\pmrelated{Subcommutative}
\pmdefines{non-commutative}

\endmetadata

\usepackage{amssymb}
\usepackage{amsmath}
\usepackage{amsfonts}

%\usepackage{psfrag}
%\usepackage{graphicx}
%%%\usepackage{xypic}
\begin{document}
Let $S$ be a set and $\circ$ a binary operation on it.  $\circ$ is said to be \emph{commutative} if

$$a \circ b = b \circ a$$

for all $a,b \in S$.

Viewing $\circ$ as a function from $S\times S$ to $S$, the commutativity of $\circ$ can be notated as $$\circ(a,b)=\circ(b,a).$$

Some common examples of commutative operations are 
\begin{itemize}
\item addition over the integers: $m+n=n+m$ for all integers $m,n$
\item multiplication over the integers: $m\cdot n=n\cdot m$ for all integers $m,n$
\item addition over $n \times n$ matrices, $A+B=B+A$ for all $n\times n$ matrices $A,B$, and
\item multiplication over the reals: $rs=sr$, for all real numbers $r,s$.
\end{itemize}

A binary operation that is not commutative is said to be \emph{non-commutative}.  A common example of a non-commutative operation is the subtraction over the integers (or more generally the real numbers).  This means that, in general, $$a-b\ne b-a.$$  For instance, $2-1=1\ne -1 = 1-2$.

Other examples of non-commutative binary operations can be found in the attachment below.

\textbf{Remark}.  The notion of commutativity can be generalized to $n$-ary operations, where $n\ge 2$.  An $n$-ary operation $f$ on a set $A$ is said to be \emph{commutative} if 
$$f(a_1,a_2,\ldots, a_n)=f(a_{\pi(1)},a_{\pi(2)},\ldots, a_{\pi(n)})$$ 
for every permutation $\pi$ on $\lbrace 1,2,\ldots, n\rbrace$, and for every choice of $n$ elements $a_i$ of $A$.
%%%%%
%%%%%
\end{document}
