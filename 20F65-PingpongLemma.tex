\documentclass[12pt]{article}
\usepackage{pmmeta}
\pmcanonicalname{PingpongLemma}
\pmcreated{2013-03-22 17:11:21}
\pmmodified{2013-03-22 17:11:21}
\pmowner{uriw}{288}
\pmmodifier{uriw}{288}
\pmtitle{ping-pong lemma}
\pmrecord{8}{39507}
\pmprivacy{1}
\pmauthor{uriw}{288}
\pmtype{Theorem}
\pmcomment{trigger rebuild}
\pmclassification{msc}{20F65}
\pmsynonym{table-tennis lemma}{PingpongLemma}

% almost certainly you want these
\usepackage{amssymb}
\usepackage{amsmath}
\usepackage{amsfonts}
\usepackage{amsthm}

\newtheorem*{theorem}{Theorem}
\newtheorem{fact}{Fact}

\begin{document}
\begin{theorem}[Ping Pong Lemma]
Let $k\geq2$ and let $G$ be a group acting on a space $X$. Suppose
we are given a class
$\mathcal{M}=\{A_1,A_2,\ldots,A_k,B_1,B_2,\ldots,B_k\}$ of $2k$
pairwise disjoint subsets of $X$ and suppose $y_1,y_2,\ldots,y_k$
are elements of $G$ such that
\[
B_i^c \subseteq y_i(A_i)\quad i=1,2,\ldots,k
\]
($B_i^c$ is the complement of $B_i$ in $X$). Then, the subgroup
of $G$ generated by $y_1,y_2,\ldots,y_k$ is free.
\end{theorem}

\PMlinkescapeword{simple}

\noindent Before turning to prove the lemma let's state three
simple facts:
\begin{fact} \label{fct1}
For all $i=1,\ldots,k$ we have $y_i(A_i^c)\subseteq B_i$ and
$y_i^{-1}(B_i^c)\subseteq A_i$
\end{fact}
\begin{proof}
$B_i^c \subseteq y_i(A_i) \implies B_i \supseteq y_i(A_i)^c =
y_i(A_i^c)$
\end{proof}
\begin{fact} \label{fct2}
If $i\neq j$ then $A_j\cup B_j \subseteq A_i^c\cap B_i^c$.
\end{fact}
\begin{proof}
$A_i$ and $A_j$ are disjoint therefore $A_j\subseteq A_i^c$.
Similarly, $A_j\subseteq B_i^c$ so $A_j\subseteq A_i^c\cap B_i^c$.
In the same way, $B_j\subseteq A_i^c\cap B_i^c$ so $A_j\cup B_j
\subseteq A_i^c\cap B_i^c$.
\end{proof}

\begin{fact} \label{fct3}
If $R,S\in\mathcal{M}$ then $R^c\not\subseteq S$
\end{fact}
\begin{proof}
Assume by contradiction that $R^c\subseteq S$. Then, $X=R\cup
S$ and therefore any element of $M$ intersects with either $R$ or
$S$. However, the elements of $M$ are pairwise disjoint and there
are at least 4 elements in $M$ so this is a contradiction.
\end{proof}

\bigskip

Using the above 3 facts, we now turn to the proof of the Ping Pong Lemma:

\begin{proof}
Suppose we are given $w=z_n^{\epsilon_n}\cdots
z_2^{\epsilon_2}z_1^{\epsilon_1}$ such that
$z_\ell\in\{y_1,y_2,\ldots,y_k\}$ and $\epsilon_\ell\in\{-1,+1\}$.
and suppose further that $w$ is freely reduced, namely, if
$z_i=z_{i+1}$ then $\epsilon_i=\epsilon_{i+1}$. We want to show
that $w\neq1$ in $G$. Assume by contradiction that $w=1$. We get a
contradiction by giving $R,S\in\mathcal{M}$ such that
$w(S^c)\subseteq R$ and therefore contradicting Fact \ref{fct3}
above since $S^c=w(S^c)\subseteq R$.

The set $S$ is chosen as follows. Assume that $z_1=y_i$ then:
\[
S=\left\{\begin{array}{ll}
  A_i & \text{if } \epsilon_1=1 \\
  B_i & \text{if } \epsilon_1=-1
\end{array} \right.
\]
Define the following subsets $P_0,P_1,\ldots,P_n$ of $X$:
\[
P_0 = S^c; \quad P_1=z_1^{\epsilon_1}(P_0), \ldots ,\;
P_n=z_n^{\epsilon_n}(P_{n-1})=w(S^c)
\]

\PMlinkescapeword{complete}

To complete the proof we show by induction that for
$\ell=1,2,\ldots,n$ if $z_\ell=y_i$ then:
\begin{enumerate}
  \item if $\epsilon_\ell=1$ then $P_\ell\subseteq B_i$.
  \item if $\epsilon_\ell=-1$ then $P_\ell\subseteq A_i$.
\end{enumerate}
For $\ell=1$ the above follows from Fact \ref{fct1} and the
specific choice of $P_0$. Assume it is true for $\ell-1$ and
assume that $z_\ell=y_i$. We have two cases to check:
\begin{enumerate}
 \item $z_{\ell-1}\neq z_\ell$: by the induction hypothesis
 $P_{\ell-1}$ is a subset of $A_j\cup B_j$ for some $j\neq i$.
 Therefore, by Fact \ref{fct2} we get that $P_{\ell-1}$ is a subset of $A_i^c\cap B_i^c$.
 Consequently, we get the following:
 \[
 P_\ell = z_\ell^{\epsilon_\ell}(P_{\ell-1}) = y_i^{\epsilon_\ell}(P_{\ell-1})
 \subseteq y_i^{\epsilon_\ell}(A_i^c\cap B_i^c)
 \]
 Hence, if $\epsilon_\ell=1$ then:
 \[
 P_\ell \subseteq y_i(A_i^c\cap B_i^c) \subseteq y_i(A_i^c) \subseteq B_i
 \]
 And if $\epsilon_\ell=-1$ then:
 \[
 P_\ell \subseteq y_i^{-1}(A_i^c\cap B_i^c) \subseteq y_i^{-1}(B_i^c) \subseteq A_i
 \]
 \item $z_{\ell-1}=z_\ell$: by the fact that $w$ is freely reduced we
 get an equality between $\epsilon_{\ell-1}$ and $\epsilon_\ell$. Hence, if
 $\epsilon_\ell=1$ then $P_{\ell-1}\subseteq B_i\subseteq A_i^c$ and
 therefore:
 \[
 P_\ell = z_\ell^{\epsilon_\ell}(P_{\ell-1}) = y_i(P_{\ell-1})
 \subseteq y_i(A_i^c) \subseteq B_i
 \]
 Similiarly, if $\epsilon_\ell=-1$ then $P_{\ell-1}\subseteq A_i\subseteq
 B_i^c$ and therefore:
 \[
 P_\ell = z_\ell^{\epsilon_\ell}(P_{\ell-1}) = y_i^{-1}(P_{\ell-1})
 \subseteq y_i^{-1}(B_i^c) \subseteq A_i
 \]
\end{enumerate}

\end{proof}

%%%%%
%%%%%
\end{document}
