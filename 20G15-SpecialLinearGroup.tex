\documentclass[12pt]{article}
\usepackage{pmmeta}
\pmcanonicalname{SpecialLinearGroup}
\pmcreated{2013-03-22 12:25:38}
\pmmodified{2013-03-22 12:25:38}
\pmowner{djao}{24}
\pmmodifier{djao}{24}
\pmtitle{special linear group}
\pmrecord{7}{32463}
\pmprivacy{1}
\pmauthor{djao}{24}
\pmtype{Definition}
\pmcomment{trigger rebuild}
\pmclassification{msc}{20G15}
\pmsynonym{SL}{SpecialLinearGroup}
\pmrelated{GeneralLinearGroup}
\pmrelated{Group}
\pmrelated{UnimodularMatrix}

% this is the default PlanetMath preamble.  as your knowledge
% of TeX increases, you will probably want to edit this, but
% it should be fine as is for beginners.

% almost certainly you want these
\usepackage{amssymb}
\usepackage{amsmath}
\usepackage{amsfonts}

% used for TeXing text within eps files
%\usepackage{psfrag}
% need this for including graphics (\includegraphics)
%\usepackage{graphicx}
% for neatly defining theorems and propositions
%\usepackage{amsthm}
% making logically defined graphics
%%%\usepackage{xypic} 

% there are many more packages, add them here as you need them

% define commands here
\newcommand{\SL}{{\operatorname{SL}}}
\begin{document}
Given a vector space $V$, the special linear group $\SL(V)$ is defined to be the subgroup of the general linear group $\operatorname{GL}(V)$ consisting of all invertible linear transformations $T: V \longrightarrow V$ in $\operatorname{GL}(V)$ that have determinant 1.

If $V = \mathbb{F}^n$ for some field $\mathbb{F}$, then the group $\SL(V)$ is often denoted $\SL(n,\mathbb{F})$ or $\SL_n(\mathbb{F})$, and if one identifies each linear transformation with its matrix with respect to the standard basis, then $\SL(n,\mathbb{F})$ consists of all $n \times n$ matrices with entries in $\mathbb{F}$ that have determinant 1.
%%%%%
%%%%%
\end{document}
