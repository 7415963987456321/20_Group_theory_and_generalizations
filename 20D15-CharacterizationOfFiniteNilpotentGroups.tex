\documentclass[12pt]{article}
\usepackage{pmmeta}
\pmcanonicalname{CharacterizationOfFiniteNilpotentGroups}
\pmcreated{2013-03-22 13:16:24}
\pmmodified{2013-03-22 13:16:24}
\pmowner{yark}{2760}
\pmmodifier{yark}{2760}
\pmtitle{characterization of finite nilpotent groups}
\pmrecord{11}{33755}
\pmprivacy{1}
\pmauthor{yark}{2760}
\pmtype{Theorem}
\pmcomment{trigger rebuild}
\pmclassification{msc}{20D15}
\pmclassification{msc}{20F18}
\pmrelated{FiniteNilpotentGroups}
\pmrelated{NilpotentGroup}
\pmrelated{NormalizerCondition}
\pmrelated{SubnormalSubgroup}
\pmrelated{LocallyNilpotentGroup}

\endmetadata


\begin{document}
\PMlinkescapeword{subgroup}

Let $G$ be a finite group.  The following are equivalent:
\begin{enumerate}
\item $G$ is nilpotent.
\item Every \PMlinkname{subgroup}{Subgroup} of $G$ is subnormal.
\item Every proper subgroup of $G$ is properly contained in its normalizer.
\item Every maximal subgroup of $G$ is normal.
\item Every Sylow subgroup of $G$ is normal.
\item $G$ is a \PMlinkname{direct product}{DirectProductAndRestrictedDirectProductOfGroups} of \PMlinkname{$p$-groups}{PGroup4}.
\end{enumerate}

For proofs, see the article on finite nilpotent groups.

Condition 3 above is the normalizer condition.
%%%%%
%%%%%
\end{document}
