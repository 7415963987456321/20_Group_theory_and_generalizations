\documentclass[12pt]{article}
\usepackage{pmmeta}
\pmcanonicalname{RationalSet}
\pmcreated{2013-03-22 19:01:16}
\pmmodified{2013-03-22 19:01:16}
\pmowner{CWoo}{3771}
\pmmodifier{CWoo}{3771}
\pmtitle{rational set}
\pmrecord{8}{41892}
\pmprivacy{1}
\pmauthor{CWoo}{3771}
\pmtype{Definition}
\pmcomment{trigger rebuild}
\pmclassification{msc}{20M35}
\pmclassification{msc}{68Q70}
\pmdefines{rational language}

\usepackage{amssymb,amscd}
\usepackage{amsmath}
\usepackage{amsfonts}
\usepackage{mathrsfs}

% used for TeXing text within eps files
%\usepackage{psfrag}
% need this for including graphics (\includegraphics)
%\usepackage{graphicx}
% for neatly defining theorems and propositions
\usepackage{amsthm}
% making logically defined graphics
%%\usepackage{xypic}
\usepackage{pst-plot}

% define commands here
\newcommand*{\abs}[1]{\left\lvert #1\right\rvert}
\newtheorem{prop}{Proposition}
\newtheorem{thm}{Theorem}
\newtheorem{ex}{Example}
\newcommand{\real}{\mathbb{R}}
\newcommand{\pdiff}[2]{\frac{\partial #1}{\partial #2}}
\newcommand{\mpdiff}[3]{\frac{\partial^#1 #2}{\partial #3^#1}}
\begin{document}
Given an alphabet $\Sigma$, recall that a regular language $R$ is a certain subset of the free monoid $M$ generated by $\Sigma$, which can be obtained by taking singleton subsets of $\Sigma$, and perform, in a finite number of steps, any of the three basic operations: taking union, string concatenation, and the Kleene star.  

The construction of a set like $R$ is still possible without $M$ being finitely generated free.

Let $M$ be a monoid, and $\mathcal{S}_M$ the set of all singleton subsets of $M$.  Consider the closure $\mathcal{R}_M$ of $S$ under the operations of union, product, and the formation of a submonoid of $M$.  In other words, $\mathcal{R}_M$ is the smallest subset of $M$ such that
\begin{itemize}
\item $\varnothing\in \mathcal{R}_M$,
\item $A,B\in \mathcal{R}_M$ imply $A\cup B \in \mathcal{R}_M$,
\item $A,B\in \mathcal{R}_M$ imply $AB\in \mathcal{R}_M$, where $AB=\lbrace ab\mid a\in A,b\in B\rbrace$,
\item $A\in \mathcal{R}_M$ implies $A^*\in \mathcal{R}_M$, where $A^*$ is the submonoid generated by $A$.
\end{itemize}
\textbf{Definition}.  A \emph{rational set} of $M$ is an element of $\mathcal{R}_M$.  

If $M$ is a finite generated free monoid, then a rational set of $M$ is also called a \emph{rational language}, more commonly known as a \emph{regular language}.

Like regular languages, rational sets can also be represented by regular expressions.  A regular expression over a monoid $M$ and the set it represents are defined inductively as follows:
\begin{itemize}
\item $\varnothing$ and $a$ are regular expressions, for any $a\in M$, representing sets $\varnothing$ and $\lbrace a\rbrace$ respectively.
\item if $a,b$ are regular expressions, so are $a\cup b$, $ab$, and $a^*$.  Furthermore, if $a,b$ represent sets $A,B$, then $a\cup b$, $ab$, and $a^*$ represent sets $A\cup B$, $AB$, and $A^*$ respectively.
\end{itemize}
Parentheses are used, as usual, to avoid ambiguity.

From the definition above, it is easy to see that a set $A\subseteq M$ is rational iff it can be represented by a regular expression over $M$.

Below are some basic properties of rational sets:
\begin{enumerate}
\item Any rational set $M$ is a subset of a finitely generated submonoid of $M$.  As a result, every rational set over $M$ is finite iff $M$ is locally finite (meaning every finitely generated submonoid of $M$ is actually finite).
\item Rationality is preserved under homomorphism: if $A$ is rational over $M$ and $f:M\to N$ is a homomorphism, then $f(A)$ is rational over $N$.
\item Conversely, if $B\in f(M)$ is rational over $N$, then there is a rational set $A$ over $M$ such that $f(A)=B$.  Thus if $f$ is onto, every rational set over $N$ is mapped by a rational set over $M$.  If $f$ fails to be onto, the statement becomes false.  In fact, inverse homomorphisms generally do not preserve rationality.
\end{enumerate}

\begin{thebibliography}{8}
\bibitem{se} S. Eilenberg, {\em Automata, Languages, and Machines, Vol. A}, Academic Press (1974).
\end{thebibliography}

%%%%%
%%%%%
\end{document}
