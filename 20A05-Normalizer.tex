\documentclass[12pt]{article}
\usepackage{pmmeta}
\pmcanonicalname{Normalizer}
\pmcreated{2013-03-22 12:36:53}
\pmmodified{2013-03-22 12:36:53}
\pmowner{yark}{2760}
\pmmodifier{yark}{2760}
\pmtitle{normalizer}
\pmrecord{15}{32873}
\pmprivacy{1}
\pmauthor{yark}{2760}
\pmtype{Definition}
\pmcomment{trigger rebuild}
\pmclassification{msc}{20A05}
\pmsynonym{normaliser}{Normalizer}
\pmrelated{Centralizer}
\pmrelated{NormalSubgroup}
\pmrelated{NormalClosure2}
\pmdefines{self-normalizing}

\endmetadata

\usepackage{amssymb}
\usepackage{amsmath}
\usepackage{amsfonts}

\begin{document}
\section*{Definitions}

Let $G$ be a group, and let $H \subseteq G$.
The {\em normalizer} of $H$ in $G$, written $N_G(H)$, is the set
\[
  \{ g \in G \mid gHg^{-1}=H \}.
\]

A subgroup $H$ of $G$ is said to be {\em self-normalizing} if $N_G(H) = H$.

\section*{Properties}

$N_G(H)$ is always a subgroup of $G$,
as it is the stabilizer of $H$ under the action $(g,H)\mapsto gHg^{-1}$
of $G$ on the set of all subsets of $G$
(or on the set of all subgroups of $G$, if $H$ is a subgroup).

If $H$ is a subgroup of $G$, then $H\leq N_G(H)$.

If $H$ is a subgroup of $G$, then $H$ is a normal subgroup of $N_G(H)$;
in fact, $N_G(H)$ is the largest subgroup of $G$
of which $H$ is a normal subgroup.
In particular, if $H$ is a subgroup of $G$,
then $H$ is normal in $G$ if and only if $N_G(H)=G$.
%%%%%
%%%%%
\end{document}
