\documentclass[12pt]{article}
\usepackage{pmmeta}
\pmcanonicalname{SubgroupsWithCoprimeOrders}
\pmcreated{2013-03-22 18:55:58}
\pmmodified{2013-03-22 18:55:58}
\pmowner{pahio}{2872}
\pmmodifier{pahio}{2872}
\pmtitle{subgroups with coprime orders}
\pmrecord{8}{41786}
\pmprivacy{1}
\pmauthor{pahio}{2872}
\pmtype{Theorem}
\pmcomment{trigger rebuild}
\pmclassification{msc}{20D99}
\pmrelated{Gcd}
\pmrelated{CycleNotation}

\endmetadata

% this is the default PlanetMath preamble.  as your knowledge
% of TeX increases, you will probably want to edit this, but
% it should be fine as is for beginners.

% almost certainly you want these
\usepackage{amssymb}
\usepackage{amsmath}
\usepackage{amsfonts}

% used for TeXing text within eps files
%\usepackage{psfrag}
% need this for including graphics (\includegraphics)
%\usepackage{graphicx}
% for neatly defining theorems and propositions
 \usepackage{amsthm}
% making logically defined graphics
%%%\usepackage{xypic}

% there are many more packages, add them here as you need them

% define commands here

\theoremstyle{definition}
\newtheorem*{thmplain}{Theorem}

\begin{document}
If the orders of two subgroups of a group are \PMlinkname{coprime}{Coprime}, the identity element is the only common element of the subgroups.\\

\emph{Proof.}\, Let $G$ and $H$ be such subgroups and $|G|$ and $|H|$ their orders.\, Then the intersection $G\!\cap\!H$ is a subgroup of both $G$ and $H$.\, By Lagrange's theorem, $|G\!\cap\!H|$ divides both $|G|$ and $|H|$ and consequently it divides also\, $\gcd(|G|,\,|H|)$\, which is 1.\, Therefore\, $|G\!\cap\!H| = 1$,\, whence the intersection contains only the identity element.\\

\textbf{Example.}\, All subgroups 
$$\{(1),\,(12)\},\quad \{(1),\,(13)\}, \quad \{(1),\,(23)\}$$
of order 2 of the symmetric group $\mathfrak{S}_3$ have only the identity element $(1)$ common with the sole subgroup
$$\{(1),\,(123),\,(132)\}$$
of order 3.
%%%%%
%%%%%
\end{document}
