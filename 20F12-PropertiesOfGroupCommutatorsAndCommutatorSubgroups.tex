\documentclass[12pt]{article}
\usepackage{pmmeta}
\pmcanonicalname{PropertiesOfGroupCommutatorsAndCommutatorSubgroups}
\pmcreated{2013-03-22 15:30:50}
\pmmodified{2013-03-22 15:30:50}
\pmowner{GrafZahl}{9234}
\pmmodifier{GrafZahl}{9234}
\pmtitle{properties of group commutators and commutator subgroups}
\pmrecord{11}{37381}
\pmprivacy{1}
\pmauthor{GrafZahl}{9234}
\pmtype{Theorem}
\pmcomment{trigger rebuild}
\pmclassification{msc}{20F12}
\pmrelated{NormalSubgroup}
\pmdefines{Hall-Witt identity}
\pmdefines{three subgroup lemma}

% this is the default PlanetMath preamble.  as your knowledge
% of TeX increases, you will probably want to edit this, but
% it should be fine as is for beginners.

% almost certainly you want these
\usepackage{amssymb}
\usepackage{amsmath}
\usepackage{amsfonts}

% used for TeXing text within eps files
%\usepackage{psfrag}
% need this for including graphics (\includegraphics)
%\usepackage{graphicx}
% for neatly defining theorems and propositions
\usepackage{amsthm}
% making logically defined graphics
%%%\usepackage{xypic}

% there are many more packages, add them here as you need them

% define commands here
\newcommand{\<}{\langle}
\renewcommand{\>}{\rangle}
\newcommand{\Bigcup}{\bigcup\limits}
\newcommand{\DirectSum}{\bigoplus\limits}
\newcommand{\Prod}{\prod\limits}
\newcommand{\Sum}{\sum\limits}
\newcommand{\h}{\widehat}
\newcommand{\mbb}{\mathbb}
\newcommand{\mbf}{\mathbf}
\newcommand{\mc}{\mathcal}
\newcommand{\mmm}[9]{\left(\begin{array}{rrr}#1&#2&#3\\#4&#5&#6\\#7&#8&#9\end{array}\right)}
\newcommand{\mf}{\mathfrak}
\newcommand{\ol}{\overline}

% Math Operators/functions
\DeclareMathOperator{\Aut}{Aut}
\DeclareMathOperator{\End}{End}
\DeclareMathOperator{\Frob}{Frob}
\DeclareMathOperator{\cwe}{cwe}
\DeclareMathOperator{\id}{id}
\DeclareMathOperator{\mult}{mult}
\DeclareMathOperator{\we}{we}
\DeclareMathOperator{\wt}{wt}
\begin{document}
\newtheorem{thm}{Theorem}
The purpose of this entry is to collect properties of \PMlinkid{group
commutators}{2812} and commutator subgroups. Feel free to add more theorems!

Let $G$ be a group.

\begin{thm}
\label{thm:inversecommutator}
Let $x,y\in G$, then $[x,y]^{-1}=[y,x]$.
\end{thm}
\begin{proof}
Direct computation yields
\begin{equation*}
[x,y]^{-1}=(x^{-1}y^{-1}xy)^{-1}=y^{-1}x^{-1}yx=[y,x].
\end{equation*}
\end{proof}

\begin{thm}
Let $X,Y$ be subsets of $G$, then $[X,Y]=[Y,X]$.
\end{thm}
\begin{proof}
By Theorem~\ref{thm:inversecommutator}, the elements from $[X,Y]$ or
$[Y,X]$ are products of commutators of the form $[x,y]$ or $[y,x]$
with $x\in X$ and $y\in Y$.
\end{proof}

\begin{thm}[Hall--Witt identity]
Let $x,y,z\in G$, then
\begin{equation*}
y^{-1}[x,y^{-1},z]yz^{-1}[y,z^{-1},x]zx^{-1}[z,x^{-1},y]x=1.
\end{equation*}
\end{thm}
\begin{proof}
This is mainly a brute-force calculation. We can easily calculate the
first factor $y^{-1}[x,y^{-1},z]y$ explicitly using
theorem~\ref{thm:inversecommutator}:
\begin{align*}
&y^{-1}[x,y^{-1},z]y\\
=&y^{-1}[y^{-1},x]z^{-1}[x,y^{-1}]zy\\
=&y^{-1}yx^{-1}y^{-1}xz^{-1}x^{-1}yxy^{-1}zy\\
=&x^{-1}y^{-1}xz^{-1}x^{-1}yxy^{-1}zy.
\end{align*}
Let $h_1:=x^{-1}y^{-1}xz^{-1}x^{-1}$, the ``first half'' of
$y^{-1}[x,y^{-1},z]y$. Let $h_2$ be the element obtained from $h_1$ by
the cyclic shift $S\colon x\mapsto y\mapsto z\mapsto x$, and $h_3$ be
the element obtained from $h_2$ by $S$. We have
\begin{equation*}
h_2^{-1}=(y^{-1}z^{-1}yx^{-1}y^{-1})^{-1}=yxy^{-1}zy
\end{equation*}
which gives us
\begin{equation*}
y^{-1}[x,y^{-1},z]y=h_1h_2^{-1},
\end{equation*}
and, by applying $S$ twice
\begin{align*}
z^{-1}[y,z^{-1},x]z&=h_2h_3^{-1},\\
x^{-1}[z,x^{-1},y]x&=h_3h_1^{-1}.
\end{align*}
In total, we have
\begin{equation*}
y^{-1}[x,y^{-1},z]yz^{-1}[y,z^{-1},x]zx^{-1}[z,x^{-1},y]x=h_1h_2^{-1}h_2h_3^{-1}h_3h_1^{-1}=1.
\end{equation*}
\end{proof}

\begin{thm}[Three subgroup lemma]
Let $N$ be a normal subgroup of $G$. Furthermore, let $X$, $Y$ and $Z$
be subgroups of $G$, such that $[X,Y,Z]$ and $[Y,Z,X]$ are contained
in $N$. Then $[Z,X,Y]$ is contained in $N$ as well.
\end{thm}
\begin{proof}
The group $[Z,X,Y]$ is generated by all elements of the form
$[z,x^{-1},y]$ with $x\in X$, $y\in Y$ and $z\in Z$. Since $N$ is
normal, $y^{-1}[x,y^{-1},z]y$ and $x^{-1}[z,x^{-1},y]x$ are elements
of $N$. The Hall--Witt identity then implies that
$x^{-1}[z,x^{-1},y]x$ is an element of $N$ as well. Again, since $N$
is normal, $[z,x^{-1},y]\in N$ which concludes the proof.
\end{proof}

\begin{thm} For any $x, y, z \in G$ we have
\begin{eqnarray*} [xy, z] & = & [x,z]^y [y,z] \\
 {[}x,yz] & = & [x,z][x,y]^z \\
 {[}x,y]^z & = & [x^z, y^z] \\
 {[}x^z, y] & = & [x, y^{z^{-1}}] 
\end{eqnarray*}
where $a^b$ denotes $b^{-1} a b$ \end{thm}
\begin{proof}
By expanding:
\begin{eqnarray*} [xy,z] & = & y^{-1}x^{-1} z^{-1} xyz \\
& = & y^{-1} x^{-1} z^{-1} \cdot xz \cdot z^{-1} x^{-1} \cdot xyz \\
&=& y^{-1} [x,z] \cdot y \cdot y^{-1} \cdot z^{-1} x^{-1} \cdot xyz \\
&=& [x,z]^y \cdot y^{-1} z^{-1} y z \\
&=& [x,z]^y [y,z]
\end{eqnarray*}
The other identities are proved similarly.
\end{proof}
%%%%%
%%%%%
\end{document}
