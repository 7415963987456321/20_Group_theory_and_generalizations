\documentclass[12pt]{article}
\usepackage{pmmeta}
\pmcanonicalname{TheKernelOfAGroupHomomorphismIsANormalSubgroup}
\pmcreated{2013-03-22 17:20:34}
\pmmodified{2013-03-22 17:20:34}
\pmowner{alozano}{2414}
\pmmodifier{alozano}{2414}
\pmtitle{the kernel of a group homomorphism is a normal subgroup}
\pmrecord{8}{39698}
\pmprivacy{1}
\pmauthor{alozano}{2414}
\pmtype{Theorem}
\pmcomment{trigger rebuild}
\pmclassification{msc}{20A05}
\pmrelated{KernelOfAGroupHomomorphism}
\pmrelated{NaturalProjection}

% this is the default PlanetMath preamble.  as your knowledge
% of TeX increases, you will probably want to edit this, but
% it should be fine as is for beginners.

% almost certainly you want these
\usepackage{amssymb}
\usepackage{amsmath}
\usepackage{amsthm}
\usepackage{amsfonts}

% used for TeXing text within eps files
%\usepackage{psfrag}
% need this for including graphics (\includegraphics)
%\usepackage{graphicx}
% for neatly defining theorems and propositions
%\usepackage{amsthm}
% making logically defined graphics
%%%\usepackage{xypic}

% there are many more packages, add them here as you need them

% define commands here

\newtheorem{thm}{Theorem}
\newtheorem{defn}{Definition}
\newtheorem{prop}{Proposition}
\newtheorem{lemma}{Lemma}
\newtheorem{cor}{Corollary}

\theoremstyle{definition}
\newtheorem*{exa}{Example}

% Some sets
\newcommand{\Nats}{\mathbb{N}}
\newcommand{\Ints}{\mathbb{Z}}
\newcommand{\Reals}{\mathbb{R}}
\newcommand{\Complex}{\mathbb{C}}
\newcommand{\Rats}{\mathbb{Q}}
\newcommand{\Gal}{\operatorname{Gal}}
\newcommand{\Cl}{\operatorname{Cl}}
\newcommand{\SL}{\operatorname{SL}}
\newcommand{\GL}{\operatorname{GL}}
\newcommand{\Ker}{\operatorname{Ker}}
\begin{document}
\PMlinkescapeword{simple} In this entry we show the following simple lemma:

\begin{lemma}
Let $G$ and $H$ be groups (with group operations $\ast_G$, $\ast_H$ and identity elements $e_G$ and $e_H$, respectively) and let $\Phi:G\to H$ be a group homomorphism. Then, the kernel of $\Phi$, i.e.
$$\Ker(\Phi)=\{ g\in G : \Phi(g)=e_H\},$$
is a normal subgroup of $G$.
\end{lemma}
\begin{proof}
Let $G,H$ and $\Phi$ be as in the statement of the lemma and let $g\in G$ and $k\in \Ker(\Phi)$. Then, $\Phi(k)=e_H$ by definition and:
\begin{eqnarray*}
\Phi(g \, \ast_G \, k \, \ast_G \, g^{-1}) &=& \Phi(g)\ast_H \Phi(k) \ast_H \Phi(g^{-1})\\
& = & \Phi(g)\ast_H ( e_H ) \ast_H \Phi(g^{-1})\\
& = & \Phi(g) \ast_H \Phi(g^{-1}) \\
& = & \Phi(g) \ast_H \Phi(g)^{-1}\\
& = & e_H,
\end{eqnarray*}
where we have used several times the properties of group homomorphisms and the properties of the identity element $e_H$. Thus, $\Phi(gkg^{-1})=e_H$ and $gkg^{-1}\in G$ is also an element of the kernel of $\Phi$. Since $g\in G$ and $k\in \Ker(\Phi)$ were arbitrary, it follows that $\Ker(\Phi)$ is normal in $G$.
\end{proof}

Conversely:

\begin{lemma}
Let $G$ be a group and let $K$ be a normal subgroup of $G$. Then there exists a group homomorphism $\Phi:G\to H$, for some group $H$, such that the kernel of $\Phi$ is precisely $K$.
\end{lemma}
\begin{proof}
Simply set $H$ equal to the quotient group $G/K$ and define $\Phi:G \to G/K$ to be the natural projection from $G$ to $G/K$ (i.e. $\Phi$ sends $g\in G$ to the coset $gK$). Then it is clear that the kernel of $\Phi$ is precisely formed by those elements of $K$.
\end{proof}

Although the first lemma is very simple, it is very useful when one tries to prove that a subgroup is normal.

\begin{exa}
Let $F$ be a field. Let us prove that the special linear group $\SL(n,F)$ is normal inside the general linear group $\GL(n,F)$, for all $n\geq 1$. By the lemmas, it suffices to construct a homomorphism of $\GL(n,F)$ with $\SL(n,F)$ as kernel. The determinant of matrices is the homomorphism we are looking for. Indeed:
$$\det : \GL(n,F) \to F^\times$$
is a group homomorphism from $\GL(n,F)$ to the multiplicative group $F^\times$ and, by definition, the kernel is precisely $\SL(n,F)$, i.e. the matrices with determinant $=1$. Hence, $\SL(n,F)$ is normal in $\GL(n,F)$.
\end{exa}
%%%%%
%%%%%
\end{document}
