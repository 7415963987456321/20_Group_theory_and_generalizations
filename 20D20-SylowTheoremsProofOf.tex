\documentclass[12pt]{article}
\usepackage{pmmeta}
\pmcanonicalname{SylowTheoremsProofOf}
\pmcreated{2013-03-22 12:51:02}
\pmmodified{2013-03-22 12:51:02}
\pmowner{Henry}{455}
\pmmodifier{Henry}{455}
\pmtitle{Sylow theorems, proof of}
\pmrecord{12}{33182}
\pmprivacy{1}
\pmauthor{Henry}{455}
\pmtype{Proof}
\pmcomment{trigger rebuild}
\pmclassification{msc}{20D20}
\pmrelated{SylowPSubgroup}
\pmrelated{SylowsThirdTheorem}

% this is the default PlanetMath preamble.  as your knowledge
% of TeX increases, you will probably want to edit this, but
% it should be fine as is for beginners.

% almost certainly you want these
\usepackage{amssymb}
\usepackage{amsmath}
\usepackage{amsfonts}

% used for TeXing text within eps files
%\usepackage{psfrag}
% need this for including graphics (\includegraphics)
%\usepackage{graphicx}
% for neatly defining theorems and propositions
\usepackage{amsthm}
% making logically defined graphics
%%%\usepackage{xypic}

% there are many more packages, add them here as you need them

% define commands here
\newtheorem{proposition}{Proposition}
\begin{document}
\PMlinkescapeword{class}
We let $G$ be a group of order $p^mk$ where $p\nmid k$ and prove Sylow's theorems.

First, a fact which will be used several times in the proof:

\begin{proposition}
If $p$ divides the size of every conjugacy class outside the center then $p$ divides the order of the center.
\end{proposition}

\begin{proof} This follows from the class equation:
\begin{displaymath}
|G|=|Z(G)|+\sum_{[a]\neq Z(G)} |[a]|
\end{displaymath}
If $p$ divides the left hand side, and divides all but one entry on the right hand side, it must divide every entry on the right side of the equation, so $p|Z(G)$.
\end{proof}

\begin{proposition}
$G$ has a Sylow p-subgroup
\end{proposition}

\begin{proof} By induction on $|G|$.  If $|G|=1$ then there is no $p$ which divides its order, so the condition is trivial.

Suppose $|G|=p^mk$, $p\nmid k$, and the \PMlinkescapetext{proposition} holds for all groups of smaller order.  Then we can consider whether $p$ divides the order of the center, $Z(G)$.

If it does, then by Cauchy's theorem, there is an element $f$ of $Z(G)$ of order $p$, and therefore a cyclic subgroup generated by $f$, $\langle f\rangle$, also of order $p$.  Since this is a subgroup of the center, it is normal, so $G/\langle f\rangle$ is well-defined and of order $p^{m-1}k$.  By the inductive hypothesis, this group has a subgroup $P/\langle f\rangle$ of order $p^{m-1}$.  Then there is a corresponding subgroup $P$ of $G$ which has $|P|=|P/\langle f\rangle|\cdot|\langle f\rangle|=p^m$.


On the other hand, if $p\nmid |Z(G)|$ then consider the conjugacy classes not in the center.  By the proposition above, since $Z(G)$ is not divisible by $p$, at least one conjugacy class can't be.  If $a$ is a representative of this class then we have $p\nmid |[a]|=[G:C(a)]$, and since $|C(a)|\cdot[G:C(a)]=|G|$, $p^m\mid |C(a)|$.  But $C(a)\neq G$, since $a\notin Z(G)$, so $C(a)$ has a subgroup of order $p^m$, and this is also a subgroup of $G$.
\end{proof}

\begin{proposition}
The intersection of a Sylow p-subgroup with the normalizer of a Sylow p-subgroup is the intersection of the subgroups.  That is, $Q\cap N_G(P)=Q\cap P$.
\end{proposition}
\begin{proof} If $P$ and $Q$ are Sylow p-subgroups, consider $R=Q\cap N_G(P)$.  Obviously $Q\cap P\subseteq R$.  In addition, since $R\subseteq N_G(P)$, the second isomorphism theorem tells us that $RP$ is a group, and $|RP|=\frac{|R|\cdot|P|}{|R\cap P|}$.  $P$ is a subgroup of $RP$, so $p^m\mid |RP|$.  But $R$ is a subgroup of $Q$ and $P$ is a Sylow p-subgroup, so $|R|\cdot |P|$ is a multiple of $p$.  Then it must be that $|RP|=p^m$, and therefore $P=RP$, and so $R\subseteq P$.  Obviously $R\subseteq Q$, so $R\subseteq Q\cap P$.
\end{proof}

The following construction will be used in the remainder of the proof:

Given any Sylow p-subgroup $P$, consider the set of its conjugates $C$.  Then 
$X\in C\leftrightarrow X=xPx^{-1}=\{xpx^{-1}|\forall p\in P\}$ for some $x\in 
G$.  Observe that every $X\in C$ is a Sylow p-subgroup (and we will show that 
the converse holds as well).  We let $G$ act on $C$ by conjugation:
\begin{displaymath}
g\cdot X=g\cdot xPx^{-1}=gxPx^{-1}g^{-1}=(gx)P(gx)^{-1}
\end{displaymath}

This is clearly a group action, so we can consider the orbits of $P$ under it; this remains true if we only consider elements from some subset of $G$.  Of course, if all $G$ is used then there is only one orbit, so we restrict the action to a Sylow p-subgroup $Q$.  \PMlinkescapetext{Name} the orbits $O_1,\ldots, O_s$, and let $P_1,\ldots, P_s$ be representatives of the corresponding orbits.  By the orbit-stabilizer theorem, the size of an orbit is the index of the stabilizer, and under this action the stabilizer of any $P_i$ is just $N_Q(P_i)=Q\cap N_G(P_i)=Q\cap P$, so $|O_i|=[Q:Q\cap P_i]$.

There are two easy results on this construction.  If $Q=P_i$ then $|O_i|=[P_i:P_i\cap P_i]=1$.  If $Q\neq P_i$ then $[Q:Q\cap P_i]>1$, and since the index of any subgroup of $Q$ divides $Q$, $p\mid |O_i|$.

\begin{proposition}
The number of conjugates of any Sylow p-subgroup of $G$ is congruent to $1$ modulo $p$
\end{proposition}
In the construction above, let $Q=P_1$.  Then $|O_1|=1$ and $p\mid |O_i|$ for $i\neq 1$.  Since the number of conjugates of $P$ is the sum of the number in each orbit, the number of conjugates is of the form $1+k_2p+k_3p+\cdots+k_sp$, which is obviously congruent to $1$ modulo $p$.

\begin{proposition}
Any two Sylow p-subgroups are conjugate
\end{proposition}

\begin{proof} Given a Sylow p-subgroup $P$ and any other Sylow p-subgroup $Q$, consider again the construction given above.  If $Q$ is not conjugate to $P$ then $Q\neq P_i$ for every $i$, and therefore $p\mid |O_i|$ for every orbit.  But then the number of conjugates of $P$ is divisible by $p$, contradicting the previous result.  Therefore $Q$ must be conjugate to $P$.
\end{proof}

\begin{proposition}
The number of subgroups of $G$ of order $p^m$ is congruent to $1$ modulo $p$ and is a factor of $k$
\end{proposition}

\begin{proof} Since conjugates of a Sylow p-subgroup are precisely the Sylow p-subgroups, and since a Sylow p-subgroup has $1$ modulo $p$ conjugates, there are $1$ modulo $p$ Sylow p-subgroups.

Since the number of conjugates is the index of the normalizer, it must be $|G:N_G(P)|$.  Since $P$ is a subgroup of its normalizer, $p^m\mid N_G(P)$, and therefore $|G:N_G(P)|\mid k$.
\end{proof}
%%%%%
%%%%%
\end{document}
