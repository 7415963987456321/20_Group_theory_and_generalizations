\documentclass[12pt]{article}
\usepackage{pmmeta}
\pmcanonicalname{ProjectPlanetMathOutlinesSeries}
\pmcreated{2013-06-03 10:17:21}
\pmmodified{2013-06-03 10:17:21}
\pmowner{unlord}{1}
\pmmodifier{unlord}{1}
\pmtitle{Project: PlanetMath Outlines Series}
\pmrecord{4}{87429}
\pmprivacy{1}
\pmauthor{unlord}{1}
\pmtype{Topic}
%\pmkeywords{Schuam's Outlines}
\pmdefines{PlanetMath Outline Series}

% this is the default PlanetMath preamble.  as your knowledge
% of TeX increases, you will probably want to edit this, but
% it should be fine as is for beginners.

% almost certainly you want these
\usepackage{amssymb}
\usepackage{amsmath}
\usepackage{amsfonts}

% need this for including graphics (\includegraphics)
\usepackage{graphicx}
% for neatly defining theorems and propositions
\usepackage{amsthm}

% making logically defined graphics
%\usepackage{xypic}
% used for TeXing text within eps files
%\usepackage{psfrag}

% there are many more packages, add them here as you need them

% define commands here

\begin{document}
We already have some ``topic articles'' which could be a good start, but long term it would be useful to add a \emph{lot} of content.  We can take the Schuam's Outline series as a model.  Replicating this material and giving it away for free would take a lot of work.

Assuming all the books are of roughly the same size, we would want to have something like 6000 expository chapters, 60000 problems, and 20000 solutions.  Presumably we have a good start on the expository texts with the current PlanetMath encyclopedia, but in order to finish the outline in a reasonable timeframe, we'll need at least 10000 new problems per year.  That sounds like a lot of work, but luckily we have an ``ace in the hole'': the PlanetMath Books project.  In short: if we find, scan, and OCR all of the \emph{public domain} mathematics books in existence, we will have tons and tons of text to work with.  Some further details are described in this \href{https://github.com/holtzermann17/planetmath-docs/issues/37}{issue} in our organizational issue tracker.

If we get a big influx of new content, we're going to have to figure out how to upload it without disturbing users.  At the same time, we will presumably want to use the existing encyclopedia as the ``framework'' within which to organize everything.

Accordingly, here is a list of outlines (copied-and-modified from the Schuam's website) that we might eventually want to provide.  As time goes by I will try to link them to new ``collections'' that walk people through the relevant content on PlanetMath.  Note that this doesn't represent all of the outlines ever published in the Schuam's series -- and we might not get to all of these.  We might also come up with our own outlines!  Accordingly, I'm making this article world editable.  Please feel free to add to the list, link to new collections, and discuss other ways to build PlanetMath's resources!

\begin{enumerate}
% \item PlanetMath Easy Outline of Calculus
% \item PlanetMath Easy Outline of College Algebra
% \item PlanetMath Easy Outline of College Mathematics
% \item PlanetMath Easy Outline of Elementary Algebra
% \item PlanetMath Easy Outline of Geometry
% \item PlanetMath Easy Outline of Mathematical Handbook of Formulas and Tables
% \item PlanetMath Easy Outline of Probability and Statistics
% \item PlanetMath Easy Outline of Programming with Java
% \item PlanetMath Easy Outline of Statistics
\item PlanetMath Outline of Abstract Algebra
\item PlanetMath Outline of Advanced Calculus
\item PlanetMath Outline of Advanced Mathematics for Engineers and Scientists
\item PlanetMath Outline of Astronomy
\item PlanetMath Outline of Basic Business Mathematics
\item PlanetMath Outline of Basic Electricity
\item PlanetMath Outline of Basic Mathematics for Electricity and Electronics
\item PlanetMath Outline of Basic Mathematics with Applications to Science and Technology
\item PlanetMath Outline of Beginning Calculus
\item PlanetMath Outline of Beginning Finite Mathematics
\item PlanetMath Outline of Beginning Statistics
\item PlanetMath Outline of Bookkeeping and Accounting
\item PlanetMath Outline of Calculus
\item PlanetMath Outline of College Algebra
\item PlanetMath Outline of College Mathematics
%\item PlanetMath Outline of Computer Architecture
%\item PlanetMath Outline of Computer Networking
%\item PlanetMath Outline of Data Structures with Java
\item PlanetMath Outline of Differential Equations
\item PlanetMath Outline of Discrete Mathematics
\item PlanetMath Outline of Electronic Devices and Circuits
\item PlanetMath Outline of Elements of Statistics I: Descriptive Statistics and Probability
\item PlanetMath Outline of Essential Computer Mathematics
%\item PlanetMath Outline of Financial Management
%\item PlanetMath Outline of Fundamentals of Relational Databases
\item PlanetMath Outline of Geometry
%\item PlanetMath Outline of HTML
%\item PlanetMath Outline of Immunology
\item PlanetMath Outline of Intermediate Algebra
\item PlanetMath Outline of Introduction to Probability and Statistics
%\item PlanetMath Outline of Introduction to Psychology
\item PlanetMath Outline of Introductory Surveying
\item PlanetMath Outline of Lagrangian Dynamics
\item PlanetMath Outline of Logic
\item PlanetMath Outline of (Mathematical Handbook of) Formulae and Tables
\item PlanetMath Outline of Mathematics for Liberal Arts Majors
\item PlanetMath Outline of Mathematics for Nurses
\item PlanetMath Outline of Mathematics for Physics Students
\item PlanetMath Outline of Mathematics of Finance
\item PlanetMath Outline of Matrix Operations
\item PlanetMath Outline of Operating Systems
\item PlanetMath Outline of Partial Differential Equations
\item PlanetMath Outline of Physics for Engineering and Science
\item PlanetMath Outline of Precalculus
\item PlanetMath Outline of Principles of Accounting I
\item PlanetMath Outline of Probability and Statistics
\item PlanetMath Outline of Probability
%\item PlanetMath Outline of Programming with C++
%\item PlanetMath Outline of Programming with Java
\item PlanetMath Outline of Review of Elementary Mathematics
%\item PlanetMath Outline of Software Engineering
\item PlanetMath Outline of Statistics
\item PlanetMath Outline of Statistics for Engineers
\item PlanetMath Outline of Thermodynamics With Chemical Applications
%\item PlanetMath Outline of Visual Basic
%\item PlanetMath Outline of XML
\item PlanetMath Outline of Mathematical Analysis
\end{enumerate}
\end{document}
