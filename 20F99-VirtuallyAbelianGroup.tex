\documentclass[12pt]{article}
\usepackage{pmmeta}
\pmcanonicalname{VirtuallyAbelianGroup}
\pmcreated{2013-03-22 14:35:58}
\pmmodified{2013-03-22 14:35:58}
\pmowner{yark}{2760}
\pmmodifier{yark}{2760}
\pmtitle{virtually abelian group}
\pmrecord{11}{36168}
\pmprivacy{1}
\pmauthor{yark}{2760}
\pmtype{Definition}
\pmcomment{trigger rebuild}
\pmclassification{msc}{20F99}
\pmclassification{msc}{20E99}
\pmsynonym{abelian-by-finite group}{VirtuallyAbelianGroup}
\pmsynonym{virtually-abelian group}{VirtuallyAbelianGroup}
\pmrelated{VirtuallyCyclicGroup}
\pmdefines{virtually abelian}
\pmdefines{abelian-by-finite}
\pmdefines{virtually nilpotent}
\pmdefines{virtually solvable}
\pmdefines{virtually polycyclic}
\pmdefines{virtually free}
\pmdefines{nilpotent-by-finite}
\pmdefines{polycyclic-by-finite}
\pmdefines{virtually nilpotent group}
\pmdefines{virtually solvable group}
\pmdefines{virtually polycyclic group}
\pmdefines{virtually free}

\endmetadata


\begin{document}
\PMlinkescapeword{index}
\PMlinkescapeword{property}
\PMlinkescapeword{subgroup}
\PMlinkescapeword{subgroups}

A group $G$ is \emph{virtually abelian} (or \emph{abelian-by-finite})
if it has an abelian \PMlinkname{subgroup}{Subgroup} of finite \PMlinkname{index}{Coset}.

More generally, let $\chi$ be a property of groups.
A group $G$ is \emph{virtually $\chi$} if it has a subgroup of finite index with the property $\chi$.
A group $G$ is \emph{$\chi$-by-finite} if it has a normal subgroup of finite index with the property $\chi$.
Note that every $\chi$-by-finite group is virtually $\chi$,
and the converse also holds if the property $\chi$ is inherited by subgroups.

These notions are obviously only of relevance to infinite groups, as all finite groups are virtually trivial (and trivial-by-finite).
%%%%%
%%%%%
\end{document}
