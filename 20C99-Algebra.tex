\documentclass[12pt]{article}
\usepackage{pmmeta}
\pmcanonicalname{Algebra}
\pmcreated{2013-03-22 11:48:37}
\pmmodified{2013-03-22 11:48:37}
\pmowner{djao}{24}
\pmmodifier{djao}{24}
\pmtitle{algebra}
\pmrecord{17}{30353}
\pmprivacy{1}
\pmauthor{djao}{24}
\pmtype{Definition}
\pmcomment{trigger rebuild}
\pmclassification{msc}{20C99}
\pmclassification{msc}{16S99}
\pmclassification{msc}{13B02}
\pmdefines{subalgebra}

\usepackage{amssymb}
\usepackage{amsmath}
\usepackage{amsfonts}
\usepackage{graphicx}
%%%%%\usepackage{xypic}
\begin{document}
In this definition, all rings are assumed to be rings with identity and all ring homomorphisms are assumed to be unital.

Let $R$ be a ring. An \emph{algebra} over $R$ is a ring $A$ together with a ring homomorphism $f\colon R \to Z(A)$, where $Z(A)$ denotes the center of $A$. A \emph{subalgebra} of $A$ is a subset of $A$ which is an algebra.

Equivalently, an algebra over a ring $R$ is an $R$--module $A$ which is a ring and satisfies the property
$$r\cdot(x*y) = (r\cdot x)*y = x*(r\cdot y)$$
for all $r \in R$ and all $x,y \in A$. Here $\cdot$ denotes $R$-module multiplication and $*$ denotes ring multiplication in $A$. One passes between the two definitions as follows: given any ring homomorphism $f\colon R \longrightarrow Z(A)$, the scalar multiplication rule
$$
r \cdot b := f(r)*b
$$
makes $A$ into an $R$-module in the sense of the second definition. Conversely, if $A$ satisfies the requirements of the second definition, then the function $f\colon R \to A$ defined by $f(r) := r \cdot 1$ is a ring homomorphism from $R$ into $Z(A)$.
%%%%%
%%%%%
%%%%%
%%%%%
%%%%%
\end{document}
