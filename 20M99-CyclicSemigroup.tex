\documentclass[12pt]{article}
\usepackage{pmmeta}
\pmcanonicalname{CyclicSemigroup}
\pmcreated{2013-03-22 13:07:30}
\pmmodified{2013-03-22 13:07:30}
\pmowner{mclase}{549}
\pmmodifier{mclase}{549}
\pmtitle{cyclic semigroup}
\pmrecord{6}{33559}
\pmprivacy{1}
\pmauthor{mclase}{549}
\pmtype{Definition}
\pmcomment{trigger rebuild}
\pmclassification{msc}{20M99}
\pmdefines{index}
\pmdefines{period}

\endmetadata

% this is the default PlanetMath preamble.  as your knowledge
% of TeX increases, you will probably want to edit this, but
% it should be fine as is for beginners.

% almost certainly you want these
\usepackage{amssymb}
\usepackage{amsmath}
\usepackage{amsfonts}

% used for TeXing text within eps files
%\usepackage{psfrag}
% need this for including graphics (\includegraphics)
%\usepackage{graphicx}
% for neatly defining theorems and propositions
%\usepackage{amsthm}
% making logically defined graphics
%%%\usepackage{xypic}

% there are many more packages, add them here as you need them

% define commands here
\begin{document}
A semigroup which is generated by a single element is called a \emph{cyclic semigroup}.

Let $S = \langle x \rangle$ be a cyclic semigroup.  Then as a set, $S = \{ x^n \mid n > 0\}$.

If all powers of $x$ are distinct, then $S = \{x, x^2, x^3, \dotsc \}$ is (countably) infinite.

Otherwise, there is a least integer $n > 0$ such that $x^n = x^m$ for some $m < n$.  It is clear then that the elements $x, x^2, \dots, x^{n-1}$ are distinct, but that for any $j \ge n$, we must have $x^j = x^i$ for some $i$, $m \le i \le n-1$.  So $S$ has $n-1$ elements.

Unlike in the group case, however, there are in general multiple non-isomorphic cyclic semigroups with the same number of elements.  In fact, there are $t$ non-isomorphic cyclic semigroups with $t$ elements:  these correspond to the different choices of $m$ in the above (with $n = t+1$).

The integer $m$ is called the \emph{index} of $S$, and $n-m$ is called the \emph{period} of $S$.

The elements $K = \{x^m, x^{m+1}, \dots, x^{n-1}\}$ are a subsemigroup of $S$.  In fact, $K$ is a cyclic group.

A concrete representation of the semigroup with index $m$ and period $r$ as a semigroup of transformations can be obtained as follows.  Let $X = \{1, 2, 3, \dots, m + r\}$.  Let
$$
\phi =
\begin{pmatrix}
1 & 2 & 3 & \dots & m+r-1 & m+r \\
2 & 3 & 4 & \dots & m+r & r+1
\end{pmatrix}.
$$

Then $\phi$ generates a subsemigroup $S$ of the full semigroup of transformations $\mathcal{T}_X$, and $S$ is cyclic with index $m$ and period $r$.
%%%%%
%%%%%
\end{document}
