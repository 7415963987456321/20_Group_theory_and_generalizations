\documentclass[12pt]{article}
\usepackage{pmmeta}
\pmcanonicalname{LawOfSignsUnderMultiplicationInARing}
\pmcreated{2013-03-22 14:14:03}
\pmmodified{2013-03-22 14:14:03}
\pmowner{alozano}{2414}
\pmmodifier{alozano}{2414}
\pmtitle{law of signs under multiplication in a ring}
\pmrecord{10}{35675}
\pmprivacy{1}
\pmauthor{alozano}{2414}
\pmtype{Derivation}
\pmcomment{trigger rebuild}
\pmclassification{msc}{20-00}
\pmclassification{msc}{16-00}
\pmclassification{msc}{13-00}
\pmsynonym{$(-x)\cdot (-y)= x\cdot y$}{LawOfSignsUnderMultiplicationInARing}
\pmrelated{Ring}

% this is the default PlanetMath preamble.  as your knowledge
% of TeX increases, you will probably want to edit this, but
% it should be fine as is for beginners.

% almost certainly you want these
\usepackage{amssymb}
\usepackage{amsmath}
\usepackage{amsthm}
\usepackage{amsfonts}

% used for TeXing text within eps files
%\usepackage{psfrag}
% need this for including graphics (\includegraphics)
%\usepackage{graphicx}
% for neatly defining theorems and propositions
%\usepackage{amsthm}
% making logically defined graphics
%%%\usepackage{xypic}

% there are many more packages, add them here as you need them

% define commands here

\newtheorem{thm}{Theorem}
\newtheorem{defn}{Definition}
\newtheorem{prop}{Proposition}
\newtheorem{lemma}{Lemma}
\newtheorem{cor}{Corollary}

% Some sets
\newcommand{\Nats}{\mathbb{N}}
\newcommand{\Ints}{\mathbb{Z}}
\newcommand{\Reals}{\mathbb{R}}
\newcommand{\Complex}{\mathbb{C}}
\newcommand{\Rats}{\mathbb{Q}}
\begin{document}
\begin{lemma}
Let $R$ be a ring with unity, which we denote by $1$. For all $x,y\in R$:
$$(-x)\cdot (-y)=x\cdot y$$
where $-x$ denotes the additive inverse of $x$ in $R$.
\end{lemma}
\begin{proof}
Here we use the fact $(-1)\cdot a = -a$ for all $a \in R$. First, we see that:

$$(-1)\cdot (-1)\cdot a=(-1)\cdot \left( (-1)\cdot a \right)=(-1)\cdot (-a)=a$$
since, clearly, the additive inverse of $-a$ is $a$ itself.

Hence:
$$(-x)\cdot (-y)=(-1)\cdot x \cdot (-1)\cdot y= (-1)\cdot (-1) \cdot x\cdot y= x \cdot y$$
where we have used several times the associativity of $\cdot$ and the fact that $(-1)\cdot x = x \cdot (-1) = -x$.
\end{proof}
%%%%%
%%%%%
\end{document}
