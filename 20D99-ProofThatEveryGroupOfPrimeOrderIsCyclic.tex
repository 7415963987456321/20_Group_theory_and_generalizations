\documentclass[12pt]{article}
\usepackage{pmmeta}
\pmcanonicalname{ProofThatEveryGroupOfPrimeOrderIsCyclic}
\pmcreated{2013-03-22 13:30:55}
\pmmodified{2013-03-22 13:30:55}
\pmowner{Wkbj79}{1863}
\pmmodifier{Wkbj79}{1863}
\pmtitle{proof that every group of prime order is cyclic}
\pmrecord{7}{34101}
\pmprivacy{1}
\pmauthor{Wkbj79}{1863}
\pmtype{Proof}
\pmcomment{trigger rebuild}
\pmclassification{msc}{20D99}
\pmrelated{ProofThatGInGImpliesThatLangleGRangleLeG}

\endmetadata

% this is the default PlanetMath preamble.  as your knowledge
% of TeX increases, you will probably want to edit this, but
% it should be fine as is for beginners.

% almost certainly you want these
\usepackage{amssymb}
\usepackage{amsmath}
\usepackage{amsfonts}

% used for TeXing text within eps files
%\usepackage{psfrag}
% need this for including graphics (\includegraphics)
%\usepackage{graphicx}
% for neatly defining theorems and propositions
%\usepackage{amsthm}
% making logically defined graphics
%%%\usepackage{xypic}

% there are many more packages, add them here as you need them

% define commands here
\begin{document}
The following is a proof that every group of prime order is cyclic.

Let $p$ be a prime and $G$ be a group such that $|G|=p$.  Then $G$ contains more than one element.  Let $g \in G$ such that $g \ne e_G$.  Then $\langle g \rangle$ contains more than one element.  Since $\langle g \rangle \le G$, by Lagrange's theorem, $|\langle g \rangle |$ divides $p$.  Since $|\langle g \rangle |>1$ and $|\langle g \rangle |$ divides a prime, $|\langle g \rangle |=p=|G|$.  Hence, $\langle g \rangle =G$.  It follows that $G$ is cyclic.
%%%%%
%%%%%
\end{document}
