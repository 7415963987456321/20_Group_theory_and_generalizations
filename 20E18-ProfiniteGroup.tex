\documentclass[12pt]{article}
\usepackage{pmmeta}
\pmcanonicalname{ProfiniteGroup}
\pmcreated{2013-03-22 12:48:50}
\pmmodified{2013-03-22 12:48:50}
\pmowner{djao}{24}
\pmmodifier{djao}{24}
\pmtitle{profinite group}
\pmrecord{9}{33134}
\pmprivacy{1}
\pmauthor{djao}{24}
\pmtype{Definition}
\pmcomment{trigger rebuild}
\pmclassification{msc}{20E18}
\pmclassification{msc}{22C05}
\pmsynonym{profinite}{ProfiniteGroup}
\pmrelated{InverseLimit}
\pmdefines{profinite topology}

% this is the default PlanetMath preamble.  as your knowledge
% of TeX increases, you will probably want to edit this, but
% it should be fine as is for beginners.

% almost certainly you want these
\usepackage{amssymb}
\usepackage{amsmath}
\usepackage{amsfonts}

% used for TeXing text within eps files
%\usepackage{psfrag}
% need this for including graphics (\includegraphics)
%\usepackage{graphicx}
% for neatly defining theorems and propositions
%\usepackage{amsthm}
% making logically defined graphics
%%%\usepackage{xypic} 

% there are many more packages, add them here as you need them

% define commands here
\newcommand{\ilim}{\,\underset{\longleftarrow}{\lim}\,}
\begin{document}
\section{Definition}
A topological group $G$ is \emph{profinite} if it is isomorphic to the inverse limit of some projective system of finite groups. In other words, $G$ is profinite if there exists a directed set $I$, a collection of finite groups $\{H_i\}_{i \in I}$, and homomorphisms $\alpha_{ij}\colon H_j \to H_i$ for each pair $i,j \in I$ with $i \leq j$, satisfying
\begin{enumerate}
\item $\alpha_{ii} = 1$ for all $i \in I$,
\item $\alpha_{ij} \circ \alpha_{jk} = \alpha_{ik}$ for all $i,j,k \in I$ with $i \leq j \leq k$,
\end{enumerate}
with the property that:
\begin{itemize}
\item $G$ is isomorphic as a group to the projective limit
$$
\ilim H_i := \left\{\left.(h_i) \in \prod_{i \in I} H_i\ \right|\ \alpha_{ij}(h_j) = h_i\ \text{ for all }\ i \leq j\right\}
$$
under componentwise multiplication.
\item The isomorphism from $G$ to $\ilim H_i$ (considered as a subspace of $\prod H_i$) is a homeomorphism of topological spaces, where each $H_i$ is given the discrete topology and $\prod H_i$ is given the product topology.
\end{itemize}
The topology on a profinite group is called the {\em profinite topology}.
\section{Properties}
One can show that a topological group is profinite if and only if it is compact and totally disconnected. Moreover, every profinite group is residually finite.

%%%%%
%%%%%
\end{document}
