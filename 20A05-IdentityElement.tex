\documentclass[12pt]{article}
\usepackage{pmmeta}
\pmcanonicalname{IdentityElement}
\pmcreated{2013-03-22 12:49:07}
\pmmodified{2013-03-22 12:49:07}
\pmowner{mclase}{549}
\pmmodifier{mclase}{549}
\pmtitle{identity element}
\pmrecord{9}{33140}
\pmprivacy{1}
\pmauthor{mclase}{549}
\pmtype{Definition}
\pmcomment{trigger rebuild}
\pmclassification{msc}{20A05}
\pmclassification{msc}{20N02}
\pmclassification{msc}{20N05}
\pmclassification{msc}{20M99}
\pmsynonym{neutral element}{IdentityElement}
\pmrelated{LeftIdentityAndRightIdentity}
\pmrelated{Group}

% this is the default PlanetMath preamble.  as your knowledge
% of TeX increases, you will probably want to edit this, but
% it should be fine as is for beginners.

% almost certainly you want these
\usepackage{amssymb}
\usepackage{amsmath}
\usepackage{amsfonts}

% used for TeXing text within eps files
%\usepackage{psfrag}
% need this for including graphics (\includegraphics)
%\usepackage{graphicx}
% for neatly defining theorems and propositions
%\usepackage{amsthm}
% making logically defined graphics
%%%\usepackage{xypic}

% there are many more packages, add them here as you need them

% define commands here
\begin{document}
\PMlinkescapeword{multiplicative}

Let $G$ be a groupoid, that is a set with a binary operation $G \times G \to G$, written muliplicatively so that $(x, y) \mapsto xy$.

An \emph{identity element} for $G$ is an element $e$ such that $ge = eg = g$ for all $g \in G$.

The symbol $e$ is most commonly used for identity elements.  Another  common symbol for an identity element is $1$, particularly in semigroup theory (and ring theory, considering the multiplicative structure as a semigroup).

Groups, monoids, and loops are classes of groupoids that, by definition, always have an identity element.
%%%%%
%%%%%
\end{document}
