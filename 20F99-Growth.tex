\documentclass[12pt]{article}
\usepackage{pmmeta}
\pmcanonicalname{Growth}
\pmcreated{2013-03-22 14:36:09}
\pmmodified{2013-03-22 14:36:09}
\pmowner{mathcam}{2727}
\pmmodifier{mathcam}{2727}
\pmtitle{growth}
\pmrecord{8}{36172}
\pmprivacy{1}
\pmauthor{mathcam}{2727}
\pmtype{Definition}
\pmcomment{trigger rebuild}
\pmclassification{msc}{20F99}
\pmclassification{msc}{20E99}
\pmrelated{GrowthOfExponentialFunction}
\pmdefines{polynomial growth}
\pmdefines{intermediate growth}
\pmdefines{exponential growth}

\endmetadata

% this is the default PlanetMath preamble.  as your knowledge
% of TeX increases, you will probably want to edit this, but
% it should be fine as is for beginners.

% almost certainly you want these
\usepackage{amssymb}
\usepackage{amsmath}
\usepackage{amsfonts}

% used for TeXing text within eps files
%\usepackage{psfrag}
% need this for including graphics (\includegraphics)
%\usepackage{graphicx}
% for neatly defining theorems and propositions
\usepackage{amsthm}
% making logically defined graphics
%%%\usepackage{xypic}

% there are many more packages, add them here as you need them

% define commands here
\begin{document}
Let $G$ be a finitely generated group with generating set $A$ (closed under inverses).

For $g=a_1a_2\ldots a_m\in G$, $a_i\in A$, let $l(g)$ be the minimum value of $m$.

Define $$\gamma(n)=\mid\{g\in G:l(g)\le n\}\mid$$.

The function $\gamma$ is called the \emph{growth function} for $G$ with generating set $A$.  If $\gamma$ is either

(a) bounded above by a polynomial function,

(b) bounded below by an exponential function, or

(c) neither,

then this condition is preserved under changing the generating set for $G$.  Respectively, then, $G$ is said to have

(a) \emph{polynomial growth},

(b) \emph{exponential growth}, or

(c) \emph{intermediate growth}.


For a survey on the topic, see:
R. I. Grigorchuk, On growth in group theory, Proceedings of the
International Congress of Mathematicians, Kyoto 1990, Volume I,
II (Math. Soc. Japan, 1991), pages 325 to 338.

Note that, as the generating set is assumed to be closed under inverses, we need only have $G$ as a semigroup - as such, the above applies equally well in semigroup theory.
%%%%%
%%%%%
\end{document}
