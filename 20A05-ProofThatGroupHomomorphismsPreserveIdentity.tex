\documentclass[12pt]{article}
\usepackage{pmmeta}
\pmcanonicalname{ProofThatGroupHomomorphismsPreserveIdentity}
\pmcreated{2013-11-16 4:44:43}
\pmmodified{2013-11-16 4:44:43}
\pmowner{jacou}{1000048}
\pmmodifier{}{0}
\pmtitle{proof that group homomorphisms preserve identity}
\pmrecord{9}{40667}
\pmprivacy{1}
\pmauthor{jacou}{0}
\pmtype{Proof}
\pmcomment{trigger rebuild}
\pmclassification{msc}{20A05}
\pmsynonym{1234}{ProofThatGroupHomomorphismsPreserveIdentity}
\pmrelated{1234}
\pmdefines{1234}

% this is the default PlanetMath preamble.  as your knowledge
% of TeX increases, you will probably want to edit this, but
% it should be fine as is for beginners.

% almost certainly you want these
\usepackage{amssymb}
\usepackage{amsmath}
\usepackage{amsfonts}

% used for TeXing text within eps files
%\usepackage{psfrag}
% need this for including graphics (\includegraphics)
%\usepackage{graphicx}
% for neatly defining theorems and propositions
\usepackage{amsthm}
% making logically defined graphics
%%%\usepackage{xypic}

% there are many more packages, add them here as you need them

% define commands here

\begin{document}
\newtheorem*{theorem}{Theorem}
\begin{theorem}
A group homomorphism preserves identity elements. 
\end{theorem}
\begin{proof}
Let $\phi:G\to K$ be a group homomorphism.
For clarity we use $\ast$ and $\star$ for the group operations of $G$ and $K$, respectively.  Also,
denote the identities by $1_G$ and $1_H$ respectively.  


By the definition of identity, 
\begin{equation}\label{eq:idG}
 1_G\ast 1_G=1_G.
\end{equation}
Applying the homomorphism $\phi$ to (\ref{eq:idG}) produces:
\begin{equation}\label{eq:idPhi}
 \phi(1_G)\star \phi(1_G)= \phi(1_G\ast 1_G) = \phi (1_G).
\end{equation}
Multiply both sides of (\ref{eq:idPhi}) by the inverse of $\phi(1_G)$ in $K$,
and use the associativity of $\star$ to produce:
\begin{equation}
  \phi(1_G)=(\phi(1_G))^{-1}\star \phi(1_G)\star \phi(1_G)  =  (\phi(1_G))^{-1}\star \phi(1_G)=1_K.
\end{equation}
\end{proof}


%%%%%
%%%%%
\end{document}
