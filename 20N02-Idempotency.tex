\documentclass[12pt]{article}
\usepackage{pmmeta}
\pmcanonicalname{Idempotency}
\pmcreated{2013-03-22 12:27:31}
\pmmodified{2013-03-22 12:27:31}
\pmowner{yark}{2760}
\pmmodifier{yark}{2760}
\pmtitle{idempotency}
\pmrecord{21}{32604}
\pmprivacy{1}
\pmauthor{yark}{2760}
\pmtype{Definition}
\pmcomment{trigger rebuild}
\pmclassification{msc}{20N02}
\pmrelated{BooleanRing}
\pmrelated{PeriodOfMapping}
\pmrelated{Idempotent2}
\pmdefines{idempotent}

\usepackage{amssymb}
\usepackage{amsmath}
\usepackage{amsfonts}
\begin{document}
\PMlinkescapeword{case}
\PMlinkescapeword{cases}
\PMlinkescapeword{lattice}
\PMlinkescapeword{words}

If $(S,*)$ is a magma, then an element $x\in S$ is said to be \emph{idempotent} if $x*x=x$.
For example, every identity element is idempotent, and in a group this is the only idempotent element.
An idempotent element is often just called an idempotent.

If every element of the magma $(S,*)$ is idempotent, then the binary operation $*$ (or the magma itself) is said to be idempotent. For example, the $\land$ and $\lor$ operations in a \PMlinkname{lattice}{Lattice} are idempotent, because $x\land x = x$ and $x\lor x = x$ for all $x$ in the lattice.

A function $f\colon D\to D$ is said to be idempotent if $f\circ f=f$. (This is just a special case of the first definition above, the magma in question being $(D^D,\circ)$, the monoid of all functions from $D$ to $D$ with the operation of function composition.) In other words, $f$ is idempotent if and only if repeated application of $f$ has the same effect as a single application: $f(f(x)) = f(x)$ for all $x\in D$. An idempotent linear transformation from a vector space to itself is called a  projection.
%%%%%
%%%%%
\end{document}
