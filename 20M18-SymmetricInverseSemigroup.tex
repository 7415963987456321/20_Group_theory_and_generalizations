\documentclass[12pt]{article}
\usepackage{pmmeta}
\pmcanonicalname{SymmetricInverseSemigroup}
\pmcreated{2013-03-22 16:11:14}
\pmmodified{2013-03-22 16:11:14}
\pmowner{Mazzu}{14365}
\pmmodifier{Mazzu}{14365}
\pmtitle{symmetric inverse semigroup}
\pmrecord{6}{38274}
\pmprivacy{1}
\pmauthor{Mazzu}{14365}
\pmtype{Definition}
\pmcomment{trigger rebuild}
\pmclassification{msc}{20M18}
%\pmkeywords{Inverse Semigroups}
\pmdefines{partial map}
\pmdefines{composition of partial maps}
\pmdefines{symmetric inverse semigroup}

% this is the default PlanetMath preamble.  as your knowledge
% of TeX increases, you will probably want to edit this, but
% it should be fine as is for beginners.

% almost certainly you want these
\usepackage{amssymb}
\usepackage{amsmath}
\usepackage{amsfonts}

% used for TeXing text within eps files
%\usepackage{psfrag}
% need this for including graphics (\includegraphics)
%\usepackage{graphicx}
% for neatly defining theorems and propositions
%\usepackage{amsthm}
% making logically defined graphics
%%%\usepackage{xypic}

% there are many more packages, add them here as you need them

% define commands here

\begin{document}
\newcommand{\domi}{\mathrm{dom}}
\newcommand{\rang}{\mathrm{ran}}
\newcommand{\FFF}{\mathfrak{F}}
\newcommand{\III}{\mathfrak{I}}
\newcommand{\cbra}[1]{\left( #1 \right)}
\newcommand{\qbra}[1]{\left[ #1 \right]}
\newcommand{\gbra}[1]{\left\{ #1 \right\}}
\newcommand{\abra}[1]{\left\langle #1 \right\rangle}

Let $X$ be a set. A \emph{partial map} on $X$ is an application defined from a subset of $X$ into $X$. We denote by $\FFF(X)$ the set of partial map on $X$. Given $\alpha\in\FFF(X)$, we denote by $\domi(\alpha)$ and $\rang(\alpha)$ respectively the domain and the range of $\alpha$, i.e. $$\domi(\alpha),\rang{\alpha}\subseteq X,\ \ \alpha:\domi(\alpha)\rightarrow X,\ \ \alpha(\domi(\alpha))=\rang(\alpha).$$
We define the composition of two partial map $\alpha,\beta\in\FFF(X)$ as the partial map $\alpha\circ\beta\in\FFF(X)$ with domain 
$$\domi(\alpha\circ\beta)=\beta^{-1}(\rang(\beta)\cap\domi(\alpha))=\gbra{x\in\domi(\beta)\,|\,\alpha(x)\in\domi(\beta)}$$
defined by the common rule
$$\alpha\circ\beta(x)=\alpha(\beta(x)),\ \ \forall x\in\domi{(\alpha\circ\beta)}.$$
It is easily verified that the $\FFF(X)$ with the composition $\circ$ is a semigroup. 

A partial map $\alpha\in\FFF(X)$ is said \emph{bijective} when it is bijective as a map $\alpha:\rang(\alpha)\rightarrow\domi(\alpha)$. It can be proved that the subset $\III(X)\subseteq\FFF(X)$ of the partial bijective maps on $X$ is an inverse semigroup (with the composition $\circ$), that is called \emph{symmetric inverse semigroup} on $X$. Note that the symmetric group on $X$ is a subgroup of $\III(X)$.
%%%%%
%%%%%
\end{document}
