\documentclass[12pt]{article}
\usepackage{pmmeta}
\pmcanonicalname{RegularSemigroup}
\pmcreated{2013-03-22 14:23:17}
\pmmodified{2013-03-22 14:23:17}
\pmowner{yark}{2760}
\pmmodifier{yark}{2760}
\pmtitle{regular semigroup}
\pmrecord{25}{35883}
\pmprivacy{1}
\pmauthor{yark}{2760}
\pmtype{Definition}
\pmcomment{trigger rebuild}
\pmclassification{msc}{20M17}
\pmclassification{msc}{20M18}
\pmrelated{ACharacterizationOfGroups}
\pmdefines{regular}
\pmdefines{$\pi$-regular}
\pmdefines{eventually regular}
\pmdefines{strongly $\pi$-regular}
\pmdefines{group-bound}
\pmdefines{inverse semigroup}
\pmdefines{Clifford semigroup}
\pmdefines{orthodox semigroup}
\pmdefines{completely regular}
\pmdefines{epigroup}
\pmdefines{regular element}
\pmdefines{inverse}
\pmdefines{relative inverse}

\endmetadata


\begin{document}
\PMlinkescapephrase{completely regular}
\PMlinkescapephrase{generated by}
\PMlinkescapeword{index}
\PMlinkescapeword{power}

Let $S$ be a semigroup.

$x\in S$ is \emph{regular} if there is a $y\in S$ such that $x=xyx$.\\
$y\in S$ is an \emph{inverse}
(or a \emph{relative inverse})  % Bruck, Survey of Binary Systems
for $x$ if $x=xyx$ and $y=yxy$.

\section{Regular semigroups}
$S$ is a \emph{regular semigroup} if all its elements are regular.
The phrase 'von Neumann regular' is sometimes used, after the definition for rings.

In a regular semigroup, every principal ideal is generated by an idempotent.

Every regular element has at least one inverse.
To show this, suppose $a\in S$ is regular,
so that $a = aba$ for some $b\in S$.
Put $c=bab$.
Then
\[
  a=aba=(aba)ba=a(bab)a=aca
\]
and
\[
  c=bab=b(aba)b=(bab)ab=cab=c(aba)b=ca(bab)=cac,
\]
so $c$ is an inverse of $a$.

\section{Inverse semigroups}
$S$ is an \emph{inverse semigroup} if for all $x\in S$ there is a \emph{unique} $y\in S$ such that $x=xyx$ and $y=yxy$.

In an inverse semigroup every principal ideal is generated by a \emph{unique} idempotent.

In an inverse semigroup the set of idempotents is a subsemigroup, in particular a \PMlinkname{commutative band}{ASemilatticeIsACommutativeBand}.

The bicyclic semigroup is an example of an inverse semigroup.
The symmetric inverse semigroup (on some set $X$) is another example.
Of course, every group is also an inverse semigroup.

\section{Motivation}
Both of these notions generalise the definition of a group.  In particular, a regular semigroup with one idempotent is a group: as such, many interesting subclasses of regular semigroups arise from putting conditions on the idempotents.  Apart from inverse semigroups, there are \emph{orthodox semigroups} where the set of idempotents is a subsemigroup, and \emph{Clifford semigroups} where the idempotents are central.

\section{Additional}
$S$ is called \emph{eventually regular} (or \emph{$\pi$-regular}) if a power of every element is regular.

$S$ is called \emph{group-bound} (or \emph{strongly $\pi$-regular}, or an \emph{epigroup}) if a power of every element is in a subgroup of $S$.

$S$ is called \emph{completely regular} if every element is in a subgroup of $S$.
%%%%%
%%%%%
\end{document}
