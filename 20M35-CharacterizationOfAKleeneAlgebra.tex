\documentclass[12pt]{article}
\usepackage{pmmeta}
\pmcanonicalname{CharacterizationOfAKleeneAlgebra}
\pmcreated{2013-03-22 17:02:40}
\pmmodified{2013-03-22 17:02:40}
\pmowner{CWoo}{3771}
\pmmodifier{CWoo}{3771}
\pmtitle{characterization of a Kleene algebra}
\pmrecord{12}{39333}
\pmprivacy{1}
\pmauthor{CWoo}{3771}
\pmtype{Definition}
\pmcomment{trigger rebuild}
\pmclassification{msc}{20M35}
\pmclassification{msc}{68Q70}
\pmdefines{$^*$-continuous}

\usepackage{amssymb,amscd}
\usepackage{amsmath}
\usepackage{amsfonts}
\usepackage{mathrsfs}
\usepackage{multicol}
\setlength\columnsep{40pt}

% used for TeXing text within eps files
%\usepackage{psfrag}
% need this for including graphics (\includegraphics)
%\usepackage{graphicx}
% for neatly defining theorems and propositions
\usepackage{amsthm}
% making logically defined graphics
%%\usepackage{xypic}
\usepackage{pst-plot}
\usepackage{psfrag}

% define commands here
\newtheorem{prop}{Proposition}
\newtheorem{thm}{Theorem}
\newtheorem{ex}{Example}
\newcommand{\real}{\mathbb{R}}
\newcommand{\pdiff}[2]{\frac{\partial #1}{\partial #2}}
\newcommand{\mpdiff}[3]{\frac{\partial^#1 #2}{\partial #3^#1}}
\begin{document}
Let $A$ be an idempotent semiring with a unary operator $*$ on $A$.  The following are equivalent
\begin{enumerate}
\item $ac+b\le c$ implies $a^*b\le c$,
\item $ab\le b$ implies $a^*b\le b$.
\end{enumerate}
\begin{proof} $(1\Rightarrow 2)$.  Assume $ab\le b$.  So $ab+b=b$.  Then $(ab+b)+b=ab+(b+b)=ab+b=b$, which implies $ab+b\le b$.  By $1$, this means $a^*b\le b$ as desired.  $(2\Rightarrow 1)$.  Assume $ac+b\le c$.  Since $0\le b$, we get $ac=ac+0\le ac+b\le c$.  Consequently $a^*c \le c$.
\end{proof}

From the above observation, we see that we get an equivalent definition of a Kleene algebra if the axioms
$$ac+b\le c\mbox{ implies }a^*b\le c\qquad \mbox{and}\qquad ca+b\le c\mbox{ implies }ba^*\le c$$
are replaced by
$$ab\le b\mbox{ implies }a^*b\le b\qquad \mbox{and}\qquad ba\le b\mbox{ implies }ba^*\le b.$$

Let $A$ be a Kleene algebra.  Some of the interesting properties of $^*$ on $A$ are the following:
\begin{enumerate}
\begin{multicols}{2}{
\item $0^*=1$.\\[-3ex]
\item $a^n\le a^*$ for all non-negative integers $n$.\\[-3ex]
\item $1^*a\le a$.\\[-3ex]
\item $1^*=1$.\\[-3ex]
\item $1+a^*=a^*$.\\[-3ex]
\item $a\le b$ implies $a^*\le b^*$.\\[-3ex]
\item $a^*a^*=a^*$.\\[-3ex]
\item $a^{**}=a^*$.\\[-3ex]
\item $1+aa^*=a^*$.\\[-3ex]
\item $1+a^*a=a^*$.\\[-3ex]
\item $ac\le cb$ implies $a^*c\le cb^*$.\\[-3ex]
\item $cb\le ac$ implies $cb^*\le a^*c$.\\[-3ex]
\item $(ab)^*a=a(ba)^*$.\\[-3ex]
\item $1+a(ba)^*b=(ab)^*$.\\[-3ex]
\item If $c=c^2$ and $1\le c$, then $c^*=c$.\\[-3ex]
\item $(a+b)^*=a^*(ba^*)^*$.\\[-3ex]
}\end{multicols}
\end{enumerate}

\textbf{Remarks}.  
\begin{itemize}
\item
Properties 9 and 10 imply that the axioms $1+aa^*\le a^*$ and $1+a^*a\le a^*$ of a Kleene algebra can be replaced by these stronger ones.
\item
Though in the example of Kleene algebras formed by regular expressions, $$a^*=\bigcup \lbrace a^i\mid i=0,1,2,\ldots \rbrace,$$ and we see that $a^*$ is the least upper bound of the $a^i$'s, this is not a property of a general Kleene algebra.  A counterexample of this can be found in Kozen's article, see below.  He calls a Kleene algebra $K$ \emph{$^*$-continuous} if $a^*\le b$ whenever $a^i\le b$ for any $i\in \mathbb{N}\cup\lbrace0\rbrace$, and $a,b\in K$.
\end{itemize}

\begin{thebibliography}{8}
\bibitem{dk} D. Kozen, {\em \PMlinkexternal{On Kleene Algebras and Closed Semirings}{http://www.cs.cornell.edu/~kozen/papers/kacs.ps}} (1990).
\end{thebibliography}


%%%%%
%%%%%
\end{document}
