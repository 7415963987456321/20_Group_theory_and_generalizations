\documentclass[12pt]{article}
\usepackage{pmmeta}
\pmcanonicalname{SemisimpleGroup}
\pmcreated{2013-03-22 13:17:07}
\pmmodified{2013-03-22 13:17:07}
\pmowner{Algeboy}{12884}
\pmmodifier{Algeboy}{12884}
\pmtitle{semisimple group}
\pmrecord{6}{33771}
\pmprivacy{1}
\pmauthor{Algeboy}{12884}
\pmtype{Definition}
\pmcomment{trigger rebuild}
\pmclassification{msc}{20D05}
\pmrelated{socle}
\pmdefines{semi-simple group}
\pmdefines{semisimple group}

\usepackage{latexsym}
\usepackage{amssymb}
\usepackage{amsmath}
\usepackage{amsfonts}
\usepackage{amsthm}

%%\usepackage{xypic}

%-----------------------------------------------------

%       Standard theoremlike environments.

%       Stolen directly from AMSLaTeX sample

%-----------------------------------------------------

%% \theoremstyle{plain} %% This is the default

\newtheorem{thm}{Theorem}

\newtheorem{coro}[thm]{Corollary}

\newtheorem{lem}[thm]{Lemma}

\newtheorem{lemma}[thm]{Lemma}

\newtheorem{prop}[thm]{Proposition}

\newtheorem{conjecture}[thm]{Conjecture}

\newtheorem{conj}[thm]{Conjecture}

\newtheorem{defn}[thm]{Definition}

\newtheorem{remark}[thm]{Remark}

\newtheorem{ex}[thm]{Example}



%\countstyle[equation]{thm}



%--------------------------------------------------

%       Item references.

%--------------------------------------------------


\newcommand{\exref}[1]{Example-\ref{#1}}

\newcommand{\thmref}[1]{Theorem-\ref{#1}}

\newcommand{\defref}[1]{Definition-\ref{#1}}

\newcommand{\eqnref}[1]{(\ref{#1})}

\newcommand{\secref}[1]{Section-\ref{#1}}

\newcommand{\lemref}[1]{Lemma-\ref{#1}}

\newcommand{\propref}[1]{Prop\-o\-si\-tion-\ref{#1}}

\newcommand{\corref}[1]{Cor\-ol\-lary-\ref{#1}}

\newcommand{\figref}[1]{Fig\-ure-\ref{#1}}

\newcommand{\conjref}[1]{Conjecture-\ref{#1}}


% Normal subgroup or equal.

\providecommand{\normaleq}{\unlhd}

% Normal subgroup.

\providecommand{\normal}{\lhd}

\providecommand{\rnormal}{\rhd}
% Divides, does not divide.

\providecommand{\divides}{\mid}

\providecommand{\ndivides}{\nmid}


\providecommand{\union}{\cup}

\providecommand{\bigunion}{\bigcup}

\providecommand{\intersect}{\cap}

\providecommand{\bigintersect}{\bigcap}










\begin{document}
In group theory the use of the phrase \emph{semi-simple group} is used sparingly.
Standard texts on group theory including \cite{Aschbacher, Gorenstein} avoid
the term altogether.  Other texts provide precise definitions which are nevertheless 
not equivalent \cite{Robinson, Suzuki}.  In general it is preferable to use 
other terms to describe the class of groups being considered as there is 
no uniform convention.  However, below is a list of possible uses of for
the phrase \emph{semi-simple group}.

\begin{enumerate}
\item A group is \emph{semi-simple} if it has no non-trivial normal abelian 
subgroups \cite[p. 89]{Robinson}.

\item A group $G$ is \emph{semi-simple} if $G'=G$ and $G/Z(G)$ is a direct 
product of non-abelian simple groups \cite[Def. 6.1]{Suzuki}.

\item A product of simple groups may be called \emph{semi-simple}.  Depending
on application, the simple groups may be further restricted to finite simple groups
and may also exclude the abelian simple groups.

\item A Lie group whose associated Lie algebra is a semi-simple Lie algebra may
be called a \emph{semi-simple} group and more specifically, a 
\emph{semi-simple Lie group}.
\end{enumerate}


\textbf{Connections with algebra}

The use of \emph{semi-simple} in the study of algebras, representation theory, and modules
is far more precise owing to the fact that the various possible definitions are generally 
equivalent.

For example.  In a finite dimensional associative algebra $A$, if $A$ it is a product of 
simple algebras then the Jacobson radical is trivial.  In contrast, if $A$ has trivial
Jacobson radical then it is a direct product of simple algebras.  Thus $A$ may be 
called \emph{semi-simple} if either: $A$ is a direct product of simple algebras or
$A$ has trivial Jacobson radical.


The analogue fails for groups.  For instance.  If a group is defined as semi-simple
by virtue of having no non-trivial normal abelian subgroups then $S_n$ is semi-simple
for all $n>5$.  However, $S_n$ is not a product of simple groups.



\bibliographystyle{amsplain}
\begin{thebibliography}{10}

\bibitem{Aschbacher}
Aschbacher, M. 
\emph{Finite group theory} Cambridge studies in advanced mathematics 10, 
Cambridge University Press, Cambridge, (1986).

\bibitem{Gorenstein}
Gorenstein, D.
\emph{Finite groups} Chelsea Publishing Company, New York, (1980).

\bibitem{Robinson}
Robinson, D. J.S.
\emph{A course in the theory of groups} Ed. 2, GTM 80, Springer, New York, (1996).

\bibitem{Suzuki}
Suzuki, M.
\emph{Group Theory I,II}, (English) Springer-verlag, Berlin (1982, 1986).


\end{thebibliography}
%%%%%
%%%%%
\end{document}
