\documentclass[12pt]{article}
\usepackage{pmmeta}
\pmcanonicalname{AbsorbingElement}
\pmcreated{2013-03-22 15:46:12}
\pmmodified{2013-03-22 15:46:12}
\pmowner{pahio}{2872}
\pmmodifier{pahio}{2872}
\pmtitle{absorbing element}
\pmrecord{20}{37727}
\pmprivacy{1}
\pmauthor{pahio}{2872}
\pmtype{Definition}
\pmcomment{trigger rebuild}
\pmclassification{msc}{20N02}
\pmsynonym{absorbant}{AbsorbingElement}
\pmsynonym{absorbing}{AbsorbingElement}
\pmrelated{RingOfSets}
\pmrelated{ZeroElements}
\pmrelated{0cdotA0}
\pmrelated{AbsorbingSet}
\pmrelated{IdentityElementIsUnique}

% this is the default PlanetMath preamble.  as your knowledge
% of TeX increases, you will probably want to edit this, but
% it should be fine as is for beginners.

% almost certainly you want these
\usepackage{amssymb}
\usepackage{amsmath}
\usepackage{amsfonts}

% used for TeXing text within eps files
%\usepackage{psfrag}
% need this for including graphics (\includegraphics)
%\usepackage{graphicx}
% for neatly defining theorems and propositions
 \usepackage{amsthm}
% making logically defined graphics
%%%\usepackage{xypic}

% there are many more packages, add them here as you need them

% define commands here

\theoremstyle{definition}
\newtheorem*{thmplain}{Theorem}
\begin{document}
An element $\zeta$ of a groupoid\, $(G,\,*)$\, is called an {\em absorbing element} (in French {\em un \'el\'ement absorbant}) for the operation ``$*$'', if it satisfies
    $$\zeta\!*\!a \;=\; a\!*\!\zeta \;=\; \zeta$$
for all elements $a$ of $G$.\\

\textbf{Examples}
\begin{itemize}
 \item The zero $0$ is the absorbing element for multiplication (or {\em multiplicatively absorbing}) in every ring\, $(R,\,+,\,\cdot)$.
 \item The zero ideal $(0)$ is absorbing for \PMlinkname{ideal multiplication}{IdealMultiplicationLaws}.
 \item The zero vector $\vec{0}$ is the absorbing element for the \PMlinkname{vectoral multiplication}{CrossProduct} ``$\times$''.
 \item The empty set $\varnothing$ is the absorbing element for the  intersection operation ``$\cap$'' and also for the Cartesian product ``$\times$''.
 \item The ``universal set'' $E$ is the absorbing element for the union operation ``$\cup$'':
             $$E\!\cup\!A \;=\; A\!\cup\!E \;=\; E \quad \forall A \;\subseteq\; E.$$
 \item In an upper semilattice, an element is absorbing iff it is the \PMlinkname{top element}{BoundedLattice}.  Dually, an element is absorbing iff it is the \PMlinkname{bottom element}{BoundedLattice} in a lower semilattice.
\end{itemize} 

As the examples give reason to believe, the absorbing element for an operation is always unique.\, Indeed, if in \PMlinkescapetext{addition} to $\zeta$ we have in $G$ another absorbing element $\eta$, then we must have\, $\eta = \zeta\!*\!\eta = \zeta$.

Because\, $\zeta\!*\!\zeta = \zeta$,\, the absorbing element is idempotent.

If a group has an absorbing element, the group is \PMlinkname{trivial}{Subgroup}.
%%%%%
%%%%%
\end{document}
