\documentclass[12pt]{article}
\usepackage{pmmeta}
\pmcanonicalname{ProofOfFirstIsomorphismTheorem}
\pmcreated{2013-03-22 12:39:19}
\pmmodified{2013-03-22 12:39:19}
\pmowner{uriw}{288}
\pmmodifier{uriw}{288}
\pmtitle{proof of first isomorphism theorem}
\pmrecord{9}{32922}
\pmprivacy{1}
\pmauthor{uriw}{288}
\pmtype{Proof}
\pmcomment{trigger rebuild}
\pmclassification{msc}{20A05}

% this is the default PlanetMath preamble.  as your knowledge
% of TeX increases, you will probably want to edit this, but
% it should be fine as is for beginners.

% almost certainly you want these
\usepackage{amssymb}
\usepackage{amsmath}
\usepackage{amsfonts}

% used for TeXing text within eps files
%\usepackage{psfrag}
% need this for including graphics (\includegraphics)
%\usepackage{graphicx}
% for neatly defining theorems and propositions
%\usepackage{amsthm}
% making logically defined graphics
%%\usepackage{xypic}

% there are many more packages, add them here as you need them

% define commands here
\begin{document}
The proof consist of several parts which we will give for
completeness. Let $K$ denote $\ker f$. The following calculation
validates that for every $g\in G$ and $k \in K$:
\[
\begin{array}{llll}
  f(gkg^{-1}) &=& f(g) \, f(k) \, f(g)^{-1} & \text{($f$ is an homomorphism)}\\
  &=& f(g) \, 1_H \, f(g)^{-1} & \text{(definition of $K$)}\\
  &=& 1_H &
\end{array}
\]
Hence, $gkg^{-1}$ is in $K$. Therefore, $K$ is a normal subgroup
of $G$ and $G/K$ is well-defined.

To prove the theorem we will define a map from $G/K$ to the image
of $f$ and show that it is a function, a homomorphism and finally
an isomorphism.

Let $\theta\colon G/K \to \operatorname{Im} f$ be a map that sends the coset
$gK$ to $f(g)$.

Since $\theta$ is defined on representatives we need to show that
it is well defined. So, let $g_1$ and $g_2$ be two elements of $G$
that belong to the same coset (i.e. $g_1K = g_2K$). Then,
$g_1^{-1}g_2$ is an element of $K$ and therefore
$f(g_1^{-1}g_2)=1$ (because $K$ is the kernel of $G$). Now, the
rules of homomorphism show that $f(g_1)^{-1}f(g_2)=1$ and that is
equivalent to $f(g_1)=f(g_2)$ which implies the equality
$\theta(g_1K)=\theta(g_2K)$.

Next we verify that $\theta$ is a homomorphism. Take two cosets
$g_1K$ and $g_2K$, then:
\[
\begin{array}{llll}
  \theta (g_1K\cdot g_2K) &=& \theta(g_1g_2K) & \text{(operation in $G/K$)}\\
  &=& f(g_1g_2) & \text{(definition of $\theta$)}\\
  &=& f(g_1)f(g_2) & \text{($f$ is an homomorphism)}\\
  &=& \theta (g_1K)\theta (g_2K)\ & \text{(definition of $\theta$)}
\end{array}
\]

Finally, we show that $\theta$ is an isomorphism (i.e. a
bijection). The kernel of $\theta$ consists of all cosets $gK$ in
$G/K$ such that $f(g)=1$ but these are exactly the elements $g$
that belong to $K$ so only the coset $K$ is in the kernel of
$\theta$ which implies that $\theta$ is an injection. Let $h$
be an element of $\operatorname{Im} f$ and $g$ its pre-image. Then,
$\theta(gK)$ equals $f(g)$ thus $\theta(gK)=h$ and therefore
$\theta$ is surjective.

The theorem is proved. Some version of the theorem also states that
the following diagram is commutative:
\[
\xymatrix{
G \ar[rd]_{f} \ar[r]^{\pi} & G/K \ar[d]_{\theta} \\
& H }
\]
were $\pi$ is the natural projection that takes $g\in G$ to $gK$.
We will conclude by verifying this. Take $g$ in $G$ then,
$\theta(\pi(g))=\theta(gK)=f(g)$ as needed.
%%%%%
%%%%%
\end{document}
