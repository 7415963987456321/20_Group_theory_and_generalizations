\documentclass[12pt]{article}
\usepackage{pmmeta}
\pmcanonicalname{CompletelySimpleSemigroup}
\pmcreated{2013-03-22 14:35:24}
\pmmodified{2013-03-22 14:35:24}
\pmowner{mathcam}{2727}
\pmmodifier{mathcam}{2727}
\pmtitle{completely simple semigroup}
\pmrecord{8}{36153}
\pmprivacy{1}
\pmauthor{mathcam}{2727}
\pmtype{Definition}
\pmcomment{trigger rebuild}
\pmclassification{msc}{20M10}
\pmdefines{primitive}
\pmdefines{completely $0$-simple}
\pmdefines{completely simple}

\endmetadata

% this is the default PlanetMath preamble.  as your knowledge
% of TeX increases, you will probably want to edit this, but
% it should be fine as is for beginners.

% almost certainly you want these
\usepackage{amssymb}
\usepackage{amsmath}
\usepackage{amsfonts}

% used for TeXing text within eps files
%\usepackage{psfrag}
% need this for including graphics (\includegraphics)
%\usepackage{graphicx}
% for neatly defining theorems and propositions
\usepackage{amsthm}
% making logically defined graphics
%%%\usepackage{xypic}

% there are many more packages, add them here as you need them

% define commands here
\begin{document}
Let $S$ be a semigroup.  An idempotent $e\in S$ is \emph{primitive} if for every other idempotent $f\in S$, $ef=fe=f\not= 0\Rightarrow e=f$

A semigroup $S$ (without zero) is \emph{completely \PMlinkescapetext{simple}} if it is simple and contains a primitive idempotent.

A semigroup $S$ is \emph{completely $0$-simple} if it is \PMlinkname{$0$-simple}{SimpleSemigroup} and contains a primitive idempotent.

Completely simple and completely $0$-simple semigroups maybe characterised by the Rees Theorem (\cite{ReesRef}, Theorem 3.2.3).

Note:

A semigroup (without zero) is completely simple if and only if it is regular and weakly cancellative.

A simple semigroup (without zero) is completely simple if and only if it is completely regular.

A $0$-simple semigroup is completely $0$-simple if and only if it is group-bound.

\begin{thebibliography}{2}
\bibitem[Ho95]{ReesRef} Howie, John M.  \emph{Fundamentals of Semigroup Theory}.  Oxford University Press, 1995.
\end{thebibliography}
%%%%%
%%%%%
\end{document}
