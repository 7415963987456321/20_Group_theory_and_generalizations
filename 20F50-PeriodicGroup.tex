\documentclass[12pt]{article}
\usepackage{pmmeta}
\pmcanonicalname{PeriodicGroup}
\pmcreated{2013-03-22 15:35:50}
\pmmodified{2013-03-22 15:35:50}
\pmowner{yark}{2760}
\pmmodifier{yark}{2760}
\pmtitle{periodic group}
\pmrecord{11}{37511}
\pmprivacy{1}
\pmauthor{yark}{2760}
\pmtype{Definition}
\pmcomment{trigger rebuild}
\pmclassification{msc}{20F50}
\pmsynonym{torsion group}{PeriodicGroup}
\pmrelated{LocallyFiniteGroup}
\pmrelated{Torsion3}
\pmdefines{periodic}
\pmdefines{torsion}

\endmetadata

\usepackage{amssymb}
\usepackage{amsmath}
\usepackage{amsthm}

\newtheorem{corollary}{Corollary}
\newtheorem{theorem}{Theorem}

\def\Z{\mathbb{Z}}

\DeclareMathOperator{\Tor}{Tor}
\begin{document}
\PMlinkescapeword{coordinate}
\PMlinkescapeword{obvious}
\PMlinkescapeword{subgroup}

A group $G$ is said to be \emph{periodic} (or \emph{torsion})
if every element of $G$ is of finite order.

All finite groups are periodic.
More generally, all locally finite groups are periodic.
Examples of periodic groups that are not locally finite include Tarski groups,
and Burnside groups $B(m,n)$ of odd exponent $n\ge665$ on $m>1$ generators.

Some easy results on periodic groups:

\begin{theorem}
\item Every \PMlinkname{subgroup}{Subgroup} of a periodic group is periodic.
\end{theorem}

\begin{theorem}
\item Every \PMlinkname{quotient}{QuotientGroup} of a periodic group is periodic.
\end{theorem}

\begin{theorem}
\item Every \PMlinkname{extension}{GroupExtension} of a periodic group by a periodic group is periodic.
\end{theorem}

\begin{theorem}
\item Every restricted direct product of periodic groups is periodic.
\end{theorem}

Note that (unrestricted) direct products of periodic groups are not necessarily periodic. For example, the direct product of all finite cyclic groups $\Z/n\Z$ is not periodic, as the element that is $1$ in every coordinate has infinite order.

Some further results on periodic groups:

\begin{theorem}
Every solvable periodic group is locally finite.
\end{theorem}

\begin{theorem}
Every periodic abelian group is the direct sum of its maximal \PMlinkname{$p$-groups}{PGroup4} over all primes $p$.
\end{theorem}
%%%%%
%%%%%
\end{document}
