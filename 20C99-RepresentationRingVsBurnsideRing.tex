\documentclass[12pt]{article}
\usepackage{pmmeta}
\pmcanonicalname{RepresentationRingVsBurnsideRing}
\pmcreated{2013-03-22 19:19:05}
\pmmodified{2013-03-22 19:19:05}
\pmowner{joking}{16130}
\pmmodifier{joking}{16130}
\pmtitle{representation ring vs burnside ring}
\pmrecord{4}{42256}
\pmprivacy{1}
\pmauthor{joking}{16130}
\pmtype{Theorem}
\pmcomment{trigger rebuild}
\pmclassification{msc}{20C99}

% this is the default PlanetMath preamble.  as your knowledge
% of TeX increases, you will probably want to edit this, but
% it should be fine as is for beginners.

% almost certainly you want these
\usepackage{amssymb}
\usepackage{amsmath}
\usepackage{amsfonts}

% used for TeXing text within eps files
%\usepackage{psfrag}
% need this for including graphics (\includegraphics)
%\usepackage{graphicx}
% for neatly defining theorems and propositions
%\usepackage{amsthm}
% making logically defined graphics
%%%\usepackage{xypic}

% there are many more packages, add them here as you need them

% define commands here

\begin{document}
Let $G$ be a finite group and let $k$ be any field. If $X$ is a $G$-set, then we may consider the vector space $V_k(X)$ over $k$ which has $X$ as a basis. In this manner $V_k(X)$ becomes a representation of $G$ via action induced from $X$ and linearly extended to $V_k(X)$. It can be shown that $V_k(X)$ only depends on the isomorphism class of $X$, so we have a well-defined mapping:
$$[X]\mapsto [V_k(X)]$$
which can be easily extended to the function
$$\beta:\Omega(G)\to R_k(G);$$
$$\beta([X])=[V_k(X)]$$
where on the left side we have the Burnside ring and on the right side the representation ring. It can be shown, that $\beta$ is actually a ring homomorphism, but in most cases it neither injective nor surjective. But the following theorem due to Segal gives us some properties of $\beta$:

\textbf{Theorem (Segal).} Let $\beta:\Omega(G)\to R_{\mathbb{Q}}(G)$ be defined as above with rationals as the underlying field. If $G$ is a $p$-group for some prime number $p$, then $\beta$ is surjective. Furthermore $\beta$ is an isomorphism if and only if $G$ is cyclic.
%%%%%
%%%%%
\end{document}
