\documentclass[12pt]{article}
\usepackage{pmmeta}
\pmcanonicalname{RepresentationsVsModules}
\pmcreated{2013-03-22 19:18:59}
\pmmodified{2013-03-22 19:18:59}
\pmowner{joking}{16130}
\pmmodifier{joking}{16130}
\pmtitle{representations vs modules}
\pmrecord{4}{42254}
\pmprivacy{1}
\pmauthor{joking}{16130}
\pmtype{Definition}
\pmcomment{trigger rebuild}
\pmclassification{msc}{20C99}

% this is the default PlanetMath preamble.  as your knowledge
% of TeX increases, you will probably want to edit this, but
% it should be fine as is for beginners.

% almost certainly you want these
\usepackage{amssymb}
\usepackage{amsmath}
\usepackage{amsfonts}

% used for TeXing text within eps files
%\usepackage{psfrag}
% need this for including graphics (\includegraphics)
%\usepackage{graphicx}
% for neatly defining theorems and propositions
%\usepackage{amsthm}
% making logically defined graphics
%%%\usepackage{xypic}

% there are many more packages, add them here as you need them

% define commands here

\begin{document}
Let $G$ be a group and $k$ a field. Recall that a pair $(V, \cdot)$ is a representation of $G$ over $k$, if $V$ is a vector space over $k$ and $\cdot:G\times V\to V$ is a linear group action (compare with parent object). On the other hand we have a group algebra $kG$, which is a vector space over $k$ with $G$ as a basis and the multiplication is induced from the multiplication in $G$. Thus we can consider modules over $kG$. These two concepts are related.

If $\mathbb{V}=(V,\cdot)$ is a representation of $G$ over $k$, then define a $kG$-module $\overline{\mathbb{V}}$ by putting $\overline{\mathbb{V}}=V$ as a vector space over $k$ and the action of $kG$ on $\overline{\mathbb{V}}$ is given by
$$(\sum \lambda_ig_i)\circ v = \sum \lambda_i(g_i\cdot v).$$
It can be easily checked that $\overline{\mathbb{V}}$ is indeed a $kG$-module.

Analogously if $M$ is a $kG$-module (with action denoted by ,,$\circ$''), then the pair $\underline{M}=(M,\cdot)$ is a representation of $G$ over $k$, where ,,$\cdot$'' is given by
$$g\cdot v =g\circ v.$$

As a simple exercise we leave the following proposition to the reader:

\textbf{Proposition.} Let $\mathbb{V}$ be a representation of $G$ over $k$ and let $M$ be a $kG$-module. Then
$$\underline{\overline{\mathbb{V}}}=\mathbb{V};$$
$$\overline{\underline{M}}=M.$$

This means that modules and representations are the same concept. One can generalize this even further by showing that $\overline{\cdot}$ and $\underline{\cdot}$ are both functors, which are (mutualy invert) isomorphisms of appropriate categories.

Therefore we can easily define such concepts as ,,direct sum of representations'' or ,,tensor product of representations'', etc.
%%%%%
%%%%%
\end{document}
