\documentclass[12pt]{article}
\usepackage{pmmeta}
\pmcanonicalname{GroupSocle}
\pmcreated{2013-03-22 15:55:12}
\pmmodified{2013-03-22 15:55:12}
\pmowner{Algeboy}{12884}
\pmmodifier{Algeboy}{12884}
\pmtitle{group socle}
\pmrecord{14}{37925}
\pmprivacy{1}
\pmauthor{Algeboy}{12884}
\pmtype{Definition}
\pmcomment{trigger rebuild}
\pmclassification{msc}{20E34}
\pmsynonym{socle}{GroupSocle}
%\pmkeywords{characteristically simple}
\pmdefines{socle}

\endmetadata

\usepackage{latexsym}
\usepackage{amssymb}
\usepackage{amsmath}
\usepackage{amsfonts}
\usepackage{amsthm}

%%\usepackage{xypic}

%-----------------------------------------------------

%       Standard theoremlike environments.

%       Stolen directly from AMSLaTeX sample

%-----------------------------------------------------

%% \theoremstyle{plain} %% This is the default

\newtheorem{thm}{Theorem}

\newtheorem{coro}[thm]{Corollary}

\newtheorem{lem}[thm]{Lemma}

\newtheorem{lemma}[thm]{Lemma}

\newtheorem{prop}[thm]{Proposition}

\newtheorem{conjecture}[thm]{Conjecture}

\newtheorem{conj}[thm]{Conjecture}

\newtheorem{defn}[thm]{Definition}

\newtheorem{remark}[thm]{Remark}

\newtheorem{ex}[thm]{Example}



%\countstyle[equation]{thm}



%--------------------------------------------------

%       Item references.

%--------------------------------------------------


\newcommand{\exref}[1]{Example-\ref{#1}}

\newcommand{\thmref}[1]{Theorem-\ref{#1}}

\newcommand{\defref}[1]{Definition-\ref{#1}}

\newcommand{\eqnref}[1]{(\ref{#1})}

\newcommand{\secref}[1]{Section-\ref{#1}}

\newcommand{\lemref}[1]{Lemma-\ref{#1}}

\newcommand{\propref}[1]{Prop\-o\-si\-tion-\ref{#1}}

\newcommand{\corref}[1]{Cor\-ol\-lary-\ref{#1}}

\newcommand{\figref}[1]{Fig\-ure-\ref{#1}}

\newcommand{\conjref}[1]{Conjecture-\ref{#1}}


% Normal subgroup or equal.

\providecommand{\normaleq}{\unlhd}

% Normal subgroup.

\providecommand{\normal}{\lhd}

\providecommand{\rnormal}{\rhd}
% Divides, does not divide.

\providecommand{\divides}{\mid}

\providecommand{\ndivides}{\nmid}


\providecommand{\union}{\cup}

\providecommand{\bigunion}{\bigcup}

\providecommand{\intersect}{\cap}

\providecommand{\bigintersect}{\bigcap}










\begin{document}
The \emph{socle} of a group is the subgroup generated by all minimal normal subgroups.  
Because the product of normal subgroups is a subgroup, it follows we can remove the word ``generated'' and replace it by ``product.''  So the socle of a group is now the product of its minimal normal subgroups.  This description can be further refined with a few observations.

\begin{prop}
If $M$ and $N$ are minimal normal subgroups then $M$ and $N$ centralize 
each other.
\end{prop}
\begin{proof}
Given two distinct minimal normal subgroup $M$ and $N$, $[M,N]$ is contained in $N$ and $M$ as both are normal.  Thus $[M,N]\leq M\cap N$.  But $M$ and $N$ are distinct minimal normal subgroups and $M\cap N$ is normal so $M\cap N=1$ thus $[M,N]=1$. 
\end{proof}

\begin{prop}
The socle of a finite group is a direct product of minimal normal subgroups.
\end{prop}
\begin{proof}
Let $S$ be the socle of $G$.  We already know $S$ is the product of its minimal normal subgroups, so let us assume $S=N_1\cdots N_k$ where each $N_i$ is a distinct minimal normal subgroup of $G$.  Thus $N_1\intersect N_2=1$ and
$N_1 N_2$ clearly contains $N_1$ and $N_2$.  Now suppose we extend this to a
subsquence $N_{i_1}=N_1, N_{i_2}=N_2,N_{i_3},\dots,N_{i_j}$ where
\[N_{i_{k}}\intersect(N_{i_1}\cdots N_{i_{k-1}})=1\]
for $1\leq k<j$ and $N_i\leq N_{i_1}\cdots N_{i_j}$ for all $1\leq i\leq i_j$.
Then consider $N_{i_j +1}$.  

As $N_{i_j +1}$ is a minimal normal subgroup and $N_{i_1}\cdots N_{i_j}$ is a
normal subgroup, $N_{i_j +1}$ is either contained in $N_{i_1}\cdots N_{i_j}$
or intersects trivially.  If $N_{i_j +1}$ is contained in $N_{i_1}\cdots N_{i_j}$ then skip to the next $N_i$, otherwise set it to be $N_{i_{j+1}}$.
The result is a squence $N_{i_1},\dots,N_{i_j}$ of minimal normal subgroups
where $S=N_{i_1}\cdots N_{i_s}$ and
\[N_{i_j}\intersect(N_{i_1}\cdots N_{i_{j-1}})=1,\quad 1\leq j\leq s.\]
As we have already seen distinct minimal normal subgroups centralize each other
we conclude that $S=N_{i_1}\times\cdots \times N_{i_s}$.
\end{proof}

\begin{prop}
A minimal normal subgroup is characteristically simple, so if it is finite then it is a product of isomorphic simple groups.
\end{prop}
\begin{proof}
If $M$ is a minimal normal subgroup of $G$ and $1<C<M$ is characteristic in $M$, then $C$ is normal in $G$ which contradicts the minimality of $M$.  Thus $M$ is characteristically simple.
\end{proof}

\begin{coro}
The socle of a finite group is a direct product of simple groups.
\end{coro}
\begin{proof}
As each $N_{i_j}$ is characteristically simple each $N_{i_j}$ is a direct product of isomorphic simple groups, thus $S$ is a direct product simple groups.
\end{proof}

%%%%%
%%%%%
\end{document}
