\documentclass[12pt]{article}
\usepackage{pmmeta}
\pmcanonicalname{GroupRing}
\pmcreated{2013-03-22 12:13:27}
\pmmodified{2013-03-22 12:13:27}
\pmowner{djao}{24}
\pmmodifier{djao}{24}
\pmtitle{group ring}
\pmrecord{8}{31595}
\pmprivacy{1}
\pmauthor{djao}{24}
\pmtype{Definition}
\pmcomment{trigger rebuild}
\pmclassification{msc}{20C05}
\pmclassification{msc}{20C07}
\pmclassification{msc}{16S34}
\pmdefines{group algebra}

% this is the default PlanetMath preamble.  as your knowledge
% of TeX increases, you will probably want to edit this, but
% it should be fine as is for beginners.

% almost certainly you want these
\usepackage{amssymb}
\usepackage{amsmath}
\usepackage{amsfonts}

% used for TeXing text within eps files
%\usepackage{psfrag}
% need this for including graphics (\includegraphics)
%\usepackage{graphicx}
% for neatly defining theorems and propositions
%\usepackage{amsthm}
% making logically defined graphics
%%%%\usepackage{xypic} 

% there are many more packages, add them here as you need them

% define commands here
\begin{document}
For any group $G$, the {\em group ring} $\mathbb{Z}[G]$ is defined to be the ring whose additive group is the abelian group of formal integer linear combinations of elements of $G$, and whose multiplication operation is defined by multiplication in $G$, extended $\mathbb{Z}$--linearly to $\mathbb{Z}[G]$.

More generally, for any ring $R$, the {\em group ring} of $G$ over $R$ is the ring $R[G]$ whose additive group is the abelian group of formal $R$--linear combinations of elements of $G$, i.e.:
$$
R[G] := \left\{\left. \sum_{i=1}^n r_i g_i\ \right|\ r_i \in R,\ g_i \in G\right\},
$$
and whose multiplication operation is defined by $R$--linearly extending the group multiplication operation of $G$. In the case where $K$ is a field, the group ring $K[G]$ is usually called a {\em group algebra}.
%%%%%
%%%%%
%%%%%
\end{document}
