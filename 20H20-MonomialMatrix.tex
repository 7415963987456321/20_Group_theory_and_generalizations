\documentclass[12pt]{article}
\usepackage{pmmeta}
\pmcanonicalname{MonomialMatrix}
\pmcreated{2013-03-22 15:15:51}
\pmmodified{2013-03-22 15:15:51}
\pmowner{GrafZahl}{9234}
\pmmodifier{GrafZahl}{9234}
\pmtitle{monomial matrix}
\pmrecord{5}{37050}
\pmprivacy{1}
\pmauthor{GrafZahl}{9234}
\pmtype{Definition}
\pmcomment{trigger rebuild}
\pmclassification{msc}{20H20}
\pmclassification{msc}{15A30}
%\pmkeywords{matrix}
%\pmkeywords{monomial}
%\pmkeywords{group}
\pmrelated{PermutationMatrix}

% this is the default PlanetMath preamble.  as your knowledge
% of TeX increases, you will probably want to edit this, but
% it should be fine as is for beginners.

% almost certainly you want these
\usepackage{amssymb}
\usepackage{amsmath}
\usepackage{amsfonts}

% used for TeXing text within eps files
%\usepackage{psfrag}
% need this for including graphics (\includegraphics)
%\usepackage{graphicx}
% for neatly defining theorems and propositions
%\usepackage{amsthm}
% making logically defined graphics
%%%\usepackage{xypic}

% there are many more packages, add them here as you need them

% define commands here
\newcommand{\Prod}{\prod\limits}
\newcommand{\Sum}{\sum\limits}
\newcommand{\mbb}{\mathbb}
\newcommand{\mbf}{\mathbf}
\newcommand{\mc}{\mathcal}
\newcommand{\ol}{\overline}

% Math Operators/functions
\DeclareMathOperator{\Frob}{Frob}
\DeclareMathOperator{\cwe}{cwe}
\DeclareMathOperator{\we}{we}
\DeclareMathOperator{\wt}{wt}
\begin{document}
Let $A$ be a matrix with entries in a field $K$. If in every \PMlinkid{row}{2464} and every
\PMlinkid{column}{2464} of $A$ there is exactly one nonzero entry, then $A$ is a
\emph{monomial matrix}.

Obviously, a monomial matrix is a square matrix and there exists a
rearrangement of \PMlinkescapetext{rows} and \PMlinkescapetext{columns} such that the result is a diagonal
matrix.

The $n\times n$ monomial matrices form a group under matrix
multiplication. This group contains the $n\times n$ permutation
matrices as a subgroup. A monomial matrix is invertible but, unlike a
permutation matrix, not necessarily \PMlinkid{orthogonal}{1176}. The only exception is
when $K=\mbb{F}_2$ (the finite field with $2$ elements), where the
$n\times n$ monomial matrices and the $n\times n$ permutation matrices
coincide.
%%%%%
%%%%%
\end{document}
