\documentclass[12pt]{article}
\usepackage{pmmeta}
\pmcanonicalname{CommutativeSemigroup}
\pmcreated{2013-03-22 13:08:09}
\pmmodified{2013-03-22 13:08:09}
\pmowner{mclase}{549}
\pmmodifier{mclase}{549}
\pmtitle{commutative semigroup}
\pmrecord{4}{33573}
\pmprivacy{1}
\pmauthor{mclase}{549}
\pmtype{Definition}
\pmcomment{trigger rebuild}
\pmclassification{msc}{20M14}
\pmsynonym{Abelian semigroup}{CommutativeSemigroup}
\pmrelated{AbelianGroup}
\pmrelated{AbelianGroup2}
\pmdefines{commutative}
\pmdefines{commutative monoid}

\endmetadata

% this is the default PlanetMath preamble.  as your knowledge
% of TeX increases, you will probably want to edit this, but
% it should be fine as is for beginners.

% almost certainly you want these
\usepackage{amssymb}
\usepackage{amsmath}
\usepackage{amsfonts}

% used for TeXing text within eps files
%\usepackage{psfrag}
% need this for including graphics (\includegraphics)
%\usepackage{graphicx}
% for neatly defining theorems and propositions
%\usepackage{amsthm}
% making logically defined graphics
%%%\usepackage{xypic}

% there are many more packages, add them here as you need them

% define commands here
\begin{document}
\PMlinkescapeword{term} 
\PMlinkescapeword{identity} 

A semigroup $S$ is \emph{commutative} if the defining binary operation is \PMlinkname{commutative}{Commutative}.  That is, for all $x, y \in S$, the identity $xy = yx$ holds.

Although the term \emph{Abelian semigroup} is sometimes used, it is more common simply to refer to such semigroups as \emph{commutative semigroups}.

A monoid which is also a commutative semigroup is called a \emph{commutative monoid}.
%%%%%
%%%%%
\end{document}
