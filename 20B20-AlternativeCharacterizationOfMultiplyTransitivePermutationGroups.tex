\documentclass[12pt]{article}
\usepackage{pmmeta}
\pmcanonicalname{AlternativeCharacterizationOfMultiplyTransitivePermutationGroups}
\pmcreated{2013-03-22 17:21:47}
\pmmodified{2013-03-22 17:21:47}
\pmowner{rm50}{10146}
\pmmodifier{rm50}{10146}
\pmtitle{alternative characterization of multiply transitive permutation groups}
\pmrecord{4}{39723}
\pmprivacy{1}
\pmauthor{rm50}{10146}
\pmtype{Derivation}
\pmcomment{trigger rebuild}
\pmclassification{msc}{20B20}
\pmdefines{doubly transitive}

\endmetadata

% this is the default PlanetMath preamble.  as your knowledge
% of TeX increases, you will probably want to edit this, but
% it should be fine as is for beginners.

% almost certainly you want these
\usepackage{amssymb}
\usepackage{amsmath}
\usepackage{amsfonts}

% used for TeXing text within eps files
%\usepackage{psfrag}
% need this for including graphics (\includegraphics)
%\usepackage{graphicx}
% for neatly defining theorems and propositions
\usepackage{amsthm}
% making logically defined graphics
%%%\usepackage{xypic}

% there are many more packages, add them here as you need them

% define commands here
\newtheorem*{thm}{Theorem}
\begin{document}
This article derives an alternative characterization of $n$-transitive groups.

\begin{thm} For $n>1$, $G$ is $n$-transitive on $X$ if and only if for all $x\in X$, $G_x$ is $(n-1)$-transitive on $X-\{x\}$.
\end{thm}
\begin{proof}
First assume $G$ is $n$-transitive on $X$, and choose $x\in X$. To show $G_x$ is $(n-1)$-transitive on $X-\{x\}$, choose $x_1,\ldots,x_{n-1},y_1,\ldots,y_{n-1}\in X$. Since $G$ is $n$-transitive on $X$, we can choose $\sigma\in G$ such that
\[\sigma\cdot(x_1,\ldots,x_{n-1},x)=(y_1,\ldots,y_{n-1},x)\]
But obviously $\sigma\in G_x$, and $\sigma$ restricted to $X-\{x\}$ is the desired permutation.

To prove the converse, choose $x_1,\ldots,x_n,y_1,\ldots,y_n\in X$. Choose $\sigma_1\in G_{x_n}$ such that
\[\sigma_1\cdot(x_1,\ldots,x_{n-1})=(y_1,\ldots,y_{n-1})\]
and choose $\sigma_2\in G_{y_1}$ such that
\[\sigma_2\cdot(y_2,\ldots,y_{n-1},x_n)=(y_2,\ldots,y_{n-1},y_n)\]
Then $\sigma_2\sigma_1$ is the desired permutation.
\end{proof}

Note that this definition of $n$-transitivity affords a straightforward proof of the statement that $A_n$ is $(n-2)$-transitive: by inspection, $A_3$ is $1$-transitive; the result follows by induction using the theorem. (The corresponding statement that $S_n$ is $n$-transitive is obvious).

Finally, note that the most common cases of $n$-transitivity are for $n=1$ (\emph{transitive}), and $n=2$ (\emph{doubly transitive}).
%%%%%
%%%%%
\end{document}
