\documentclass[12pt]{article}
\usepackage{pmmeta}
\pmcanonicalname{DefectTheorem}
\pmcreated{2013-03-22 18:21:43}
\pmmodified{2013-03-22 18:21:43}
\pmowner{Ziosilvio}{18733}
\pmmodifier{Ziosilvio}{18733}
\pmtitle{defect theorem}
\pmrecord{6}{41000}
\pmprivacy{1}
\pmauthor{Ziosilvio}{18733}
\pmtype{Theorem}
\pmcomment{trigger rebuild}
\pmclassification{msc}{20M10}
\pmclassification{msc}{20M05}
\pmdefines{$x^m=y^n$ has only trivial solutions over $A^\ast$}

\endmetadata

% this is the default PlanetMath preamble.  as your knowledge
% of TeX increases, you will probably want to edit this, but
% it should be fine as is for beginners.

% almost certainly you want these
\usepackage{amssymb}
\usepackage{amsmath}
\usepackage{amsfonts}

% used for TeXing text within eps files
%\usepackage{psfrag}
% need this for including graphics (\includegraphics)
%\usepackage{graphicx}
% for neatly defining theorems and propositions
%\usepackage{amsthm}
% making logically defined graphics
%%%\usepackage{xypic}

% there are many more packages, add them here as you need them

% define commands here

\begin{document}
Let $A$ be an arbitrary set,
let $A^\ast$ be the free monoid on $A$,
and let $X$ be a finite subset of $A^\ast$ which is not a code.
Then the free hull $H$ of $X$ is itself finite
and $|H|<|X|$.

Observe how from the defect theorem
follows that the equation $x^m=y^n$ over $A^\ast$
has only the trivial solutions where $x$ and $y$ are powers of the same word:
in fact, $x^m=y^n$ iff $\{x,y\}$ is not a code.

\textit{Proof.}
It is sufficient to prove the thesis in the case
when $X$ does not contain the empty word on $A$.

Define $f:X\to H$ such that $f(x)$ is the only $h\in H$
such that $x\in hH^\ast$:
$f$ is well defined because of the choice of $X$ and $H$.
Since $X$ is finite, the thesis shall be established
once we prove that $f$ is surjective but not injective.

By hypothesis, $X$ is not a code.
Then there exists an equation $x_1\cdots x_n=x'_1\cdots x'_m$ over $X$
with a nontrivial solution:
it is not restrictive to suppose that $x_1\neq x'_1$.
Since however the factorization \emph{over $H$} is unique,
the $h$ such that $x_1\in hH^\ast$
is the same as the $h'$ such that $x'_1\in h'H^\ast$:
then $f(x_1)=f(x'_1)$ with $x_1\neq x'_1$,
so $f$ is not injective.

Now suppose, for the sake of contradiction,
that $h\in H\setminus f(X)$ exists.
Let $K=(H\setminus\{h\})h^\ast$.
Then any equation
\begin{equation} \label{eq:K}
k_1\cdots k_n=k'_1\cdots k'_m\;,k_1,\cdots,k_n,k'_1,\ldots,k'_m\in K
\end{equation}
can be rewritten as
\begin{displaymath}
h_1h^{r_1}\cdots h_nh^{r_n}=h'_1h^{s_1}\cdots h'_mh^{s_m}\;,
h_1,\ldots,h_n,h'_1,\ldots,h'_m\in H\setminus\{h\}\;,
\end{displaymath}
with $k_i=h_ih^{r_i}$, $k'_i=h'_ih^{s_i}$:
this is an equation over $H$,
and as such, has only the trivial solution $n=m$,
$h_i=h'_i$, $r_i=s_i$ for all $i$.
This implies that (\ref{eq:K}) only has trivial solutions over $K$:
by the characterization of free submonoids, $K$ is a code.
However, $X\subseteq K^\ast$ because no element of $x$ ``starts with $h$'',
and $K^\ast$ is a proper subset of $H^\ast$
because the former does not contain $h$ and the latter does.
Then $K^\ast$ is a free submonoid of $A^\ast$
which contains $X$ and is properly contained in $H^\ast$,
against the definition of $H$ as the free hull of $X$.

\begin{thebibliography}{99}

\bibitem{l97}
M. Lothaire.
\textit{Combinatorics on words.}
Cambridge University Press 1997.

\end{thebibliography}

%%%%%
%%%%%
\end{document}
