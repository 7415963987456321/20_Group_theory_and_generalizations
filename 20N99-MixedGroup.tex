\documentclass[12pt]{article}
\usepackage{pmmeta}
\pmcanonicalname{MixedGroup}
\pmcreated{2013-03-22 18:42:32}
\pmmodified{2013-03-22 18:42:32}
\pmowner{CWoo}{3771}
\pmmodifier{CWoo}{3771}
\pmtitle{mixed group}
\pmrecord{7}{41474}
\pmprivacy{1}
\pmauthor{CWoo}{3771}
\pmtype{Definition}
\pmcomment{trigger rebuild}
\pmclassification{msc}{20N99}
\pmdefines{kernel}

\endmetadata

\usepackage{amssymb,amscd}
\usepackage{amsmath}
\usepackage{amsfonts}
\usepackage{mathrsfs}

% used for TeXing text within eps files
%\usepackage{psfrag}
% need this for including graphics (\includegraphics)
%\usepackage{graphicx}
% for neatly defining theorems and propositions
\usepackage{amsthm}
% making logically defined graphics
%%\usepackage{xypic}
\usepackage{pst-plot}

% define commands here
\newcommand*{\abs}[1]{\left\lvert #1\right\rvert}
\newtheorem{prop}{Proposition}
\newtheorem{thm}{Theorem}
\newtheorem{ex}{Example}
\newcommand{\real}{\mathbb{R}}
\newcommand{\pdiff}[2]{\frac{\partial #1}{\partial #2}}
\newcommand{\mpdiff}[3]{\frac{\partial^#1 #2}{\partial #3^#1}}
\begin{document}
A \emph{mixed group} is a partial groupoid $G$ such that $G$ contains a non-empty subset $K$, called the \emph{kernel} of $G$, with the following conditions:
\begin{enumerate}
\item if $a,b\in G$, then $ab$ is defined iff $a\in K$,
\item if $a,b\in K$ and $c\in G$, then $(ab)c=a(bc)$,
\item if $a\in K$, then $K\subseteq aK\cap Ka$,
\item if $a\in K$ and $b\in G$ such that $ab=b$, then $ac=c$ for all $c\in G$.
\end{enumerate}

Mixed groups are generalizations of groups, as the following proposition illustrates:

\begin{prop} If $K=G$, then $G$ is a group. \end{prop}

\begin{proof}
$G$ is a groupoid by condition 1, and a semigroup by condition 2.  

Now, by condition 3, given $a\in G$, there is $b\in G$ such that $ba=a$, so that $bc=c$ for all $c\in G$ by condition 4.  In other words, $b$ is a left identity of $G$.  Again, by condition 3, for every $a\in G$, there is a $d\in G$ such that $b=da$.  So $ad= a(bd)=a(da)d=(ad)^2$, so, by condition 4, $adx=x$ for all $x\in G$.  In particular, set $x=a$, we get $a=(ad)a=a(da)=ab$.  Hence, $b$ is a two-sided identity, and $G$ is a monoid.  

Finally, by condition 3, for every $a\in G$, there are $c,d\in G$, such that $b=ac=da$.  So, $c=bc=(da)c=d(ac)=db=d$, showing that $a$ has a two-sided inverse.  This means that $G$ is a group.
\end{proof}

For a non-trivial example of a mixed group, let $G$ be a group and $H$ a subgroup of $G$.  Define a new multiplication $\cdot$ on $G$ as follows: $a\cdot b$ is defined iff $a\in H$, and if $a\cdot b$ is defined, it is defined as $ab$, the group multiplication of $a$ and $b$.  Then $(G,\cdot)$ is a mixed group.  Clearly, associativity of $\cdot$ is automatically satisfied.  Next, pick any $a\in H$, then, for any $b\in H$, $a^{-1}\cdot b$ and $b\cdot a^{-1}$ are both elements of $H$, so that $b\in a\cdot H\cap H\cdot a$, and condition 3 is also satisfied.  Finally, if $a\in H$ and $b\in G$ such that $a\cdot b=b$, then $a$ is the multiplicative identity of $G$, clearly $a\cdot c=c$ for all $c\in G$.

\begin{thebibliography}{0}
\bibitem{HB}
R. H. Bruck,
{\it A Survey of Binary Systems}, Springer-Verlag, 1966
\bibitem{HP}
R. Baer,
{\it Zur Einordnung der Theorie der Mischgruppen in die Gruppentheorie}, S.-B. Heidelberg. Akad. Wiss., Math.-naturwiss. KI. 1928, 4, 13 pp
\bibitem{HP}
R. Baer,
{\it \"{U}ber die Zerlegungen einer Mischgruppe nach einer Untermischgruppe}, S.-B. Heidelberg. Akad. Wiss., Math.-naturwiss. KI. 1928, 5, 13 pp
\bibitem{HP}
A. Loewy,
{\it \"{U}ber abstrakt definierte Transmutationssysteme oder Mischgruppen}, J. reine angew. Math. 157, pp 239-254, 1927
\end{thebibliography}
%%%%%
%%%%%
\end{document}
