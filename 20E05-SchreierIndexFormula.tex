\documentclass[12pt]{article}
\usepackage{pmmeta}
\pmcanonicalname{SchreierIndexFormula}
\pmcreated{2013-03-22 13:56:18}
\pmmodified{2013-03-22 13:56:18}
\pmowner{yark}{2760}
\pmmodifier{yark}{2760}
\pmtitle{Schreier index formula}
\pmrecord{14}{34699}
\pmprivacy{1}
\pmauthor{yark}{2760}
\pmtype{Theorem}
\pmcomment{trigger rebuild}
\pmclassification{msc}{20E05}
\pmrelated{ProofOfNielsenSchreierTheoremAndSchreierIndexFormula}

\usepackage{amssymb}
\usepackage{amsmath}
\usepackage{amsfonts}

\def\rank{\operatorname{rank}}
\begin{document}
\PMlinkescapeword{index}
\PMlinkescapeword{rank}
\PMlinkescapeword{states}
\PMlinkescapeword{subgroup}

Let $F$ be a free group of finite rank, and let $H$ be a \PMlinkname{subgroup}{Subgroup} of finite index in $F$.
By the Nielsen-Schreier theorem, $H$ is free.
The \emph{Schreier index formula} states that
\[\rank(H)=|F:H|\cdot(\rank(F)-1)+1.\]

This implies more generally that if $G$ is a group generated by $m$ elements, then any subgroup of index $n$ in $G$ can be generated by at most $nm-n+1$ elements.
%%%%%
%%%%%
\end{document}
